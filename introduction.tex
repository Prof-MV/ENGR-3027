% Options for packages loaded elsewhere
\PassOptionsToPackage{unicode}{hyperref}
\PassOptionsToPackage{hyphens}{url}
\documentclass[
]{book}
\usepackage{xcolor}
\usepackage{amsmath,amssymb}
\setcounter{secnumdepth}{5}
\usepackage{iftex}
\ifPDFTeX
  \usepackage[T1]{fontenc}
  \usepackage[utf8]{inputenc}
  \usepackage{textcomp} % provide euro and other symbols
\else % if luatex or xetex
  \usepackage{unicode-math} % this also loads fontspec
  \defaultfontfeatures{Scale=MatchLowercase}
  \defaultfontfeatures[\rmfamily]{Ligatures=TeX,Scale=1}
\fi
\usepackage{lmodern}
\ifPDFTeX\else
  % xetex/luatex font selection
\fi
% Use upquote if available, for straight quotes in verbatim environments
\IfFileExists{upquote.sty}{\usepackage{upquote}}{}
\IfFileExists{microtype.sty}{% use microtype if available
  \usepackage[]{microtype}
  \UseMicrotypeSet[protrusion]{basicmath} % disable protrusion for tt fonts
}{}
\makeatletter
\@ifundefined{KOMAClassName}{% if non-KOMA class
  \IfFileExists{parskip.sty}{%
    \usepackage{parskip}
  }{% else
    \setlength{\parindent}{0pt}
    \setlength{\parskip}{6pt plus 2pt minus 1pt}}
}{% if KOMA class
  \KOMAoptions{parskip=half}}
\makeatother
\usepackage{color}
\usepackage{fancyvrb}
\newcommand{\VerbBar}{|}
\newcommand{\VERB}{\Verb[commandchars=\\\{\}]}
\DefineVerbatimEnvironment{Highlighting}{Verbatim}{commandchars=\\\{\}}
% Add ',fontsize=\small' for more characters per line
\usepackage{framed}
\definecolor{shadecolor}{RGB}{248,248,248}
\newenvironment{Shaded}{\begin{snugshade}}{\end{snugshade}}
\newcommand{\AlertTok}[1]{\textcolor[rgb]{0.94,0.16,0.16}{#1}}
\newcommand{\AnnotationTok}[1]{\textcolor[rgb]{0.56,0.35,0.01}{\textbf{\textit{#1}}}}
\newcommand{\AttributeTok}[1]{\textcolor[rgb]{0.13,0.29,0.53}{#1}}
\newcommand{\BaseNTok}[1]{\textcolor[rgb]{0.00,0.00,0.81}{#1}}
\newcommand{\BuiltInTok}[1]{#1}
\newcommand{\CharTok}[1]{\textcolor[rgb]{0.31,0.60,0.02}{#1}}
\newcommand{\CommentTok}[1]{\textcolor[rgb]{0.56,0.35,0.01}{\textit{#1}}}
\newcommand{\CommentVarTok}[1]{\textcolor[rgb]{0.56,0.35,0.01}{\textbf{\textit{#1}}}}
\newcommand{\ConstantTok}[1]{\textcolor[rgb]{0.56,0.35,0.01}{#1}}
\newcommand{\ControlFlowTok}[1]{\textcolor[rgb]{0.13,0.29,0.53}{\textbf{#1}}}
\newcommand{\DataTypeTok}[1]{\textcolor[rgb]{0.13,0.29,0.53}{#1}}
\newcommand{\DecValTok}[1]{\textcolor[rgb]{0.00,0.00,0.81}{#1}}
\newcommand{\DocumentationTok}[1]{\textcolor[rgb]{0.56,0.35,0.01}{\textbf{\textit{#1}}}}
\newcommand{\ErrorTok}[1]{\textcolor[rgb]{0.64,0.00,0.00}{\textbf{#1}}}
\newcommand{\ExtensionTok}[1]{#1}
\newcommand{\FloatTok}[1]{\textcolor[rgb]{0.00,0.00,0.81}{#1}}
\newcommand{\FunctionTok}[1]{\textcolor[rgb]{0.13,0.29,0.53}{\textbf{#1}}}
\newcommand{\ImportTok}[1]{#1}
\newcommand{\InformationTok}[1]{\textcolor[rgb]{0.56,0.35,0.01}{\textbf{\textit{#1}}}}
\newcommand{\KeywordTok}[1]{\textcolor[rgb]{0.13,0.29,0.53}{\textbf{#1}}}
\newcommand{\NormalTok}[1]{#1}
\newcommand{\OperatorTok}[1]{\textcolor[rgb]{0.81,0.36,0.00}{\textbf{#1}}}
\newcommand{\OtherTok}[1]{\textcolor[rgb]{0.56,0.35,0.01}{#1}}
\newcommand{\PreprocessorTok}[1]{\textcolor[rgb]{0.56,0.35,0.01}{\textit{#1}}}
\newcommand{\RegionMarkerTok}[1]{#1}
\newcommand{\SpecialCharTok}[1]{\textcolor[rgb]{0.81,0.36,0.00}{\textbf{#1}}}
\newcommand{\SpecialStringTok}[1]{\textcolor[rgb]{0.31,0.60,0.02}{#1}}
\newcommand{\StringTok}[1]{\textcolor[rgb]{0.31,0.60,0.02}{#1}}
\newcommand{\VariableTok}[1]{\textcolor[rgb]{0.00,0.00,0.00}{#1}}
\newcommand{\VerbatimStringTok}[1]{\textcolor[rgb]{0.31,0.60,0.02}{#1}}
\newcommand{\WarningTok}[1]{\textcolor[rgb]{0.56,0.35,0.01}{\textbf{\textit{#1}}}}
\usepackage{longtable,booktabs,array}
\usepackage{calc} % for calculating minipage widths
% Correct order of tables after \paragraph or \subparagraph
\usepackage{etoolbox}
\makeatletter
\patchcmd\longtable{\par}{\if@noskipsec\mbox{}\fi\par}{}{}
\makeatother
% Allow footnotes in longtable head/foot
\IfFileExists{footnotehyper.sty}{\usepackage{footnotehyper}}{\usepackage{footnote}}
\makesavenoteenv{longtable}
\usepackage{graphicx}
\makeatletter
\newsavebox\pandoc@box
\newcommand*\pandocbounded[1]{% scales image to fit in text height/width
  \sbox\pandoc@box{#1}%
  \Gscale@div\@tempa{\textheight}{\dimexpr\ht\pandoc@box+\dp\pandoc@box\relax}%
  \Gscale@div\@tempb{\linewidth}{\wd\pandoc@box}%
  \ifdim\@tempb\p@<\@tempa\p@\let\@tempa\@tempb\fi% select the smaller of both
  \ifdim\@tempa\p@<\p@\scalebox{\@tempa}{\usebox\pandoc@box}%
  \else\usebox{\pandoc@box}%
  \fi%
}
% Set default figure placement to htbp
\def\fps@figure{htbp}
\makeatother
\setlength{\emergencystretch}{3em} % prevent overfull lines
\providecommand{\tightlist}{%
  \setlength{\itemsep}{0pt}\setlength{\parskip}{0pt}}
\usepackage[]{natbib}
\bibliographystyle{plainnat}
\usepackage{booktabs}
\usepackage{booktabs}
\usepackage{longtable}
\usepackage{array}
\usepackage{multirow}
\usepackage{wrapfig}
\usepackage{float}
\usepackage{colortbl}
\usepackage{pdflscape}
\usepackage{tabu}
\usepackage{threeparttable}
\usepackage{threeparttablex}
\usepackage[normalem]{ulem}
\usepackage{makecell}
\usepackage{xcolor}
\usepackage{bookmark}
\IfFileExists{xurl.sty}{\usepackage{xurl}}{} % add URL line breaks if available
\urlstyle{same}
\hypersetup{
  pdftitle={ENGR-3027 Process Engineering},
  pdfauthor={Martin Volkening},
  hidelinks,
  pdfcreator={LaTeX via pandoc}}

\title{ENGR-3027 Process Engineering}
\author{Martin Volkening}
\date{2026-01-22}

\begin{document}
\maketitle

{
\setcounter{tocdepth}{1}
\tableofcontents
}
\chapter*{Introduction}\label{introduction}
\addcontentsline{toc}{chapter}{Introduction}

Why are we studying this?

\begin{itemize}
\item
  Defining and identifying manufacturing processes is critical to understanding the role that automation plays and how to program an industrial process line.
\item
  What are some examples of a manufacturing process?
\item
  What do we have or use that is manufactured?
\item
  What is the biggest manufacturing change that you have experienced so far?
\item
  Processes have changed throughout the centuries and have been driven by the four Industrial Revolutions.
\item
  First Industrial Revolution -- Coal in 1765

  \begin{itemize}
  \tightlist
  \item
    This transformed society from an agricultural base to an industrial base. Processes became mechanized and rather than hand made, products were manufactured. The discovery of coal and the development of the steam engine and metal forging transformed the way goods were created.
  \end{itemize}
\item
  Second Industrial Revolution -- Gas in 1870
  *The discovery of oil and electricity fueled the second industrial revolution. The discovery of oil lead to the invention of the combustion engine and new steel and chemical based products entered the manufacturing streams. Technological advances in communication were driven by the invention of the telegraph and later the telephone. Modes of transportation developed with the invention of the airplane and automobile. These discoveries helped to drive the advent of mass production which forever changed how items were produced.
\item
  Third Industrial Revolution -- Electronics and Nuclear Energy in 1969
  \emph{The third industrial revolution is dominated by the development of electronics and the beginning of the age of computers.
  }Technological developments were centered around four pillars of science: telecommunications, biotechnology, information technology and energy engineering.
\item
  Fourth Industrial Revolution -- Internet and AI

  \begin{itemize}
  \tightlist
  \item
    The fourth industrial revolution describes the current era of digitalization, automation, and advanced manufacturing technology. It's characterized by the convergence of emerging technologies, such as robotics, the Internet of Things (IoT), and 3D printing.
  \end{itemize}
\end{itemize}

Check out these videos:

and

What is a Fifth Industrial Revolution going to look like?

Further Reading
\href{https://www.intel.com/content/www/us/en/newsroom/tech101/manufacturing-101-what-is-manufacturing.html\#gs.hgplkf}{Whitepaper `What is Manufacturing' by Intel}

\begin{longtable}[]{@{}
  >{\raggedright\arraybackslash}p{(\linewidth - 8\tabcolsep) * \real{0.2596}}
  >{\raggedright\arraybackslash}p{(\linewidth - 8\tabcolsep) * \real{0.2788}}
  >{\raggedright\arraybackslash}p{(\linewidth - 8\tabcolsep) * \real{0.2212}}
  >{\raggedright\arraybackslash}p{(\linewidth - 8\tabcolsep) * \real{0.1250}}
  >{\raggedright\arraybackslash}p{(\linewidth - 8\tabcolsep) * \real{0.1154}}@{}}
\toprule\noalign{}
\begin{minipage}[b]{\linewidth}\raggedright
Category
\end{minipage} & \begin{minipage}[b]{\linewidth}\raggedright
Grade Item
\end{minipage} & \begin{minipage}[b]{\linewidth}\raggedright
Type
\end{minipage} & \begin{minipage}[b]{\linewidth}\raggedright
Max. Points
\end{minipage} & \begin{minipage}[b]{\linewidth}\raggedright
Weight (\%)
\end{minipage} \\
\midrule\noalign{}
\endhead
\bottomrule\noalign{}
\endlastfoot
\textbf{Lab-Tutorial} & \textbf{Lab-Tutorial} & \textbf{Association} & \textbf{48} & \textbf{40} \\
\textbf{Labs} & Lab-1 & NumericSubmissions & 48.33 & 8.33 \\
& Lab-2 & NumericSubmissions & 48.33 & 8.33 \\
& Lab-3 & NumericSubmissions & 48.33 & 8.33 \\
& Lab-4 & NumericSubmissions & 48.33 & 8.33 \\
& Lab-5 & NumericSubmissions & 48.33 & 8.33 \\
& Lab-6 & NumericSubmissions & 48.33 & 8.33 \\
& Lab-7 & NumericSubmissions & 48.33 & 8.33 \\
& Lab-8 & NumericSubmissions & 48.33 & 8.33 \\
& Lab-9 & NumericSubmissions & 48.33 & 8.33 \\
& Lab-10 & NumericSubmissions & 48.33 & 8.33 \\
& Lab-11 & NumericSubmissions & 48.33 & 8.33 \\
& Lab-12 & NumericSubmissions & 48.33 & 8.33 \\
\textbf{Assignments} & \textbf{Assignments} & & \textbf{12} & \textbf{12} \\
& Assignment-1 & NumericQuizzes & 2 & 16.67 \\
& Assignment-2 & NumericQuizzes & 2 & 16.67 \\
& Assignment-3 & NumericQuizzes & 2 & 16.67 \\
& Assignment-4 & NumericQuizzes & 2 & 16.67 \\
& Assignment-5 & NumericQuizzes & 2 & 16.66 \\
& Assignment-6 & NumericQuizzes & 2 & 16.66 \\
\textbf{Lab Pop-up Quizzes} & \textbf{Lab Quizes} & & \textbf{3} & \textbf{3} \\
& Lab Quiz-1 & NumericQuizzes & 0.5 & 16.66 \\
& Lab Quiz-2 & NumericQuizzes & 0.5 & 16.66 \\
& Lab Quiz-3 & NumericQuizzes & 0.5 & 16.67 \\
& Lab Quiz-4 & NumericQuizzes & 0.5 & 16.67 \\
& Lab Quiz-5 & NumericQuizzes & 0.5 & 16.67 \\
& Lab Quiz-6 & NumericQuizzes & 0.5 & 16.67 \\
\textbf{Lecture Pop-up Quizzes} & \textbf{Lecture Quizes} & & \textbf{3} & \textbf{3 } \\
& Lecture Quiz-1 & NumericQuizzes & 0.5 & 16.66 \\
& Lecture Quiz-2 & NumericQuizzes & 0.5 & 16.66 \\
& Lecture Quiz-3 & NumericQuizzes & 0.5 & 16.67 \\
& Lecture Quiz-4 & NumericQuizzes & 0.5 & 16.67 \\
& Lecture Quiz-5 & NumericQuizzes & 0.5 & 16.67 \\
& Lecture Quiz-6 & NumericQuizzes & 0.5 & 16.67 \\
\textbf{Midterm Test} & \textbf{Mid Term Test} & \textbf{NumericQuizzes} & \textbf{40.5} & \textbf{17} \\
\textbf{Final Test} & Final Test & NumericQuizzes & 50 & 25 \\
\textbf{Calculated Grades} & Midterm Grade & Calculated & --- & --- \\
& Final Calculated Grade & --- & 156.5 & 100 \\
\end{longtable}

\section*{Labs}\label{labs}
\addcontentsline{toc}{section}{Labs}

\begin{itemize}
\tightlist
\item
  Late penalty is 25\% per lab session.
\item
  Labs 1-5 must be completed before Week 7.
\item
  Bring your lab kit and PPE to every lab session.
\end{itemize}

\section*{Assignments}\label{assignments}
\addcontentsline{toc}{section}{Assignments}

\begin{itemize}
\tightlist
\item
  6 Assignments worth 2\% each, for a total of 12\%.
\item
  All are FOL quizzes, with at least one week open.
\end{itemize}

\section*{Quizzes}\label{quizzes}
\addcontentsline{toc}{section}{Quizzes}

\begin{itemize}
\tightlist
\item
  6 Lecture Quizzes and 6 Lab Quizzes.
\item
  Quizzes are worth 0.5\% each, for a total of 6\%.
\item
  Quizzes are given randomly throughout the term as a ``pop quiz''.
\item
  Quizzes are given at the start of the lab session (10 minute log in window) or at the end of the lecture.
\item
  \textbf{You must be in attendance to write the quizzes.}
\end{itemize}

\section*{Tests}\label{tests}
\addcontentsline{toc}{section}{Tests}

\begin{itemize}
\tightlist
\item
  During in-person lecture time (90 minutes). One midterm test worth 17\%, one final test worth 25\%.
\item
  Both tests are cumulative and online, in-person.
\item
  Closed-book with handwritten cheat sheet; details will be posted on FOL closer to the date.
\item
  Kit calculators allowed.
\end{itemize}

\section*{Other Grade Considerations}\label{other-grade-considerations}
\addcontentsline{toc}{section}{Other Grade Considerations}

In order to pass, two conditions must be met:
* Overall passing grade (50\%).
* Attendance and submission of the mandatory labs (labs 1 though 10)

\section*{How to succeed}\label{how-to-succeed}
\addcontentsline{toc}{section}{How to succeed}

\begin{itemize}
\tightlist
\item
  Come to class and lab sessions Participate
\item
  Complete examples given in class
\item
  Do the homework and assigned reading
\item
  Attend office hours
\end{itemize}

\section*{Cheating and Plagiarism}\label{cheating-and-plagiarism}
\addcontentsline{toc}{section}{Cheating and Plagiarism}

Students must write their reports and assignments in their \textbf{own words}. Whenever students take an idea or a passage from another author, they must acknowledge their debt both by using quotation marks where appropriate and by proper referencing such as footnotes or citations. This includes non-published sources, such as work completed in previous years or a classmate's work.
\textbf{This includes the use of ``AI'' Technologies.}

\section*{FOL}\label{fol}
\addcontentsline{toc}{section}{FOL}

\begin{itemize}
\tightlist
\item
  Lectures will be posted after class, as well as supplementary material.
\item
  Assignments and lab manuals will be posted ahead of time. You will always have at least one week's notice.
\item
  All announcements will be made through the FOL course.
\end{itemize}

\section*{Accessibility}\label{accessibility}
\addcontentsline{toc}{section}{Accessibility}

Please contact me if you require material in an alternate format or if any other arrangements can make this course more accessible to you.

Consider contacting Accessibility Services for specifics on accommodations that you may be eligible for.
M-Fr 8:30-4:30
F2010, 519-452-4282, \href{mailto:accessibility@fanshawec.ca}{\nolinkurl{accessibility@fanshawec.ca}}

\section*{Fanshawe's Recording Policy}\label{fanshawes-recording-policy}
\addcontentsline{toc}{section}{Fanshawe's Recording Policy}

It is forbidden for students to record without written permission from the instructor (and) as an approved accommodation. Recording without permission or sharing a recording will result in disciplinary action.

If a class is being recorded with permission: it will be announced, and alternate means to participation will be made available to protect students' privacy.

\chapter{Types of Manufacturing Processes}\label{types-of-manufacturing-processes}

\begin{center}\rule{0.5\linewidth}{0.5pt}\end{center}

\section{Learning Objectives}\label{learning-objectives}

By the end of this chapter, you will be able to:

\begin{enumerate}
\def\labelenumi{\arabic{enumi}.}
\tightlist
\item
  Identify and classify the seven main types of manufacturing processes
\item
  Select appropriate manufacturing processes based on material, volume, and precision requirements
\item
  Compare cost, speed, and quality trade-offs between process types
\item
  Recognize manufacturing processes used in automotive, food, and defense industries
\item
  Apply process selection criteria to real-world manufacturing scenarios
\end{enumerate}

\begin{center}\rule{0.5\linewidth}{0.5pt}\end{center}

\section{Introduction to Manufacturing Processes}\label{introduction-to-manufacturing-processes}

Manufacturing is defined as the conversion of raw materials into finished goods on a large scale using man and machine. \textbf{Manufacturing processes} are the specific methods used for this conversion. Based on product requirements, different types of manufacturing processes are selected to achieve the required output with optimal cost, quality, and efficiency.

\begin{verbatim}
## Warning: Using `size` aesthetic for lines was deprecated in
## ggplot2 3.4.0.
## i Please use `linewidth` instead.
## This warning is displayed once per session.
## Call `lifecycle::last_lifecycle_warnings()` to see
## where this warning was generated.
\end{verbatim}

\begin{figure}

{\centering \includegraphics{introduction_files/figure-latex/manufacturing-overview-1} 

}

\caption{Overview of Manufacturing Process Categories}\label{fig:manufacturing-overview}
\end{figure}

Why do we need different manufacturing processes?

Different products require different manufacturing approaches based on:

\begin{itemize}
\tightlist
\item
  \textbf{Material type}: Metals, plastics, ceramics, composites
\item
  \textbf{Production volume}: One-off prototypes vs.~millions of units
\item
  \textbf{Geometric complexity}: Simple shapes vs.~intricate details
\item
  \textbf{Precision requirements}: Loose tolerances vs.~micron-level accuracy
\item
  \textbf{Cost constraints}: Budget limitations affect process selection
\item
  \textbf{Lead time}: How quickly parts are needed
\end{itemize}

No single process is optimal for all situations!

A good video summarizing 6 of the 7 manufacturing processes is linked below (Coating/Plating is not covered):

\begin{center}\rule{0.5\linewidth}{0.5pt}\end{center}

\section{Process Selection Framework}\label{process-selection-framework}

Before examining each process type, it's important to understand how engineers select the right manufacturing process. The following visualization shows the key factors:

\begin{figure}

{\centering \includegraphics{introduction_files/figure-latex/process-selection-matrix-1} 

}

\caption{Process Selection Based on Volume and Complexity}\label{fig:process-selection-matrix}
\end{figure}

\label{tab:process-comparison-table}Manufacturing Process Comparison Summary

Process Type

Typical Materials

Production Volume

Typical Tolerance

Tooling Cost

Unit Cost at Volume

Casting

Metals, alloys

Low to High

±0.5-2mm

Medium-High

Low

Molding

Plastics, rubber

Medium to Very High

±0.1-0.5mm

High

Very Low

Forming

Metals, sheet

Medium to Very High

±0.1-1mm

Medium-High

Low

Machining

All materials

Low to Medium

±0.01-0.1mm

Low

High

Joining

All materials

All volumes

Varies

Low

Medium

Coating

All materials

All volumes

N/A

Low

Low

Additive

Plastics, metals, ceramics

Prototype to Low

±0.1-0.3mm

None

High

\begin{center}\rule{0.5\linewidth}{0.5pt}\end{center}

\section{1. Casting Processes}\label{casting-processes}

\textbf{Casting} is a process of pouring liquid metal into a mold containing the hollow shape of the desired outcome. It uses sprue, gates, and runners to direct the molten metal flow.

\begin{figure}

{\centering \includegraphics{introduction_files/figure-latex/casting-comparison-1} 

}

\caption{Comparison of Casting Processes}\label{fig:casting-comparison}
\end{figure}

\subsection{Sand Casting}\label{sand-casting}

Sand casting is a metal casting process that uses sand as the mold material. This process has a low production rate as the sand mold must be destroyed to remove the part. However, it is economical for low-volume production.

\textbf{Process Steps:}

\begin{enumerate}
\def\labelenumi{\arabic{enumi}.}
\tightlist
\item
  Create a pattern (usually wood or metal)
\item
  Pack sand around the pattern to form the mold
\item
  Remove the pattern, leaving a cavity
\item
  Pour molten metal into the cavity
\item
  Allow solidification
\item
  Break the sand mold to remove the casting
\end{enumerate}

\textbf{Advantages:} Low tooling cost, can produce very large parts, suitable for all metals

\textbf{Disadvantages:} Poor surface finish, loose tolerances, slow cycle time

\textbf{Examples:} Engine blocks, machine tool bases, cylinder heads, pump housings, large gears

Discussion: Why is sand casting still widely used despite its limitations?

Sand casting remains popular because:

\begin{enumerate}
\def\labelenumi{\arabic{enumi}.}
\tightlist
\item
  \textbf{Low startup cost} - A wood pattern costs hundreds vs.~steel dies costing tens of thousands
\item
  \textbf{Flexibility} - Easy to modify patterns for design changes
\item
  \textbf{Size capability} - Can cast parts weighing several tons
\item
  \textbf{Material versatility} - Works with nearly any metal
\item
  \textbf{Short lead time} - Parts can be produced in days vs.~weeks for die casting
\end{enumerate}

For prototypes and low-volume production (\textless{} 500 parts), sand casting is often the most economical choice even with its lower quality.

\subsection{Die Casting}\label{die-casting}

Die casting is used for producing metal parts where high precision is required. Molten metal is forced under high pressure into reusable metal dies, enabling high-volume production.

\textbf{Process:}

\begin{enumerate}
\def\labelenumi{\arabic{enumi}.}
\tightlist
\item
  Steel molds (dies) are machined to the part shape
\item
  Dies are mounted in a die casting machine
\item
  Molten metal is injected under high pressure (10-175 MPa)
\item
  Metal solidifies rapidly due to die cooling
\item
  Part is ejected and the cycle repeats
\end{enumerate}

\textbf{Advantages:} Excellent dimensional accuracy, smooth surface finish, high production rates (up to 400 shots/hour)

\textbf{Disadvantages:} High die cost (\$50,000-\$500,000), limited to non-ferrous metals, porosity issues

\textbf{Examples:} Engine components, transmission housings, electronic enclosures, power tool housings

\subsection{Investment Casting (Lost-Wax)}\label{investment-casting-lost-wax}

Investment casting produces complex, precision parts by creating a wax pattern that is ``invested'' (surrounded) with ceramic slurry, then melted out to leave a mold cavity.

\textbf{Advantages:} Excellent surface finish, tight tolerances (±0.1\%), complex geometries, any castable metal

\textbf{Disadvantages:} Expensive, slow cycle time, limited part size

\textbf{Examples:} Turbine blades, aerospace components, medical implants, jewelry, firearm components

Why do aerospace companies pay premium prices for investment casting?

Investment casting is essential for aerospace because:

\begin{enumerate}
\def\labelenumi{\arabic{enumi}.}
\tightlist
\item
  \textbf{Complex internal passages} - Turbine blades have intricate cooling channels
\item
  \textbf{Superior metallurgy} - Controlled solidification produces better grain structure
\item
  \textbf{Near-net shape} - Minimizes machining of expensive superalloys
\item
  \textbf{Weight savings} - Can create hollow structures impossible with other methods
\item
  \textbf{Material options} - Works with titanium, Inconel, and other aerospace alloys
\end{enumerate}

A single turbine blade might cost \$1,000-\$5,000, but there's no other process that can produce the required geometry and performance.

\subsection{Centrifugal Casting}\label{centrifugal-casting}

Centrifugal casting spins molten metal in a rotating die, using centrifugal force to distribute the metal and float impurities to the inner surface.

\textbf{Applications:} Pipes, tubes, cylinder liners, bushings, rings

\subsection{Permanent Mold Casting}\label{permanent-mold-casting}

Uses reusable metal molds with gravity or low-pressure filling. Good for medium-volume production of aluminum and copper alloy parts.

\textbf{Applications:} Pistons, gear housings, wheels, pipe fittings

\subsection{Shell Molding}\label{shell-molding}

Uses resin-coated sand that forms a thin, strong shell around a heated pattern. Better dimensional accuracy than sand casting.

\textbf{Applications:} Gear housings, cylinder heads, connecting rods

\begin{center}\rule{0.5\linewidth}{0.5pt}\end{center}

\section{2. Molding Processes}\label{molding-processes}

\textbf{Molding} uses a rigid frame to shape hot liquid or ductile raw material. It is predominantly used for plastics and rubber products.

\begin{figure}

{\centering \includegraphics{introduction_files/figure-latex/molding-volume-chart-1} 

}

\caption{Molding Process Selection by Production Volume}\label{fig:molding-volume-chart}
\end{figure}

\subsection{Injection Molding}\label{injection-molding}

Injection molding is the most common process for manufacturing plastic parts. Plastic pellets are melted and injected under high pressure into a mold cavity.

\textbf{Process Cycle:}

\begin{enumerate}
\def\labelenumi{\arabic{enumi}.}
\tightlist
\item
  \textbf{Clamping} - Mold halves are clamped together
\item
  \textbf{Injection} - Molten plastic is injected into the cavity
\item
  \textbf{Cooling} - Part solidifies in the mold
\item
  \textbf{Ejection} - Part is pushed out by ejector pins
\end{enumerate}

\textbf{Cycle times:} 10-60 seconds typical, as fast as 2 seconds for small parts

\textbf{Examples:} Automotive dashboards, phone cases, medical devices, food containers, toys

\begin{Shaded}
\begin{Highlighting}[]
\CommentTok{\# Injection Molding Cost Analysis Example}
\CommentTok{\# Comparing costs at different production volumes}

\NormalTok{mold\_cost }\OtherTok{\textless{}{-}} \DecValTok{50000}  \CommentTok{\# Typical mold cost in dollars}
\NormalTok{material\_cost\_per\_kg }\OtherTok{\textless{}{-}} \FloatTok{2.50}  \CommentTok{\# ABS plastic}
\NormalTok{part\_weight\_kg }\OtherTok{\textless{}{-}} \FloatTok{0.1}
\NormalTok{cycle\_time\_sec }\OtherTok{\textless{}{-}} \DecValTok{30}
\NormalTok{machine\_rate\_per\_hour }\OtherTok{\textless{}{-}} \DecValTok{75}  \CommentTok{\# Machine + operator cost}

\CommentTok{\# Calculate cost per part at different volumes}
\NormalTok{volumes }\OtherTok{\textless{}{-}} \FunctionTok{c}\NormalTok{(}\DecValTok{1000}\NormalTok{, }\DecValTok{10000}\NormalTok{, }\DecValTok{50000}\NormalTok{, }\DecValTok{100000}\NormalTok{, }\DecValTok{500000}\NormalTok{)}

\NormalTok{cost\_analysis }\OtherTok{\textless{}{-}} \FunctionTok{data.frame}\NormalTok{(}
  \AttributeTok{Volume =}\NormalTok{ volumes,}
  \AttributeTok{Mold\_Per\_Part =}\NormalTok{ mold\_cost }\SpecialCharTok{/}\NormalTok{ volumes,}
  \AttributeTok{Material =}\NormalTok{ part\_weight\_kg }\SpecialCharTok{*}\NormalTok{ material\_cost\_per\_kg,}
  \AttributeTok{Processing =}\NormalTok{ (cycle\_time\_sec }\SpecialCharTok{/} \DecValTok{3600}\NormalTok{) }\SpecialCharTok{*}\NormalTok{ machine\_rate\_per\_hour}
\NormalTok{)}

\NormalTok{cost\_analysis}\SpecialCharTok{$}\NormalTok{Total\_Per\_Part }\OtherTok{\textless{}{-}}\NormalTok{ cost\_analysis}\SpecialCharTok{$}\NormalTok{Mold\_Per\_Part }\SpecialCharTok{+}
\NormalTok{                                 cost\_analysis}\SpecialCharTok{$}\NormalTok{Material }\SpecialCharTok{+}
\NormalTok{                                 cost\_analysis}\SpecialCharTok{$}\NormalTok{Processing}

\FunctionTok{cat}\NormalTok{(}\StringTok{"Injection Molding Cost Breakdown:}\SpecialCharTok{\textbackslash{}n}\StringTok{"}\NormalTok{)}
\end{Highlighting}
\end{Shaded}

\begin{verbatim}
## Injection Molding Cost Breakdown:
\end{verbatim}

\begin{Shaded}
\begin{Highlighting}[]
\FunctionTok{cat}\NormalTok{(}\StringTok{"================================}\SpecialCharTok{\textbackslash{}n}\StringTok{"}\NormalTok{)}
\end{Highlighting}
\end{Shaded}

\begin{verbatim}
## ================================
\end{verbatim}

\begin{Shaded}
\begin{Highlighting}[]
\ControlFlowTok{for}\NormalTok{(i }\ControlFlowTok{in} \DecValTok{1}\SpecialCharTok{:}\FunctionTok{nrow}\NormalTok{(cost\_analysis)) \{}
  \FunctionTok{cat}\NormalTok{(}\FunctionTok{sprintf}\NormalTok{(}\StringTok{"At \%s parts: $\%.2f per part (Tooling: $\%.2f, Material: $\%.2f, Processing: $\%.2f)}\SpecialCharTok{\textbackslash{}n}\StringTok{"}\NormalTok{,}
              \FunctionTok{format}\NormalTok{(cost\_analysis}\SpecialCharTok{$}\NormalTok{Volume[i], }\AttributeTok{big.mark=}\StringTok{","}\NormalTok{),}
\NormalTok{              cost\_analysis}\SpecialCharTok{$}\NormalTok{Total\_Per\_Part[i],}
\NormalTok{              cost\_analysis}\SpecialCharTok{$}\NormalTok{Mold\_Per\_Part[i],}
\NormalTok{              cost\_analysis}\SpecialCharTok{$}\NormalTok{Material[i],}
\NormalTok{              cost\_analysis}\SpecialCharTok{$}\NormalTok{Processing[i]))}
\NormalTok{\}}
\end{Highlighting}
\end{Shaded}

\begin{verbatim}
## At 1,000 parts: $50.88 per part (Tooling: $50.00, Material: $0.25, Processing: $0.62)
## At 10,000 parts: $5.88 per part (Tooling: $5.00, Material: $0.25, Processing: $0.62)
## At 50,000 parts: $1.88 per part (Tooling: $1.00, Material: $0.25, Processing: $0.62)
## At 1e+05 parts: $1.38 per part (Tooling: $0.50, Material: $0.25, Processing: $0.62)
## At 5e+05 parts: $0.97 per part (Tooling: $0.10, Material: $0.25, Processing: $0.62)
\end{verbatim}

Discussion: When does injection molding become cost-effective vs.~3D printing?

\textbf{Break-even analysis:}

\begin{itemize}
\tightlist
\item
  3D printing: \textasciitilde\$5-20 per part (no tooling, slow production)
\item
  Injection molding: \$50,000 tooling + \$0.50-2 per part
\end{itemize}

\textbf{Break-even point:} Approximately 5,000-10,000 parts

\textbf{Rule of thumb:}
- \textless{} 500 parts: 3D printing or machining
- 500-5,000 parts: Consider soft tooling (aluminum molds)
- \textgreater{} 5,000 parts: Steel injection molds become economical

\subsection{Blow Molding}\label{blow-molding}

Blow molding creates hollow plastic parts by inflating a heated plastic tube (parison) inside a mold cavity.

\textbf{Types:}
- \textbf{Extrusion blow molding} - Parison extruded, then blown
- \textbf{Injection blow molding} - Preform injection molded, then blown
- \textbf{Stretch blow molding} - Preform stretched and blown (PET bottles)

\textbf{Examples:} Bottles, containers, fuel tanks, automotive ducts, toys

\subsection{Rotational Molding}\label{rotational-molding}

Creates hollow parts by rotating a mold on two axes while heating, causing plastic powder to melt and coat the interior surface.

\textbf{Advantages:} Low tooling cost, no internal stresses, large hollow parts

\textbf{Examples:} Kayaks, playground equipment, tanks, coolers

\subsection{Thermoforming}\label{thermoforming}

Heats a plastic sheet until pliable, then forms it over a mold using vacuum, pressure, or mechanical force.

\textbf{Examples:} Packaging (blister packs), disposable cups, refrigerator liners

\subsection{Powder Metallurgy}\label{powder-metallurgy}

Though involving metal, powder metallurgy uses molding principles. Metal powders are compacted in dies and sintered (heated below melting point) to fuse particles.

\textbf{Advantages:} Near-net shape, complex geometries, controlled porosity

\textbf{Examples:} Self-lubricating bearings, gears, filters, magnets

\begin{center}\rule{0.5\linewidth}{0.5pt}\end{center}

\section{3. Forming Processes}\label{forming-processes}

\textbf{Forming} uses mechanical forces (compression, tension, shear) to shape material without removing any material. This preserves material grain structure and can improve mechanical properties.

\begin{figure}

{\centering \includegraphics{introduction_files/figure-latex/forming-forces-1} 

}

\caption{Types of Forming Forces}\label{fig:forming-forces}
\end{figure}

\subsection{Forging}\label{forging}

Forging uses compressive forces from hammers or presses to shape heated metal. Forged parts have superior strength due to aligned grain structure.

\textbf{Types of Forging:}
- \textbf{Open-die forging} - Simple shapes, large parts
- \textbf{Closed-die (impression) forging} - Complex shapes, high volume
- \textbf{Cold forging} - Room temperature, excellent finish
- \textbf{Hot forging} - Heated material, easier deformation

\textbf{Examples:} Crankshafts, connecting rods, gears, aircraft structural components, hand tools

Why are aircraft landing gear forgings so expensive?

Aircraft landing gear forgings can cost \$100,000+ because:

\begin{enumerate}
\def\labelenumi{\arabic{enumi}.}
\tightlist
\item
  \textbf{Material} - High-strength steel or titanium alloys
\item
  \textbf{Size} - Large parts require massive presses (50,000+ tons)
\item
  \textbf{Certification} - Extensive testing and documentation
\item
  \textbf{Grain flow} - Must be optimized for fatigue resistance
\item
  \textbf{NDT requirements} - 100\% ultrasonic and magnetic particle inspection
\item
  \textbf{Low volume} - Only a few hundred per aircraft type per year
\end{enumerate}

The forged grain structure provides 20-40\% better fatigue life than equivalent castings or machined parts.

\subsection{Stamping}\label{stamping}

Stamping uses dies and presses to form, cut, or shape sheet metal. Modern stamping lines can produce 1,000+ parts per hour.

\textbf{Operations include:}
- \textbf{Blanking} - Cutting flat shapes from sheet
- \textbf{Piercing} - Punching holes
- \textbf{Drawing} - Forming cup shapes
- \textbf{Bending} - Creating angles
- \textbf{Coining} - Fine detail forming

\textbf{Examples:} Automotive body panels, appliance housings, electronic enclosures

\subsection{Bending}\label{bending}

Bending uses press brakes to create angular shapes in sheet metal.

\textbf{Key formula:}
\[\text{Bend Allowance} = \frac{\pi \times A \times (R + K \times T)}{180}\]

Where: A = bend angle, R = inside radius, T = material thickness, K = K-factor (typically 0.3-0.5)

\subsection{Shearing}\label{shearing}

Shearing cuts sheet metal without chip formation using opposed cutting edges.

\textbf{Types:} Blanking, piercing, trimming, notching, slitting

\subsection{Rolling}\label{rolling}

Reduces material thickness by passing through rotating rolls. Used for sheet, plate, and structural shapes.

\subsection{Extrusion (Metal)}\label{extrusion-metal}

Forces material through a die to create constant cross-section profiles.

\textbf{Examples:} Aluminum window frames, heat sinks, structural shapes

\begin{center}\rule{0.5\linewidth}{0.5pt}\end{center}

\section{4. Machining Processes}\label{machining-processes}

\textbf{Machining} removes material using cutting tools to achieve precise dimensions and surface finishes. It is the most versatile manufacturing process but typically has higher per-part costs.

\begin{figure}

{\centering \includegraphics{introduction_files/figure-latex/machining-comparison-1} 

}

\caption{Machining Process Capabilities}\label{fig:machining-comparison}
\end{figure}

\subsection{Turning}\label{turning}

Rotates the workpiece while a cutting tool moves to create cylindrical shapes.

\textbf{Performed on lathes or turning centers}

\textbf{Operations:} Facing, turning, boring, threading, grooving, parting

\subsection{Milling}\label{milling}

Rotates a cutting tool while moving the workpiece to remove material. The most versatile machining process.

\textbf{Types:}
- \textbf{Peripheral milling} - Cutter axis parallel to surface
- \textbf{Face milling} - Cutter axis perpendicular to surface
- \textbf{End milling} - Creates slots, pockets, contours

\subsection{Drilling}\label{drilling}

Creates cylindrical holes using rotating drill bits.

\textbf{Related operations:} Reaming (sizing), boring (enlarging), tapping (threading)

\subsection{Grinding}\label{grinding}

Uses abrasive wheels for precision finishing. Achieves the finest tolerances and surface finishes.

\textbf{Types:} Surface grinding, cylindrical grinding, centerless grinding

\subsection{Electrical Discharge Machining (EDM)}\label{electrical-discharge-machining-edm}

Uses electrical sparks to erode conductive materials. Essential for hardened materials and complex geometries.

\textbf{Applications:} Injection mold cavities, dies, aerospace components

When should you use EDM instead of conventional machining?

\textbf{Choose EDM when:}

\begin{enumerate}
\def\labelenumi{\arabic{enumi}.}
\tightlist
\item
  \textbf{Hardened materials} - Can machine 60+ HRC steels that would destroy cutting tools
\item
  \textbf{Complex internal features} - Wire EDM cuts intricate shapes
\item
  \textbf{Tight corners} - Can achieve sharp internal corners impossible with rotary tools
\item
  \textbf{Fragile parts} - No cutting forces (non-contact process)
\item
  \textbf{Thin walls} - No deflection from tool pressure
\end{enumerate}

\textbf{Limitations:}
- Slow material removal (grams/hour vs.~kg/hour for milling)
- Only works on conductive materials
- Recast layer may need removal
- Higher cost per part

\begin{center}\rule{0.5\linewidth}{0.5pt}\end{center}

\section{5. Joining Processes}\label{joining-processes}

\textbf{Joining} combines two or more components into a single assembly. Methods range from permanent (welding) to removable (fasteners).

\label{tab:joining-comparison}Joining Process Comparison

Process

Joint Strength

Dissimilar Metals

Automation

Typical Applications

Arc Welding

Excellent

Limited

Medium

Structural steel, pressure vessels

Resistance Welding

Excellent

Limited

High

Auto body panels, wire mesh

Laser Welding

Excellent

Limited

High

Electronics, medical devices

Brazing

Good

Yes

Medium

Carbide tooling, HVAC

Soldering

Fair

Yes

High

Electronics, plumbing

Adhesive Bonding

Good

Yes

High

Composites, glass, plastics

Mechanical Fastening

Excellent

Yes

High

All industries

\subsection{Welding}\label{welding}

Melts and fuses base materials together, often with added filler metal.

\textbf{Common Types:}
- \textbf{MIG (GMAW)} - Gas Metal Arc Welding - fast, easy automation
- \textbf{TIG (GTAW)} - Gas Tungsten Arc Welding - precise, clean welds
- \textbf{Stick (SMAW)} - Shielded Metal Arc Welding - versatile, field use
- \textbf{Resistance} - Uses electrical resistance heating
- \textbf{Laser/Electron Beam} - High precision, deep penetration

\subsection{Brazing and Soldering}\label{brazing-and-soldering}

Joins metals using a filler metal with a lower melting point than the base materials. The base materials do not melt.

\begin{itemize}
\tightlist
\item
  \textbf{Brazing} - Filler melts above 450°C (840°F)
\item
  \textbf{Soldering} - Filler melts below 450°C
\end{itemize}

\subsection{Adhesive Bonding}\label{adhesive-bonding}

Uses chemical adhesives to join materials. Essential for composites, glass, and dissimilar materials.

\textbf{Advantages:} Distributes stress, joins dissimilar materials, seals against fluids

\textbf{Disadvantages:} Surface preparation critical, limited temperature resistance, aging concerns

\subsection{Mechanical Fastening}\label{mechanical-fastening}

Uses bolts, screws, rivets, or clips. Allows disassembly for maintenance.

\begin{center}\rule{0.5\linewidth}{0.5pt}\end{center}

\section{6. Surface Treatment and Coating}\label{surface-treatment-and-coating}

\textbf{Coating and surface treatment} processes protect parts from corrosion, wear, or environmental damage, and can improve appearance or electrical properties.

\label{tab:coating-processes}Surface Treatment and Coating Processes

Process

Typical Thickness

Base Material

Primary Benefit

Industry

Electroplating

1-50 μm

Conductive metals

Corrosion, appearance

Automotive, electronics

Anodizing

5-100 μm

Aluminum

Corrosion, hardness

Aerospace, construction

Powder Coating

50-150 μm

Any metal

Corrosion, appearance

Appliances, furniture

Painting

25-75 μm

Any

Appearance

All

Hot Dip Galvanizing

45-85 μm

Steel

Corrosion

Construction, automotive

PVD/CVD

1-10 μm

Any

Wear, hardness

Tooling, aerospace

Thermal Spray

100-500 μm

Any

Wear, thermal barrier

Aerospace, energy

\subsection{Electroplating}\label{electroplating}

Deposits metal coating using electrical current. Common coatings include chrome, nickel, zinc, gold, and silver.

\subsection{Anodizing}\label{anodizing}

Electrochemically grows an oxide layer on aluminum. Provides corrosion resistance and can be dyed various colors.

\subsection{Powder Coating}\label{powder-coating}

Applies dry powder electrostatically, then cures with heat. Produces thick, durable, attractive finishes.

\subsection{Galvanizing}\label{galvanizing}

Coats steel with zinc for corrosion protection. Hot-dip galvanizing provides excellent outdoor durability.

\begin{center}\rule{0.5\linewidth}{0.5pt}\end{center}

\section{7. Additive Manufacturing (3D Printing)}\label{additive-manufacturing-3d-printing}

\textbf{Additive manufacturing} builds parts layer-by-layer from digital models. It's the opposite of machining (subtractive manufacturing).

\begin{figure}

{\centering \includegraphics{introduction_files/figure-latex/am-processes-1} 

}

\caption{Additive Manufacturing Process Comparison}\label{fig:am-processes}
\end{figure}

\subsection{FDM (Fused Deposition Modeling)}\label{fdm-fused-deposition-modeling}

Extrudes thermoplastic filament through a heated nozzle. Most common desktop 3D printing technology.

\textbf{Materials:} PLA, ABS, PETG, Nylon, TPU

\subsection{SLA (Stereolithography)}\label{sla-stereolithography}

Uses UV laser to cure liquid photopolymer resin. Excellent surface finish and detail.

\subsection{SLS (Selective Laser Sintering)}\label{sls-selective-laser-sintering}

Fuses powder material with a laser. Produces strong functional parts without support structures.

\subsection{Metal 3D Printing (DMLS/SLM)}\label{metal-3d-printing-dmlsslm}

Fuses metal powder with high-power lasers. Used for aerospace, medical, and tooling applications.

When does metal 3D printing make sense vs.~traditional manufacturing?

\textbf{Metal 3D printing is advantageous when:}

\begin{enumerate}
\def\labelenumi{\arabic{enumi}.}
\tightlist
\item
  \textbf{Complex internal features} - Conformal cooling channels in molds
\item
  \textbf{Topology optimization} - Organic shapes that minimize weight
\item
  \textbf{Consolidation} - Combining multiple parts into one
\item
  \textbf{Low volume} - Too few parts to justify tooling
\item
  \textbf{Custom/one-off} - Medical implants, prototypes
\end{enumerate}

\textbf{Traditional manufacturing is better when:}
- High volumes (\textgreater{} 100-1000 parts depending on complexity)
- Simple geometries
- Large parts (3D print beds are limited)
- Maximum strength required (forging still superior)

\textbf{Cost comparison:}
- Simple bracket: \$5 machined vs.~\$50+ 3D printed
- Complex aerospace part: \$10,000 machined vs.~\$3,000 3D printed

\begin{center}\rule{0.5\linewidth}{0.5pt}\end{center}

\section{Industry Applications}\label{industry-applications}

\label{tab:industry-table}Manufacturing Process Applications by Industry

Process

Automotive

Food Industry

Defense/Aerospace

Die Casting

Engine blocks, transmission cases

Equipment housings

Aircraft structural components

Injection Molding

Dashboard, interior trim, connectors

Packaging, containers, caps

Connectors, housings, switches

Forging

Crankshafts, connecting rods, gears

Mixer blades, pump impellers

Landing gear, turbine disks

Stamping

Body panels, structural components

Cans, trays, lids

Aircraft skins, missile bodies

Machining

Engine components, precision parts

Valves, fittings

Precision weapons components

Welding

Body structure, exhaust systems

Tanks, piping, conveyors

Aircraft structures, armor

Powder Coating

Wheels, chassis components

Equipment frames

Military vehicles, equipment

3D Printing

Prototypes, custom fixtures

Custom tooling, fixtures

Turbine blades, brackets, tooling

\begin{center}\rule{0.5\linewidth}{0.5pt}\end{center}

\section{Sustainability in Manufacturing}\label{sustainability-in-manufacturing}

Modern manufacturing must consider environmental impact. Here's how different processes compare:

\begin{figure}

{\centering \includegraphics{introduction_files/figure-latex/sustainability-1} 

}

\caption{Environmental Considerations by Process Type}\label{fig:sustainability}
\end{figure}

Discussion: How can manufacturers reduce environmental impact?

\textbf{Strategies for sustainable manufacturing:}

\begin{enumerate}
\def\labelenumi{\arabic{enumi}.}
\tightlist
\item
  \textbf{Near-net-shape processes} - Casting, forging, and molding reduce machining waste
\item
  \textbf{Closed-loop recycling} - Reprocess scrap material on-site
\item
  \textbf{Energy-efficient equipment} - Modern machines use 30-50\% less energy
\item
  \textbf{Additive manufacturing} - Only uses material where needed
\item
  \textbf{Water-based coatings} - Replace solvent-based paints
\item
  \textbf{Lean manufacturing} - Reduces overproduction and waste
\item
  \textbf{Life cycle assessment} - Consider total environmental impact
\end{enumerate}

\textbf{Example:} Aluminum die casting recycles 95\%+ of sprues and runners, while machining from billet may waste 80\% of material.

\begin{center}\rule{0.5\linewidth}{0.5pt}\end{center}

\section{Review Questions}\label{review-questions}

Question 1: Process Identification

\textbf{A company needs to produce 50,000 aluminum housings per year. Each housing is a complex shape with internal fins for heat dissipation. The surface must be smooth and dimensions must be within ±0.3mm. Which process would you recommend?}

\textbf{Answer:} \textbf{Die Casting} is the best choice because:
- High volume (50,000/year) justifies tooling cost
- Aluminum is ideal for die casting
- Complex internal features are possible
- ±0.3mm tolerance is achievable
- Good surface finish without secondary operations

Alternative: If volumes were lower (\textless{} 5,000), investment casting might be considered.

Question 2: Cost Comparison

\textbf{Calculate the break-even point between 3D printing and injection molding for a small plastic part:}
- 3D printing: \$8 per part (no tooling)
- Injection molding: \$25,000 mold + \$0.50 per part

\textbf{Answer:}
Let n = number of parts

Break-even: \$8n = \$25,000 + \$0.50n

\$7.50n = \$25,000

n = 3,333 parts

\textbf{Below 3,333 parts:} 3D printing is cheaper
\textbf{Above 3,333 parts:} Injection molding is cheaper

At 10,000 parts:
- 3D printing: \$80,000
- Injection molding: \$30,000

Question 3: Material Selection

\textbf{Match each material/product to the most appropriate manufacturing process:}

\begin{enumerate}
\def\labelenumi{\alph{enumi})}
\tightlist
\item
  Stainless steel surgical instrument
\item
  Large plastic kayak
\item
  Titanium jet engine bracket
\item
  Aluminum beverage can
\end{enumerate}

\textbf{Answers:}

\begin{enumerate}
\def\labelenumi{\alph{enumi})}
\tightlist
\item
  \textbf{Investment casting} or \textbf{machining} - Complex shapes, high precision, excellent surface finish
\item
  \textbf{Rotational molding} - Large hollow plastic part, low tooling cost
\item
  \textbf{Additive manufacturing (DMLS)} or \textbf{forging} - Complex geometry, lightweight, strength critical
\item
  \textbf{Deep drawing (stamping)} - Very high volume, thin-wall cylinder
\end{enumerate}

Question 4: Process Selection Scenario

\textbf{You need to manufacture a new product with these requirements:}
- Quantity: 200 units for initial market test
- Material: ABS plastic
- Size: 150mm × 100mm × 50mm
- Tolerance: ±0.5mm
- Surface: Smooth, will be painted

\textbf{What process would you recommend and why?}

\textbf{Answer:} \textbf{3D Printing (FDM or SLA)} would be most appropriate because:

\begin{enumerate}
\def\labelenumi{\arabic{enumi}.}
\tightlist
\item
  \textbf{Low quantity (200)} - Injection molding tooling (\$30,000+) not justified
\item
  \textbf{Market test phase} - Design may change based on feedback
\item
  \textbf{Quick turnaround} - Parts in days vs.~weeks for tooling
\item
  \textbf{Tolerances achievable} - ±0.5mm is within FDM capabilities
\item
  \textbf{Post-processing acceptable} - Parts will be painted anyway
\end{enumerate}

\textbf{Cost comparison:}
- FDM: 200 × \$15 = \$3,000
- Injection molding: \$30,000 + (200 × \$1) = \$30,200

If the product succeeds and volume increases to 10,000+, transition to injection molding.

Question 5: Hybrid Manufacturing

\textbf{What is hybrid manufacturing and when is it used?}

\textbf{Answer:} \textbf{Hybrid manufacturing} combines additive and subtractive processes in a single machine or workflow.

\textbf{Common combinations:}
- 3D print near-net shape → CNC machine critical surfaces
- Cast blank → Machine precision features
- Forge rough shape → Machine final dimensions

\textbf{Benefits:}
- Reduced material waste vs.~pure machining
- Better precision than pure additive
- Faster than pure machining for complex parts

\textbf{Example:} Aerospace companies 3D print titanium brackets, then machine mounting surfaces to achieve tight tolerances. This can reduce material waste by 90\% compared to machining from billet.

\begin{center}\rule{0.5\linewidth}{0.5pt}\end{center}

\section{Summary}\label{summary}

\label{tab:summary-table-mfg}Manufacturing Processes Summary

Process Category

Key Principle

Best For

Typical Industries

Casting

Pour liquid into mold

Complex metal shapes, medium-high volume

Automotive, aerospace

Molding

Shape heated/softened material

Plastic parts, very high volume

Consumer products, packaging

Forming

Deform solid with force

Strong metal parts, high volume

Automotive, construction

Machining

Remove material with tools

Precision parts, any material

All industries

Joining

Combine multiple parts

Assembly of components

All industries

Coating

Apply protective layer

Corrosion/wear protection

All industries

Additive

Build layer by layer

Prototypes, complex geometry

Aerospace, medical, prototyping

\subsection{Key Takeaways}\label{key-takeaways}

\begin{enumerate}
\def\labelenumi{\arabic{enumi}.}
\tightlist
\item
  \textbf{No single process is best for everything} - Selection depends on material, volume, complexity, and cost constraints
\item
  \textbf{Volume drives economics} - High tooling costs require high volumes to amortize
\item
  \textbf{Consider the complete supply chain} - Secondary operations, assembly, and finishing affect total cost
\item
  \textbf{Sustainability matters} - Modern manufacturing must minimize waste and energy use
\item
  \textbf{Hybrid approaches are growing} - Combining processes optimizes cost and performance
\end{enumerate}

\begin{center}\rule{0.5\linewidth}{0.5pt}\end{center}

\section{References}\label{references}

\begin{enumerate}
\def\labelenumi{\arabic{enumi}.}
\tightlist
\item
  Groover, M.P. (2020). \emph{Fundamentals of Modern Manufacturing} (7th ed.). Wiley.
\item
  Kalpakjian, S., \& Schmid, S.R. (2020). \emph{Manufacturing Engineering and Technology} (8th ed.). Pearson.
\item
  DeGarmo, E.P., Black, J.T., \& Kohser, R.A. (2019). \emph{DeGarmo's Materials and Processes in Manufacturing} (13th ed.). Wiley.
\item
  Afzal, M. ``Types of Manufacturing Processes.'' \emph{Engineering Articles}.
\end{enumerate}

\chapter{Process and Design}\label{process-and-design}

\begin{center}\rule{0.5\linewidth}{0.5pt}\end{center}

\section{Learning Objectives}\label{learning-objectives-1}

By the end of this chapter, you will be able to:

\begin{enumerate}
\def\labelenumi{\arabic{enumi}.}
\tightlist
\item
  Describe the stages of process plant project development from concept to operation
\item
  Differentiate between fixed, programmable, and flexible automation systems
\item
  Interpret Process Flow Diagrams (PFDs) and Piping \& Instrumentation Diagrams (P\&IDs)
\item
  Explain the purpose and content of Front-End Engineering Design (FEED)
\item
  Apply Design for Manufacturing (DFM) and Design for Assembly (DFA) principles
\item
  Understand the role of HAZOP studies in process safety
\item
  Identify key engineering deliverables for each project phase
\end{enumerate}

\begin{center}\rule{0.5\linewidth}{0.5pt}\end{center}

\section{Introduction to Process Engineering}\label{introduction-to-process-engineering}

\textbf{Process engineering} is a specialized branch of engineering that focuses on the design, optimization, and implementation of manufacturing processes and production systems. Process engineers work at the intersection of design and production, ensuring that products can be manufactured efficiently, safely, and economically.

What is the role of a process engineer?

Process engineers are responsible for:

\begin{itemize}
\tightlist
\item
  \textbf{Designing production processes} - Selecting equipment, defining sequences, optimizing flow
\item
  \textbf{Improving existing processes} - Reducing waste, increasing throughput, improving quality
\item
  \textbf{Scaling up from lab to production} - Taking R\&D concepts to full-scale manufacturing
\item
  \textbf{Ensuring safety and compliance} - Meeting regulatory and environmental requirements
\item
  \textbf{Supporting production} - Troubleshooting issues, training operators
\item
  \textbf{Cost optimization} - Reducing manufacturing costs while maintaining quality
\end{itemize}

\begin{center}\rule{0.5\linewidth}{0.5pt}\end{center}

\section{Types of Manufacturing Automation}\label{types-of-manufacturing-automation}

Automation is essential for achieving high productivity, consistent quality, and competitive manufacturing costs. There are three fundamental types of automation, each suited to different production scenarios.

\begin{figure}

{\centering \includegraphics{introduction_files/figure-latex/automation-comparison-1} 

}

\caption{Comparison of Automation Types}\label{fig:automation-comparison}
\end{figure}

\subsection{Fixed Automation}\label{fixed-automation}

\textbf{Fixed automation} (also called ``hard automation'') uses specialized equipment designed for a specific product. The sequence of operations is determined by the equipment configuration and cannot be easily changed.

\textbf{Characteristics:}
- Very high production rates (thousands per hour)
- Low per-unit cost at high volumes
- High initial capital investment
- Inflexible - dedicated to one product

\textbf{Examples:}
- Automotive transfer lines for engine blocks
- Beverage bottling lines
- Chemical processing plants
- Paper manufacturing

\begin{Shaded}
\begin{Highlighting}[]
\CommentTok{\# Fixed Automation Economic Analysis}
\CommentTok{\# When does fixed automation make sense?}

\CommentTok{\# Example: Automotive transfer line for cylinder heads}
\NormalTok{fixed\_automation\_cost }\OtherTok{\textless{}{-}} \DecValTok{5000000}  \CommentTok{\# Initial equipment cost}
\NormalTok{annual\_production }\OtherTok{\textless{}{-}} \DecValTok{500000}       \CommentTok{\# Parts per year}
\NormalTok{fixed\_cost\_per\_part }\OtherTok{\textless{}{-}} \FloatTok{2.50}       \CommentTok{\# Operating cost per part}
\NormalTok{equipment\_life\_years }\OtherTok{\textless{}{-}} \DecValTok{10}

\CommentTok{\# Calculate total cost per part including amortization}
\NormalTok{annual\_equipment\_cost }\OtherTok{\textless{}{-}}\NormalTok{ fixed\_automation\_cost }\SpecialCharTok{/}\NormalTok{ equipment\_life\_years}
\NormalTok{total\_annual\_cost }\OtherTok{\textless{}{-}}\NormalTok{ annual\_equipment\_cost }\SpecialCharTok{+}\NormalTok{ (fixed\_cost\_per\_part }\SpecialCharTok{*}\NormalTok{ annual\_production)}
\NormalTok{cost\_per\_part }\OtherTok{\textless{}{-}}\NormalTok{ total\_annual\_cost }\SpecialCharTok{/}\NormalTok{ annual\_production}

\FunctionTok{cat}\NormalTok{(}\StringTok{"Fixed Automation Analysis:}\SpecialCharTok{\textbackslash{}n}\StringTok{"}\NormalTok{)}
\end{Highlighting}
\end{Shaded}

\begin{verbatim}
## Fixed Automation Analysis:
\end{verbatim}

\begin{Shaded}
\begin{Highlighting}[]
\FunctionTok{cat}\NormalTok{(}\StringTok{"==========================}\SpecialCharTok{\textbackslash{}n}\StringTok{"}\NormalTok{)}
\end{Highlighting}
\end{Shaded}

\begin{verbatim}
## ==========================
\end{verbatim}

\begin{Shaded}
\begin{Highlighting}[]
\FunctionTok{cat}\NormalTok{(}\FunctionTok{sprintf}\NormalTok{(}\StringTok{"Equipment cost: $\%s}\SpecialCharTok{\textbackslash{}n}\StringTok{"}\NormalTok{, }\FunctionTok{format}\NormalTok{(fixed\_automation\_cost, }\AttributeTok{big.mark=}\StringTok{","}\NormalTok{)))}
\end{Highlighting}
\end{Shaded}

\begin{verbatim}
## Equipment cost: $5e+06
\end{verbatim}

\begin{Shaded}
\begin{Highlighting}[]
\FunctionTok{cat}\NormalTok{(}\FunctionTok{sprintf}\NormalTok{(}\StringTok{"Annual production: \%s parts}\SpecialCharTok{\textbackslash{}n}\StringTok{"}\NormalTok{, }\FunctionTok{format}\NormalTok{(annual\_production, }\AttributeTok{big.mark=}\StringTok{","}\NormalTok{)))}
\end{Highlighting}
\end{Shaded}

\begin{verbatim}
## Annual production: 5e+05 parts
\end{verbatim}

\begin{Shaded}
\begin{Highlighting}[]
\FunctionTok{cat}\NormalTok{(}\FunctionTok{sprintf}\NormalTok{(}\StringTok{"Operating cost per part: $\%.2f}\SpecialCharTok{\textbackslash{}n}\StringTok{"}\NormalTok{, fixed\_cost\_per\_part))}
\end{Highlighting}
\end{Shaded}

\begin{verbatim}
## Operating cost per part: $2.50
\end{verbatim}

\begin{Shaded}
\begin{Highlighting}[]
\FunctionTok{cat}\NormalTok{(}\FunctionTok{sprintf}\NormalTok{(}\StringTok{"Amortized equipment cost per part: $\%.2f}\SpecialCharTok{\textbackslash{}n}\StringTok{"}\NormalTok{, annual\_equipment\_cost}\SpecialCharTok{/}\NormalTok{annual\_production))}
\end{Highlighting}
\end{Shaded}

\begin{verbatim}
## Amortized equipment cost per part: $1.00
\end{verbatim}

\begin{Shaded}
\begin{Highlighting}[]
\FunctionTok{cat}\NormalTok{(}\FunctionTok{sprintf}\NormalTok{(}\StringTok{"Total cost per part: $\%.2f}\SpecialCharTok{\textbackslash{}n}\StringTok{"}\NormalTok{, cost\_per\_part))}
\end{Highlighting}
\end{Shaded}

\begin{verbatim}
## Total cost per part: $3.50
\end{verbatim}

\subsection{Programmable Automation}\label{programmable-automation}

\textbf{Programmable automation} allows the production equipment to be reprogrammed for different products. It's suited for batch production where products change periodically.

\textbf{Characteristics:}
- Medium production rates
- Significant changeover time between products
- Lower initial cost than fixed automation
- Suitable for batch sizes of 100-10,000

\textbf{Examples:}
- CNC machining centers
- Industrial robots
- PLC-controlled assembly machines
- Batch chemical reactors

Discussion: When should you choose programmable over fixed automation?

\textbf{Choose programmable automation when:}

\begin{enumerate}
\def\labelenumi{\arabic{enumi}.}
\tightlist
\item
  \textbf{Product variety} - You make multiple products on the same equipment
\item
  \textbf{Volume uncertainty} - Demand may shift between products
\item
  \textbf{Product evolution} - Frequent design changes expected
\item
  \textbf{Lower volumes} - Not enough volume to justify dedicated equipment
\item
  \textbf{Capital constraints} - Can't afford multiple dedicated lines
\end{enumerate}

\textbf{Example scenario:}
A machine shop that makes 20 different part numbers for aerospace customers would use CNC machines (programmable) rather than dedicated transfer lines (fixed) because:
- Volumes per part number are low (100-1,000/year)
- New part designs are introduced regularly
- Flexibility to respond to customer changes is critical

\subsection{Flexible Automation}\label{flexible-automation}

\textbf{Flexible automation} can produce a variety of parts with minimal changeover time. It combines the flexibility of programmable automation with the productivity approaching fixed automation.

\textbf{Characteristics:}
- Medium-high production rates
- Very fast changeover (minutes, not hours)
- High initial investment
- Suitable for medium variety, medium volume

\textbf{Examples:}
- Flexible Manufacturing Systems (FMS)
- Robotic assembly cells
- Automated guided vehicle (AGV) systems
- Modern automotive final assembly lines

\label{tab:automation-selection}Automation Type Selection Guide

Production Characteristic

Fixed Automation

Programmable Automation

Flexible Automation

Annual Volume

\textgreater{} 1,000,000 units

1,000 - 100,000 units

10,000 - 500,000 units

Product Variety

Single product

Several products in batches

Many products, mixed

Changeover Frequency

Rarely (years)

Weekly to monthly

Hours to minutes

Initial Investment

Very High (\$5M+)

Medium (\$500K-\$2M)

High (\$2M-\$10M)

Operating Cost

Lowest per unit

Medium per unit

Low-medium per unit

Best Industries

Automotive, Beverage, Chemical

Job shops, Aerospace, Medical

Electronics, Automotive suppliers

List industries where you can find various automated process systems

\textbf{By automation type:}

\textbf{Fixed Automation:}
- Oil refineries
- Chemical plants
- Food processing (high-volume)
- Beverage bottling
- Automotive powertrain
- Steel mills
- Paper/pulp mills

\textbf{Programmable Automation:}
- Aerospace manufacturing
- Medical device production
- Pharmaceutical (batch)
- Job shops
- Tool \& die making
- Defense contractors

\textbf{Flexible Automation:}
- Electronics assembly
- Automotive final assembly
- Warehouse/distribution
- Semiconductor fabrication
- Consumer appliances
- Furniture manufacturing

\subsection{Automation Advantages and Disadvantages}\label{automation-advantages-and-disadvantages}

\label{tab:automation-pros-cons}Advantages and Disadvantages of Manufacturing Automation

Category

Point

Advantages

Lowered Operating Costs - Reduced labor per unit

Advantages

Improved Worker Safety - Removes humans from hazardous operations

Advantages

Reduced Factory Lead Times - Faster, more predictable production

Advantages

Increased Production Output - 24/7 operation possible

Advantages

Smaller Environmental Footprint - Optimized resource usage

Advantages

Increased Productivity and Efficiency - Consistent cycle times

Advantages

Increased System Versatility - Modern systems adapt to changes

Disadvantages

Higher Start-up Cost - Significant capital investment required

Disadvantages

Higher Cost of Maintenance - Specialized technicians needed

Disadvantages

Obsolescence/Depreciation - Technology changes rapidly

Disadvantages

Workforce Displacement - Fewer direct labor jobs

Disadvantages

Not Economical for Small Scale - Minimum volumes required

\begin{center}\rule{0.5\linewidth}{0.5pt}\end{center}

\section{Project Development Phases}\label{project-development-phases}

Manufacturing projects follow a structured development process from initial concept to full operation. Understanding these phases is essential for process engineers.

\begin{figure}

{\centering \includegraphics{introduction_files/figure-latex/project-phases-1} 

}

\caption{Project Development Phases and Gate Reviews}\label{fig:project-phases}
\end{figure}

\subsection{Phase 1: Concept Development}\label{phase-1-concept-development}

The initial phase defines the project scope, evaluates feasibility, and develops preliminary cost estimates.

\textbf{Key Deliverables:}
- Project charter and business case
- Preliminary process flow diagrams
- Order-of-magnitude cost estimate (±30-50\%)
- Initial schedule
- Risk assessment

\subsection{Phase 2: Front-End Engineering Design (FEED)}\label{phase-2-front-end-engineering-design-feed}

FEED (also called FEL - Front-End Loading) develops the project definition to a level sufficient for final investment decision.

\textbf{FEED Deliverables include:}

\begin{itemize}
\tightlist
\item
  Project organization chart
\item
  Defined project scope
\item
  Process Flow Diagrams (PFDs)
\item
  Piping \& Instrumentation Diagrams (P\&IDs)
\item
  Equipment datasheets and specifications
\item
  Major equipment list
\item
  2D and 3D preliminary models
\item
  Equipment layout and installation plan
\item
  HAZOP and safety studies
\item
  Automation strategy
\item
  Budget estimate (±10-15\%)
\item
  Project timeline
\item
  Fixed-bid quote capability
\end{itemize}

What are the steps of Process System Design and Engineering?

\begin{enumerate}
\def\labelenumi{\arabic{enumi}.}
\tightlist
\item
  \textbf{Process Development}

  \begin{itemize}
  \tightlist
  \item
    Develop PFDs \& P\&IDs
  \item
    Process simulation
  \item
    Heat and mass balance
  \end{itemize}
\item
  \textbf{Equipment Engineering}

  \begin{itemize}
  \tightlist
  \item
    2D \& 3D models
  \item
    Equipment layout and installation plan
  \item
    Skid design
  \item
    Mechanical \& structural design
  \end{itemize}
\item
  \textbf{Systems Engineering}

  \begin{itemize}
  \tightlist
  \item
    Piping design
  \item
    Electrical design
  \item
    Instrumentation design
  \item
    Control system architecture
  \end{itemize}
\item
  \textbf{Installation Planning}

  \begin{itemize}
  \tightlist
  \item
    Lifts \& installation planning
  \item
    Construction sequencing
  \item
    Tie-in planning
  \end{itemize}
\end{enumerate}

\subsection{Phase 3: Detailed Engineering}\label{phase-3-detailed-engineering}

Detailed engineering produces all documents needed for procurement and construction.

\label{tab:engineering-disciplines}Engineering Disciplines and Deliverables

Discipline

Key Deliverables

Systems Provided

Process

P\&IDs, datasheets, process packages, operating procedures

Process units, utility systems

Mechanical

Vessel drawings, heat exchanger specs, rotating equipment specs

Pressure vessels, tanks, exchangers, pumps

Piping

Isometrics, pipe specs, stress analysis, 3D models

Process piping, utility piping

Electrical

Single-line diagrams, load lists, cable schedules, lighting

Power distribution, motor control, lighting

I\&C

Logic diagrams, instrument index, control narratives, HMI specs

DCS/PLC, safety systems, instruments

Civil/Structural

Foundation drawings, structural steel, buildings, roads

Foundations, structures, buildings

Automation and Controls Engineering needs to include which tasks?

\textbf{Control System Design:}
- Automation architecture design
- Control panel fabrication drawings
- PLC/DCS programming
- HMI/SCADA design
- Safety system design (SIS)

\textbf{Documentation:}
- Control narratives
- Logic diagrams (cause \& effect)
- Instrument datasheets
- Loop diagrams
- Cable schedules

\textbf{Integration:}
- Controls integration services
- Network architecture
- Cybersecurity
- Historian configuration
- MES integration

\subsection{Phase 4-7: Construction, Commissioning, and Start-up}\label{phase-4-7-construction-commissioning-and-start-up}

\label{tab:project-activities}Later Project Phases

Phase

Key Activities

Process Engineer Role

Typical Duration

Procurement \& Fabrication

Purchase equipment, fabricate skids, manufacture custom items

Review vendor documents, approve equipment

3-12 months

Construction

Civil work, install equipment, piping, electrical, instrumentation

Supervise installation, resolve field issues

6-24 months

Commissioning

System checkout, loop testing, safety system validation

Verify system operation, support testing

1-3 months

Start-up

Initial operation, performance testing, operator training

Optimize process, troubleshoot issues, train operators

1-3 months

What needs to be planned for Testing \& Commissioning?

\textbf{Pre-Commissioning:}
- Mechanical completion checks
- Piping pressure tests
- Electrical continuity tests
- Instrument calibration

\textbf{Commissioning:}
- Loop checking
- Control system function tests
- Safety system (SIS) validation
- Utility systems start-up
- Dry run testing

\textbf{Performance Testing:}
- Factory Acceptance Tests (FAT)
- Site Acceptance Tests (SAT)
- Performance guarantee tests
- Reliability testing

\begin{center}\rule{0.5\linewidth}{0.5pt}\end{center}

\section{Process Documentation}\label{process-documentation}

\subsection{Process Flow Diagrams (PFDs)}\label{process-flow-diagrams-pfds}

PFDs show the overall process flow, major equipment, and key operating conditions. They are the ``big picture'' view of the process.

\label{tab:pfd-elements}Process Flow Diagram (PFD) Content

Element

What's Shown

What's NOT Shown

Major Equipment

Reactors, vessels, heat exchangers, pumps, compressors

Pipe sizes, valve types, instrument details

Process Streams

Main process flow lines with arrows

Minor bypasses, drain lines, vent lines

Operating Conditions

Temperature, pressure, flow rates at key points

All intermediate conditions

Control Loops

Major control loops only

Local instruments, minor controls

Utility Streams

Steam, cooling water, nitrogen (simplified)

Detailed utility connections

Stream Data

Stream numbers with heat/mass balance table

All intermediate streams

\subsection{Piping \& Instrumentation Diagrams (P\&IDs)}\label{piping-instrumentation-diagrams-pids}

P\&IDs are the detailed ``roadmap'' of the process, showing all equipment, piping, valves, and instrumentation.

\begin{figure}

{\centering \includegraphics{introduction_files/figure-latex/pid-symbols-1} 

}

\caption{Common P&ID Symbols (Simplified)}\label{fig:pid-symbols}
\end{figure}

\label{tab:instrument-tags}ISA Instrument Tag Structure

First Letter

Measured Variable

Modifier Letters

Example Tag

F

Flow

T = Transmitter

FT-101 (Flow Transmitter)

L

Level

C = Controller

LC-201 (Level Controller)

P

Pressure

I = Indicator

PI-301 (Pressure Indicator)

T

Temperature

E = Element/Sensor

TE-401 (Temperature Element)

A

Analysis

V = Valve

AT-501 (Analyzer Transmitter)

S

Speed

A = Alarm

SI-601 (Speed Indicator)

Quick Quiz: Decode these instrument tags

\textbf{Decode the following P\&ID tags:}

\begin{enumerate}
\def\labelenumi{\arabic{enumi}.}
\tightlist
\item
  \textbf{FIC-101} = ?
\item
  \textbf{TT-205} = ?
\item
  \textbf{PSV-301} = ?
\item
  \textbf{LCV-402} = ?
\end{enumerate}

\textbf{Answers:}

\begin{enumerate}
\def\labelenumi{\arabic{enumi}.}
\tightlist
\item
  \textbf{FIC-101} = Flow Indicating Controller (loop 101)
\item
  \textbf{TT-205} = Temperature Transmitter (loop 205)
\item
  \textbf{PSV-301} = Pressure Safety Valve (loop 301)
\item
  \textbf{LCV-402} = Level Control Valve (loop 402)
\end{enumerate}

\begin{center}\rule{0.5\linewidth}{0.5pt}\end{center}

\section{Design for Manufacturing (DFM) and Assembly (DFA)}\label{design-for-manufacturing-dfm-and-assembly-dfa}

\textbf{DFM} and \textbf{DFA} are methodologies that consider manufacturing and assembly constraints during product design, reducing cost and improving quality.

\subsection{DFM Principles}\label{dfm-principles}

\begin{figure}

{\centering \includegraphics{introduction_files/figure-latex/dfm-principles-1} 

}

\caption{DFM Guidelines}\label{fig:dfm-principles}
\end{figure}

\label{tab:dfm-guidelines}Design for Manufacturing Guidelines

DFM Guideline

Why It Matters

Example

Minimize total number of parts

Fewer parts = fewer chances for failure, less inventory, lower cost

Combine bracket and housing into single molded part

Use standard components

Standard parts are cheaper, faster to procure, easier to replace

Use M6 bolts instead of custom fasteners

Design for ease of fabrication

Complex features increase manufacturing cost and defect rates

Avoid undercuts, deep pockets, thin walls

Design for ease of assembly

Reduces assembly time and errors

Use snap-fits instead of screws where appropriate

Minimize assembly directions

Multiple directions require repositioning, increasing time

All parts insert from top-down in assembly

Maximize compliance in assembly

Tight tolerances increase cost; allow self-alignment where possible

Use tapered holes for self-centering pins

Design for ease of service

Products will need maintenance; design for technician access

Locate service points on accessible panels

\subsection{DFA Analysis}\label{dfa-analysis}

DFA systematically evaluates assembly operations to identify improvement opportunities.

\begin{Shaded}
\begin{Highlighting}[]
\CommentTok{\# DFA Analysis Example: Simple Assembly}
\CommentTok{\# Calculate assembly efficiency using Boothroyd{-}Dewhurst method}

\CommentTok{\# Assembly data (simplified)}
\NormalTok{parts }\OtherTok{\textless{}{-}} \FunctionTok{data.frame}\NormalTok{(}
  \AttributeTok{Part =} \FunctionTok{c}\NormalTok{(}\StringTok{"Base"}\NormalTok{, }\StringTok{"PCB"}\NormalTok{, }\StringTok{"Cover"}\NormalTok{, }\StringTok{"Screw 1"}\NormalTok{, }\StringTok{"Screw 2"}\NormalTok{, }\StringTok{"Screw 3"}\NormalTok{, }\StringTok{"Screw 4"}\NormalTok{, }\StringTok{"Label"}\NormalTok{),}
  \AttributeTok{Handling\_Time =} \FunctionTok{c}\NormalTok{(}\FloatTok{2.5}\NormalTok{, }\FloatTok{3.0}\NormalTok{, }\FloatTok{2.5}\NormalTok{, }\FloatTok{3.5}\NormalTok{, }\FloatTok{3.5}\NormalTok{, }\FloatTok{3.5}\NormalTok{, }\FloatTok{3.5}\NormalTok{, }\FloatTok{4.0}\NormalTok{),  }\CommentTok{\# seconds}
  \AttributeTok{Insertion\_Time =} \FunctionTok{c}\NormalTok{(}\FloatTok{3.0}\NormalTok{, }\FloatTok{4.0}\NormalTok{, }\FloatTok{3.5}\NormalTok{, }\FloatTok{6.0}\NormalTok{, }\FloatTok{6.0}\NormalTok{, }\FloatTok{6.0}\NormalTok{, }\FloatTok{6.0}\NormalTok{, }\FloatTok{2.0}\NormalTok{),  }\CommentTok{\# seconds}
  \AttributeTok{Theoretically\_Necessary =} \FunctionTok{c}\NormalTok{(}\ConstantTok{TRUE}\NormalTok{, }\ConstantTok{TRUE}\NormalTok{, }\ConstantTok{TRUE}\NormalTok{, }\ConstantTok{FALSE}\NormalTok{, }\ConstantTok{FALSE}\NormalTok{, }\ConstantTok{FALSE}\NormalTok{, }\ConstantTok{FALSE}\NormalTok{, }\ConstantTok{FALSE}\NormalTok{)}
\NormalTok{)}

\CommentTok{\# Calculations}
\NormalTok{total\_parts }\OtherTok{\textless{}{-}} \FunctionTok{nrow}\NormalTok{(parts)}
\NormalTok{total\_time }\OtherTok{\textless{}{-}} \FunctionTok{sum}\NormalTok{(parts}\SpecialCharTok{$}\NormalTok{Handling\_Time }\SpecialCharTok{+}\NormalTok{ parts}\SpecialCharTok{$}\NormalTok{Insertion\_Time)}
\NormalTok{min\_parts }\OtherTok{\textless{}{-}} \FunctionTok{sum}\NormalTok{(parts}\SpecialCharTok{$}\NormalTok{Theoretically\_Necessary)}
\NormalTok{ideal\_time }\OtherTok{\textless{}{-}}\NormalTok{ min\_parts }\SpecialCharTok{*} \DecValTok{3}  \CommentTok{\# 3 seconds is theoretical minimum per part}

\NormalTok{efficiency }\OtherTok{\textless{}{-}}\NormalTok{ (ideal\_time }\SpecialCharTok{/}\NormalTok{ total\_time) }\SpecialCharTok{*} \DecValTok{100}

\FunctionTok{cat}\NormalTok{(}\StringTok{"DFA Analysis Results:}\SpecialCharTok{\textbackslash{}n}\StringTok{"}\NormalTok{)}
\end{Highlighting}
\end{Shaded}

\begin{verbatim}
## DFA Analysis Results:
\end{verbatim}

\begin{Shaded}
\begin{Highlighting}[]
\FunctionTok{cat}\NormalTok{(}\StringTok{"=====================}\SpecialCharTok{\textbackslash{}n}\StringTok{"}\NormalTok{)}
\end{Highlighting}
\end{Shaded}

\begin{verbatim}
## =====================
\end{verbatim}

\begin{Shaded}
\begin{Highlighting}[]
\FunctionTok{cat}\NormalTok{(}\FunctionTok{sprintf}\NormalTok{(}\StringTok{"Total parts: \%d}\SpecialCharTok{\textbackslash{}n}\StringTok{"}\NormalTok{, total\_parts))}
\end{Highlighting}
\end{Shaded}

\begin{verbatim}
## Total parts: 8
\end{verbatim}

\begin{Shaded}
\begin{Highlighting}[]
\FunctionTok{cat}\NormalTok{(}\FunctionTok{sprintf}\NormalTok{(}\StringTok{"Theoretically minimum parts: \%d}\SpecialCharTok{\textbackslash{}n}\StringTok{"}\NormalTok{, min\_parts))}
\end{Highlighting}
\end{Shaded}

\begin{verbatim}
## Theoretically minimum parts: 3
\end{verbatim}

\begin{Shaded}
\begin{Highlighting}[]
\FunctionTok{cat}\NormalTok{(}\FunctionTok{sprintf}\NormalTok{(}\StringTok{"Total assembly time: \%.1f seconds}\SpecialCharTok{\textbackslash{}n}\StringTok{"}\NormalTok{, total\_time))}
\end{Highlighting}
\end{Shaded}

\begin{verbatim}
## Total assembly time: 62.5 seconds
\end{verbatim}

\begin{Shaded}
\begin{Highlighting}[]
\FunctionTok{cat}\NormalTok{(}\FunctionTok{sprintf}\NormalTok{(}\StringTok{"Ideal assembly time: \%.1f seconds}\SpecialCharTok{\textbackslash{}n}\StringTok{"}\NormalTok{, ideal\_time))}
\end{Highlighting}
\end{Shaded}

\begin{verbatim}
## Ideal assembly time: 9.0 seconds
\end{verbatim}

\begin{Shaded}
\begin{Highlighting}[]
\FunctionTok{cat}\NormalTok{(}\FunctionTok{sprintf}\NormalTok{(}\StringTok{"Design Efficiency: \%.1f\%\%}\SpecialCharTok{\textbackslash{}n}\StringTok{"}\NormalTok{, efficiency))}
\end{Highlighting}
\end{Shaded}

\begin{verbatim}
## Design Efficiency: 14.4%
\end{verbatim}

\begin{Shaded}
\begin{Highlighting}[]
\FunctionTok{cat}\NormalTok{(}\StringTok{"}\SpecialCharTok{\textbackslash{}n}\StringTok{Opportunity: Combine 4 screws into snap{-}fit design}\SpecialCharTok{\textbackslash{}n}\StringTok{"}\NormalTok{)}
\end{Highlighting}
\end{Shaded}

\begin{verbatim}
## 
## Opportunity: Combine 4 screws into snap-fit design
\end{verbatim}

Case Study: Redesigning for DFM/DFA

\textbf{Original Design:}
- 15 parts
- 4 different screw types
- Assembly time: 180 seconds
- Multiple assembly directions
- Special tools required

\textbf{Redesigned using DFM/DFA:}
- 8 parts (47\% reduction)
- 1 standard screw type
- Assembly time: 75 seconds (58\% reduction)
- Single assembly direction (top-down)
- No special tools

\textbf{Cost Impact:}
- Part cost reduced 35\%
- Assembly labor reduced 58\%
- Quality defects reduced 60\%
- Total product cost reduced 40\%

\begin{center}\rule{0.5\linewidth}{0.5pt}\end{center}

\section{HAZOP Studies}\label{hazop-studies}

\textbf{HAZOP} (Hazard and Operability Study) is a systematic method to identify hazards and operability problems in process plants.

\subsection{HAZOP Methodology}\label{hazop-methodology}

\label{tab:hazop-keywords}HAZOP Guide Words and Examples

Guide Word

Meaning

Example with FLOW

NO/NOT

Complete negation of design intent

No flow - pump failure, blocked line

MORE

Quantitative increase

More flow - control valve fails open

LESS

Quantitative decrease

Less flow - partial blockage, pump wear

AS WELL AS

Qualitative modification/addition

Flow + contamination - upstream leak

PART OF

Qualitative decrease

Only part of intended flow path active

REVERSE

Logical opposite of intent

Reverse flow - check valve failure

OTHER THAN

Complete substitution

Wrong material flowing

\subsection{HAZOP Example}\label{hazop-example}

\label{tab:hazop-example}HAZOP Worksheet Example

Node

Parameter

Guide Word

Deviation

Cause

Consequence

Safeguard

Action

Feed Pump P-101 to Reactor R-101

Flow

No

No Flow

Pump failure, blocked suction

Reactor starved, runaway possible

Low flow alarm, interlock

Verify interlock function

Feed Pump P-101 to Reactor R-101

Flow

More

More Flow

Control valve fails open

Reactor overflow, environmental release

High level alarm, overflow protection

Add independent high-high level trip

Feed Pump P-101 to Reactor R-101

Pressure

More

High Pressure

Downstream blockage

Pipe rupture, personnel injury

Relief valve, high pressure trip

Verify relief valve sizing

Feed Pump P-101 to Reactor R-101

Temperature

Less

Low Temperature

Cooling water leak into feed

Reaction incomplete, off-spec product

Low temp alarm, feed analysis

Add temperature interlock

Discussion: Why is HAZOP done during FEED, not later?

HAZOP should be completed during FEED because:

\begin{enumerate}
\def\labelenumi{\arabic{enumi}.}
\tightlist
\item
  \textbf{Cost of changes} - Changes during FEED cost 1x; during detailed design cost 10x; during construction cost 100x
\item
  \textbf{Design influence} - P\&IDs are still being developed and can be modified
\item
  \textbf{Equipment selection} - Safety requirements may affect equipment sizing
\item
  \textbf{Budget accuracy} - Safety systems must be included in project budget
\item
  \textbf{Regulatory approval} - Many jurisdictions require HAZOP before construction permits
\end{enumerate}

\textbf{Best practice:} Conduct preliminary HAZOP early in FEED, then detailed HAZOP when P\&IDs are 60-80\% complete.

\begin{center}\rule{0.5\linewidth}{0.5pt}\end{center}

\section{Cost Estimation}\label{cost-estimation}

Accurate cost estimation is critical for project approval and budget management.

\label{tab:cost-estimate-classes}Cost Estimate Classification (AACE International)

AACE Class

Project Phase

Engineering Complete

Accuracy Range

Purpose

Class 5

Concept Screening

0-2\%

-50\% to +100\%

Screening alternatives

Class 4

Concept Study

1-15\%

-30\% to +50\%

Feasibility

Class 3

FEED/Budget

10-40\%

-20\% to +30\%

Authorization

Class 2

Detailed Design

30-75\%

-15\% to +20\%

Control

Class 1

Construction

65-100\%

-10\% to +15\%

Bid/Tender

\begin{figure}

{\centering \includegraphics{introduction_files/figure-latex/cost-breakdown-1} 

}

\caption{Typical Process Plant Cost Breakdown}\label{fig:cost-breakdown}
\end{figure}

\begin{center}\rule{0.5\linewidth}{0.5pt}\end{center}

\section{Case Study: Automotive Paint Line Project}\label{case-study-automotive-paint-line-project}

\begin{figure}

{\centering \includegraphics{introduction_files/figure-latex/case-study-timeline-1} 

}

\caption{Automotive Paint Line Project Timeline}\label{fig:case-study-timeline}
\end{figure}

Case Study Details: Paint Line Engineering Challenges

\textbf{Project Overview:}
- New automotive paint line for SUV body shop
- Capacity: 60 jobs per hour (JPH)
- Process: E-coat, primer, basecoat, clearcoat

\textbf{Key Engineering Challenges:}

\textbf{1. Environmental Control}
- Temperature: ±1°C throughout paint booth
- Humidity: 65\% ±5\% RH
- Air cleanliness: Class 10,000

\textbf{2. Automation Requirements}
- 48 paint robots (8 per zone)
- Conveyor system with 2m pitch
- Zone tracking and recipe management

\textbf{3. Safety Systems}
- Explosion-proof electrical classification
- Fire suppression throughout
- VOC monitoring and control

\textbf{4. Quality Systems}
- Inline thickness measurement
- Automated visual inspection
- Recipe management and traceability

\textbf{Project Outcome:}
- On-time completion
- Under budget by 3\%
- Quality metrics exceeded targets
- 99.2\% first-pass yield achieved

\begin{center}\rule{0.5\linewidth}{0.5pt}\end{center}

\section{Review Questions}\label{review-questions-1}

Question 1: Automation Selection

\textbf{A medical device company needs to produce 5,000 units per year of a surgical instrument. The design is mature and not expected to change. Volumes could grow to 50,000 units if successful. Which automation type should they initially choose?}

\textbf{Answer:} \textbf{Programmable automation} (CNC machining, robotic cells)

\textbf{Reasoning:}
- Current volume (5,000/year) doesn't justify fixed automation
- Flexibility allows for minor design refinements
- Can scale up by adding shifts or machines
- Lower initial investment preserves capital
- If volume grows to 50,000, can justify dedicated equipment later

Question 2: P\&ID Interpretation

\textbf{Decode the following instrument tags and explain their function:}

\begin{enumerate}
\def\labelenumi{\alph{enumi})}
\tightlist
\item
  FIC-201
\item
  PSV-305
\item
  TT-401
\item
  LCV-150
\end{enumerate}

\textbf{Answers:}

\begin{enumerate}
\def\labelenumi{\alph{enumi})}
\tightlist
\item
  \textbf{FIC-201} - Flow Indicating Controller (loop 201)

  \begin{itemize}
  \tightlist
  \item
    Measures flow, displays value, and controls a valve
  \end{itemize}
\item
  \textbf{PSV-305} - Pressure Safety Valve (loop 305)

  \begin{itemize}
  \tightlist
  \item
    Relieves pressure if it exceeds setpoint (safety device)
  \end{itemize}
\item
  \textbf{TT-401} - Temperature Transmitter (loop 401)

  \begin{itemize}
  \tightlist
  \item
    Converts temperature measurement to 4-20mA signal
  \end{itemize}
\item
  \textbf{LCV-150} - Level Control Valve (loop 150)

  \begin{itemize}
  \tightlist
  \item
    Valve that is manipulated by a level controller
  \end{itemize}
\end{enumerate}

Question 3: Project Phases

\textbf{Why is FEED often considered the most critical project phase?}

\textbf{Answer:} FEED is critical because:

\begin{enumerate}
\def\labelenumi{\arabic{enumi}.}
\tightlist
\item
  \textbf{Defines scope} - All subsequent work based on FEED deliverables
\item
  \textbf{Locks in 80\% of cost} - Design decisions determine equipment and construction costs
\item
  \textbf{Lowest cost to change} - Changes cost 10-100x more in later phases
\item
  \textbf{Authorization basis} - Management approves project based on FEED estimate
\item
  \textbf{Risk identification} - HAZOP and risk studies identify safety requirements
\item
  \textbf{Contractor selection} - FEED package used for competitive bidding
\end{enumerate}

\textbf{Poor FEED = project overruns, scope changes, and budget problems.}

Question 4: DFM Application

\textbf{A product currently uses 12 screws of 3 different sizes for assembly. Apply DFM principles to improve this design.}

\textbf{Answer:} Apply these DFM principles:

\begin{enumerate}
\def\labelenumi{\arabic{enumi}.}
\item
  \textbf{Minimize parts}: Can some screws be eliminated by using integral snap-fits?
\item
  \textbf{Standardize}: Use one screw size instead of three

  \begin{itemize}
  \tightlist
  \item
    Reduces inventory complexity
  \item
    Single tool for assembly
  \item
    Lower procurement cost
  \end{itemize}
\item
  \textbf{Design for ease}: Consider alternatives:

  \begin{itemize}
  \tightlist
  \item
    Snap-fit connections (no tools)
  \item
    Twist-lock features
  \item
    Press-fit pins
  \end{itemize}
\end{enumerate}

\textbf{Possible redesign:}
- Reduce from 12 screws to 4 screws (for serviceability)
- All same size (M3 × 8)
- Add 4 snap-fits for initial positioning
- Result: 67\% fewer fasteners, single tool, faster assembly

Question 5: HAZOP Analysis

\textbf{For a reactor feed pump, apply the HAZOP guide word ``REVERSE'' to the parameter ``FLOW''. What could cause this? What are the consequences? What safeguards would you recommend?}

\textbf{Answer:}

\textbf{Deviation:} Reverse flow through feed pump

\textbf{Causes:}
- Pump stops while discharge valve open
- Check valve fails (stuck open or missing)
- Parallel pump creates backflow
- Reactor pressure exceeds pump shutoff head

\textbf{Consequences:}
- Reactor contents backflow into feed tank
- Contamination of feed material
- Possible chemical reaction in feed system
- Pump damage (running backward)

\textbf{Safeguards:}
- Check valve on pump discharge
- Automatic pump discharge block valve (closes on pump stop)
- Low flow alarm and trip
- Reactor pressure monitoring
- Backflow-preventing orifice

\begin{center}\rule{0.5\linewidth}{0.5pt}\end{center}

\section{Summary}\label{summary-1}

\label{tab:summary-ch2}Chapter 2 Summary

Topic

Key Takeaway

Automation Types

Fixed for high volume single product; Programmable for batch; Flexible for variety with quick changeover

Project Phases

Concept → FEED → Detailed Design → Construction → Commissioning → Start-up

PFDs vs P\&IDs

PFD = big picture (equipment, flows); P\&ID = detailed (all instruments, valves, piping)

DFM/DFA

Design for easy manufacturing and assembly reduces cost 20-40\%

HAZOP

Systematic hazard identification using guide words; do it during FEED

Cost Estimation

Estimate accuracy improves with engineering completion; AACE Class 1-5

\subsection{Key Takeaways}\label{key-takeaways-1}

\begin{enumerate}
\def\labelenumi{\arabic{enumi}.}
\tightlist
\item
  \textbf{Automation selection} depends on production volume, variety, and changeover requirements
\item
  \textbf{FEED phase} defines 80\% of project cost with only 10-15\% of engineering effort
\item
  \textbf{P\&IDs} are the critical document connecting design to construction
\item
  \textbf{DFM/DFA} should be applied early in design to maximize impact
\item
  \textbf{HAZOP} identifies hazards systematically and must be done before construction
\item
  \textbf{Cost estimates} improve in accuracy as engineering progresses
\end{enumerate}

\begin{center}\rule{0.5\linewidth}{0.5pt}\end{center}

\section{References}\label{references-1}

\begin{enumerate}
\def\labelenumi{\arabic{enumi}.}
\tightlist
\item
  Towler, G., \& Sinnott, R. (2021). \emph{Chemical Engineering Design} (7th ed.). Elsevier.
\item
  AACE International. (2020). \emph{Cost Estimate Classification System}.
\item
  ISA-5.1. (2022). \emph{Instrumentation Symbols and Identification}.
\item
  Boothroyd, G., Dewhurst, P., \& Knight, W.A. (2011). \emph{Product Design for Manufacture and Assembly} (3rd ed.). CRC Press.
\item
  CCPS. (2008). \emph{Guidelines for Hazard Evaluation Procedures} (3rd ed.). Wiley.
\end{enumerate}

\chapter{Integrated Manufacturing Systems}\label{integrated-manufacturing-systems}

\begin{center}\rule{0.5\linewidth}{0.5pt}\end{center}

\section{Learning Objectives}\label{learning-objectives-2}

By the end of this chapter, you will be able to:

\begin{enumerate}
\def\labelenumi{\arabic{enumi}.}
\tightlist
\item
  Compare different types of production facility layouts and material handling methods
\item
  Calculate Manufacturing Cycle Time (MCT) and Manufacturing Cycle Efficiency (MCE)
\item
  Analyze production line performance including the impact of station failures
\item
  Apply line balancing techniques to optimize workstation assignments
\item
  Explain the concepts of takt time, cycle time, and their relationship
\item
  Describe Flexible Manufacturing Systems (FMS) and their applications
\item
  Understand modern concepts including digital twins and Industry 4.0
\end{enumerate}

\begin{center}\rule{0.5\linewidth}{0.5pt}\end{center}

\section{Material Handling Methods \& Systems}\label{material-handling-methods-systems}

There are four common types of production facility layouts, each with distinct features and material handling requirements:

\label{tab:layout-comparison-table}Material Handling Methods by Plant Layout

Layout Type

Product Characteristics

Material Handling

Examples

Fixed-Position

Large, heavy, or immobile products

Cranes, hoists, fork lifts bring materials to product

Ships, aircraft, buildings, large machinery

Process (Job Shop)

High variety, low volume, custom work

Fork lifts, AGVs, carts move WIP between departments

Machine shops, hospitals, universities

Cellular

Part families, medium variety/volume

Conveyors within cells, carts between cells

Electronics assembly, engine components

Product (Flow Line)

Standardized products, high volume

Conveyors, automated transfer systems

Automotive assembly, bottling, food processing

Discussion: How do you choose between layout types?

\textbf{Selection criteria:}

\begin{longtable}[]{@{}lll@{}}
\toprule\noalign{}
Factor & Favors Process Layout & Favors Product Layout \\
\midrule\noalign{}
\endhead
\bottomrule\noalign{}
\endlastfoot
Volume & Low (\textless{} 1,000/year) & High (\textgreater{} 10,000/year) \\
Variety & High (many products) & Low (1-2 products) \\
Routing & Variable by product & Fixed, same for all \\
Equipment & General-purpose & Specialized \\
Worker skills & High, varied & Specific, repetitive \\
WIP inventory & High & Low \\
Lead time & Long & Short \\
\end{longtable}

\textbf{When in doubt:} If volume and variety are moderate, consider cellular manufacturing as a compromise.

\begin{center}\rule{0.5\linewidth}{0.5pt}\end{center}

\section{Fundamentals of Production Lines}\label{fundamentals-of-production-lines}

A \textbf{production line} consists of workstations arranged so that the product moves from one station to the next, with a portion of the total work performed at each location.

\begin{figure}

{\centering \includegraphics{introduction_files/figure-latex/production-line-viz-1} 

}

\caption{Production Line Concept}\label{fig:production-line-viz}
\end{figure}

\subsection{Methods of Work Transport}\label{methods-of-work-transport}

\label{tab:transport-methods}Work Transport Methods in Production Lines

Method

Description

Advantages

Disadvantages

Applications

Manual

Workers pass parts by hand or using simple carts

Low cost, flexible, easy to implement

Variable pace, potential blocking/starving

Low-volume assembly, repair shops

Continuous

Conveyor moves at constant speed; work done while moving

Smooth flow, no transfer time between stations

Work must be done on moving product

Beverage bottling, canning lines

Synchronous (Intermittent)

All units move simultaneously in discrete steps

Fixed cycle time, easy to control

Rigid timing, all stations must finish together

Automotive body welding, transfer lines

Asynchronous

Units move independently when released by worker

Absorbs variation, no blocking/starving

Requires buffers, more complex control

Electronics assembly, flexible lines

\begin{verbatim}
## Warning in geom_segment(aes(x = 0.5, xend = 5.5, y = 3, yend = 3), color = "#3498DB", : All aesthetics have length 1, but the data has 15
## rows.
## i Please consider using `annotate()` or provide
##   this layer with data containing a single row.
\end{verbatim}

\begin{verbatim}
## Warning in geom_segment(aes(x = 0.7, xend = 1.3, y = 2, yend = 2), color = "#E74C3C", : All aesthetics have length 1, but the data has 15
## rows.
## i Please consider using `annotate()` or provide
##   this layer with data containing a single row.
\end{verbatim}

\begin{verbatim}
## Warning in geom_segment(aes(x = 1.7, xend = 2.3, y = 2, yend = 2), color = "#E74C3C", : All aesthetics have length 1, but the data has 15
## rows.
## i Please consider using `annotate()` or provide
##   this layer with data containing a single row.
\end{verbatim}

\begin{verbatim}
## Warning in geom_segment(aes(x = 2.7, xend = 3.3, y = 2, yend = 2), color = "#E74C3C", : All aesthetics have length 1, but the data has 15
## rows.
## i Please consider using `annotate()` or provide
##   this layer with data containing a single row.
\end{verbatim}

\begin{verbatim}
## Warning in geom_segment(aes(x = 3.7, xend = 4.3, y = 2, yend = 2), color = "#E74C3C", : All aesthetics have length 1, but the data has 15
## rows.
## i Please consider using `annotate()` or provide
##   this layer with data containing a single row.
\end{verbatim}

\begin{verbatim}
## Warning in geom_segment(aes(x = 4.7, xend = 5.3, y = 2, yend = 2), color = "#E74C3C", : All aesthetics have length 1, but the data has 15
## rows.
## i Please consider using `annotate()` or provide
##   this layer with data containing a single row.
\end{verbatim}

\begin{verbatim}
## Warning in geom_segment(aes(x = 0.7, xend = 1.1, y = 1, yend = 1), color = "#27AE60", : All aesthetics have length 1, but the data has 15
## rows.
## i Please consider using `annotate()` or provide
##   this layer with data containing a single row.
\end{verbatim}

\begin{verbatim}
## Warning in geom_segment(aes(x = 1.9, xend = 2.5, y = 1, yend = 1), color = "#27AE60", : All aesthetics have length 1, but the data has 15
## rows.
## i Please consider using `annotate()` or provide
##   this layer with data containing a single row.
\end{verbatim}

\begin{verbatim}
## Warning in geom_segment(aes(x = 2.7, xend = 3.1, y = 1, yend = 1), color = "#27AE60", : All aesthetics have length 1, but the data has 15
## rows.
## i Please consider using `annotate()` or provide
##   this layer with data containing a single row.
\end{verbatim}

\begin{verbatim}
## Warning in geom_segment(aes(x = 4, xend = 4.4, y = 1, yend = 1), color = "#27AE60", : All aesthetics have length 1, but the data has 15
## rows.
## i Please consider using `annotate()` or provide
##   this layer with data containing a single row.
\end{verbatim}

\begin{verbatim}
## Warning in geom_segment(aes(x = 4.8, xend = 5.3, y = 1, yend = 1), color = "#27AE60", : All aesthetics have length 1, but the data has 15
## rows.
## i Please consider using `annotate()` or provide
##   this layer with data containing a single row.
\end{verbatim}

\begin{figure}

{\centering \includegraphics{introduction_files/figure-latex/transport-diagrams-1} 

}

\caption{Work Transport System Types}\label{fig:transport-diagrams}
\end{figure}

What causes blocking and starving in production lines?

\textbf{Starving:} A station is ready to work but has no parts to process.
- \textbf{Causes:} Upstream station is slower, upstream breakdown, material shortage

\textbf{Blocking:} A station finishes but cannot release the part.
- \textbf{Causes:} Downstream station still working, downstream breakdown, no buffer space

\textbf{Solutions:}
1. \textbf{Buffers} - Storage between stations absorbs variation
2. \textbf{Line balancing} - Equalize work content across stations
3. \textbf{Parallel stations} - Add capacity at bottlenecks
4. \textbf{Preventive maintenance} - Reduce breakdowns

\subsection{Work Cycle Types}\label{work-cycle-types}

\begin{verbatim}
## Warning in geom_tile(aes(x = 50, y = 3), width = 100, height = 0.5, fill = "#F39C12"): All aesthetics have length 1, but the data has 3
## rows.
## i Please consider using `annotate()` or provide
##   this layer with data containing a single row.
\end{verbatim}

\begin{verbatim}
## Warning in geom_tile(aes(x = 20, y = 2), width = 40, height = 0.5, fill = "#F39C12"): All aesthetics have length 1, but the data has 3
## rows.
## i Please consider using `annotate()` or provide
##   this layer with data containing a single row.
\end{verbatim}

\begin{verbatim}
## Warning in geom_tile(aes(x = 70, y = 2), width = 60, height = 0.5, fill = "#3498DB"): All aesthetics have length 1, but the data has 3
## rows.
## i Please consider using `annotate()` or provide
##   this layer with data containing a single row.
\end{verbatim}

\begin{verbatim}
## Warning in geom_tile(aes(x = 2.5, y = 1), width = 5, height = 0.5, fill = "#F39C12"): All aesthetics have length 1, but the data has 3
## rows.
## i Please consider using `annotate()` or provide
##   this layer with data containing a single row.
\end{verbatim}

\begin{verbatim}
## Warning in geom_tile(aes(x = 52.5, y = 1), width = 95, height = 0.5, fill = "#3498DB"): All aesthetics have length 1, but the data has 3
## rows.
## i Please consider using `annotate()` or provide
##   this layer with data containing a single row.
\end{verbatim}

\begin{figure}

{\centering \includegraphics{introduction_files/figure-latex/work-cycle-viz-1} 

}

\caption{Manual, Semi-Automated, and Fully Automated Machine Cycles}\label{fig:work-cycle-viz}
\end{figure}

\begin{center}\rule{0.5\linewidth}{0.5pt}\end{center}

\section{Key Time Metrics}\label{key-time-metrics}

Understanding time metrics is essential for analyzing and improving production systems.

\label{tab:time-metrics-table}Key Manufacturing Time Metrics

Metric

Definition

Formula

Purpose

Takt Time

Available production time ÷ Customer demand

\(T_{takt} = \frac{\text{Available Time}}{\text{Demand}}\)

Sets the pace of production to match demand

Cycle Time

Time between units coming off the line

\(T_c = \frac{60}{R_p}\) (min, if Rp in units/hr)

Actual pace of the production line

Lead Time

Total time from order to delivery

Order processing + Manufacturing + Delivery

Customer's wait time

Throughput Time

Time from raw material to finished goods

Processing + Waiting + Moving + Inspection

Internal manufacturing efficiency

\subsection{Takt Time vs.~Cycle Time}\label{takt-time-vs.-cycle-time}

\begin{Shaded}
\begin{Highlighting}[]
\CommentTok{\# Takt Time Calculation Example}
\CommentTok{\# Automotive assembly plant}

\CommentTok{\# Given data}
\NormalTok{shift\_length\_min }\OtherTok{\textless{}{-}} \DecValTok{480}      \CommentTok{\# 8 hours = 480 minutes}
\NormalTok{breaks\_min }\OtherTok{\textless{}{-}} \DecValTok{30}             \CommentTok{\# Two 15{-}minute breaks}
\NormalTok{planned\_downtime\_min }\OtherTok{\textless{}{-}} \DecValTok{20}   \CommentTok{\# Changeover, meetings}
\NormalTok{customer\_demand }\OtherTok{\textless{}{-}} \DecValTok{60}        \CommentTok{\# Cars per shift}

\CommentTok{\# Calculate available time}
\NormalTok{available\_time }\OtherTok{\textless{}{-}}\NormalTok{ shift\_length\_min }\SpecialCharTok{{-}}\NormalTok{ breaks\_min }\SpecialCharTok{{-}}\NormalTok{ planned\_downtime\_min}
\FunctionTok{cat}\NormalTok{(}\StringTok{"Available Production Time:"}\NormalTok{, available\_time, }\StringTok{"minutes per shift}\SpecialCharTok{\textbackslash{}n}\StringTok{"}\NormalTok{)}
\end{Highlighting}
\end{Shaded}

\begin{verbatim}
## Available Production Time: 430 minutes per shift
\end{verbatim}

\begin{Shaded}
\begin{Highlighting}[]
\CommentTok{\# Calculate Takt Time}
\NormalTok{takt\_time }\OtherTok{\textless{}{-}}\NormalTok{ available\_time }\SpecialCharTok{/}\NormalTok{ customer\_demand}
\FunctionTok{cat}\NormalTok{(}\StringTok{"Takt Time:"}\NormalTok{, }\FunctionTok{round}\NormalTok{(takt\_time, }\DecValTok{2}\NormalTok{), }\StringTok{"minutes per car}\SpecialCharTok{\textbackslash{}n}\StringTok{"}\NormalTok{)}
\end{Highlighting}
\end{Shaded}

\begin{verbatim}
## Takt Time: 7.17 minutes per car
\end{verbatim}

\begin{Shaded}
\begin{Highlighting}[]
\FunctionTok{cat}\NormalTok{(}\StringTok{"         ="}\NormalTok{, }\FunctionTok{round}\NormalTok{(takt\_time }\SpecialCharTok{*} \DecValTok{60}\NormalTok{, }\DecValTok{1}\NormalTok{), }\StringTok{"seconds per car}\SpecialCharTok{\textbackslash{}n\textbackslash{}n}\StringTok{"}\NormalTok{)}
\end{Highlighting}
\end{Shaded}

\begin{verbatim}
##          = 430 seconds per car
\end{verbatim}

\begin{Shaded}
\begin{Highlighting}[]
\CommentTok{\# Compare to actual cycle time}
\NormalTok{actual\_cycle\_time }\OtherTok{\textless{}{-}} \FloatTok{7.5}  \CommentTok{\# minutes (current performance)}
\FunctionTok{cat}\NormalTok{(}\StringTok{"Actual Cycle Time:"}\NormalTok{, actual\_cycle\_time, }\StringTok{"minutes per car}\SpecialCharTok{\textbackslash{}n}\StringTok{"}\NormalTok{)}
\end{Highlighting}
\end{Shaded}

\begin{verbatim}
## Actual Cycle Time: 7.5 minutes per car
\end{verbatim}

\begin{Shaded}
\begin{Highlighting}[]
\ControlFlowTok{if}\NormalTok{ (actual\_cycle\_time }\SpecialCharTok{\textless{}=}\NormalTok{ takt\_time) \{}
  \FunctionTok{cat}\NormalTok{(}\StringTok{"Status: MEETING DEMAND {-} Cycle time is within takt time}\SpecialCharTok{\textbackslash{}n}\StringTok{"}\NormalTok{)}
  \FunctionTok{cat}\NormalTok{(}\StringTok{"Capacity margin:"}\NormalTok{, }\FunctionTok{round}\NormalTok{((}\DecValTok{1} \SpecialCharTok{{-}}\NormalTok{ actual\_cycle\_time}\SpecialCharTok{/}\NormalTok{takt\_time) }\SpecialCharTok{*} \DecValTok{100}\NormalTok{, }\DecValTok{1}\NormalTok{), }\StringTok{"\%}\SpecialCharTok{\textbackslash{}n}\StringTok{"}\NormalTok{)}
\NormalTok{\} }\ControlFlowTok{else}\NormalTok{ \{}
  \FunctionTok{cat}\NormalTok{(}\StringTok{"Status: CANNOT MEET DEMAND {-} Cycle time exceeds takt time}\SpecialCharTok{\textbackslash{}n}\StringTok{"}\NormalTok{)}
  \FunctionTok{cat}\NormalTok{(}\StringTok{"Shortfall:"}\NormalTok{, }\FunctionTok{round}\NormalTok{((actual\_cycle\_time}\SpecialCharTok{/}\NormalTok{takt\_time }\SpecialCharTok{{-}} \DecValTok{1}\NormalTok{) }\SpecialCharTok{*} \DecValTok{100}\NormalTok{, }\DecValTok{1}\NormalTok{), }\StringTok{"\% over capacity}\SpecialCharTok{\textbackslash{}n}\StringTok{"}\NormalTok{)}
\NormalTok{\}}
\end{Highlighting}
\end{Shaded}

\begin{verbatim}
## Status: CANNOT MEET DEMAND - Cycle time exceeds takt time
## Shortfall: 4.7 % over capacity
\end{verbatim}

Discussion: What happens when cycle time exceeds takt time?

\textbf{When Cycle Time \textgreater{} Takt Time, you cannot meet customer demand.}

\textbf{Options to fix this:}

\begin{enumerate}
\def\labelenumi{\arabic{enumi}.}
\tightlist
\item
  \textbf{Reduce cycle time:}

  \begin{itemize}
  \tightlist
  \item
    Kaizen improvements at bottleneck
  \item
    Better tools/fixtures
  \item
    Automation of slow tasks
  \end{itemize}
\item
  \textbf{Increase available time:}

  \begin{itemize}
  \tightlist
  \item
    Add overtime
  \item
    Add shifts
  \item
    Reduce breaks (carefully!)
  \end{itemize}
\item
  \textbf{Add capacity:}

  \begin{itemize}
  \tightlist
  \item
    Parallel stations at bottleneck
  \item
    Additional production lines
  \item
    Outsourcing
  \end{itemize}
\item
  \textbf{Manage demand:}

  \begin{itemize}
  \tightlist
  \item
    Negotiate delivery schedules
  \item
    Build inventory in advance
  \item
    Redirect to other facilities
  \end{itemize}
\end{enumerate}

\textbf{Key insight:} Takt time is the ``heartbeat'' of lean manufacturing - everything should be synchronized to it.

\subsection{Manufacturing Cycle Time and Efficiency}\label{manufacturing-cycle-time-and-efficiency}

\begin{Shaded}
\begin{Highlighting}[]
\CommentTok{\# Manufacturing Cycle Time (MCT) and Efficiency (MCE) Example}

\CommentTok{\# Given production data}
\NormalTok{production\_time\_min }\OtherTok{\textless{}{-}} \DecValTok{480}   \CommentTok{\# Total time equipment ran}
\NormalTok{units\_produced }\OtherTok{\textless{}{-}} \DecValTok{60}         \CommentTok{\# Good units produced}

\CommentTok{\# MCT Calculation}
\NormalTok{MCT }\OtherTok{\textless{}{-}}\NormalTok{ production\_time\_min }\SpecialCharTok{/}\NormalTok{ units\_produced}
\FunctionTok{cat}\NormalTok{(}\StringTok{"Manufacturing Cycle Time (MCT):"}\NormalTok{, MCT, }\StringTok{"minutes per unit}\SpecialCharTok{\textbackslash{}n\textbackslash{}n}\StringTok{"}\NormalTok{)}
\end{Highlighting}
\end{Shaded}

\begin{verbatim}
## Manufacturing Cycle Time (MCT): 8 minutes per unit
\end{verbatim}

\begin{Shaded}
\begin{Highlighting}[]
\CommentTok{\# MCE Calculation requires understanding of value{-}added time}
\CommentTok{\# Typical breakdown of production time:}
\NormalTok{value\_added\_time }\OtherTok{\textless{}{-}} \FloatTok{3.5}      \CommentTok{\# Actual processing time (minutes)}
\NormalTok{inspection\_time }\OtherTok{\textless{}{-}} \FloatTok{0.5}       \CommentTok{\# Quality checks}
\NormalTok{wait\_time }\OtherTok{\textless{}{-}} \FloatTok{2.5}             \CommentTok{\# Queuing between operations}
\NormalTok{move\_time }\OtherTok{\textless{}{-}} \FloatTok{1.5}             \CommentTok{\# Transport between stations}

\NormalTok{total\_time\_per\_unit }\OtherTok{\textless{}{-}}\NormalTok{ value\_added\_time }\SpecialCharTok{+}\NormalTok{ inspection\_time }\SpecialCharTok{+}\NormalTok{ wait\_time }\SpecialCharTok{+}\NormalTok{ move\_time}
\FunctionTok{cat}\NormalTok{(}\StringTok{"Time breakdown per unit:}\SpecialCharTok{\textbackslash{}n}\StringTok{"}\NormalTok{)}
\end{Highlighting}
\end{Shaded}

\begin{verbatim}
## Time breakdown per unit:
\end{verbatim}

\begin{Shaded}
\begin{Highlighting}[]
\FunctionTok{cat}\NormalTok{(}\StringTok{"  Value{-}added (processing):"}\NormalTok{, value\_added\_time, }\StringTok{"min}\SpecialCharTok{\textbackslash{}n}\StringTok{"}\NormalTok{)}
\end{Highlighting}
\end{Shaded}

\begin{verbatim}
##   Value-added (processing): 3.5 min
\end{verbatim}

\begin{Shaded}
\begin{Highlighting}[]
\FunctionTok{cat}\NormalTok{(}\StringTok{"  Inspection:"}\NormalTok{, inspection\_time, }\StringTok{"min}\SpecialCharTok{\textbackslash{}n}\StringTok{"}\NormalTok{)}
\end{Highlighting}
\end{Shaded}

\begin{verbatim}
##   Inspection: 0.5 min
\end{verbatim}

\begin{Shaded}
\begin{Highlighting}[]
\FunctionTok{cat}\NormalTok{(}\StringTok{"  Waiting:"}\NormalTok{, wait\_time, }\StringTok{"min}\SpecialCharTok{\textbackslash{}n}\StringTok{"}\NormalTok{)}
\end{Highlighting}
\end{Shaded}

\begin{verbatim}
##   Waiting: 2.5 min
\end{verbatim}

\begin{Shaded}
\begin{Highlighting}[]
\FunctionTok{cat}\NormalTok{(}\StringTok{"  Moving:"}\NormalTok{, move\_time, }\StringTok{"min}\SpecialCharTok{\textbackslash{}n}\StringTok{"}\NormalTok{)}
\end{Highlighting}
\end{Shaded}

\begin{verbatim}
##   Moving: 1.5 min
\end{verbatim}

\begin{Shaded}
\begin{Highlighting}[]
\FunctionTok{cat}\NormalTok{(}\StringTok{"  Total:"}\NormalTok{, total\_time\_per\_unit, }\StringTok{"min}\SpecialCharTok{\textbackslash{}n\textbackslash{}n}\StringTok{"}\NormalTok{)}
\end{Highlighting}
\end{Shaded}

\begin{verbatim}
##   Total: 8 min
\end{verbatim}

\begin{Shaded}
\begin{Highlighting}[]
\CommentTok{\# MCE Calculation}
\NormalTok{MCE }\OtherTok{\textless{}{-}}\NormalTok{ (value\_added\_time }\SpecialCharTok{/}\NormalTok{ total\_time\_per\_unit) }\SpecialCharTok{*} \DecValTok{100}
\FunctionTok{cat}\NormalTok{(}\StringTok{"Manufacturing Cycle Efficiency (MCE):"}\NormalTok{, }\FunctionTok{round}\NormalTok{(MCE, }\DecValTok{1}\NormalTok{), }\StringTok{"\%}\SpecialCharTok{\textbackslash{}n}\StringTok{"}\NormalTok{)}
\end{Highlighting}
\end{Shaded}

\begin{verbatim}
## Manufacturing Cycle Efficiency (MCE): 43.8 %
\end{verbatim}

\begin{Shaded}
\begin{Highlighting}[]
\FunctionTok{cat}\NormalTok{(}\StringTok{"}\SpecialCharTok{\textbackslash{}n}\StringTok{Interpretation: Only"}\NormalTok{, }\FunctionTok{round}\NormalTok{(MCE, }\DecValTok{1}\NormalTok{), }\StringTok{"\% of time adds value.}\SpecialCharTok{\textbackslash{}n}\StringTok{"}\NormalTok{)}
\end{Highlighting}
\end{Shaded}

\begin{verbatim}
## 
## Interpretation: Only 43.8 % of time adds value.
\end{verbatim}

\begin{Shaded}
\begin{Highlighting}[]
\FunctionTok{cat}\NormalTok{(}\StringTok{"Opportunity:"}\NormalTok{, }\FunctionTok{round}\NormalTok{(}\DecValTok{100} \SpecialCharTok{{-}}\NormalTok{ MCE, }\DecValTok{1}\NormalTok{), }\StringTok{"\% of time is waste that could be reduced.}\SpecialCharTok{\textbackslash{}n}\StringTok{"}\NormalTok{)}
\end{Highlighting}
\end{Shaded}

\begin{verbatim}
## Opportunity: 56.2 % of time is waste that could be reduced.
\end{verbatim}

\begin{figure}

{\centering \includegraphics{introduction_files/figure-latex/mce-visual-1} 

}

\caption{Manufacturing Cycle Efficiency - Where Does Time Go?}\label{fig:mce-visual}
\end{figure}

\begin{center}\rule{0.5\linewidth}{0.5pt}\end{center}

\section{Manual Assembly Lines}\label{manual-assembly-lines}

Manual assembly lines use human workers at each station. The efficiency depends on how well work is balanced across stations.

\subsection{Line Balancing}\label{line-balancing}

\textbf{Line balancing} assigns tasks to workstations to minimize idle time while meeting production rate requirements.

\textbf{Key formulas:}

\[T_c = \frac{\text{Available Time}}{\text{Required Output}} = \frac{60}{R_p}\]

\[TM = \frac{\sum t_i}{T_c} = \frac{T_{wc}}{T_c}\]

\[E_b = \frac{T_{wc}}{n \times T_c} \times 100\%\]

Where:
- \(T_c\) = Cycle time (time available per unit)
- \(R_p\) = Production rate (units per hour)
- \(TM\) = Theoretical minimum stations
- \(T_{wc}\) = Total work content (sum of all task times)
- \(n\) = Actual number of stations
- \(E_b\) = Line efficiency (balance efficiency)

\subsection{Line Balancing Example: Electronics Assembly}\label{line-balancing-example-electronics-assembly}

\begin{Shaded}
\begin{Highlighting}[]
\CommentTok{\# Line Balancing Example: Tablet Computer Assembly}

\CommentTok{\# Task data}
\NormalTok{tasks }\OtherTok{\textless{}{-}} \FunctionTok{data.frame}\NormalTok{(}
  \AttributeTok{Task =} \FunctionTok{c}\NormalTok{(}\StringTok{"A"}\NormalTok{, }\StringTok{"B"}\NormalTok{, }\StringTok{"C"}\NormalTok{, }\StringTok{"D"}\NormalTok{, }\StringTok{"E"}\NormalTok{, }\StringTok{"F"}\NormalTok{, }\StringTok{"G"}\NormalTok{, }\StringTok{"H"}\NormalTok{),}
  \AttributeTok{Description =} \FunctionTok{c}\NormalTok{(}\StringTok{"Install battery"}\NormalTok{, }\StringTok{"Mount display"}\NormalTok{, }\StringTok{"Connect cables"}\NormalTok{,}
                  \StringTok{"Install motherboard"}\NormalTok{, }\StringTok{"Add memory"}\NormalTok{, }\StringTok{"Install camera"}\NormalTok{,}
                  \StringTok{"Attach case back"}\NormalTok{, }\StringTok{"Final test"}\NormalTok{),}
  \AttributeTok{Time\_sec =} \FunctionTok{c}\NormalTok{(}\DecValTok{30}\NormalTok{, }\DecValTok{45}\NormalTok{, }\DecValTok{20}\NormalTok{, }\DecValTok{55}\NormalTok{, }\DecValTok{15}\NormalTok{, }\DecValTok{25}\NormalTok{, }\DecValTok{35}\NormalTok{, }\DecValTok{40}\NormalTok{),}
  \AttributeTok{Predecessors =} \FunctionTok{c}\NormalTok{(}\StringTok{"{-}"}\NormalTok{, }\StringTok{"A"}\NormalTok{, }\StringTok{"A"}\NormalTok{, }\StringTok{"B,C"}\NormalTok{, }\StringTok{"D"}\NormalTok{, }\StringTok{"D"}\NormalTok{, }\StringTok{"E,F"}\NormalTok{, }\StringTok{"G"}\NormalTok{)}
\NormalTok{)}

\CommentTok{\# Display task data}
\FunctionTok{cat}\NormalTok{(}\StringTok{"Assembly Tasks:}\SpecialCharTok{\textbackslash{}n}\StringTok{"}\NormalTok{)}
\end{Highlighting}
\end{Shaded}

\begin{verbatim}
## Assembly Tasks:
\end{verbatim}

\begin{Shaded}
\begin{Highlighting}[]
\FunctionTok{print}\NormalTok{(tasks[, }\FunctionTok{c}\NormalTok{(}\StringTok{"Task"}\NormalTok{, }\StringTok{"Description"}\NormalTok{, }\StringTok{"Time\_sec"}\NormalTok{, }\StringTok{"Predecessors"}\NormalTok{)])}
\end{Highlighting}
\end{Shaded}

\begin{verbatim}
##   Task         Description Time_sec Predecessors
## 1    A     Install battery       30            -
## 2    B       Mount display       45            A
## 3    C      Connect cables       20            A
## 4    D Install motherboard       55          B,C
## 5    E          Add memory       15            D
## 6    F      Install camera       25            D
## 7    G    Attach case back       35          E,F
## 8    H          Final test       40            G
\end{verbatim}

\begin{Shaded}
\begin{Highlighting}[]
\CommentTok{\# Calculate metrics}
\NormalTok{total\_work\_content }\OtherTok{\textless{}{-}} \FunctionTok{sum}\NormalTok{(tasks}\SpecialCharTok{$}\NormalTok{Time\_sec)}
\FunctionTok{cat}\NormalTok{(}\StringTok{"}\SpecialCharTok{\textbackslash{}n}\StringTok{Total Work Content:"}\NormalTok{, total\_work\_content, }\StringTok{"seconds}\SpecialCharTok{\textbackslash{}n}\StringTok{"}\NormalTok{)}
\end{Highlighting}
\end{Shaded}

\begin{verbatim}
## 
## Total Work Content: 265 seconds
\end{verbatim}

\begin{Shaded}
\begin{Highlighting}[]
\CommentTok{\# Production requirements}
\NormalTok{desired\_output }\OtherTok{\textless{}{-}} \DecValTok{45}  \CommentTok{\# units per hour}
\NormalTok{available\_time }\OtherTok{\textless{}{-}} \DecValTok{3600}  \CommentTok{\# seconds per hour}

\CommentTok{\# Step 1: Calculate cycle time}
\NormalTok{cycle\_time }\OtherTok{\textless{}{-}}\NormalTok{ available\_time }\SpecialCharTok{/}\NormalTok{ desired\_output}
\FunctionTok{cat}\NormalTok{(}\StringTok{"}\SpecialCharTok{\textbackslash{}n}\StringTok{Step 1 {-} Cycle Time:"}\NormalTok{)}
\end{Highlighting}
\end{Shaded}

\begin{verbatim}
## 
## Step 1 - Cycle Time:
\end{verbatim}

\begin{Shaded}
\begin{Highlighting}[]
\FunctionTok{cat}\NormalTok{(}\StringTok{"}\SpecialCharTok{\textbackslash{}n}\StringTok{  Required output:"}\NormalTok{, desired\_output, }\StringTok{"units/hour"}\NormalTok{)}
\end{Highlighting}
\end{Shaded}

\begin{verbatim}
## 
##   Required output: 45 units/hour
\end{verbatim}

\begin{Shaded}
\begin{Highlighting}[]
\FunctionTok{cat}\NormalTok{(}\StringTok{"}\SpecialCharTok{\textbackslash{}n}\StringTok{  Cycle time = 3600 /"}\NormalTok{, desired\_output, }\StringTok{"="}\NormalTok{, cycle\_time, }\StringTok{"seconds}\SpecialCharTok{\textbackslash{}n}\StringTok{"}\NormalTok{)}
\end{Highlighting}
\end{Shaded}

\begin{verbatim}
## 
##   Cycle time = 3600 / 45 = 80 seconds
\end{verbatim}

\begin{Shaded}
\begin{Highlighting}[]
\CommentTok{\# Step 2: Calculate theoretical minimum stations}
\NormalTok{TM }\OtherTok{\textless{}{-}}\NormalTok{ total\_work\_content }\SpecialCharTok{/}\NormalTok{ cycle\_time}
\FunctionTok{cat}\NormalTok{(}\StringTok{"}\SpecialCharTok{\textbackslash{}n}\StringTok{Step 2 {-} Theoretical Minimum Stations:"}\NormalTok{)}
\end{Highlighting}
\end{Shaded}

\begin{verbatim}
## 
## Step 2 - Theoretical Minimum Stations:
\end{verbatim}

\begin{Shaded}
\begin{Highlighting}[]
\FunctionTok{cat}\NormalTok{(}\StringTok{"}\SpecialCharTok{\textbackslash{}n}\StringTok{  TM ="}\NormalTok{, total\_work\_content, }\StringTok{"/"}\NormalTok{, cycle\_time, }\StringTok{"="}\NormalTok{, }\FunctionTok{round}\NormalTok{(TM, }\DecValTok{2}\NormalTok{))}
\end{Highlighting}
\end{Shaded}

\begin{verbatim}
## 
##   TM = 265 / 80 = 3.31
\end{verbatim}

\begin{Shaded}
\begin{Highlighting}[]
\FunctionTok{cat}\NormalTok{(}\StringTok{"}\SpecialCharTok{\textbackslash{}n}\StringTok{  Minimum stations required:"}\NormalTok{, }\FunctionTok{ceiling}\NormalTok{(TM), }\StringTok{"}\SpecialCharTok{\textbackslash{}n}\StringTok{"}\NormalTok{)}
\end{Highlighting}
\end{Shaded}

\begin{verbatim}
## 
##   Minimum stations required: 4
\end{verbatim}

\begin{Shaded}
\begin{Highlighting}[]
\CommentTok{\# Step 3: Assign tasks to stations (using largest task time rule)}
\FunctionTok{cat}\NormalTok{(}\StringTok{"}\SpecialCharTok{\textbackslash{}n}\StringTok{Step 3 {-} Station Assignments (respecting precedence):}\SpecialCharTok{\textbackslash{}n}\StringTok{"}\NormalTok{)}
\end{Highlighting}
\end{Shaded}

\begin{verbatim}
## 
## Step 3 - Station Assignments (respecting precedence):
\end{verbatim}

\begin{Shaded}
\begin{Highlighting}[]
\FunctionTok{cat}\NormalTok{(}\StringTok{"  Station 1: A (30s) + C (20s) + B (45s) = 95s \textgreater{} 80s... EXCEEDS}\SpecialCharTok{\textbackslash{}n}\StringTok{"}\NormalTok{)}
\end{Highlighting}
\end{Shaded}

\begin{verbatim}
##   Station 1: A (30s) + C (20s) + B (45s) = 95s > 80s... EXCEEDS
\end{verbatim}

\begin{Shaded}
\begin{Highlighting}[]
\FunctionTok{cat}\NormalTok{(}\StringTok{"  Let\textquotesingle{}s try: A (30s) + C (20s) = 50s [30s idle]}\SpecialCharTok{\textbackslash{}n}\StringTok{"}\NormalTok{)}
\end{Highlighting}
\end{Shaded}

\begin{verbatim}
##   Let's try: A (30s) + C (20s) = 50s [30s idle]
\end{verbatim}

\begin{Shaded}
\begin{Highlighting}[]
\FunctionTok{cat}\NormalTok{(}\StringTok{"  Station 2: B (45s) + E (15s) = 60s... wait, E needs D}\SpecialCharTok{\textbackslash{}n}\StringTok{"}\NormalTok{)}
\end{Highlighting}
\end{Shaded}

\begin{verbatim}
##   Station 2: B (45s) + E (15s) = 60s... wait, E needs D
\end{verbatim}

\begin{Shaded}
\begin{Highlighting}[]
\FunctionTok{cat}\NormalTok{(}\StringTok{"  Station 2: B (45s) + D (55s) = 100s \textgreater{} 80s... EXCEEDS}\SpecialCharTok{\textbackslash{}n}\StringTok{"}\NormalTok{)}
\end{Highlighting}
\end{Shaded}

\begin{verbatim}
##   Station 2: B (45s) + D (55s) = 100s > 80s... EXCEEDS
\end{verbatim}

\begin{Shaded}
\begin{Highlighting}[]
\FunctionTok{cat}\NormalTok{(}\StringTok{"  Station 2: B (45s) = 45s [35s idle]}\SpecialCharTok{\textbackslash{}n}\StringTok{"}\NormalTok{)}
\end{Highlighting}
\end{Shaded}

\begin{verbatim}
##   Station 2: B (45s) = 45s [35s idle]
\end{verbatim}

\begin{Shaded}
\begin{Highlighting}[]
\FunctionTok{cat}\NormalTok{(}\StringTok{"  Station 3: D (55s) = 55s [25s idle]}\SpecialCharTok{\textbackslash{}n}\StringTok{"}\NormalTok{)}
\end{Highlighting}
\end{Shaded}

\begin{verbatim}
##   Station 3: D (55s) = 55s [25s idle]
\end{verbatim}

\begin{Shaded}
\begin{Highlighting}[]
\FunctionTok{cat}\NormalTok{(}\StringTok{"  Station 4: E (15s) + F (25s) + G (35s) = 75s [5s idle]}\SpecialCharTok{\textbackslash{}n}\StringTok{"}\NormalTok{)}
\end{Highlighting}
\end{Shaded}

\begin{verbatim}
##   Station 4: E (15s) + F (25s) + G (35s) = 75s [5s idle]
\end{verbatim}

\begin{Shaded}
\begin{Highlighting}[]
\FunctionTok{cat}\NormalTok{(}\StringTok{"  Station 5: H (40s) = 40s [40s idle]}\SpecialCharTok{\textbackslash{}n}\StringTok{"}\NormalTok{)}
\end{Highlighting}
\end{Shaded}

\begin{verbatim}
##   Station 5: H (40s) = 40s [40s idle]
\end{verbatim}

\begin{Shaded}
\begin{Highlighting}[]
\NormalTok{n\_stations }\OtherTok{\textless{}{-}} \DecValTok{5}
\NormalTok{efficiency }\OtherTok{\textless{}{-}}\NormalTok{ (total\_work\_content }\SpecialCharTok{/}\NormalTok{ (n\_stations }\SpecialCharTok{*}\NormalTok{ cycle\_time)) }\SpecialCharTok{*} \DecValTok{100}
\NormalTok{balance\_delay }\OtherTok{\textless{}{-}} \DecValTok{100} \SpecialCharTok{{-}}\NormalTok{ efficiency}

\FunctionTok{cat}\NormalTok{(}\StringTok{"}\SpecialCharTok{\textbackslash{}n}\StringTok{Step 4 {-} Results:"}\NormalTok{)}
\end{Highlighting}
\end{Shaded}

\begin{verbatim}
## 
## Step 4 - Results:
\end{verbatim}

\begin{Shaded}
\begin{Highlighting}[]
\FunctionTok{cat}\NormalTok{(}\StringTok{"}\SpecialCharTok{\textbackslash{}n}\StringTok{  Number of stations:"}\NormalTok{, n\_stations)}
\end{Highlighting}
\end{Shaded}

\begin{verbatim}
## 
##   Number of stations: 5
\end{verbatim}

\begin{Shaded}
\begin{Highlighting}[]
\FunctionTok{cat}\NormalTok{(}\StringTok{"}\SpecialCharTok{\textbackslash{}n}\StringTok{  Line efficiency:"}\NormalTok{, }\FunctionTok{round}\NormalTok{(efficiency, }\DecValTok{1}\NormalTok{), }\StringTok{"\%"}\NormalTok{)}
\end{Highlighting}
\end{Shaded}

\begin{verbatim}
## 
##   Line efficiency: 66.2 %
\end{verbatim}

\begin{Shaded}
\begin{Highlighting}[]
\FunctionTok{cat}\NormalTok{(}\StringTok{"}\SpecialCharTok{\textbackslash{}n}\StringTok{  Balance delay (idle time):"}\NormalTok{, }\FunctionTok{round}\NormalTok{(balance\_delay, }\DecValTok{1}\NormalTok{), }\StringTok{"\%"}\NormalTok{)}
\end{Highlighting}
\end{Shaded}

\begin{verbatim}
## 
##   Balance delay (idle time): 33.8 %
\end{verbatim}

\begin{Shaded}
\begin{Highlighting}[]
\FunctionTok{cat}\NormalTok{(}\StringTok{"}\SpecialCharTok{\textbackslash{}n\textbackslash{}n}\StringTok{  This is poor balance {-} consider redesigning tasks or adding parallel stations."}\NormalTok{)}
\end{Highlighting}
\end{Shaded}

\begin{verbatim}
## 
## 
##   This is poor balance - consider redesigning tasks or adding parallel stations.
\end{verbatim}

\begin{figure}

{\centering \includegraphics{introduction_files/figure-latex/precedence-diagram-1} 

}

\caption{Precedence Diagram for Tablet Assembly}\label{fig:precedence-diagram}
\end{figure}

Try It: Can you find a better balance?

\textbf{Challenge:} Reassign tasks to improve efficiency above 80\%.

\textbf{Hints:}
- Can task A be split into two smaller tasks?
- Could parallel workstations handle the bottleneck task D?
- What if we allow 85 seconds per station (slower line, fewer stations)?

\textbf{One solution with parallel stations:}
- Station 1: A + C = 50s
- Station 2a: B = 45s (parallel)
- Station 2b: B = 45s (parallel)
- Station 3: D = 55s
- Station 4: E + F = 40s
- Station 5: G + H = 75s

With parallelism at Station 2, effective cycle time stays at 55s (Station 3 bottleneck), but we can now achieve higher throughput.

\begin{center}\rule{0.5\linewidth}{0.5pt}\end{center}

\section{Automated Production Lines}\label{automated-production-lines}

Automated lines reduce human intervention using mechanized transfer systems and automated workstations. They're used for high-volume production with well-defined work content.

\subsection{Impact of Station Failures}\label{impact-of-station-failures}

The performance of automated lines is degraded by station failures. Even small failure probabilities can significantly reduce output.

\textbf{Actual Cycle Time:}
\[T_p = T_c + pT\]

\textbf{Actual Production Rate:}
\[R_p = \frac{60}{T_p} = \frac{60}{T_c + pT}\]

Where:
- \(T_c\) = Ideal cycle time (minutes)
- \(p\) = Probability of station failure per cycle
- \(T\) = Average downtime per failure (minutes)

\begin{Shaded}
\begin{Highlighting}[]
\CommentTok{\# Automated Line Performance Analysis}

\CommentTok{\# Line parameters}
\NormalTok{ideal\_cycle\_time }\OtherTok{\textless{}{-}} \FloatTok{1.0}    \CommentTok{\# minutes per part (ideal)}
\NormalTok{n\_stations }\OtherTok{\textless{}{-}} \DecValTok{10}           \CommentTok{\# Number of stations}

\CommentTok{\# Scenario comparison}
\NormalTok{scenarios }\OtherTok{\textless{}{-}} \FunctionTok{data.frame}\NormalTok{(}
  \AttributeTok{Scenario =} \FunctionTok{c}\NormalTok{(}\StringTok{"World Class"}\NormalTok{, }\StringTok{"Good"}\NormalTok{, }\StringTok{"Average"}\NormalTok{, }\StringTok{"Poor"}\NormalTok{),}
  \AttributeTok{Failure\_Prob =} \FunctionTok{c}\NormalTok{(}\FloatTok{0.005}\NormalTok{, }\FloatTok{0.02}\NormalTok{, }\FloatTok{0.05}\NormalTok{, }\FloatTok{0.10}\NormalTok{),  }\CommentTok{\# Per station per cycle}
  \AttributeTok{Repair\_Time =} \FunctionTok{c}\NormalTok{(}\DecValTok{2}\NormalTok{, }\DecValTok{5}\NormalTok{, }\DecValTok{10}\NormalTok{, }\DecValTok{15}\NormalTok{)  }\CommentTok{\# Minutes average}
\NormalTok{)}

\CommentTok{\# Calculate line performance for each scenario}
\FunctionTok{cat}\NormalTok{(}\StringTok{"Automated Line Performance Analysis}\SpecialCharTok{\textbackslash{}n}\StringTok{"}\NormalTok{)}
\end{Highlighting}
\end{Shaded}

\begin{verbatim}
## Automated Line Performance Analysis
\end{verbatim}

\begin{Shaded}
\begin{Highlighting}[]
\FunctionTok{cat}\NormalTok{(}\StringTok{"===================================}\SpecialCharTok{\textbackslash{}n}\StringTok{"}\NormalTok{)}
\end{Highlighting}
\end{Shaded}

\begin{verbatim}
## ===================================
\end{verbatim}

\begin{Shaded}
\begin{Highlighting}[]
\FunctionTok{cat}\NormalTok{(}\StringTok{"Ideal cycle time:"}\NormalTok{, ideal\_cycle\_time, }\StringTok{"min/part}\SpecialCharTok{\textbackslash{}n}\StringTok{"}\NormalTok{)}
\end{Highlighting}
\end{Shaded}

\begin{verbatim}
## Ideal cycle time: 1 min/part
\end{verbatim}

\begin{Shaded}
\begin{Highlighting}[]
\FunctionTok{cat}\NormalTok{(}\StringTok{"Number of stations:"}\NormalTok{, n\_stations, }\StringTok{"}\SpecialCharTok{\textbackslash{}n}\StringTok{"}\NormalTok{)}
\end{Highlighting}
\end{Shaded}

\begin{verbatim}
## Number of stations: 10
\end{verbatim}

\begin{Shaded}
\begin{Highlighting}[]
\FunctionTok{cat}\NormalTok{(}\StringTok{"Ideal production rate:"}\NormalTok{, }\DecValTok{60}\SpecialCharTok{/}\NormalTok{ideal\_cycle\_time, }\StringTok{"parts/hour}\SpecialCharTok{\textbackslash{}n\textbackslash{}n}\StringTok{"}\NormalTok{)}
\end{Highlighting}
\end{Shaded}

\begin{verbatim}
## Ideal production rate: 60 parts/hour
\end{verbatim}

\begin{Shaded}
\begin{Highlighting}[]
\ControlFlowTok{for}\NormalTok{(i }\ControlFlowTok{in} \DecValTok{1}\SpecialCharTok{:}\FunctionTok{nrow}\NormalTok{(scenarios)) \{}
\NormalTok{  p }\OtherTok{\textless{}{-}}\NormalTok{ scenarios}\SpecialCharTok{$}\NormalTok{Failure\_Prob[i]}
\NormalTok{  T }\OtherTok{\textless{}{-}}\NormalTok{ scenarios}\SpecialCharTok{$}\NormalTok{Repair\_Time[i]}

  \CommentTok{\# Line failure probability (any station)}
\NormalTok{  p\_line }\OtherTok{\textless{}{-}} \DecValTok{1} \SpecialCharTok{{-}}\NormalTok{ (}\DecValTok{1} \SpecialCharTok{{-}}\NormalTok{ p)}\SpecialCharTok{\^{}}\NormalTok{n\_stations}

  \CommentTok{\# Actual cycle time}
\NormalTok{  Tp }\OtherTok{\textless{}{-}}\NormalTok{ ideal\_cycle\_time }\SpecialCharTok{+}\NormalTok{ p\_line }\SpecialCharTok{*}\NormalTok{ T}

  \CommentTok{\# Actual production rate}
\NormalTok{  Rp }\OtherTok{\textless{}{-}} \DecValTok{60} \SpecialCharTok{/}\NormalTok{ Tp}

  \CommentTok{\# Efficiency}
\NormalTok{  efficiency }\OtherTok{\textless{}{-}}\NormalTok{ (ideal\_cycle\_time }\SpecialCharTok{/}\NormalTok{ Tp) }\SpecialCharTok{*} \DecValTok{100}

  \FunctionTok{cat}\NormalTok{(}\FunctionTok{sprintf}\NormalTok{(}\StringTok{"\%s Performance:}\SpecialCharTok{\textbackslash{}n}\StringTok{"}\NormalTok{, scenarios}\SpecialCharTok{$}\NormalTok{Scenario[i]))}
  \FunctionTok{cat}\NormalTok{(}\FunctionTok{sprintf}\NormalTok{(}\StringTok{"  Station failure prob: \%.1f\%\%}\SpecialCharTok{\textbackslash{}n}\StringTok{"}\NormalTok{, p }\SpecialCharTok{*} \DecValTok{100}\NormalTok{))}
  \FunctionTok{cat}\NormalTok{(}\FunctionTok{sprintf}\NormalTok{(}\StringTok{"  Line failure prob: \%.1f\%\%}\SpecialCharTok{\textbackslash{}n}\StringTok{"}\NormalTok{, p\_line }\SpecialCharTok{*} \DecValTok{100}\NormalTok{))}
  \FunctionTok{cat}\NormalTok{(}\FunctionTok{sprintf}\NormalTok{(}\StringTok{"  Average repair time: \%d min}\SpecialCharTok{\textbackslash{}n}\StringTok{"}\NormalTok{, T))}
  \FunctionTok{cat}\NormalTok{(}\FunctionTok{sprintf}\NormalTok{(}\StringTok{"  Actual cycle time: \%.2f min}\SpecialCharTok{\textbackslash{}n}\StringTok{"}\NormalTok{, Tp))}
  \FunctionTok{cat}\NormalTok{(}\FunctionTok{sprintf}\NormalTok{(}\StringTok{"  Actual production: \%.1f parts/hour}\SpecialCharTok{\textbackslash{}n}\StringTok{"}\NormalTok{, Rp))}
  \FunctionTok{cat}\NormalTok{(}\FunctionTok{sprintf}\NormalTok{(}\StringTok{"  Line efficiency: \%.1f\%\%}\SpecialCharTok{\textbackslash{}n\textbackslash{}n}\StringTok{"}\NormalTok{, efficiency))}
\NormalTok{\}}
\end{Highlighting}
\end{Shaded}

\begin{verbatim}
## World Class Performance:
##   Station failure prob: 0.5%
##   Line failure prob: 4.9%
##   Average repair time: 2 min
##   Actual cycle time: 1.10 min
##   Actual production: 54.7 parts/hour
##   Line efficiency: 91.1%
## 
## Good Performance:
##   Station failure prob: 2.0%
##   Line failure prob: 18.3%
##   Average repair time: 5 min
##   Actual cycle time: 1.91 min
##   Actual production: 31.3 parts/hour
##   Line efficiency: 52.2%
## 
## Average Performance:
##   Station failure prob: 5.0%
##   Line failure prob: 40.1%
##   Average repair time: 10 min
##   Actual cycle time: 5.01 min
##   Actual production: 12.0 parts/hour
##   Line efficiency: 19.9%
## 
## Poor Performance:
##   Station failure prob: 10.0%
##   Line failure prob: 65.1%
##   Average repair time: 15 min
##   Actual cycle time: 10.77 min
##   Actual production: 5.6 parts/hour
##   Line efficiency: 9.3%
\end{verbatim}

\begin{figure}

{\centering \includegraphics{introduction_files/figure-latex/downtime-visual-1} 

}

\caption{Impact of Station Failures on Production Rate}\label{fig:downtime-visual}
\end{figure}

Discussion: Why is preventive maintenance critical for automated lines?

\textbf{The math shows why:}
- At 5\% failure probability with 10-minute repairs, you lose 25\% of capacity
- This costs more than the maintenance to prevent it

\textbf{Cost comparison (example):}
- Lost production: 15 parts/hour × \$50/part × 8 hours = \$6,000/day
- Preventive maintenance: \$500/day
- \textbf{ROI of PM: 12:1}

\textbf{Best practices:}
1. Track failure data by station
2. Prioritize PM on high-failure stations
3. Stock critical spare parts
4. Train operators on quick changeover/repair
5. Consider redundant stations for bottlenecks

\subsection{Buffer Sizing}\label{buffer-sizing}

Buffers between stations decouple operations and improve overall line availability.

\begin{figure}

{\centering \includegraphics{introduction_files/figure-latex/buffer-analysis-1} 

}

\caption{Effect of Buffers on Line Availability}\label{fig:buffer-analysis}
\end{figure}

\begin{center}\rule{0.5\linewidth}{0.5pt}\end{center}

\section{Cellular Manufacturing}\label{cellular-manufacturing}

\textbf{Cellular manufacturing} applies Group Technology to organize production into cells that specialize in ``families'' of similar parts.

\subsection{Part Families and Machine Cells}\label{part-families-and-machine-cells}

\begin{figure}

{\centering \includegraphics{introduction_files/figure-latex/cellular-concept-1} 

}

\caption{Cellular Manufacturing: From Process Layout to Cells}\label{fig:cellular-concept}
\end{figure}

\label{tab:cellular-benefits}Cellular Manufacturing Benefits vs.~Traditional Process Layout

Metric

Traditional (Process)

Cellular

Typical Improvement

Setup time

High - frequent changeovers

Low - dedicated to family

50-90\% reduction

Work-in-process

High - batches waiting

Low - continuous flow

50-90\% reduction

Lead time

Long - complex routing

Short - U-cell flow

50-80\% reduction

Quality defects

Higher - multiple handoffs

Lower - cell ownership

30-50\% reduction

Material handling

High - long distances

Low - within cell

40-70\% reduction

Floor space

More - large aisles

Less - compact cells

20-50\% reduction

Worker skills

Specialized by machine

Cross-trained on cell

Increased flexibility

\begin{center}\rule{0.5\linewidth}{0.5pt}\end{center}

\section{Flexible Manufacturing Systems (FMS)}\label{flexible-manufacturing-systems-fms}

An \textbf{FMS} is a highly automated cell consisting of CNC machines interconnected by automated material handling, all controlled by a central computer.

\subsection{FMS Components}\label{fms-components}

\begin{verbatim}
## Warning in geom_segment(aes(x = 4, xend = 2.5, y = 2.4, yend = 2.3), linetype = "dotted"): All aesthetics have length 1, but the data has 6
## rows.
## i Please consider using `annotate()` or provide
##   this layer with data containing a single row.
\end{verbatim}

\begin{verbatim}
## Warning in geom_segment(aes(x = 4, xend = 4, y = 2.4, yend = 2.6), linetype = "dotted"): All aesthetics have length 1, but the data has 6
## rows.
## i Please consider using `annotate()` or provide
##   this layer with data containing a single row.
\end{verbatim}

\begin{verbatim}
## Warning in geom_segment(aes(x = 4, xend = 5.5, y = 2.4, yend = 2.3), linetype = "dotted"): All aesthetics have length 1, but the data has 6
## rows.
## i Please consider using `annotate()` or provide
##   this layer with data containing a single row.
\end{verbatim}

\begin{verbatim}
## Warning in geom_segment(aes(x = 4, xend = 4, y = 1.6, yend = 1.4), linetype = "dotted"): All aesthetics have length 1, but the data has 6
## rows.
## i Please consider using `annotate()` or provide
##   this layer with data containing a single row.
\end{verbatim}

\begin{figure}

{\centering \includegraphics{introduction_files/figure-latex/fms-components-1} 

}

\caption{Flexible Manufacturing System Components}\label{fig:fms-components}
\end{figure}

\label{tab:fms-applications}FMS Characteristics and Applications

Characteristic

FMS Capability

Production Volume

Medium (2,000 - 100,000/year per part type)

Part Variety

Medium (4-100 different part types)

Batch Size

Small to medium (1-500 per batch)

Setup Time

Minimal (automatic pallet/fixture changes)

Typical Parts

Prismatic parts (housings, brackets, covers)

Investment

\$5M - \$50M+ depending on size

Best Suited For

Aerospace, automotive components, job shops with repeat parts

\begin{center}\rule{0.5\linewidth}{0.5pt}\end{center}

\section{Modern Manufacturing Concepts}\label{modern-manufacturing-concepts}

\subsection{SMED (Single-Minute Exchange of Die)}\label{smed-single-minute-exchange-of-die}

\textbf{SMED} is a lean manufacturing technique to reduce changeover time to under 10 minutes (``single digit minutes'').

\label{tab:smed-process}SMED Methodology for Setup Reduction

Step

Description

Examples

Typical Improvement

\begin{enumerate}
\def\labelenumi{\arabic{enumi}.}
\tightlist
\item
  Separate

  Distinguish internal setup (machine stopped) from external setup (machine running)

  Pre-stage tools while machine runs; Pre-heat molds; Prepare fixtures

  30-50\% reduction

  \begin{enumerate}
  \def\labelenumii{\arabic{enumii}.}
  \setcounter{enumii}{1}
  \tightlist
  \item
    Convert

    Convert as much internal setup to external setup as possible

    Use quick-release clamps instead of bolts; Standardize tool heights

    Additional 20-30\%

    \begin{enumerate}
    \def\labelenumiii{\arabic{enumiii}.}
    \setcounter{enumiii}{2}
    \tightlist
    \item
      Streamline

      Reduce time for remaining internal and external activities

      Eliminate adjustments; Use one-turn fasteners; Parallel operations with 2 people

      Additional 10-20\%
    \end{enumerate}
  \end{enumerate}
\end{enumerate}

Case Study: Injection Molding Changeover

\textbf{Before SMED:}
- Changeover time: 90 minutes
- Machine stopped for entire process
- One operator working alone

\textbf{After SMED analysis:}

\begin{longtable}[]{@{}
  >{\raggedright\arraybackslash}p{(\linewidth - 8\tabcolsep) * \real{0.2564}}
  >{\raggedright\arraybackslash}p{(\linewidth - 8\tabcolsep) * \real{0.2051}}
  >{\raggedright\arraybackslash}p{(\linewidth - 8\tabcolsep) * \real{0.1538}}
  >{\raggedright\arraybackslash}p{(\linewidth - 8\tabcolsep) * \real{0.1795}}
  >{\raggedright\arraybackslash}p{(\linewidth - 8\tabcolsep) * \real{0.2051}}@{}}
\toprule\noalign{}
\begin{minipage}[b]{\linewidth}\raggedright
Activity
\end{minipage} & \begin{minipage}[b]{\linewidth}\raggedright
Before
\end{minipage} & \begin{minipage}[b]{\linewidth}\raggedright
Type
\end{minipage} & \begin{minipage}[b]{\linewidth}\raggedright
After
\end{minipage} & \begin{minipage}[b]{\linewidth}\raggedright
Change
\end{minipage} \\
\midrule\noalign{}
\endhead
\bottomrule\noalign{}
\endlastfoot
Find tools & 10 min & Internal & 0 & External - tools pre-staged \\
Remove old mold & 15 min & Internal & 5 min & Quick-release clamps \\
Get new mold & 10 min & Internal & 0 & External - mold pre-staged at machine \\
Install new mold & 20 min & Internal & 8 min & Standardized mold heights \\
Connect utilities & 10 min & Internal & 4 min & Quick-connect fittings \\
Adjust settings & 15 min & Internal & 3 min & Pre-programmed recipes \\
First article & 10 min & Internal & 5 min & Improved documentation \\
\end{longtable}

\textbf{Results:}
- Total changeover: 25 minutes (72\% reduction)
- Additional capacity: +15 changeovers/week possible
- Smaller batch sizes now economical

\subsection{AGV and AMR Systems}\label{agv-and-amr-systems}

\label{tab:agv-amr-comparison}AGV vs.~AMR Comparison

Feature

AGV (Automated Guided Vehicle)

AMR (Autonomous Mobile Robot)

Navigation

Fixed path (wires, magnets, tape)

Dynamic (SLAM, vision, sensors)

Flexibility

Low - follows predetermined routes

High - calculates optimal routes

Infrastructure

Requires installation (floor modifications)

Minimal (maps environment)

Cost

Lower vehicle cost, higher infrastructure

Higher vehicle cost, lower infrastructure

Path Changes

Requires physical modification

Software update only

Obstacle Handling

Stops and waits

Navigates around obstacles

Best Application

High-volume, repetitive routes

Variable routes, dynamic environments

\subsection{Digital Twin}\label{digital-twin}

A \textbf{digital twin} is a virtual representation of a physical manufacturing system that is updated in real-time with sensor data.

\begin{figure}

{\centering \includegraphics{introduction_files/figure-latex/digital-twin-viz-1} 

}

\caption{Digital Twin Concept}\label{fig:digital-twin-viz}
\end{figure}

\textbf{Digital Twin Applications:}
- Predictive maintenance (predict failures before they occur)
- Process optimization (simulate changes before implementation)
- Training (virtual commissioning and operator training)
- Quality prediction (correlate parameters with outcomes)

\begin{center}\rule{0.5\linewidth}{0.5pt}\end{center}

\section{Industry Applications}\label{industry-applications-1}

\label{tab:industry-applications-ch3}Manufacturing System Applications by Industry

System Type

Automotive

Food Processing

Aerospace/Defense

Transfer Line

Engine block machining, cylinder head lines

Beverage bottling, canning lines

High-volume fastener production

Manual Assembly

Final vehicle assembly, interior trim

Packaging, quality inspection

Aircraft final assembly, inspection

FMS

Transmission components, brake parts

Multi-product bakery equipment

Structural component machining

Cellular

Subassemblies, wiring harnesses

Specialty product lines

Avionics assembly, repair cells

AGV/AMR

Body shop material delivery, parts sequencing

Ingredient delivery, packaging transport

Large assembly transport, tool delivery

\begin{center}\rule{0.5\linewidth}{0.5pt}\end{center}

\section{Review Questions}\label{review-questions-2}

Question 1: Takt Time Calculation

\textbf{A plant operates 8-hour shifts with 30 minutes of breaks. Daily demand is 400 units. Calculate the takt time.}

\textbf{Solution:}

Available time = 8 hours × 60 min - 30 min = 450 minutes

Takt time = 450 min / 400 units = \textbf{1.125 minutes per unit} = 67.5 seconds

If actual cycle time is 60 seconds, the plant has (67.5 - 60)/67.5 = \textbf{11\% capacity margin}.

Question 2: Line Efficiency

\textbf{A 6-station assembly line has a cycle time of 2 minutes. The task times at each station are: 1.8, 1.9, 2.0, 1.7, 1.6, and 1.5 minutes. Calculate the line efficiency.}

\textbf{Solution:}

Total work content = 1.8 + 1.9 + 2.0 + 1.7 + 1.6 + 1.5 = 10.5 minutes

Line efficiency = (10.5) / (6 × 2.0) × 100\% = 10.5/12 × 100\% = \textbf{87.5\%}

Balance delay (idle time) = 100\% - 87.5\% = \textbf{12.5\%}

Question 3: Downtime Impact

\textbf{An automated line has 8 stations, each with 3\% failure probability per cycle and 8-minute average repair time. The ideal cycle time is 0.5 minutes. What is the actual production rate?}

\textbf{Solution:}

Line failure probability = 1 - (1 - 0.03)\^{}8 = 1 - 0.97\^{}8 = 1 - 0.784 = \textbf{21.6\%}

Actual cycle time = 0.5 + (0.216 × 8) = 0.5 + 1.73 = \textbf{2.23 minutes}

Actual production rate = 60 / 2.23 = \textbf{26.9 parts/hour}

(vs.~ideal of 120 parts/hour - only 22\% of ideal capacity!)

This shows why reliability is critical in automated systems.

Question 4: FMS Selection

\textbf{A company makes 15 different part numbers with annual volumes of 500-5,000 each. Parts are machined from aluminum and require milling, drilling, and tapping. Which system is most appropriate: dedicated transfer line, FMS, or CNC job shop?}

\textbf{Answer:} \textbf{FMS is most appropriate} because:

\begin{itemize}
\tightlist
\item
  \textbf{Volume range (500-5,000)} is ideal for FMS
\item
  \textbf{Part variety (15 types)} requires flexibility
\item
  \textbf{Similar operations} (milling, drilling, tapping) can share equipment
\item
  \textbf{Material (aluminum)} machines well on CNC
\end{itemize}

Transfer line would require too much volume per part. Job shop would have high setup times and low efficiency for these volumes.

Question 5: SMED Application

\textbf{A stamping press has a 45-minute die change. Internal activities are 35 minutes, external are 10 minutes but currently done with machine stopped. How much can SMED reduce this?}

\textbf{Answer:}

\textbf{Step 1 - Separate:} Do the 10 minutes of external work while press runs.
- New changeover: 35 minutes (30\% reduction)

\textbf{Step 2 - Convert:} Analyze the 35 minutes:
- If 15 minutes can be converted to external (pre-staging, etc.): New changeover = 20 minutes

\textbf{Step 3 - Streamline:} Quick-release clamps, standardization might save another 5 minutes.
- Final changeover: \textbf{15 minutes (67\% reduction)}

This enables smaller batch sizes and more frequent changeovers.

\begin{center}\rule{0.5\linewidth}{0.5pt}\end{center}

\section{Summary}\label{summary-2}

\label{tab:summary-ch3}Chapter 3 Summary

Topic

Key Points

Production Lines

Manual, continuous, synchronous, or asynchronous transfer; each has trade-offs

Time Metrics

Takt time = demand pace; Cycle time = actual pace; MCE measures value-added \%

Line Balancing

Assign tasks to minimize idle time; Efficiency = work content / (stations × cycle time)

Automated Lines

Reliability is critical; Small failure probabilities cause large capacity losses

Cellular Manufacturing

Group Technology groups part families; Reduces setup, WIP, lead time 50-90\%

FMS

Automated cells for medium variety/volume; Computer-controlled flexibility

Modern Concepts

SMED for quick changeover; AGV/AMR for material handling; Digital twins for optimization

\subsection{Key Takeaways}\label{key-takeaways-2}

\begin{enumerate}
\def\labelenumi{\arabic{enumi}.}
\tightlist
\item
  \textbf{Match system to product} - Volume and variety determine optimal manufacturing system
\item
  \textbf{Takt time sets the pace} - Everything should be designed to meet customer demand rate
\item
  \textbf{Balance matters} - Unbalanced lines waste capacity through idle time
\item
  \textbf{Reliability is critical} - Small failures compound to large losses in automated systems
\item
  \textbf{Flexibility has value} - FMS and cellular systems enable quick response to change
\item
  \textbf{Technology evolves} - Digital twins and AMRs are transforming manufacturing
\end{enumerate}

\begin{center}\rule{0.5\linewidth}{0.5pt}\end{center}

\section{References}\label{references-2}

\begin{enumerate}
\def\labelenumi{\arabic{enumi}.}
\tightlist
\item
  Groover, M.P. (2020). \emph{Automation, Production Systems, and Computer-Integrated Manufacturing} (5th ed.). Pearson.
\item
  Liker, J.K. (2004). \emph{The Toyota Way}. McGraw-Hill.
\item
  Shingo, S. (1985). \emph{A Revolution in Manufacturing: The SMED System}. Productivity Press.
\item
  Black, J.T. (1991). \emph{The Design of the Factory with a Future}. McGraw-Hill.
\end{enumerate}

\chapter{Plant Layout and Facility Design}\label{plant-layout-and-facility-design}

\begin{center}\rule{0.5\linewidth}{0.5pt}\end{center}

\section{Learning Objectives}\label{learning-objectives-3}

By the end of this chapter, you will be able to:

\begin{itemize}
\tightlist
\item
  Identify and describe the four basic types of facility layouts
\item
  Compare the advantages and disadvantages of each layout type
\item
  Apply the steps involved in designing a process layout
\item
  Calculate cycle time, theoretical minimum stations, and line efficiency for product layouts
\item
  Understand the principles of warehouse and office layout design
\end{itemize}

Why is Plant Layout Important?

Plant layout directly impacts:

\begin{itemize}
\tightlist
\item
  \textbf{Productivity:} Efficient layouts minimize wasted movement and time
\item
  \textbf{Cost:} Poor layouts increase material handling costs by 20-50\%
\item
  \textbf{Safety:} Well-designed layouts reduce accidents and injuries
\item
  \textbf{Flexibility:} The right layout allows adaptation to changing demands
\item
  \textbf{Employee Morale:} Good layouts improve working conditions
\end{itemize}

\begin{center}\rule{0.5\linewidth}{0.5pt}\end{center}

\section{Types of Facility Layouts}\label{types-of-facility-layouts}

There are \textbf{four basic layout types}, each suited to different production requirements:

\label{tab:layout-types-table}Overview of Facility Layout Types

Layout Type

Also Known As

Best For

Example Industries

Process Layout

Functional Layout, Job Shop

High variety, low volume

Hospitals, Machine Shops, Universities

Product Layout

Flow Line, Assembly Line

Low variety, high volume

Automotive Assembly, Food Processing

Hybrid/Cellular Layout

Group Technology, Cell Layout

Medium variety, medium volume

Electronics Manufacturing, Furniture

Fixed-Position Layout

Project Layout

Very large or immobile products

Shipbuilding, Aircraft, Construction

\subsection{Visualizing the Layout Spectrum}\label{visualizing-the-layout-spectrum}

\begin{figure}

{\centering \includegraphics{introduction_files/figure-latex/layout-spectrum-1} 

}

\caption{Product Variety vs. Production Volume for Different Layouts}\label{fig:layout-spectrum}
\end{figure}

\begin{center}\rule{0.5\linewidth}{0.5pt}\end{center}

\section{Process Layout (Functional Layout)}\label{process-layout-functional-layout}

In a \textbf{process layout}, similar resources (machines, equipment, workers with similar skills) are grouped together. Work flows between departments based on the specific requirements of each job.

\begin{figure}

{\centering \includegraphics{introduction_files/figure-latex/process-layout-diagram-1} 

}

\caption{Example Process Layout - Machine Shop}\label{fig:process-layout-diagram}
\end{figure}

\subsection{Characteristics of Process Layouts}\label{characteristics-of-process-layouts}

\label{tab:process-characteristics}Process Layout Characteristics

Characteristic

Description

Equipment

General-purpose machines

Labor

Skilled workers required

Flexibility

High - can handle variety

Processing Speed

Slower due to varied routing

Material Handling

Higher costs (complex paths)

Scheduling

Complex - different routes

Space Requirements

Higher - WIP inventory storage

Discussion Question: Can you identify a process layout in your daily life?

\textbf{Think about:}
- A hospital emergency room (patients routed based on their needs)
- A university campus (students move between buildings for different classes)
- A grocery store (customers choose their own path)
- A machine shop (parts routed based on required operations)

\textbf{What makes these process layouts?} Similar functions are grouped together, and ``products'' (patients, students, customers) follow different paths based on their individual needs.

\subsection{Real-World Examples}\label{real-world-examples}

\begin{center}\rule{0.5\linewidth}{0.5pt}\end{center}

\section{Product Layout (Assembly Line)}\label{product-layout-assembly-line}

In a \textbf{product layout}, equipment and workstations are arranged in a line according to the sequence of operations needed to produce the product. Every unit follows the same path.

\begin{figure}

{\centering \includegraphics{introduction_files/figure-latex/product-layout-diagram-1} 

}

\caption{Product Layout - Assembly Line}\label{fig:product-layout-diagram}
\end{figure}

\subsection{Characteristics of Product Layouts}\label{characteristics-of-product-layouts}

\label{tab:product-characteristics}Product Layout: Advantages vs.~Disadvantages

Advantage

Disadvantage

High production rate

High initial investment

Low unit cost

Low flexibility

Low material handling

Line stops affect all

Simple scheduling

Monotonous work

Less WIP inventory

Requires balanced workloads

Quick Quiz: Product Layout

\textbf{Question:} A car wash where vehicles move through a series of cleaning stations is an example of which layout type?

\textbf{Answer:} Product Layout - all cars follow the same path through the same sequence of operations.

\begin{center}\rule{0.5\linewidth}{0.5pt}\end{center}

\section{Hybrid Layout (Cellular Manufacturing)}\label{hybrid-layout-cellular-manufacturing}

\textbf{Hybrid layouts} combine the flexibility of process layouts with the efficiency of product layouts. The most common type is the \textbf{cellular layout} using Group Technology.

\subsection{Group Technology Concept}\label{group-technology-concept}

Group Technology (GT) identifies parts with similar characteristics and groups them into \textbf{part families}. Machines are then arranged into \textbf{cells} to process these families.

\begin{figure}

{\centering \includegraphics{introduction_files/figure-latex/cellular-layout-1} 

}

\caption{Cellular Layout with Manufacturing Cells}\label{fig:cellular-layout}
\end{figure}

Discussion Question: Benefits of Cellular Manufacturing

\textbf{What are the main benefits of cellular manufacturing over traditional process layouts?}

\begin{enumerate}
\def\labelenumi{\arabic{enumi}.}
\tightlist
\item
  \textbf{Reduced material handling} - Parts stay within the cell
\item
  \textbf{Shorter lead times} - No waiting between departments
\item
  \textbf{Less WIP inventory} - Smaller batches, faster flow
\item
  \textbf{Improved quality} - Workers responsible for complete part
\item
  \textbf{Team accountability} - Cell teams own their output
\end{enumerate}

\begin{center}\rule{0.5\linewidth}{0.5pt}\end{center}

\section{Fixed-Position Layout}\label{fixed-position-layout}

In a \textbf{fixed-position layout}, the product remains stationary while workers, equipment, and materials are brought to it. This is used when the product is too large or heavy to move.

\subsection{Examples of Fixed-Position Layouts}\label{examples-of-fixed-position-layouts}

\begin{itemize}
\tightlist
\item
  \textbf{Shipbuilding} - Ships built in dry docks
\item
  \textbf{Aircraft manufacturing} - Large aircraft assembled in hangars
\item
  \textbf{Construction} - Buildings, bridges, dams
\item
  \textbf{Surgery} - Patient remains stationary
\end{itemize}

Challenge Question: Managing a Fixed-Position Layout

\textbf{What are the unique challenges of managing a fixed-position layout?}

Think about:
- Scheduling deliveries of materials and equipment
- Coordinating multiple contractors/teams
- Limited space around the product
- Weather and environmental factors (outdoor projects)
- Ensuring safety with many workers in one area

\begin{center}\rule{0.5\linewidth}{0.5pt}\end{center}

\section{Designing Process Layouts}\label{designing-process-layouts}

Designing an effective process layout involves \textbf{three main steps}:

\subsection{Step 1: Gather Information}\label{step-1-gather-information}

You need to determine:
- \textbf{Space requirements} for each department
- \textbf{Available space} in the facility
- \textbf{Closeness relationships} between departments

\subsubsection{From-To Matrix (Load Matrix)}\label{from-to-matrix-load-matrix}

A From-To matrix shows the number of trips or loads between departments:

\label{tab:from-to-matrix}From-To Matrix: Loads per Day Between Departments

Receiving

Machining

Assembly

Painting

Shipping

Receiving

\begin{itemize}
\item
  100

  50

  0

  0

  Machining

  0

  \begin{itemize}
  \item
    200

    50

    0

    Assembly

    0

    0

    \begin{itemize}
    \item
      100

      60

      Painting

      0

      0

      0

      \begin{itemize}
      \item
        150

        Shipping

        0

        0

        0

        0

        \begin{itemize}
        \item
        \end{itemize}
      \end{itemize}
    \end{itemize}
  \end{itemize}
\end{itemize}

\subsection{Step 2: Develop Block Plans}\label{step-2-develop-block-plans}

Use the \textbf{Load-Distance Model} to evaluate layout alternatives:

\[\text{Total Cost} = \sum_{i=1}^{n} \sum_{j=1}^{n} L_{ij} \times D_{ij} \times C\]

Where:
- \(L_{ij}\) = Number of loads between departments \(i\) and \(j\)
- \(D_{ij}\) = Distance between departments \(i\) and \(j\)
- \(C\) = Cost per unit distance

\subsection{Interactive Example: Comparing Two Layouts}\label{interactive-example-comparing-two-layouts}

\begin{figure}

{\centering \includegraphics{introduction_files/figure-latex/layout-comparison-1} 

}

\caption{Comparing Two Process Layout Alternatives}\label{fig:layout-comparison}
\end{figure}

\begin{Shaded}
\begin{Highlighting}[]
\CommentTok{\# Calculate Load{-}Distance scores for both layouts}
\CommentTok{\# From{-}To data: Receiving{-}\textgreater{}Machining=100, Machining{-}\textgreater{}Assembly=200,}
\CommentTok{\#               Assembly{-}\textgreater{}Painting=100, Painting{-}\textgreater{}Shipping=150}

\CommentTok{\# Layout A distances (linear): all adjacent = 1 unit}
\NormalTok{layout\_A\_score }\OtherTok{\textless{}{-}} \DecValTok{100}\SpecialCharTok{*}\DecValTok{1} \SpecialCharTok{+} \DecValTok{200}\SpecialCharTok{*}\DecValTok{1} \SpecialCharTok{+} \DecValTok{100}\SpecialCharTok{*}\DecValTok{1} \SpecialCharTok{+} \DecValTok{150}\SpecialCharTok{*}\DecValTok{1}
\FunctionTok{cat}\NormalTok{(}\StringTok{"Layout A Total Load{-}Distance:"}\NormalTok{, layout\_A\_score, }\StringTok{"}\SpecialCharTok{\textbackslash{}n}\StringTok{"}\NormalTok{)}
\end{Highlighting}
\end{Shaded}

\begin{verbatim}
## Layout A Total Load-Distance: 550
\end{verbatim}

\begin{Shaded}
\begin{Highlighting}[]
\CommentTok{\# Layout B distances (compact): most departments closer together}
\CommentTok{\# Receiving to Machining = 1.12 (diagonal)}
\CommentTok{\# Machining to Assembly = 1}
\CommentTok{\# Assembly to Painting = 1.12 (diagonal)}
\CommentTok{\# Painting to Shipping = 1}
\NormalTok{layout\_B\_score }\OtherTok{\textless{}{-}} \DecValTok{100}\SpecialCharTok{*}\FloatTok{1.12} \SpecialCharTok{+} \DecValTok{200}\SpecialCharTok{*}\DecValTok{1} \SpecialCharTok{+} \DecValTok{100}\SpecialCharTok{*}\FloatTok{1.12} \SpecialCharTok{+} \DecValTok{150}\SpecialCharTok{*}\DecValTok{1}
\FunctionTok{cat}\NormalTok{(}\StringTok{"Layout B Total Load{-}Distance:"}\NormalTok{, }\FunctionTok{round}\NormalTok{(layout\_B\_score, }\DecValTok{1}\NormalTok{), }\StringTok{"}\SpecialCharTok{\textbackslash{}n}\StringTok{"}\NormalTok{)}
\end{Highlighting}
\end{Shaded}

\begin{verbatim}
## Layout B Total Load-Distance: 574
\end{verbatim}

\begin{Shaded}
\begin{Highlighting}[]
\FunctionTok{cat}\NormalTok{(}\StringTok{"}\SpecialCharTok{\textbackslash{}n}\StringTok{Layout A is better by:"}\NormalTok{, }\FunctionTok{round}\NormalTok{(layout\_B\_score }\SpecialCharTok{{-}}\NormalTok{ layout\_A\_score, }\DecValTok{1}\NormalTok{), }\StringTok{"units"}\NormalTok{)}
\end{Highlighting}
\end{Shaded}

\begin{verbatim}
## 
## Layout A is better by: 24 units
\end{verbatim}

Why does the linear layout win in this case?

In this example, the linear layout (A) wins because the process flow is essentially sequential (Receiving → Machining → Assembly → Painting → Shipping).

When flow is \textbf{sequential}, a \textbf{product layout} (linear arrangement) is more efficient. Process layouts are better when there's \textbf{varied routing} between departments.

\subsection{Step 3: Develop Detailed Layout}\label{step-3-develop-detailed-layout}

Once the block plan is selected:
- Determine exact sizes and shapes of departments
- Plan aisle locations and widths
- Consider utilities, safety exits, and accessibility
- Use CAD software or 3D modeling for visualization

\begin{center}\rule{0.5\linewidth}{0.5pt}\end{center}

\section{Designing Product Layouts (Line Balancing)}\label{designing-product-layouts-line-balancing}

The key challenge in product layout design is \textbf{line balancing} - assigning tasks to workstations to minimize idle time while meeting production requirements.

\subsection{Line Balancing Steps}\label{line-balancing-steps}

\begin{enumerate}
\def\labelenumi{\arabic{enumi}.}
\tightlist
\item
  \textbf{Identify tasks and precedence relationships}
\item
  \textbf{Calculate required cycle time}
\item
  \textbf{Calculate theoretical minimum number of stations}
\item
  \textbf{Assign tasks to stations}
\item
  \textbf{Calculate efficiency}
\end{enumerate}

\subsection{Key Formulas}\label{key-formulas}

\textbf{Cycle Time:}
\[C = \frac{\text{Available Production Time}}{\text{Desired Output}}\]

\textbf{Theoretical Minimum Stations:}
\[TM = \frac{\sum t_i}{C}\]

\textbf{Line Efficiency:}
\[\text{Efficiency} = \frac{\sum t_i}{n \times C} \times 100\%\]

Where:
- \(C\) = Cycle time
- \(\sum t_i\) = Total task time
- \(n\) = Number of workstations

\subsection{Interactive Example: Pizza Assembly Line}\label{interactive-example-pizza-assembly-line}

\label{tab:pizza-example}Pizza Assembly Tasks

Task

Description

Time (sec)

Predecessors

A

Roll dough

40

\begin{itemize}
\item
  B

  Spread sauce

  25

  A

  C

  Add cheese

  20

  B

  D

  Add toppings

  35

  B

  E

  Season

  15

  C, D

  F

  Box pizza

  30

  E
\end{itemize}

\begin{figure}

{\centering \includegraphics{introduction_files/figure-latex/pizza-precedence-1} 

}

\caption{Precedence Diagram for Pizza Assembly}\label{fig:pizza-precedence}
\end{figure}

\subsection{Line Balancing Calculation}\label{line-balancing-calculation}

\begin{Shaded}
\begin{Highlighting}[]
\CommentTok{\# Given data}
\NormalTok{total\_task\_time }\OtherTok{\textless{}{-}} \DecValTok{40} \SpecialCharTok{+} \DecValTok{25} \SpecialCharTok{+} \DecValTok{20} \SpecialCharTok{+} \DecValTok{35} \SpecialCharTok{+} \DecValTok{15} \SpecialCharTok{+} \DecValTok{30}  \CommentTok{\# seconds}
\NormalTok{desired\_output }\OtherTok{\textless{}{-}} \DecValTok{60}  \CommentTok{\# pizzas per hour}
\NormalTok{available\_time }\OtherTok{\textless{}{-}} \DecValTok{3600}  \CommentTok{\# seconds per hour}

\CommentTok{\# Step 1: Calculate Cycle Time}
\NormalTok{cycle\_time }\OtherTok{\textless{}{-}}\NormalTok{ available\_time }\SpecialCharTok{/}\NormalTok{ desired\_output}
\FunctionTok{cat}\NormalTok{(}\StringTok{"Cycle Time:"}\NormalTok{, cycle\_time, }\StringTok{"seconds per pizza}\SpecialCharTok{\textbackslash{}n}\StringTok{"}\NormalTok{)}
\end{Highlighting}
\end{Shaded}

\begin{verbatim}
## Cycle Time: 60 seconds per pizza
\end{verbatim}

\begin{Shaded}
\begin{Highlighting}[]
\CommentTok{\# Step 2: Calculate Theoretical Minimum Stations}
\NormalTok{TM }\OtherTok{\textless{}{-}}\NormalTok{ total\_task\_time }\SpecialCharTok{/}\NormalTok{ cycle\_time}
\FunctionTok{cat}\NormalTok{(}\StringTok{"Theoretical Minimum Stations:"}\NormalTok{, TM, }\StringTok{"="}\NormalTok{, }\FunctionTok{ceiling}\NormalTok{(TM), }\StringTok{"stations}\SpecialCharTok{\textbackslash{}n}\StringTok{"}\NormalTok{)}
\end{Highlighting}
\end{Shaded}

\begin{verbatim}
## Theoretical Minimum Stations: 2.75 = 3 stations
\end{verbatim}

\begin{Shaded}
\begin{Highlighting}[]
\CommentTok{\# Step 3: Assign tasks to stations (one possible solution)}
\FunctionTok{cat}\NormalTok{(}\StringTok{"}\SpecialCharTok{\textbackslash{}n}\StringTok{{-}{-}{-} Workstation Assignments {-}{-}{-}}\SpecialCharTok{\textbackslash{}n}\StringTok{"}\NormalTok{)}
\end{Highlighting}
\end{Shaded}

\begin{verbatim}
## 
## --- Workstation Assignments ---
\end{verbatim}

\begin{Shaded}
\begin{Highlighting}[]
\FunctionTok{cat}\NormalTok{(}\StringTok{"Station 1: A (40s) + B (25s) = 65s \textgreater{} 60s ... EXCEEDS!}\SpecialCharTok{\textbackslash{}n}\StringTok{"}\NormalTok{)}
\end{Highlighting}
\end{Shaded}

\begin{verbatim}
## Station 1: A (40s) + B (25s) = 65s > 60s ... EXCEEDS!
\end{verbatim}

\begin{Shaded}
\begin{Highlighting}[]
\FunctionTok{cat}\NormalTok{(}\StringTok{"Let\textquotesingle{}s try again:}\SpecialCharTok{\textbackslash{}n}\StringTok{"}\NormalTok{)}
\end{Highlighting}
\end{Shaded}

\begin{verbatim}
## Let's try again:
\end{verbatim}

\begin{Shaded}
\begin{Highlighting}[]
\FunctionTok{cat}\NormalTok{(}\StringTok{"Station 1: A (40s) = 40s [Idle: 20s]}\SpecialCharTok{\textbackslash{}n}\StringTok{"}\NormalTok{)}
\end{Highlighting}
\end{Shaded}

\begin{verbatim}
## Station 1: A (40s) = 40s [Idle: 20s]
\end{verbatim}

\begin{Shaded}
\begin{Highlighting}[]
\FunctionTok{cat}\NormalTok{(}\StringTok{"Station 2: B (25s) + D (35s) = 60s [Idle: 0s]}\SpecialCharTok{\textbackslash{}n}\StringTok{"}\NormalTok{)}
\end{Highlighting}
\end{Shaded}

\begin{verbatim}
## Station 2: B (25s) + D (35s) = 60s [Idle: 0s]
\end{verbatim}

\begin{Shaded}
\begin{Highlighting}[]
\FunctionTok{cat}\NormalTok{(}\StringTok{"Station 3: C (20s) + E (15s) + F (30s) = 65s \textgreater{} 60s... EXCEEDS!}\SpecialCharTok{\textbackslash{}n}\StringTok{"}\NormalTok{)}
\end{Highlighting}
\end{Shaded}

\begin{verbatim}
## Station 3: C (20s) + E (15s) + F (30s) = 65s > 60s... EXCEEDS!
\end{verbatim}

\begin{Shaded}
\begin{Highlighting}[]
\FunctionTok{cat}\NormalTok{(}\StringTok{"Revised:}\SpecialCharTok{\textbackslash{}n}\StringTok{"}\NormalTok{)}
\end{Highlighting}
\end{Shaded}

\begin{verbatim}
## Revised:
\end{verbatim}

\begin{Shaded}
\begin{Highlighting}[]
\FunctionTok{cat}\NormalTok{(}\StringTok{"Station 3: C (20s) + E (15s) = 35s [Idle: 25s]}\SpecialCharTok{\textbackslash{}n}\StringTok{"}\NormalTok{)}
\end{Highlighting}
\end{Shaded}

\begin{verbatim}
## Station 3: C (20s) + E (15s) = 35s [Idle: 25s]
\end{verbatim}

\begin{Shaded}
\begin{Highlighting}[]
\FunctionTok{cat}\NormalTok{(}\StringTok{"Station 4: F (30s) = 30s [Idle: 30s]}\SpecialCharTok{\textbackslash{}n}\StringTok{"}\NormalTok{)}
\end{Highlighting}
\end{Shaded}

\begin{verbatim}
## Station 4: F (30s) = 30s [Idle: 30s]
\end{verbatim}

\begin{Shaded}
\begin{Highlighting}[]
\CommentTok{\# With 4 stations}
\NormalTok{n\_stations }\OtherTok{\textless{}{-}} \DecValTok{4}
\NormalTok{efficiency }\OtherTok{\textless{}{-}}\NormalTok{ (total\_task\_time }\SpecialCharTok{/}\NormalTok{ (n\_stations }\SpecialCharTok{*}\NormalTok{ cycle\_time)) }\SpecialCharTok{*} \DecValTok{100}
\FunctionTok{cat}\NormalTok{(}\StringTok{"}\SpecialCharTok{\textbackslash{}n}\StringTok{{-}{-}{-} Results {-}{-}{-}}\SpecialCharTok{\textbackslash{}n}\StringTok{"}\NormalTok{)}
\end{Highlighting}
\end{Shaded}

\begin{verbatim}
## 
## --- Results ---
\end{verbatim}

\begin{Shaded}
\begin{Highlighting}[]
\FunctionTok{cat}\NormalTok{(}\StringTok{"Number of Stations:"}\NormalTok{, n\_stations, }\StringTok{"}\SpecialCharTok{\textbackslash{}n}\StringTok{"}\NormalTok{)}
\end{Highlighting}
\end{Shaded}

\begin{verbatim}
## Number of Stations: 4
\end{verbatim}

\begin{Shaded}
\begin{Highlighting}[]
\FunctionTok{cat}\NormalTok{(}\StringTok{"Line Efficiency:"}\NormalTok{, }\FunctionTok{round}\NormalTok{(efficiency, }\DecValTok{1}\NormalTok{), }\StringTok{"\%}\SpecialCharTok{\textbackslash{}n}\StringTok{"}\NormalTok{)}
\end{Highlighting}
\end{Shaded}

\begin{verbatim}
## Line Efficiency: 68.8 %
\end{verbatim}

\begin{Shaded}
\begin{Highlighting}[]
\FunctionTok{cat}\NormalTok{(}\StringTok{"Balance Delay (Idle Time):"}\NormalTok{, }\FunctionTok{round}\NormalTok{(}\DecValTok{100} \SpecialCharTok{{-}}\NormalTok{ efficiency, }\DecValTok{1}\NormalTok{), }\StringTok{"\%}\SpecialCharTok{\textbackslash{}n}\StringTok{"}\NormalTok{)}
\end{Highlighting}
\end{Shaded}

\begin{verbatim}
## Balance Delay (Idle Time): 31.2 %
\end{verbatim}

Try It Yourself: Can you find a better balance?

\textbf{Challenge:} Can you assign tasks to only 3 stations while respecting the precedence constraints?

\textbf{Hint:} Consider these assignments:
- Station 1: A (40s) + ?
- Station 2: B (25s) + ? + ?
- Station 3: ? + ?

Remember: Total time per station cannot exceed 60 seconds!

\textbf{Solution:} It's actually possible with careful planning:
- Station 1: A (40s) = 40s
- Station 2: B (25s) + C (20s) + E (15s) = 60s (but E needs C AND D complete!)

So 3 stations is NOT feasible due to precedence constraints. The minimum is 4 stations for this problem.

\begin{center}\rule{0.5\linewidth}{0.5pt}\end{center}

\section{Warehouse Layout Design}\label{warehouse-layout-design}

Warehouse layouts are a special case of process layouts focused on \textbf{minimizing material handling}.

\subsection{Key Principles}\label{key-principles}

\begin{enumerate}
\def\labelenumi{\arabic{enumi}.}
\tightlist
\item
  \textbf{Place high-volume items near the dock}
\item
  \textbf{Group items that are often picked together}
\item
  \textbf{Use ABC analysis} for storage assignment:

  \begin{itemize}
  \tightlist
  \item
    \textbf{A items} (20\% of SKUs, 80\% of picks): Prime locations
  \item
    \textbf{B items} (30\% of SKUs, 15\% of picks): Secondary locations
  \item
    \textbf{C items} (50\% of SKUs, 5\% of picks): Remote locations
  \end{itemize}
\end{enumerate}

\begin{figure}

{\centering \includegraphics{introduction_files/figure-latex/warehouse-viz-1} 

}

\caption{Warehouse Layout: ABC Storage Strategy}\label{fig:warehouse-viz}
\end{figure}

\begin{center}\rule{0.5\linewidth}{0.5pt}\end{center}

\section{Office Layout Design}\label{office-layout-design}

Office layouts balance \textbf{communication needs} with \textbf{privacy requirements}.

\subsection{Two Main Approaches}\label{two-main-approaches}

\label{tab:office-comparison}Office Layout Comparison

Aspect

Open Plan

Traditional/Closed

Space

Shared workspaces

Private offices

Walls

Partitions/dividers

Permanent walls

Flexibility

High - easy to reconfigure

Low - expensive to change

Communication

Excellent

Limited

Privacy

Low

High

Cost

Lower

Higher

\subsection{Office Ergonomics Checklist}\label{office-ergonomics-checklist}

Click to expand: Office Ergonomics Best Practices

\textbf{Posture:}
- Support the small of the back
- Keep shoulders relaxed
- Maintain neutral wrist position
- Feet flat on floor or footrest

\textbf{Lighting:}
- Position monitor to reduce glare
- Adjust brightness for screen vs.~paper work
- Use task lighting when needed

\textbf{Workstation:}
- Monitor at arm's length
- Top of screen at or below eye level
- Keyboard and mouse at elbow height

\textbf{Work Habits:}
- Take breaks every 30-60 minutes
- Vary tasks to prevent repetitive strain
- Use keyboard shortcuts to reduce mouse use

\subsection{The Future of Office Design}\label{the-future-of-office-design}

\begin{center}\rule{0.5\linewidth}{0.5pt}\end{center}

\section{Summary: Choosing the Right Layout}\label{summary-choosing-the-right-layout}

\begin{figure}

{\centering \includegraphics{introduction_files/figure-latex/decision-flowchart-1} 

}

\caption{Layout Selection Decision Guide}\label{fig:decision-flowchart}
\end{figure}

\subsection{Key Takeaways}\label{key-takeaways-3}

\begin{enumerate}
\def\labelenumi{\arabic{enumi}.}
\tightlist
\item
  \textbf{Process layouts} offer flexibility for varied products but have higher material handling costs
\item
  \textbf{Product layouts} are efficient for high-volume, standardized products
\item
  \textbf{Cellular layouts} provide a middle ground with benefits of both
\item
  \textbf{Fixed-position layouts} are necessary when products cannot be moved
\item
  \textbf{Line balancing} is critical for efficient product layouts
\item
  \textbf{Warehouse layouts} should minimize travel distance for high-frequency items
\end{enumerate}

\begin{center}\rule{0.5\linewidth}{0.5pt}\end{center}

\section{Industry Case Studies}\label{industry-case-studies}

\subsection{Case Study 1: Food Processing Plant Layout}\label{case-study-1-food-processing-plant-layout}

Food processing facilities have unique layout requirements due to food safety, hygiene, and regulatory compliance.

\begin{figure}

{\centering \includegraphics{introduction_files/figure-latex/food-processing-layout-1} 

}

\caption{Food Processing Plant Layout Example}\label{fig:food-processing-layout}
\end{figure}

\textbf{Key Design Principles for Food Processing:}

\label{tab:food-design-principles}Food Processing Layout Design Principles

Principle

Description

Regulatory Reference

Linear Flow

Raw materials → Processing → Packaging (no backtracking)

FSMA

Zone Separation

Physical barriers between raw and ready-to-eat areas

HACCP

Cleanability

Smooth, non-porous surfaces; rounded corners; proper drainage

3-A Standards

Temperature Control

Maintain cold chain; separate refrigerated zones

FDA 21 CFR 110

Sanitary Design

Handwashing stations, foot baths, air curtains at zone transitions

USDA FSIS

Allergen Control

Dedicated lines or thorough cleaning between allergen products

FALCPA

Discussion Question: Cross-Contamination Prevention

\textbf{Question:} A food processing plant produces both peanut butter products and tree nut-free products. What layout considerations are essential?

\textbf{Key Considerations:}

\begin{enumerate}
\def\labelenumi{\arabic{enumi}.}
\tightlist
\item
  \textbf{Physical Separation:} Completely separate production lines with physical barriers (walls, not just distance)
\item
  \textbf{Air Handling:} Separate HVAC systems to prevent airborne allergen transfer
\item
  \textbf{Personnel Flow:} Dedicated workers for each line, or strict gowning/cleaning procedures when crossing zones
\item
  \textbf{Scheduling:} If shared equipment is unavoidable, schedule allergen products last before deep cleaning
\item
  \textbf{Traffic Patterns:} Forklift and cart paths should not cross between allergen zones
\item
  \textbf{Verification:} Air sampling and surface swabs to verify effectiveness
\end{enumerate}

\subsection{Case Study 2: Aerospace Manufacturing Facility}\label{case-study-2-aerospace-manufacturing-facility}

Aerospace facilities often use \textbf{fixed-position layouts} combined with \textbf{process layouts} for component fabrication.

\begin{figure}

{\centering \includegraphics{introduction_files/figure-latex/aerospace-layout-1} 

}

\caption{Aerospace Assembly Facility Layout}\label{fig:aerospace-layout}
\end{figure}

\textbf{Aerospace Layout Characteristics:}

\begin{itemize}
\tightlist
\item
  \textbf{Clean rooms} for precision components (hydraulics, avionics)
\item
  \textbf{High-bay areas} for large assembly operations
\item
  \textbf{Overhead cranes} for moving heavy components
\item
  \textbf{Flow-line final assembly} (aircraft moves through stations) for high-volume commercial aircraft
\item
  \textbf{Fixed-position} for military aircraft or low-volume production
\item
  \textbf{Strict FOD (Foreign Object Debris)} control throughout
\end{itemize}

Video: Boeing 737 Assembly

Watch how Boeing uses a moving assembly line for high-volume aircraft production:

\subsection{Case Study 3: Automotive Supplier - Hybrid Cell Layout}\label{case-study-3-automotive-supplier---hybrid-cell-layout}

\begin{figure}

{\centering \includegraphics{introduction_files/figure-latex/auto-supplier-layout-1} 

}

\caption{Automotive Supplier Facility with Cellular Layout}\label{fig:auto-supplier-layout}
\end{figure}

\textbf{JIT Delivery Requirements:}

Automotive suppliers must often deliver directly to the assembly line in \textbf{sequence} (JIS - Just-In-Sequence):

\label{tab:jit-table}JIT/JIS Requirements and Layout Implications

Requirement

Typical Standard

Layout Impact

Delivery Window

±30 minutes of scheduled time

Dedicated shipping lane, staging area near dock

Sequence Accuracy

100\% (0 sequence errors)

Pick-to-light systems, visual management

Quality Level

0 PPM target (Six Sigma)

Inline inspection, mistake-proofing (poka-yoke)

Packaging

Returnable containers to line side

Container cleaning/storage area needed

Traceability

Full lot/serial traceability required

Barcode/RFID scanning stations throughout

\begin{center}\rule{0.5\linewidth}{0.5pt}\end{center}

\section{Ergonomics in Manufacturing}\label{ergonomics-in-manufacturing}

Ergonomics (human factors engineering) is critical for plant layout design to ensure worker safety, reduce injuries, and improve productivity.

\subsection{The NIOSH Lifting Equation}\label{the-niosh-lifting-equation}

The National Institute for Occupational Safety and Health (NIOSH) developed an equation to evaluate manual lifting tasks:

\[RWL = LC \times HM \times VM \times DM \times AM \times FM \times CM\]

Where \textbf{RWL} = Recommended Weight Limit (in pounds or kg)

\label{tab:niosh-factors-plant}NIOSH Lifting Equation Factors

Factor

Name

Formula

Description

LC

Load Constant

51 lb (23 kg)

Maximum weight under ideal conditions

HM

Horizontal Multiplier

10/H

H = horizontal distance from spine to load (inches)

VM

Vertical Multiplier

1 - 0.0075\textbar V - 30\textbar{}

V = vertical height of hands at start (inches)

DM

Distance Multiplier

0.82 + 1.8/D

D = vertical travel distance (inches)

AM

Asymmetric Multiplier

1 - 0.0032A

A = angle of asymmetry (degrees)

FM

Frequency Multiplier

Table lookup

Based on lift frequency and duration

CM

Coupling Multiplier

Table lookup

Based on grip quality (good/fair/poor)

\subsection{Lifting Index Calculation}\label{lifting-index-calculation}

\[LI = \frac{\text{Actual Weight}}{\text{RWL}}\]

\label{tab:li-interpretation-plant}Lifting Index Interpretation

Lifting Index

Risk Level

Action Required

LI ≤ 1.0

Low

Task is acceptable for most workers

1.0 \textless{} LI ≤ 3.0

Moderate

Task poses risk; engineering controls recommended

LI \textgreater{} 3.0

High

Task is unacceptable; immediate redesign required

\subsection{NIOSH Lifting Example}\label{niosh-lifting-example}

\begin{Shaded}
\begin{Highlighting}[]
\CommentTok{\# Example: Worker lifts 35 lb boxes from pallet to conveyor}
\CommentTok{\# Conditions:}
\CommentTok{\# {-} Horizontal distance (H) = 15 inches}
\CommentTok{\# {-} Vertical height at start (V) = 10 inches}
\CommentTok{\# {-} Vertical travel distance (D) = 20 inches}
\CommentTok{\# {-} Asymmetry angle (A) = 30 degrees}
\CommentTok{\# {-} Frequency = 4 lifts per minute, 2 hours}
\CommentTok{\# {-} Coupling = Fair (handles present)}

\NormalTok{LC }\OtherTok{\textless{}{-}} \DecValTok{51}  \CommentTok{\# Load constant (lb)}

\CommentTok{\# Calculate multipliers}
\NormalTok{HM }\OtherTok{\textless{}{-}} \DecValTok{10} \SpecialCharTok{/} \DecValTok{15}  \CommentTok{\# Horizontal multiplier}
\NormalTok{VM }\OtherTok{\textless{}{-}} \DecValTok{1} \SpecialCharTok{{-}} \FloatTok{0.0075} \SpecialCharTok{*} \FunctionTok{abs}\NormalTok{(}\DecValTok{10} \SpecialCharTok{{-}} \DecValTok{30}\NormalTok{)  }\CommentTok{\# Vertical multiplier}
\NormalTok{DM }\OtherTok{\textless{}{-}} \FloatTok{0.82} \SpecialCharTok{+}\NormalTok{ (}\FloatTok{1.8} \SpecialCharTok{/} \DecValTok{20}\NormalTok{)  }\CommentTok{\# Distance multiplier}
\NormalTok{AM }\OtherTok{\textless{}{-}} \DecValTok{1} \SpecialCharTok{{-}} \FloatTok{0.0032} \SpecialCharTok{*} \DecValTok{30}  \CommentTok{\# Asymmetric multiplier}
\NormalTok{FM }\OtherTok{\textless{}{-}} \FloatTok{0.72}  \CommentTok{\# From NIOSH table: 4 lifts/min, 1{-}2 hours}
\NormalTok{CM }\OtherTok{\textless{}{-}} \FloatTok{0.95}  \CommentTok{\# Fair coupling}

\FunctionTok{cat}\NormalTok{(}\StringTok{"Multiplier Values:}\SpecialCharTok{\textbackslash{}n}\StringTok{"}\NormalTok{)}
\end{Highlighting}
\end{Shaded}

\begin{verbatim}
## Multiplier Values:
\end{verbatim}

\begin{Shaded}
\begin{Highlighting}[]
\FunctionTok{cat}\NormalTok{(}\StringTok{"  HM (Horizontal):"}\NormalTok{, }\FunctionTok{round}\NormalTok{(HM, }\DecValTok{3}\NormalTok{), }\StringTok{"}\SpecialCharTok{\textbackslash{}n}\StringTok{"}\NormalTok{)}
\end{Highlighting}
\end{Shaded}

\begin{verbatim}
##   HM (Horizontal): 0.667
\end{verbatim}

\begin{Shaded}
\begin{Highlighting}[]
\FunctionTok{cat}\NormalTok{(}\StringTok{"  VM (Vertical):"}\NormalTok{, }\FunctionTok{round}\NormalTok{(VM, }\DecValTok{3}\NormalTok{), }\StringTok{"}\SpecialCharTok{\textbackslash{}n}\StringTok{"}\NormalTok{)}
\end{Highlighting}
\end{Shaded}

\begin{verbatim}
##   VM (Vertical): 0.85
\end{verbatim}

\begin{Shaded}
\begin{Highlighting}[]
\FunctionTok{cat}\NormalTok{(}\StringTok{"  DM (Distance):"}\NormalTok{, }\FunctionTok{round}\NormalTok{(DM, }\DecValTok{3}\NormalTok{), }\StringTok{"}\SpecialCharTok{\textbackslash{}n}\StringTok{"}\NormalTok{)}
\end{Highlighting}
\end{Shaded}

\begin{verbatim}
##   DM (Distance): 0.91
\end{verbatim}

\begin{Shaded}
\begin{Highlighting}[]
\FunctionTok{cat}\NormalTok{(}\StringTok{"  AM (Asymmetry):"}\NormalTok{, }\FunctionTok{round}\NormalTok{(AM, }\DecValTok{3}\NormalTok{), }\StringTok{"}\SpecialCharTok{\textbackslash{}n}\StringTok{"}\NormalTok{)}
\end{Highlighting}
\end{Shaded}

\begin{verbatim}
##   AM (Asymmetry): 0.904
\end{verbatim}

\begin{Shaded}
\begin{Highlighting}[]
\FunctionTok{cat}\NormalTok{(}\StringTok{"  FM (Frequency):"}\NormalTok{, FM, }\StringTok{"}\SpecialCharTok{\textbackslash{}n}\StringTok{"}\NormalTok{)}
\end{Highlighting}
\end{Shaded}

\begin{verbatim}
##   FM (Frequency): 0.72
\end{verbatim}

\begin{Shaded}
\begin{Highlighting}[]
\FunctionTok{cat}\NormalTok{(}\StringTok{"  CM (Coupling):"}\NormalTok{, CM, }\StringTok{"}\SpecialCharTok{\textbackslash{}n}\StringTok{"}\NormalTok{)}
\end{Highlighting}
\end{Shaded}

\begin{verbatim}
##   CM (Coupling): 0.95
\end{verbatim}

\begin{Shaded}
\begin{Highlighting}[]
\CommentTok{\# Calculate RWL}
\NormalTok{RWL }\OtherTok{\textless{}{-}}\NormalTok{ LC }\SpecialCharTok{*}\NormalTok{ HM }\SpecialCharTok{*}\NormalTok{ VM }\SpecialCharTok{*}\NormalTok{ DM }\SpecialCharTok{*}\NormalTok{ AM }\SpecialCharTok{*}\NormalTok{ FM }\SpecialCharTok{*}\NormalTok{ CM}
\FunctionTok{cat}\NormalTok{(}\StringTok{"}\SpecialCharTok{\textbackslash{}n}\StringTok{Recommended Weight Limit (RWL):"}\NormalTok{, }\FunctionTok{round}\NormalTok{(RWL, }\DecValTok{1}\NormalTok{), }\StringTok{"lb}\SpecialCharTok{\textbackslash{}n}\StringTok{"}\NormalTok{)}
\end{Highlighting}
\end{Shaded}

\begin{verbatim}
## 
## Recommended Weight Limit (RWL): 16.3 lb
\end{verbatim}

\begin{Shaded}
\begin{Highlighting}[]
\CommentTok{\# Calculate Lifting Index}
\NormalTok{actual\_weight }\OtherTok{\textless{}{-}} \DecValTok{35}
\NormalTok{LI }\OtherTok{\textless{}{-}}\NormalTok{ actual\_weight }\SpecialCharTok{/}\NormalTok{ RWL}
\FunctionTok{cat}\NormalTok{(}\StringTok{"Lifting Index (LI):"}\NormalTok{, }\FunctionTok{round}\NormalTok{(LI, }\DecValTok{2}\NormalTok{), }\StringTok{"}\SpecialCharTok{\textbackslash{}n}\StringTok{"}\NormalTok{)}
\end{Highlighting}
\end{Shaded}

\begin{verbatim}
## Lifting Index (LI): 2.15
\end{verbatim}

\begin{Shaded}
\begin{Highlighting}[]
\ControlFlowTok{if}\NormalTok{ (LI }\SpecialCharTok{\textless{}=} \FloatTok{1.0}\NormalTok{) \{}
  \FunctionTok{cat}\NormalTok{(}\StringTok{"}\SpecialCharTok{\textbackslash{}n}\StringTok{Risk Level: LOW {-} Task is acceptable}\SpecialCharTok{\textbackslash{}n}\StringTok{"}\NormalTok{)}
\NormalTok{\} }\ControlFlowTok{else} \ControlFlowTok{if}\NormalTok{ (LI }\SpecialCharTok{\textless{}=} \FloatTok{3.0}\NormalTok{) \{}
  \FunctionTok{cat}\NormalTok{(}\StringTok{"}\SpecialCharTok{\textbackslash{}n}\StringTok{Risk Level: MODERATE {-} Consider engineering controls}\SpecialCharTok{\textbackslash{}n}\StringTok{"}\NormalTok{)}
\NormalTok{\} }\ControlFlowTok{else}\NormalTok{ \{}
  \FunctionTok{cat}\NormalTok{(}\StringTok{"}\SpecialCharTok{\textbackslash{}n}\StringTok{Risk Level: HIGH {-} Immediate redesign required}\SpecialCharTok{\textbackslash{}n}\StringTok{"}\NormalTok{)}
\NormalTok{\}}
\end{Highlighting}
\end{Shaded}

\begin{verbatim}
## 
## Risk Level: MODERATE - Consider engineering controls
\end{verbatim}

Discussion Question: Reducing the Lifting Index

\textbf{Based on the example above, what layout or workstation changes could reduce the Lifting Index?}

\textbf{Solutions (in order of effectiveness):}

\begin{enumerate}
\def\labelenumi{\arabic{enumi}.}
\tightlist
\item
  \textbf{Reduce horizontal distance (H):} Move conveyor closer to pallet position

  \begin{itemize}
  \tightlist
  \item
    If H = 10'': HM improves from 0.67 to 1.0 → RWL increases by 50\%
  \end{itemize}
\item
  \textbf{Raise the starting height (V):} Use elevated pallet positions or pallet lifters

  \begin{itemize}
  \tightlist
  \item
    If V = 30'' (optimal): VM improves from 0.85 to 1.0 → RWL increases by 18\%
  \end{itemize}
\item
  \textbf{Eliminate asymmetry (A):} Reposition conveyor directly in front of pallet

  \begin{itemize}
  \tightlist
  \item
    If A = 0°: AM improves from 0.90 to 1.0 → RWL increases by 11\%
  \end{itemize}
\item
  \textbf{Reduce frequency:} Add a second worker or use automation

  \begin{itemize}
  \tightlist
  \item
    If 1 lift/min: FM improves from 0.72 to 0.94 → RWL increases by 31\%
  \end{itemize}
\item
  \textbf{Improve coupling:} Use boxes with good handles

  \begin{itemize}
  \tightlist
  \item
    If good coupling: CM improves from 0.95 to 1.0 → RWL increases by 5\%
  \end{itemize}
\end{enumerate}

\textbf{Best solution:} Implement a scissor lift for the pallet (adjustable height) and move the conveyor closer. This addresses the largest multiplier reductions.

\subsection{Ergonomic Workstation Design}\label{ergonomic-workstation-design}

\begin{figure}

{\centering \includegraphics{introduction_files/figure-latex/ergonomic-zones-1} 

}

\caption{Ergonomic Work Zones for Standing Operations}\label{fig:ergonomic-zones}
\end{figure}

\textbf{Layout Implications:}

\begin{itemize}
\tightlist
\item
  \textbf{Primary zone:} Most frequently used tools and parts
\item
  \textbf{Secondary zone:} Regularly used items, occasional reaches
\item
  \textbf{Tertiary zone:} Rarely needed items, should trigger redesign if frequent use required
\end{itemize}

\begin{center}\rule{0.5\linewidth}{0.5pt}\end{center}

\section{Modern Layout Trends and Industry 4.0}\label{modern-layout-trends-and-industry-4.0}

\subsection{Flexible Manufacturing Layouts}\label{flexible-manufacturing-layouts}

Modern facilities must adapt quickly to changing product demands. Key trends include:

\label{tab:flexible-layout-trends}Modern Flexible Layout Trends

Trend

Description

Benefits

Modular Workstations

Pre-built workstation modules that can be relocated in hours

Rapid response to volume changes, easy expansion

Reconfigurable Cells

Cells with standardized machine interfaces for quick changeover

Switch between product families without major capital

Mobile Equipment

Equipment on wheels or air bearings for rapid repositioning

Eliminate fixed foundations, reduce floor damage

Flexible Utilities

Overhead utility drops and floor trenches allow equipment mobility

No hardwired constraints on equipment placement

Digital Layout Planning

Digital twins enable virtual layout testing before physical changes

Reduce trial-and-error, optimize before implementation

\subsection{Industry 4.0 Layout Considerations}\label{industry-4.0-layout-considerations}

\begin{figure}

{\centering \includegraphics{introduction_files/figure-latex/industry-4-layout-1} 

}

\caption{Industry 4.0 Enabled Plant Layout Elements}\label{fig:industry-4-layout}
\end{figure}

\subsection{AGV and AMR Traffic Planning}\label{agv-and-amr-traffic-planning}

Automated Guided Vehicles (AGVs) and Autonomous Mobile Robots (AMRs) require careful layout planning:

\label{tab:agv-considerations}AGV/AMR Layout Planning Guidelines

Consideration

Design Guidelines

Route Design

Unidirectional loops preferred; minimize intersections; avoid steep grades

Charging Locations

Opportunity charging at pickup/dropoff points; dedicated charging bays for breaks

Traffic Intersections

Yield zones, traffic signals, or elevation changes at crossings

Floor Requirements

Smooth, level surfaces; magnetic tape or QR codes for navigation; no oil/water

Safety Zones

Minimum 1m clearance from pedestrian paths; light curtains at crossings

Staging Areas

Buffer areas for queue management; waiting positions clear of traffic

Video: AGV Systems in Modern Warehouses

\begin{center}\rule{0.5\linewidth}{0.5pt}\end{center}

\section{Layout Simulation Software}\label{layout-simulation-software}

Modern plant layout design relies heavily on simulation software to test alternatives before physical implementation.

\subsection{Common Layout Planning Software}\label{common-layout-planning-software}

\label{tab:software-comparison}Plant Layout Planning Software Comparison

Software

Primary Use

Key Features

Typical Users

AutoCAD Factory Design Suite

2D/3D CAD layout design

Extensive equipment libraries, interference checking, material flow analysis

Facility engineers, architects

FlexSim

3D discrete event simulation

Drag-and-drop modeling, VR capability, real-time 3D animation

Industrial engineers, consultants

Arena

Process simulation

Statistical analysis, what-if scenarios, bottleneck identification

Process engineers, academics

Plant Simulation (Siemens)

Digital factory simulation

Full Siemens integration, AGV simulation, energy analysis

Large manufacturers (automotive, aerospace)

SketchUp + Extensions

Quick 3D visualization

Low cost, easy learning curve, walkthrough capability

Small businesses, initial concepts

\subsection{Simulation Benefits}\label{simulation-benefits}

\begin{figure}

{\centering \includegraphics{introduction_files/figure-latex/simulation-benefits-1} 

}

\caption{Benefits of Layout Simulation}\label{fig:simulation-benefits}
\end{figure}

\textbf{ROI of Simulation:}

\begin{itemize}
\tightlist
\item
  \textbf{10-20x} return on simulation investment typical
\item
  \textbf{30-50\%} reduction in layout change costs
\item
  \textbf{Weeks to months} saved in ramp-up time
\item
  \textbf{Stakeholder alignment} - visual communication of concepts
\end{itemize}

Video: Plant Simulation Demo

\begin{center}\rule{0.5\linewidth}{0.5pt}\end{center}

\section{Practice Problems}\label{practice-problems}

Problem 1: Layout Selection

\textbf{Scenario:} A company manufactures custom kitchen cabinets. Each order is unique, with different dimensions, materials, and finishes. Production volume is about 50 units per month.

\textbf{Question:} Which layout type would you recommend and why?

\textbf{Answer:} A \textbf{process layout} would be most appropriate because:
- High product variety (custom orders)
- Low production volume
- Different operations needed for each order
- Flexibility is more important than efficiency

Problem 2: Line Balancing

\textbf{Given:}
- Total task time: 240 seconds
- Desired output: 30 units per hour
- Available time: 3600 seconds per hour

\textbf{Calculate:}
1. Cycle time
2. Theoretical minimum number of stations
3. If you use 3 stations, what is the efficiency?

\textbf{Solution:}
1. Cycle time = 3600 / 30 = \textbf{120 seconds}
2. TM = 240 / 120 = \textbf{2 stations}
3. Efficiency = (240 / (3 × 120)) × 100 = \textbf{66.7\%}

Problem 3: Warehouse Layout

\textbf{Scenario:} A warehouse has the following items:
- Item X: 500 picks/day
- Item Y: 100 picks/day
- Item Z: 50 picks/day

\textbf{Question:} How should these items be positioned relative to the shipping dock?

\textbf{Answer:} Using ABC analysis:
- Item X (high frequency): \textbf{Closest to dock} (A zone)
- Item Y (medium frequency): \textbf{Middle distance} (B zone)
- Item Z (low frequency): \textbf{Farthest from dock} (C zone)

\begin{center}\rule{0.5\linewidth}{0.5pt}\end{center}

\begin{center}\rule{0.5\linewidth}{0.5pt}\end{center}

\section{Review Questions}\label{review-questions-3}

Question 1: What distinguishes a cellular layout from both process and product layouts?

\textbf{Answer:} A cellular layout combines characteristics of both:

\begin{itemize}
\tightlist
\item
  \textbf{From process layout:} Equipment variety within each cell, flexibility to handle part variations
\item
  \textbf{From product layout:} Sequential flow within the cell, dedicated equipment for part families
\item
  \textbf{Unique characteristics:}

  \begin{itemize}
  \tightlist
  \item
    Parts grouped into families using Group Technology
  \item
    Cells are self-contained mini-factories
  \item
    Workers often cross-trained to operate multiple machines
  \item
    Reduced material handling compared to process layout
  \item
    More flexibility than pure product layout
  \end{itemize}
\end{itemize}

Question 2: Calculate the Lifting Index for the following scenario

\textbf{Scenario:} A worker lifts 25 kg boxes from floor level (V=0) to a shelf at waist height. The horizontal distance is 40 cm, vertical travel is 75 cm, no asymmetry, frequency is 2 lifts per minute for 4 hours, and coupling is poor (no handles).

\textbf{Given multipliers:}
- LC = 23 kg
- HM = 25/H (H in cm)
- VM = 1 - 0.003\textbar V - 75\textbar{} (V in cm)
- DM = 0.82 + 4.5/D (D in cm)
- AM = 1.0 (no asymmetry)
- FM = 0.65 (from tables)
- CM = 0.90 (poor coupling)

\textbf{Solution:}

\begin{verbatim}
HM = 25/40 = 0.625
VM = 1 - 0.003|0 - 75| = 1 - 0.225 = 0.775
DM = 0.82 + 4.5/75 = 0.82 + 0.06 = 0.88
AM = 1.0
FM = 0.65
CM = 0.90

RWL = 23 × 0.625 × 0.775 × 0.88 × 1.0 × 0.65 × 0.90
RWL = 23 × 0.625 × 0.775 × 0.88 × 0.65 × 0.90
RWL = 5.7 kg

LI = 25/5.7 = 4.4

Risk Level: HIGH (LI > 3.0) - Immediate redesign required
\end{verbatim}

\textbf{Recommended actions:} Raise starting height (use pallet lifter), reduce horizontal distance, add handles to boxes, reduce frequency.

Question 3: List three Industry 4.0 technologies that impact plant layout design

\textbf{Answer:} (Any three of the following)

\begin{enumerate}
\def\labelenumi{\arabic{enumi}.}
\tightlist
\item
  \textbf{AGVs/AMRs:} Require route planning, charging stations, floor markings, safety zones
\item
  \textbf{Collaborative robots (cobots):} Need safety zones, but smaller footprint than traditional robot cells
\item
  \textbf{Digital twins:} Enable virtual layout testing, reduce physical prototyping
\item
  \textbf{IoT sensors:} Require network infrastructure (5G/WiFi), edge computing locations
\item
  \textbf{AR/VR systems:} Training stations, remote assistance capabilities
\item
  \textbf{Flexible automation:} Modular equipment, quick-connect utilities
\end{enumerate}

Question 4: What are the key differences between food processing plant layouts and general manufacturing?

\textbf{Answer:}

\begin{longtable}[]{@{}
  >{\raggedright\arraybackslash}p{(\linewidth - 4\tabcolsep) * \real{0.1739}}
  >{\raggedright\arraybackslash}p{(\linewidth - 4\tabcolsep) * \real{0.3478}}
  >{\raggedright\arraybackslash}p{(\linewidth - 4\tabcolsep) * \real{0.4783}}@{}}
\toprule\noalign{}
\begin{minipage}[b]{\linewidth}\raggedright
Aspect
\end{minipage} & \begin{minipage}[b]{\linewidth}\raggedright
Food Processing
\end{minipage} & \begin{minipage}[b]{\linewidth}\raggedright
General Manufacturing
\end{minipage} \\
\midrule\noalign{}
\endhead
\bottomrule\noalign{}
\endlastfoot
Flow direction & Strictly one-way (raw → finished) & Can have backtracking \\
Zone separation & Physical barriers between raw/clean & Logical separation often sufficient \\
Surfaces & Sanitary design, smooth, cleanable & Durability focused \\
Temperature & Multiple temperature zones (cold chain) & Generally ambient \\
Regulatory & HACCP, FSMA, FDA oversight & OSHA, industry standards \\
Allergen control & Dedicated lines or validated cleaning & Not typically a concern \\
Traceability & Lot-level required by regulation & Often batch-level sufficient \\
\end{longtable}

Question 5: A warehouse has three product categories with the following pick frequencies. Design the storage zones.

\textbf{Data:}
- Category A: 1200 picks/day, 500 SKUs
- Category B: 600 picks/day, 1500 SKUs
- Category C: 200 picks/day, 3000 SKUs

\textbf{Solution:}

Using ABC analysis based on pick frequency:
- \textbf{Category A} (60\% of picks, 10\% of SKUs): Prime locations closest to shipping dock
- \textbf{Category B} (30\% of picks, 30\% of SKUs): Secondary locations, moderate distance
- \textbf{Category C} (10\% of picks, 60\% of SKUs): Remote locations, maximum distance from dock

\textbf{Additional considerations:}
- Within each zone, use cube utilization for dense storage of slow movers
- Consider pick path optimization (serpentine vs.~return routing)
- High-frequency items at ergonomic pick heights (waist level)
- Reserve space near dock for cross-docking high-velocity items

\begin{center}\rule{0.5\linewidth}{0.5pt}\end{center}

\section{Chapter Summary}\label{chapter-summary}

\label{tab:chapter-summary}Chapter 4 Summary: Plant Layout and Facility Design

Topic

Key Concepts

Key Formulas

Layout Types

Fixed-position, Process, Product, Hybrid/Cellular - selected based on volume and variety

Layout selection: Volume × Variety matrix

Process Layout Design

From-To matrix, Load-Distance model, Block plans, Closeness relationships

Cost = Σ Lij × Dij × C

Product Layout (Line Balancing)

Cycle time, Theoretical minimum stations, Precedence diagrams, Line efficiency

C = Time/Output; TM = Σti/C; Eff = Σti/(n×C)

Cellular Manufacturing

Group Technology, Part families, Self-contained cells, Reduced material handling

Cell efficiency = Output/(Input × Time)

Warehouse Design

ABC analysis, High-frequency items near dock, Pick path optimization

Travel distance = Σ(picks × distance)

Ergonomics

NIOSH lifting equation, RWL, Lifting Index, Work zones, Workstation design

RWL = LC×HM×VM×DM×AM×FM×CM; LI = Weight/RWL

Modern Trends

Flexible layouts, Industry 4.0, AGV/AMR planning, Digital twins

ROI = (Savings - Investment)/Investment

Simulation

FlexSim, Arena, Plant Simulation - virtual testing before physical changes

Simulation accuracy \textgreater{} 90\% typical

\begin{center}\rule{0.5\linewidth}{0.5pt}\end{center}

\section{References}\label{references-3}

\begin{itemize}
\tightlist
\item
  Reid, R. D., \& Sanders, N. R. (2019). \emph{Operations Management: An Integrated Approach} (7th ed.). Wiley. Chapter 10.
\item
  Groover, M. P. (2020). \emph{Automation, Production Systems, and Computer-Integrated Manufacturing} (5th ed.). Pearson.
\item
  Waters, T. R., Putz-Anderson, V., \& Garg, A. (1994). \emph{Applications Manual for the Revised NIOSH Lifting Equation}. NIOSH Publication No.~94-110.
\item
  Tompkins, J. A., White, J. A., Bozer, Y. A., \& Tanchoco, J. M. A. (2010). \emph{Facilities Planning} (4th ed.). Wiley.
\item
  FDA. (2021). \emph{Current Good Manufacturing Practice, Hazard Analysis, and Risk-Based Preventive Controls for Human Food}. 21 CFR Part 117.
\item
  OSHA. (2021). \emph{Ergonomics: The Study of Work}. OSHA Publication 3125.
\end{itemize}

\chapter{Lean Manufacturing}\label{lean-manufacturing}

\begin{center}\rule{0.5\linewidth}{0.5pt}\end{center}

\section{Learning Objectives}\label{learning-objectives-4}

By the end of this chapter, you will be able to:

\begin{itemize}
\tightlist
\item
  Define Lean Manufacturing and explain its origins
\item
  Identify and describe the 3 M's of Lean (Muda, Mura, Muri)
\item
  Recognize and eliminate the seven types of waste
\item
  Apply the 5S methodology to workplace organization
\item
  Understand Just-in-Time (JIT) principles and Kanban systems
\item
  Calculate Takt time and apply line balancing concepts
\item
  Implement setup reduction techniques (SMED)
\end{itemize}

\begin{quote}
``Perfection is not attainable. But if we chase perfection, we can catch excellence.''
--- Vince Lombardi
\end{quote}

\begin{center}\rule{0.5\linewidth}{0.5pt}\end{center}

\section{Introduction to Lean Manufacturing}\label{introduction-to-lean-manufacturing}

\textbf{Lean Manufacturing} is the systematic elimination of waste. The term can be traced to Jim Womack, Daniel Jones, and Daniel Roos' book \emph{The Machine that Changed the World} (1991), which examined the Toyota Production System.

What does ``Lean'' actually mean?

The term ``Lean'' refers to cutting ``fat'' from production activities --- eliminating anything that doesn't add value to the customer. Think of it like a lean athlete: no excess weight, maximum efficiency, peak performance.

\subsection{Definition of Lean}\label{definition-of-lean}

\label{tab:lean-definition}Lean Manufacturing Targets (Womack, Jones, Roos 1990)

Metric

Lean Target

Human effort in factory

Half the hours

Defects in finished product

Half the defects

Engineering effort

One-third the hours

Factory space for same output

Half the space

In-process inventories

A tenth or less

\begin{center}\rule{0.5\linewidth}{0.5pt}\end{center}

\section{Evolution of Manufacturing Systems}\label{evolution-of-manufacturing-systems}

Manufacturing has evolved through three major paradigms. Understanding this evolution helps us appreciate why Lean emerged as the dominant philosophy.

\begin{figure}

{\centering \includegraphics{introduction_files/figure-latex/manufacturing-evolution-1} 

}

\caption{Evolution of Manufacturing Systems}\label{fig:manufacturing-evolution}
\end{figure}

\subsection{Comparison of Manufacturing Systems}\label{comparison-of-manufacturing-systems}

\label{tab:mfg-comparison-table}Evolution of Manufacturing Systems

Characteristic

Craft Manufacturing

Mass Manufacturing

Lean Manufacturing

Time Period

Late 1800s

1920s (Ford)

1970s - Present

Process

Built on blocks, workers walk around

Assembly line

Cells or flexible lines

Workforce

Craftsmen with pride

Low skilled, simplistic jobs

Highly skilled, proud of product

Components

Hand-crafted, hand-fitted

Interchangeable parts

Interchangeable, high variety

Quality

Excellent

Lower

Excellent (mandatory)

Cost

Very expensive

Affordable

Continuously decreasing

Volume

Few produced

Billions - identical

High volume, high variety

Flexibility

Very high

Very low

High

Discussion: Why did Mass Manufacturing dominate for so long?

Mass manufacturing dominated because:

\begin{enumerate}
\def\labelenumi{\arabic{enumi}.}
\tightlist
\item
  \textbf{Economies of scale} - Higher volume = lower unit cost
\item
  \textbf{Predictable demand} - Post-war boom created stable markets
\item
  \textbf{Limited competition} - Few global competitors
\item
  \textbf{Consumer acceptance} - ``Any color as long as it's black'' (Henry Ford)
\end{enumerate}

\textbf{What changed?}
- Global competition (especially from Japan)
- Customers demanded variety and quality
- Product lifecycles shortened
- Technology enabled flexibility

\begin{center}\rule{0.5\linewidth}{0.5pt}\end{center}

\section{The Toyota Production System (TPS)}\label{the-toyota-production-system-tps}

The most influential lean manufacturing model is the \textbf{Toyota Production System (TPS)}, developed by Taiichi Ohno and Shigeo Shingo at Toyota.

\subsection{Historical Context}\label{historical-context}

\begin{figure}

{\centering \includegraphics{introduction_files/figure-latex/tps-timeline-1} 

}

\caption{Toyota's Journey to Lean Excellence}\label{fig:tps-timeline}
\end{figure}

\textbf{In 1949}, Toyota was on the brink of bankruptcy. Ford's car production was at least \textbf{8 times more efficient} than Toyota's.

\begin{quote}
\textbf{Kiichiro Toyoda's Challenge:} ``To achieve the same rate of production as the United States in three years.''
\end{quote}

\textbf{Taiichi Ohno} accepted this challenge. Inspired by American supermarkets, he developed the Just-in-Time method.

\subsection{Two Pillars of TPS}\label{two-pillars-of-tps}

\begin{figure}

{\centering \includegraphics{introduction_files/figure-latex/tps-pillars-1} 

}

\caption{The Two Pillars of Toyota Production System}\label{fig:tps-pillars}
\end{figure}

\textbf{1. Just-in-Time (JIT):} Produce only what is needed, when it is needed, in the amount needed.

\textbf{2. Jidoka (Built-in Quality):} Automation with a human touch --- stop production immediately when a defect is detected.

\begin{center}\rule{0.5\linewidth}{0.5pt}\end{center}

\section{The 3 M's of Lean}\label{the-3-ms-of-lean}

Lean manufacturing focuses on eliminating three types of inefficiency, known as the \textbf{3 M's}:

\begin{figure}

{\centering \includegraphics{introduction_files/figure-latex/three-ms-1} 

}

\caption{The 3 M's of Lean Manufacturing}\label{fig:three-ms}
\end{figure}

\label{tab:three-ms-table}The 3 M's Explained

Japanese

English

Description

Example

\textbf{Muda} (無駄)

Waste

Activities that consume resources without adding value

Excess inventory, waiting, defects

\textbf{Mura} (斑)

Inconsistency/Unevenness

Variation and lack of uniformity in processes

Uneven production schedules, variable quality

\textbf{Muri} (無理)

Overburden/Unreasonableness

Overburdening people or equipment beyond capacity

Unrealistic deadlines, overworked employees

\begin{center}\rule{0.5\linewidth}{0.5pt}\end{center}

\section{The Seven Types of Waste (MUDA)}\label{the-seven-types-of-waste-muda}

\textbf{Shigeo Shingo} identified seven categories of waste common to manufacturing. These are activities that add cost without adding value.

\begin{quote}
``Waste is everything that is not absolutely essential.''
--- Hiroyuki Hirano
\end{quote}

\begin{figure}

{\centering \includegraphics{introduction_files/figure-latex/seven-wastes-viz-1} 

}

\caption{The Seven Types of Waste (TIMWOOD)}\label{fig:seven-wastes-viz}
\end{figure}

\subsection{Detailed Breakdown of the Seven Wastes}\label{detailed-breakdown-of-the-seven-wastes}

\label{tab:seven-wastes-table}The Seven Wastes: Detailed Analysis

\#

Waste Type

Description

Examples

Solution

1

\textbf{Overproduction}

Producing more than needed, earlier than needed

Building to forecast, large batch sizes

Pull systems, smaller batches

2

\textbf{Inventory}

Excess raw materials, WIP, or finished goods

Safety stock, buffer inventory, obsolete stock

JIT delivery, kanban systems

3

\textbf{Waiting}

Idle time waiting for materials, information, or equipment

Machine downtime, waiting for approvals

TPM, better scheduling

4

\textbf{Transportation}

Unnecessary movement of materials between processes

Long distances between operations, poor layout

Cellular layout, co-location

5

\textbf{Over-processing}

Doing more work than the customer requires

Tighter tolerances than needed, redundant inspections

Value engineering, standardization

6

\textbf{Motion}

Unnecessary movement of people

Reaching, bending, walking to get tools

5S, ergonomic workstations

7

\textbf{Defects}

Products that don't meet specifications

Scrap, rework, warranty claims

Poka-yoke, root cause analysis

Memory Aid: TIMWOOD

Use \textbf{TIMWOOD} to remember the seven wastes:

\begin{itemize}
\tightlist
\item
  \textbf{T}ransportation
\item
  \textbf{I}nventory
\item
  \textbf{M}otion
\item
  \textbf{W}aiting
\item
  \textbf{O}verproduction
\item
  \textbf{O}ver-processing
\item
  \textbf{D}efects
\end{itemize}

Some people also use \textbf{DOWNTIME} (adding ``Non-utilized talent'' as an 8th waste).

Interactive Exercise: Identify the Waste

\textbf{Scenario 1:} A worker walks 50 feet to get a tool, uses it for 30 seconds, then walks back.

\textbf{Answer:} This is \textbf{Motion} waste. The solution would be to place tools at the point of use (5S).

\textbf{Scenario 2:} A machine produces 1000 parts, but only 800 are needed this week.

\textbf{Answer:} This is \textbf{Overproduction} waste. The solution would be to implement pull production based on actual demand.

\textbf{Scenario 3:} Parts sit in a queue for 3 days before being processed at the next station.

\textbf{Answer:} This is both \textbf{Inventory} and \textbf{Waiting} waste. The solution would be to balance the line and implement one-piece flow.

\begin{center}\rule{0.5\linewidth}{0.5pt}\end{center}

\section{The 5S Methodology}\label{the-5s-methodology}

\textbf{5S} is a workplace organization methodology that creates a clean, efficient, and safe working environment.

\begin{figure}

{\centering \includegraphics{introduction_files/figure-latex/five-s-viz-1} 

}

\caption{The 5S Methodology}\label{fig:five-s-viz}
\end{figure}

\label{tab:five-s-table}The 5S Steps Explained

Step

Japanese

English

Action

Key Question

1S

\textbf{Seiri}

Sort

Remove unnecessary items from the workplace

Do we need this?

2S

\textbf{Seiton}

Set in Order

Organize items so they are easy to find and use

Where does this belong?

3S

\textbf{Seiso}

Shine

Clean the workplace regularly

Is everything clean?

4S

\textbf{Seiketsu}

Standardize

Create standards for maintaining 1S-3S

How do we keep it this way?

5S

\textbf{Shitsuke}

Sustain

Build discipline to maintain standards

Are we following our standards?

\subsection{Visual Factory}\label{visual-factory}

\begin{quote}
``The ability to understand the status of a production area in \textbf{5 minutes or less} by simple observation without use of computers or speaking to anyone.''
\end{quote}

A well-implemented 5S creates a \textbf{visual factory} where problems are immediately apparent.

\begin{center}\rule{0.5\linewidth}{0.5pt}\end{center}

\section{Just-in-Time (JIT) Production}\label{just-in-time-jit-production}

\subsection{The JIT Philosophy}\label{the-jit-philosophy}

\begin{quote}
``Deliver the right material, in the exact quantity, with perfect quality, in the right place, just before it is needed.''
--- Ohno and Shingo
\end{quote}

\begin{verbatim}
## Warning in geom_segment(aes(x = 5.4, xend = 5.9, y = 1, yend = 1), arrow = arrow(length = unit(0.3, : All aesthetics have length 1, but the data has 5
## rows.
## i Please consider using `annotate()` or provide
##   this layer with data containing a single row.
\end{verbatim}

\begin{figure}

{\centering \includegraphics{introduction_files/figure-latex/jit-concept-1} 

}

\caption{Just-in-Time: The Right Everything}\label{fig:jit-concept}
\end{figure}

\subsection{Push vs.~Pull Systems}\label{push-vs.-pull-systems}

\begin{figure}

{\centering \includegraphics{introduction_files/figure-latex/push-pull-viz-1} 

}

\caption{Push vs. Pull Production Systems}\label{fig:push-pull-viz}
\end{figure}

\label{tab:push-pull-table}Push vs.~Pull System Comparison

Aspect

Push System

Pull System

Production trigger

Forecast/Schedule

Actual demand/Kanban signal

Information flow

Forward (upstream to downstream)

Backward (downstream to upstream)

Inventory levels

High (buffers at each station)

Low (minimal buffers)

Flexibility

Low

High

Problem visibility

Problems hidden by inventory

Problems immediately visible

Customer responsiveness

Slow

Fast

\subsection{Kanban System}\label{kanban-system}

\textbf{Kanban} (看板) is Japanese for ``signboard'' or ``billboard.'' It's a scheduling system for lean and JIT production.

\begin{figure}

{\centering \includegraphics{introduction_files/figure-latex/kanban-example-1} 

}

\caption{Example Kanban Card}\label{fig:kanban-example}
\end{figure}

\begin{center}\rule{0.5\linewidth}{0.5pt}\end{center}

\section{Key Time Metrics}\label{key-time-metrics-1}

Understanding time metrics is essential for lean manufacturing analysis and improvement.

\subsection{Definitions}\label{definitions}

\begin{figure}

{\centering \includegraphics{introduction_files/figure-latex/time-metrics-viz-1} 

}

\caption{Relationship Between Time Metrics}\label{fig:time-metrics-viz}
\end{figure}

\subsection{Formulas and Calculations}\label{formulas-and-calculations}

\textbf{Takt Time} --- The pace of production to meet customer demand:

\[\text{Takt Time} = \frac{\text{Available Production Time}}{\text{Customer Demand}}\]

\textbf{Cycle Time} --- Time to complete one unit at a workstation:

\[\text{Cycle Time} = \frac{\text{Net Operating Time}}{\text{Units Produced}}\]

\textbf{Throughput Time} --- Total time from start to finish:

\[\text{Throughput Time} = \text{Processing} + \text{Inspection} + \text{Queue} + \text{Move Time}\]

\subsection{Interactive Example: Calculating Takt Time}\label{interactive-example-calculating-takt-time}

\begin{Shaded}
\begin{Highlighting}[]
\CommentTok{\# Given data}
\NormalTok{available\_time\_per\_day }\OtherTok{\textless{}{-}} \DecValTok{8} \SpecialCharTok{*} \DecValTok{60}  \CommentTok{\# 8 hours = 480 minutes}
\NormalTok{breaks\_and\_meetings }\OtherTok{\textless{}{-}} \DecValTok{60}         \CommentTok{\# 60 minutes for breaks/meetings}
\NormalTok{customer\_demand }\OtherTok{\textless{}{-}} \DecValTok{240}            \CommentTok{\# 240 units per day}

\CommentTok{\# Calculate net available time}
\NormalTok{net\_available\_time }\OtherTok{\textless{}{-}}\NormalTok{ available\_time\_per\_day }\SpecialCharTok{{-}}\NormalTok{ breaks\_and\_meetings}
\FunctionTok{cat}\NormalTok{(}\StringTok{"Net Available Time:"}\NormalTok{, net\_available\_time, }\StringTok{"minutes/day}\SpecialCharTok{\textbackslash{}n}\StringTok{"}\NormalTok{)}
\end{Highlighting}
\end{Shaded}

\begin{verbatim}
## Net Available Time: 420 minutes/day
\end{verbatim}

\begin{Shaded}
\begin{Highlighting}[]
\CommentTok{\# Calculate Takt Time}
\NormalTok{takt\_time }\OtherTok{\textless{}{-}}\NormalTok{ net\_available\_time }\SpecialCharTok{/}\NormalTok{ customer\_demand}
\FunctionTok{cat}\NormalTok{(}\StringTok{"Takt Time:"}\NormalTok{, takt\_time, }\StringTok{"minutes per unit}\SpecialCharTok{\textbackslash{}n}\StringTok{"}\NormalTok{)}
\end{Highlighting}
\end{Shaded}

\begin{verbatim}
## Takt Time: 1.75 minutes per unit
\end{verbatim}

\begin{Shaded}
\begin{Highlighting}[]
\FunctionTok{cat}\NormalTok{(}\StringTok{"Takt Time:"}\NormalTok{, takt\_time }\SpecialCharTok{*} \DecValTok{60}\NormalTok{, }\StringTok{"seconds per unit}\SpecialCharTok{\textbackslash{}n}\StringTok{"}\NormalTok{)}
\end{Highlighting}
\end{Shaded}

\begin{verbatim}
## Takt Time: 105 seconds per unit
\end{verbatim}

\begin{Shaded}
\begin{Highlighting}[]
\CommentTok{\# Interpretation}
\FunctionTok{cat}\NormalTok{(}\StringTok{"}\SpecialCharTok{\textbackslash{}n}\StringTok{{-}{-}{-} Interpretation {-}{-}{-}}\SpecialCharTok{\textbackslash{}n}\StringTok{"}\NormalTok{)}
\end{Highlighting}
\end{Shaded}

\begin{verbatim}
## 
## --- Interpretation ---
\end{verbatim}

\begin{Shaded}
\begin{Highlighting}[]
\FunctionTok{cat}\NormalTok{(}\StringTok{"To meet customer demand, we must produce 1 unit every"}\NormalTok{,}
\NormalTok{    takt\_time, }\StringTok{"minutes ("}\NormalTok{, takt\_time }\SpecialCharTok{*} \DecValTok{60}\NormalTok{, }\StringTok{"seconds).}\SpecialCharTok{\textbackslash{}n}\StringTok{"}\NormalTok{)}
\end{Highlighting}
\end{Shaded}

\begin{verbatim}
## To meet customer demand, we must produce 1 unit every 1.75 minutes ( 105 seconds).
\end{verbatim}

Practice Problem: Calculate Your Takt Time

\textbf{Given:}
- Shift length: 10 hours
- Two 15-minute breaks and one 30-minute lunch
- Customer requires 300 units per shift

\textbf{Calculate:}
1. Net available production time
2. Takt time in minutes
3. Takt time in seconds

\textbf{Solution:}

\begin{verbatim}
Net time = (10 × 60) - (15 + 15 + 30) = 600 - 60 = 540 minutes
Takt time = 540 / 300 = 1.8 minutes = 108 seconds per unit
\end{verbatim}

\begin{center}\rule{0.5\linewidth}{0.5pt}\end{center}

\section{Setup Reduction (SMED)}\label{setup-reduction-smed}

\textbf{SMED} (Single Minute Exchange of Dies) is a methodology developed by Shigeo Shingo to dramatically reduce changeover time.

\subsection{The SMED Concept}\label{the-smed-concept}

\begin{figure}

{\centering \includegraphics{introduction_files/figure-latex/smed-viz-1} 

}

\caption{SMED: Internal vs External Setup}\label{fig:smed-viz}
\end{figure}

\subsection{SMED Methodology}\label{smed-methodology}

\label{tab:smed-steps}The Four Steps of SMED

Step

Action

Description

Example

1

\textbf{Document current process}

Video the entire changeover, identify all steps

Time each step of die change

2

\textbf{Separate internal and external}

Internal = machine must be stopped; External = can be done while running

Getting tools = external; Removing die = internal

3

\textbf{Convert internal to external}

Find ways to perform `internal' tasks while machine is running

Pre-stage next die, pre-heat tools

4

\textbf{Streamline all operations}

Reduce time for remaining internal tasks through better methods

Use quick-release clamps instead of bolts

Case Study: Press Setup Reduction

\textbf{Problem:} Presses were idle more than 50\% of the time due to setup.
- Average production run: 1-2 days
- Setup time: 1-2 days

\textbf{Root Cause Analysis:}
- Using standard bolt fasteners with open-end wrenches
- No pre-staging of tools or dies
- Sequential operations that could be parallel

\textbf{Solutions Implemented:}

\begin{longtable}[]{@{}ll@{}}
\toprule\noalign{}
Change & Time Saved \\
\midrule\noalign{}
\endhead
\bottomrule\noalign{}
\endlastfoot
Socket sets + power tools & 10\% \\
Quick-release clamps & 50\% \\
Pre-staging materials & 20\% \\
Parallel operations & 20\% \\
\end{longtable}

\textbf{Result:} Setup time reduced from 1-2 days to 4-6 hours, with a target of under 1 hour.

\begin{center}\rule{0.5\linewidth}{0.5pt}\end{center}

\section{Kaizen: Continuous Improvement}\label{kaizen-continuous-improvement}

\textbf{Kaizen} (改善) means ``change for the better'' --- the philosophy of continuous, incremental improvement.

\subsection{Kaizen Principles}\label{kaizen-principles}

\begin{figure}

{\centering \includegraphics{introduction_files/figure-latex/kaizen-cycle-1} 

}

\caption{The Kaizen Continuous Improvement Cycle}\label{fig:kaizen-cycle}
\end{figure}

\subsection{Key Kaizen Concepts}\label{key-kaizen-concepts}

\begin{itemize}
\tightlist
\item
  \textbf{Everyone participates} --- from CEO to frontline workers
\item
  \textbf{Small, incremental changes} --- not major overhauls
\item
  \textbf{Focus on process} --- blame the process, not the person
\item
  \textbf{Data-driven decisions} --- measure before and after
\item
  \textbf{Standardize improvements} --- lock in gains
\end{itemize}

\begin{center}\rule{0.5\linewidth}{0.5pt}\end{center}

\section{Value Stream Mapping (VSM)}\label{value-stream-mapping-vsm}

\textbf{Value Stream Mapping} is a lean tool for analyzing the flow of materials and information required to bring a product to the customer.

\subsection{VSM Symbols}\label{vsm-symbols}

\label{tab:vsm-metrics}Key Metrics in Value Stream Maps

Metric

Description

Typical Location

\textbf{Process Time (PT)}

Time to actually process one unit

In process box

\textbf{Lead Time (LT)}

Total time from raw material to finished good

Timeline at bottom

\textbf{Changeover Time (C/O)}

Time to switch between products

In process box

\textbf{Uptime}

Percentage of time equipment is available

In process box

\textbf{Takt Time}

Required pace to meet demand

Noted on map

\textbf{WIP Inventory}

Units waiting between processes

Triangle symbol between processes

\begin{center}\rule{0.5\linewidth}{0.5pt}\end{center}

\section{Summary}\label{summary-3}

\label{tab:summary-table-lean}Lean Manufacturing: Key Concepts Summary

Concept

Definition

Key Benefit

\textbf{Lean Manufacturing}

Systematic elimination of waste

Reduced cost, improved quality

\textbf{3 M's}

Muda (waste), Mura (inconsistency), Muri (overburden)

Framework for identifying improvement opportunities

\textbf{7 Wastes}

TIMWOOD: Transport, Inventory, Motion, Waiting, Overproduction, Over-processing, Defects

Systematic waste identification

\textbf{5S}

Sort, Set in Order, Shine, Standardize, Sustain

Organized, efficient workplace

\textbf{JIT}

Right part, right quantity, right quality, right place, right time

Minimal inventory, fast response

\textbf{Kanban}

Visual signal system for pull production

Controlled WIP, smooth flow

\textbf{SMED}

Single Minute Exchange of Dies - rapid changeover

Flexibility, reduced batch sizes

\textbf{Kaizen}

Continuous incremental improvement

Engaged workforce, sustained improvement

\begin{center}\rule{0.5\linewidth}{0.5pt}\end{center}

\section{Review Questions}\label{review-questions-4}

Question 1: What are the seven types of waste, and which is often considered the most impactful?

The seven types of waste (TIMWOOD) are:
1. \textbf{T}ransportation
2. \textbf{I}nventory
3. \textbf{M}otion
4. \textbf{W}aiting
5. \textbf{O}verproduction
6. \textbf{O}ver-processing
7. \textbf{D}efects

\textbf{Inventory} is often considered the most impactful because it hides other problems, ties up capital, and can become obsolete.

Question 2: Explain the difference between Push and Pull production systems.

\textbf{Push System:}
- Production based on forecasts
- Materials pushed through the system
- Results in high inventory
- Problems hidden by buffer stock

\textbf{Pull System:}
- Production triggered by actual demand
- Materials pulled by downstream operations
- Minimal inventory
- Problems immediately visible

Question 3: Calculate the Takt time for a facility with 450 minutes of available time and demand of 150 units.

\[\text{Takt Time} = \frac{450 \text{ minutes}}{150 \text{ units}} = 3 \text{ minutes per unit}\]

This means we need to produce one unit every 3 minutes (180 seconds) to meet customer demand.

Question 4: What is the difference between internal and external setup in SMED?

\textbf{Internal Setup:} Activities that can ONLY be performed when the machine is stopped (e.g., physically removing the old die).

\textbf{External Setup:} Activities that CAN be performed while the machine is still running (e.g., getting tools, pre-staging the next die, preparing materials).

The key to SMED is converting as many internal activities to external as possible.

\begin{center}\rule{0.5\linewidth}{0.5pt}\end{center}

\section{References}\label{references-4}

\begin{itemize}
\tightlist
\item
  Womack, J., Jones, D., \& Roos, D. (1990). \emph{The Machine that Changed the World}.
\item
  Ohno, T. (1988). \emph{Toyota Production System: Beyond Large-Scale Production}.
\item
  Shingo, S. (1985). \emph{A Revolution in Manufacturing: The SMED System}.
\item
  Liker, J. (2004). \emph{The Toyota Way}.
\item
  Wang, H.P., Chang, T.C., \& Wysk, R.A. (2007). \emph{Computer Aided Manufacturing}, Chapter 18.
\end{itemize}

\chapter{Machine Guarding and Functional Safety}\label{machine-guarding-and-functional-safety}

\begin{center}\rule{0.5\linewidth}{0.5pt}\end{center}

\section{Learning Objectives}\label{learning-objectives-5}

By the end of this chapter, you will be able to:

\begin{itemize}
\tightlist
\item
  Apply the Hierarchy of Controls to manufacturing safety design
\item
  Identify and specify appropriate physical guarding and interlock systems
\item
  Select and position presence-sensing devices (light curtains, area scanners)
\item
  Calculate safety distances for protective device placement
\item
  Understand collaborative robot safety requirements
\item
  Design emergency stop systems according to standards
\item
  Differentiate between safety integrity levels (SIL) and performance levels (PL)
\end{itemize}

\begin{quote}
``Safety isn't expensive, it's priceless.''
--- Anonymous
\end{quote}

\begin{center}\rule{0.5\linewidth}{0.5pt}\end{center}

\section{Introduction: The Safety Hierarchy}\label{introduction-the-safety-hierarchy}

In process engineering, safety is not merely a feature added at the end of a project; it is a \textbf{fundamental design constraint}. Before deploying safety devices, the engineer must apply the \textbf{Hierarchy of Controls} --- a systematic approach to eliminating or reducing workplace hazards.

\begin{figure}

{\centering \includegraphics{introduction_files/figure-latex/hierarchy-controls-viz-1} 

}

\caption{The Hierarchy of Controls (Most to Least Effective)}\label{fig:hierarchy-controls-viz}
\end{figure}

\label{tab:hierarchy-table}Hierarchy of Controls: Detailed Breakdown

Priority

Control Level

Description

Example

Effectiveness

1

\textbf{Elimination}

Physically remove the hazard entirely

Eliminate pinch point by redesigning mechanism

Most Effective

2

\textbf{Substitution}

Replace with something less hazardous

Use lower voltage (24V DC instead of 120V AC)

Very Effective

3

\textbf{Engineering Controls}

Isolate people from the hazard through design

Install machine guarding, light curtains, interlocks

Effective

4

\textbf{Administrative Controls}

Change the way people work

Safety procedures, training, warning signs, lock-out/tag-out

Less Effective

5

\textbf{PPE}

Protect the worker with equipment

Safety glasses, gloves, steel-toe boots, hearing protection

Least Effective

Discussion: Why not just rely on PPE?

\textbf{PPE (Personal Protective Equipment)} is the least effective control because:

\begin{enumerate}
\def\labelenumi{\arabic{enumi}.}
\tightlist
\item
  \textbf{It doesn't eliminate the hazard} --- the danger still exists
\item
  \textbf{It depends on human compliance} --- workers may forget or choose not to wear it
\item
  \textbf{It can fail} --- equipment degrades, doesn't fit properly, or is damaged
\item
  \textbf{It's the last line of defense} --- if PPE fails, the worker is exposed
\end{enumerate}

\textbf{Key Principle:} The more we can design out hazards through engineering controls, the less we depend on human behavior to keep people safe.

\begin{quote}
``If you have to post a warning sign, you've already failed at design.''
\end{quote}

\begin{center}\rule{0.5\linewidth}{0.5pt}\end{center}

\section{Physical Guarding and Interlocks}\label{physical-guarding-and-interlocks}

Physical barriers are the most intuitive form of protection. They create a ``hard'' separation between the operator and moving parts, ensuring that humans cannot enter the danger zone during machine operation.

\begin{figure}

{\centering \includegraphics{introduction_files/figure-latex/guarding-types-viz-1} 

}

\caption{Types of Physical Guarding}\label{fig:guarding-types-viz}
\end{figure}

\subsection{Fixed Guards}\label{fixed-guards}

\textbf{Fixed guards} are permanent barriers that require tools to remove. They are the simplest and most reliable form of protection.

\label{tab:fixed-guard-table}Fixed Guards: Key Characteristics

Characteristic

Description

Construction

Wire mesh, polycarbonate sheets, or solid metal panels

Removal

Requires tools (screws, bolts) to remove

Best Application

Areas that need infrequent access for maintenance only

Advantages

Simple, reliable, low cost, no moving parts to fail

Disadvantages

Impedes visibility, slows maintenance, can be removed and not replaced

\subsection{Interlocked Guards}\label{interlocked-guards}

\textbf{Interlocked guards} are access points equipped with sensors. When the guard is opened, the safety controller initiates a stop.

\begin{figure}

{\centering \includegraphics{introduction_files/figure-latex/interlock-diagram-1} 

}

\caption{Interlocked Guard Operation Principle}\label{fig:interlock-diagram}
\end{figure}

\subsection{Stop Categories}\label{stop-categories}

When an interlock is triggered, the machine must stop. The \textbf{stop category} determines how this happens:

\label{tab:stop-categories}Machine Stop Categories (IEC 60204-1)

Category

Type

Method

Application

Example

\textbf{Category 0}

Uncontrolled Stop

Immediate removal of power to actuators

Emergency stops, dangerous machines

Press brake E-stop

\textbf{Category 1}

Controlled Stop

Controlled deceleration, then power removed

High-inertia machines, robots

Industrial robot guard open

\textbf{Category 2}

Controlled Stop

Controlled stop with power maintained

Process that cannot be interrupted abruptly

CNC spindle during tool change

\subsection{Trapped Key Systems (Lock-Out/Tag-Out)}\label{trapped-key-systems-lock-outtag-out}

\textbf{Trapped key systems} use a mechanical sequence to ensure that a machine cannot be energized unless all access keys are returned to the control panel.

\begin{figure}

{\centering \includegraphics{introduction_files/figure-latex/loto-sequence-1} 

}

\caption{Trapped Key System Sequence}\label{fig:loto-sequence}
\end{figure}

Case Study: The Fatal Bypass

\textbf{Incident:} A maintenance worker was killed when a robot arm unexpectedly activated while he was inside a robotic cell.

\textbf{What Happened:}
- The guard interlock had been bypassed with a ``defeat device'' (a magnet taped to fool the sensor)
- Workers had bypassed the interlock because it slowed down troubleshooting
- The bypass had been in place for months without management knowledge

\textbf{Root Causes:}
1. Interlock caused inconvenience during frequent troubleshooting
2. No monitoring system to detect bypasses
3. Culture that tolerated ``workarounds''

\textbf{Lessons Learned:}
- Design interlocks that don't impede normal operations
- Use tamper-evident or tamper-resistant devices
- Regular safety audits to detect bypasses
- Zero-tolerance policy for defeating safety devices

\begin{quote}
\textbf{Never bypass a safety device, even temporarily.}
\end{quote}

\begin{center}\rule{0.5\linewidth}{0.5pt}\end{center}

\section{Presence-Sensing Devices}\label{presence-sensing-devices}

Presence-sensing devices allow for a \textbf{``fence-less'' environment}, improving visibility and ergonomics while maintaining high safety levels. They detect when a person enters a hazardous zone and trigger a machine stop.

\begin{figure}

{\centering \includegraphics{introduction_files/figure-latex/presence-sensing-overview-1} 

}

\caption{Types of Presence-Sensing Devices}\label{fig:presence-sensing-overview}
\end{figure}

\subsection{Light Curtains}\label{light-curtains}

\textbf{Light curtains} use a transmitter and receiver to create an array of infrared beams. If any beam is broken, the safety circuit is interrupted.

\begin{figure}

{\centering \includegraphics{introduction_files/figure-latex/light-curtain-diagram-1} 

}

\caption{Light Curtain Operating Principle}\label{fig:light-curtain-diagram}
\end{figure}

\label{tab:light-curtain-specs}Light Curtain Specifications by Application

Parameter

Finger Detection (14mm)

Hand Detection (30mm)

Body Detection (40mm+)

Resolution

14mm beam spacing

30mm beam spacing

40mm+ beam spacing

Height

150-1800mm

300-1800mm

900-2000mm

Range

0.5-20m

0.5-20m

0.5-30m

Response Time

5-15ms

10-20ms

15-30ms

Application

Point of operation guarding

Perimeter guarding

Area access detection

\subsection{Laser Area Scanners}\label{laser-area-scanners}

\textbf{Laser area scanners} use LiDAR technology to monitor a 2D plane. They are highly flexible and allow for programmable zones.

\begin{figure}

{\centering \includegraphics{introduction_files/figure-latex/area-scanner-diagram-1} 

}

\caption{Laser Area Scanner Zones}\label{fig:area-scanner-diagram}
\end{figure}

\label{tab:scanner-comparison}Light Curtains vs.~Area Scanners

Feature

Light Curtains

Area Scanners

Detection Type

Vertical/Linear Barrier

Horizontal/Area Coverage

Flexibility

Fixed installation

Programmable zones

Zone Configuration

Single detection plane

Multiple warning + safety zones

Cost

Generally lower (\$1,000-5,000)

Higher (\$3,000-10,000)

Best Use Case

Clear entry/exit points, press brakes

Large open floors, AGV paths, flexible cells

Environmental Limits

Dust/debris can cause false trips

More tolerant of dust, outdoor rated available

\subsection{Pressure-Sensitive Safety Mats}\label{pressure-sensitive-safety-mats}

\textbf{Pressure-sensitive mats} detect the presence of a person standing on them. They are used primarily in legacy systems or specific zones where optical sensors may be obscured.

\begin{figure}

{\centering \includegraphics{introduction_files/figure-latex/pressure-mat-diagram-1} 

}

\caption{Pressure-Sensitive Safety Mat Construction}\label{fig:pressure-mat-diagram}
\end{figure}

When to Use Pressure Mats vs.~Optical Sensors

\textbf{Use Pressure Mats When:}
- Environment has heavy dust, smoke, or steam (blinds optical sensors)
- Detection area is well-defined and doesn't change
- Workers need to stand in a specific location
- Budget is limited

\textbf{Use Optical Sensors When:}
- Flexibility in zone configuration is needed
- Environment is relatively clean
- Detection area is large or needs to change
- Higher integrity level is required

\textbf{Hybrid Approach:} Many modern systems use both --- area scanners for perimeter protection and pressure mats for specific danger zones near the machine.

\begin{center}\rule{0.5\linewidth}{0.5pt}\end{center}

\section{Control-Actuated Safety Devices}\label{control-actuated-safety-devices}

These devices ensure that the operator's body---specifically their hands---is in a predetermined \textbf{safe location} before the machine can cycle.

\subsection{Two-Hand Controls}\label{two-hand-controls}

\textbf{Two-hand controls} are common in pressing or stamping operations. To initiate a cycle, the operator must press two buttons simultaneously.

\begin{figure}

{\centering \includegraphics{introduction_files/figure-latex/two-hand-diagram-1} 

}

\caption{Two-Hand Control System}\label{fig:two-hand-diagram}
\end{figure}

\begin{quote}
\textbf{Anti-Tie-Down Requirement:} Modern systems require both buttons to be pressed within \textbf{0.5 seconds} of each other. This prevents operators from taping one button down to work with one hand.
\end{quote}

\label{tab:two-hand-requirements}Two-Hand Control Requirements

Requirement

Standard Value

Purpose

Synchronous activation

Within 0.5 seconds

Prevents one-hand operation

Continuous pressure

Must be held throughout cycle

Ensures hands stay on controls

Release before restart

Both must be released before next cycle

Prevents automatic cycling

Button spacing

Minimum 260mm (10 inches) apart

Cannot press both with one hand/arm

Guards around buttons

Prevent accidental activation

Prevents hip, elbow, or object activation

\subsection{Enable/Deadman Switches}\label{enabledeadman-switches}

\textbf{Enable switches} (also called deadman switches) are used during \textbf{Teach Mode} for robotics. The operator must hold a three-position trigger in the \textbf{middle} position.

\begin{figure}

{\centering \includegraphics{introduction_files/figure-latex/enable-switch-diagram-1} 

}

\caption{Three-Position Enable Switch Operation}\label{fig:enable-switch-diagram}
\end{figure}

Why Three Positions?

The three-position design accounts for \textbf{two natural human reactions} to danger:

\begin{enumerate}
\def\labelenumi{\arabic{enumi}.}
\tightlist
\item
  \textbf{Panic (Release):} When frightened, people often release what they're holding --- robot stops.
\item
  \textbf{Startle (Squeeze):} When startled, people often grip tighter --- robot stops.
\end{enumerate}

Only a deliberate, controlled \textbf{middle hold} allows the robot to move. This is fundamental to safe teach pendant operation.

\begin{center}\rule{0.5\linewidth}{0.5pt}\end{center}

\section{Collaborative Robotics (Cobots)}\label{collaborative-robotics-cobots}

\textbf{Collaborative robots (cobots)} represent a shift from ``separation'' to ``collaboration.'' They rely on sophisticated internal sensors rather than external cages, allowing humans and robots to work in the same space.

\begin{figure}

{\centering \includegraphics{introduction_files/figure-latex/cobot-comparison-guarding-1} 

}

\caption{Traditional Robot vs. Collaborative Robot}\label{fig:cobot-comparison-guarding}
\end{figure}

\subsection{Collaborative Operation Methods}\label{collaborative-operation-methods}

ISO 10218 and ISO/TS 15066 define \textbf{four methods} for safe collaborative operation:

\label{tab:cobot-methods}Four Methods of Collaborative Robot Operation

Method

How It Works

Sensors Used

Application

\textbf{Safety-Rated Monitored Stop}

Robot stops when human enters workspace, resumes when clear

Area scanners, light curtains

Large robot needing occasional access

\textbf{Hand Guiding}

Operator physically guides robot through motions

Force/torque sensors in joints

Teaching, positioning

\textbf{Speed and Separation Monitoring}

Robot slows or stops based on human proximity

Area scanners, radar, cameras

Shared workspace with varying proximity

\textbf{Power and Force Limiting}

Robot limits force/power to safe levels on contact

Joint torque sensors

Direct contact applications

\subsection{Power and Force Limiting (PFL)}\label{power-and-force-limiting-pfl}

The most common cobot safety method is \textbf{Power and Force Limiting}. The robot detects a change in torque in its joints. If it bumps into a human, it stops before a regulated amount of force is exceeded.

\label{tab:pfl-limits}ISO/TS 15066 Biomechanical Limits (Quasi-static contact)

Body Region

Maximum Pressure (N/cm²)

Maximum Force (N)

Risk Level

Skull/Forehead

130

130

Critical

Face

65

65

Critical

Neck

145

150

Critical

Back/Shoulders

210

210

Moderate

Chest

140

140

Moderate

Abdomen

110

110

Moderate

Hand/Fingers

300

140

Lower

Discussion: Are Cobots Always Safe?

\textbf{Common Misconception:} ``Cobots are safe, so we don't need to do a risk assessment.''

\textbf{Reality:} Cobots are \emph{potentially} safe, but a \textbf{risk assessment is always required}.

\textbf{Factors that can make a cobot unsafe:}

\begin{enumerate}
\def\labelenumi{\arabic{enumi}.}
\tightlist
\item
  \textbf{End-of-arm tooling} --- A soft gripper is different from a sharp blade
\item
  \textbf{Workpiece} --- Carrying a heavy or sharp object increases risk
\item
  \textbf{Speed} --- Even force-limited robots can cause injury at high speed
\item
  \textbf{Application} --- Some tasks bring humans closer to hazards
\item
  \textbf{Environment} --- Slippery floors, confined spaces, etc.
\end{enumerate}

\begin{quote}
\textbf{Every collaborative application requires a documented risk assessment per ISO 12100.}
\end{quote}

\begin{center}\rule{0.5\linewidth}{0.5pt}\end{center}

\section{Safety Distance Calculation}\label{safety-distance-calculation}

A safety device cannot be placed arbitrarily. It must be far enough away that the machine comes to a \textbf{complete halt} before the human can reach the hazard.

\subsection{The Safety Distance Formula}\label{the-safety-distance-formula}

The minimum safety distance \(S\) is calculated as:

\[S = (K \times T) + C\]

Where:

\begin{itemize}
\tightlist
\item
  \textbf{\(K\)}: The hand/body approach speed (standardized values)
\item
  \textbf{\(T\)}: The total stopping time of the system (device response + machine stop time)
\item
  \textbf{\(C\)}: The ``penetration depth'' (how far a hand can reach through before being detected)
\end{itemize}

\begin{figure}

{\centering \includegraphics{introduction_files/figure-latex/safety-distance-viz-1} 

}

\caption{Safety Distance Components}\label{fig:safety-distance-viz}
\end{figure}

\subsection{Standard Values for K}\label{standard-values-for-k}

\label{tab:k-values}Standard Approach Speed Values (ISO 13855)

Approach Type

K Value (mm/s)

When to Use

Hand/Arm Approach (fast)

2000

Close approach to point of operation

Hand/Arm Approach (normal)

1600

General approach to machines

Walking Approach

1600

Area access detection

Standing Reach

0 (static)

Worker already in position

\subsection{Penetration Depth (C)}\label{penetration-depth-c}

The penetration depth depends on the \textbf{resolution} of the sensing device:

\label{tab:c-values}Penetration Depth Values Based on Resolution

Resolution (mm)

C Value (mm)

Detection Type

≤14

0

Finger detection

\textgreater14 to ≤20

80

Finger/Hand

\textgreater20 to ≤30

130

Hand detection

\textgreater30 to ≤40

240

Hand/Arm

\textgreater40

850

Body detection

\subsection{Example Calculation}\label{example-calculation}

\begin{Shaded}
\begin{Highlighting}[]
\CommentTok{\# Given parameters}
\NormalTok{K }\OtherTok{\textless{}{-}} \DecValTok{2000}        \CommentTok{\# Approach speed (mm/s) {-} fast hand approach}
\NormalTok{T\_device }\OtherTok{\textless{}{-}} \FloatTok{0.020}  \CommentTok{\# Light curtain response time (20 ms = 0.020 s)}
\NormalTok{T\_machine }\OtherTok{\textless{}{-}} \FloatTok{0.150} \CommentTok{\# Machine stopping time (150 ms = 0.150 s)}
\NormalTok{C }\OtherTok{\textless{}{-}} \DecValTok{0}           \CommentTok{\# Penetration depth for 14mm resolution}

\CommentTok{\# Calculate total stopping time}
\NormalTok{T\_total }\OtherTok{\textless{}{-}}\NormalTok{ T\_device }\SpecialCharTok{+}\NormalTok{ T\_machine}
\FunctionTok{cat}\NormalTok{(}\StringTok{"Total stopping time (T):"}\NormalTok{, T\_total }\SpecialCharTok{*} \DecValTok{1000}\NormalTok{, }\StringTok{"ms}\SpecialCharTok{\textbackslash{}n}\StringTok{"}\NormalTok{)}
\end{Highlighting}
\end{Shaded}

\begin{verbatim}
## Total stopping time (T): 170 ms
\end{verbatim}

\begin{Shaded}
\begin{Highlighting}[]
\CommentTok{\# Calculate minimum safety distance}
\NormalTok{S }\OtherTok{\textless{}{-}}\NormalTok{ (K }\SpecialCharTok{*}\NormalTok{ T\_total) }\SpecialCharTok{+}\NormalTok{ C}
\FunctionTok{cat}\NormalTok{(}\StringTok{"Minimum safety distance (S):"}\NormalTok{, S, }\StringTok{"mm}\SpecialCharTok{\textbackslash{}n}\StringTok{"}\NormalTok{)}
\end{Highlighting}
\end{Shaded}

\begin{verbatim}
## Minimum safety distance (S): 340 mm
\end{verbatim}

\begin{Shaded}
\begin{Highlighting}[]
\FunctionTok{cat}\NormalTok{(}\StringTok{"Minimum safety distance (S):"}\NormalTok{, S }\SpecialCharTok{/} \FloatTok{25.4}\NormalTok{, }\StringTok{"inches}\SpecialCharTok{\textbackslash{}n}\StringTok{"}\NormalTok{)}
\end{Highlighting}
\end{Shaded}

\begin{verbatim}
## Minimum safety distance (S): 13.38583 inches
\end{verbatim}

\begin{Shaded}
\begin{Highlighting}[]
\CommentTok{\# Interpretation}
\FunctionTok{cat}\NormalTok{(}\StringTok{"}\SpecialCharTok{\textbackslash{}n}\StringTok{{-}{-}{-} Interpretation {-}{-}{-}}\SpecialCharTok{\textbackslash{}n}\StringTok{"}\NormalTok{)}
\end{Highlighting}
\end{Shaded}

\begin{verbatim}
## 
## --- Interpretation ---
\end{verbatim}

\begin{Shaded}
\begin{Highlighting}[]
\FunctionTok{cat}\NormalTok{(}\StringTok{"The light curtain must be at least"}\NormalTok{, S, }\StringTok{"mm from the hazard.}\SpecialCharTok{\textbackslash{}n}\StringTok{"}\NormalTok{)}
\end{Highlighting}
\end{Shaded}

\begin{verbatim}
## The light curtain must be at least 340 mm from the hazard.
\end{verbatim}

\begin{Shaded}
\begin{Highlighting}[]
\FunctionTok{cat}\NormalTok{(}\StringTok{"This ensures the machine stops before a person can reach the danger zone.}\SpecialCharTok{\textbackslash{}n}\StringTok{"}\NormalTok{)}
\end{Highlighting}
\end{Shaded}

\begin{verbatim}
## This ensures the machine stops before a person can reach the danger zone.
\end{verbatim}

Practice Problem: Calculate Safety Distance

\textbf{Given:}
- Light curtain resolution: 30mm (hand detection)
- Light curtain response time: 15ms
- Machine stopping time: 200ms
- Application: Operator hand approach

\textbf{Calculate the minimum safety distance.}

\textbf{Solution:}

\begin{verbatim}
K = 1600 mm/s (normal hand approach)
T = 0.015 + 0.200 = 0.215 seconds
C = 130 mm (for 30mm resolution)

S = (K × T) + C
S = (1600 × 0.215) + 130
S = 344 + 130
S = 474 mm (approximately 18.7 inches)
\end{verbatim}

The light curtain must be installed at least \textbf{474mm} from the nearest hazard point.

\begin{center}\rule{0.5\linewidth}{0.5pt}\end{center}

\section{Emergency Stop (E-Stop) Systems}\label{emergency-stop-e-stop-systems}

The \textbf{Emergency Stop} is the ``last resort'' --- it is not a safety device but a \textbf{complementary protective measure}. It must be available to stop the machine when all other measures have failed.

\begin{figure}

{\centering \includegraphics{introduction_files/figure-latex/estop-design-1} 

}

\caption{E-Stop Design Requirements}\label{fig:estop-design}
\end{figure}

\label{tab:estop-requirements}E-Stop Requirements per IEC 60204-1

Requirement

Standard

Color

Red (RAL 3000 or equivalent)

Shape

Mushroom head (palm or fist operated)

Background

Yellow (high contrast)

Operation

Self-latching (stays activated when pressed)

Reset

Manual reset required (twist, pull, or key)

Accessibility

Within easy reach of all operators

Wiring

Normally closed contacts (fail-safe)

Function

Category 0 or Category 1 stop

\subsection{E-Stop Pull Cords}\label{e-stop-pull-cords}

For long conveyors or machines where an operator cannot quickly reach a fixed button, \textbf{pull cords} provide continuous E-Stop coverage.

\begin{figure}

{\centering \includegraphics{introduction_files/figure-latex/pull-cord-diagram-1} 

}

\caption{E-Stop Pull Cord System}\label{fig:pull-cord-diagram}
\end{figure}

Common E-Stop Mistakes

\textbf{Mistake 1: Using E-Stop for normal stopping}
- E-Stops are for emergencies only
- Normal stops should use a separate control
- Frequent E-Stop use causes wear and desensitizes operators

\textbf{Mistake 2: E-Stop that doesn't stop everything}
- All hazardous motion must stop
- Auxiliary equipment (conveyors, feeders) often forgotten
- Review entire system, not just the primary machine

\textbf{Mistake 3: Automatic restart after E-Stop reset}
- Machine must NOT automatically restart when E-Stop is released
- Separate ``Start'' action required
- This prevents unexpected motion after reset

\textbf{Mistake 4: Hidden or blocked E-Stops}
- E-Stops must be visible and accessible
- Guard placement sometimes blocks access
- Regular audits needed to ensure accessibility

\begin{center}\rule{0.5\linewidth}{0.5pt}\end{center}

\section{Functional Safety Standards}\label{functional-safety-standards}

Modern safety systems must meet specific \textbf{performance levels} to ensure reliability. Two main frameworks exist:

\subsection{Safety Integrity Levels (SIL) - IEC 62061}\label{safety-integrity-levels-sil---iec-62061}

\label{tab:sil-levels}Safety Integrity Levels (PFH = Probability of Dangerous Failure per Hour)

SIL Level

PFH Range

Approximate Meaning

Typical Application

SIL 1

≥10⁻⁶ to \textless10⁻⁵

Dangerous failure less than once per 11 years

Low-risk tasks

SIL 2

≥10⁻⁷ to \textless10⁻⁶

Dangerous failure less than once per 114 years

Most industrial machinery

SIL 3

≥10⁻⁸ to \textless10⁻⁷

Dangerous failure less than once per 1,140 years

High-risk processes

\subsection{Performance Levels (PL) - ISO 13849-1}\label{performance-levels-pl---iso-13849-1}

\label{tab:pl-levels}Performance Levels per ISO 13849-1

Performance Level

PFH Range

Risk Reduction

Equivalent SIL

PL a

≥10⁻⁵ to \textless10⁻⁴

Lowest

\textless{} SIL 1

PL b

≥3×10⁻⁶ to \textless10⁻⁵

Low

SIL 1

PL c

≥10⁻⁶ to \textless3×10⁻⁶

Medium

SIL 1

PL d

≥10⁻⁷ to \textless10⁻⁶

High

SIL 2

PL e

≥10⁻⁸ to \textless10⁻⁷

Highest

SIL 3

\begin{figure}

{\centering \includegraphics{introduction_files/figure-latex/sil-pl-relationship-1} 

}

\caption{Relationship Between SIL and PL}\label{fig:sil-pl-relationship}
\end{figure}

Which Standard to Use: SIL or PL?

\textbf{Use ISO 13849-1 (Performance Levels) when:}
- Designing safety functions for standard machinery
- Using off-the-shelf safety components
- Simpler systems with well-defined safety functions

\textbf{Use IEC 62061 (SIL) when:}
- Complex programmable systems (safety PLCs)
- Process industries
- When customer or regulation specifies SIL

\textbf{Note:} Many modern systems use both standards, as they are harmonized and can be used together. The choice often depends on industry practice and customer requirements.

\begin{center}\rule{0.5\linewidth}{0.5pt}\end{center}

\section{Risk Assessment Process}\label{risk-assessment-process}

Before selecting any safety device, a \textbf{risk assessment} must be performed. This is required by ISO 12100.

\begin{figure}

{\centering \includegraphics{introduction_files/figure-latex/risk-assessment-process-1} 

}

\caption{Risk Assessment Process (ISO 12100)}\label{fig:risk-assessment-process}
\end{figure}

\subsection{Risk Estimation}\label{risk-estimation}

Risk is typically estimated using three factors:

\begin{figure}

{\centering \includegraphics{introduction_files/figure-latex/risk-matrix-1} 

}

\caption{Risk Estimation Matrix}\label{fig:risk-matrix}
\end{figure}

\begin{center}\rule{0.5\linewidth}{0.5pt}\end{center}

\section{Summary}\label{summary-4}

\label{tab:summary-table-ch6}Machine Guarding and Safety: Key Concepts Summary

Topic

Key Points

Critical Remember

\textbf{Hierarchy of Controls}

Elimination \textgreater{} Substitution \textgreater{} Engineering \textgreater{} Administrative \textgreater{} PPE

Always start at the top of the hierarchy

\textbf{Physical Guarding}

Fixed guards, interlocked gates, trapped key systems

Stop category (0, 1, 2) determines how machine stops

\textbf{Presence Sensing}

Light curtains, area scanners, pressure mats

Resolution determines detection capability

\textbf{Control-Actuated Devices}

Two-hand controls, enable/deadman switches

Anti-tie-down prevents one-hand bypass

\textbf{Collaborative Robots}

PFL, speed/separation monitoring, hand guiding

Risk assessment required for every cobot application

\textbf{Safety Distance}

S = (K × T) + C formula for device placement

Include device response AND machine stop time

\textbf{E-Stop Systems}

Red mushroom on yellow, self-latching, manual reset

E-Stop is last resort, not normal operation

\textbf{Functional Safety}

SIL (IEC 62061) and PL (ISO 13849-1) standards

Required performance level from risk assessment

\begin{center}\rule{0.5\linewidth}{0.5pt}\end{center}

\section{Review Questions}\label{review-questions-5}

Question 1: What are the five levels of the Hierarchy of Controls, and why is PPE the least effective?

The five levels are (from most to least effective):

\begin{enumerate}
\def\labelenumi{\arabic{enumi}.}
\tightlist
\item
  \textbf{Elimination} --- Physically remove the hazard
\item
  \textbf{Substitution} --- Replace with something less hazardous
\item
  \textbf{Engineering Controls} --- Isolate people from the hazard
\item
  \textbf{Administrative Controls} --- Change how people work
\item
  \textbf{PPE} --- Protect the worker with equipment
\end{enumerate}

\textbf{PPE is least effective because:}
- The hazard still exists
- It depends on human compliance
- Equipment can fail or be worn incorrectly
- It's the last line of defense

Question 2: A light curtain has a response time of 12ms and resolution of 14mm. The machine takes 180ms to stop. Calculate the minimum safety distance for hand approach.

\textbf{Given:}
- K = 2000 mm/s (hand approach, use higher value for safety)
- T = 0.012 + 0.180 = 0.192 seconds
- C = 0 mm (14mm resolution)

\textbf{Calculation:}
\[S = (K \times T) + C\]
\[S = (2000 \times 0.192) + 0\]
\[S = 384 \text{ mm}\]

The light curtain must be installed at least \textbf{384mm} (approximately 15 inches) from the hazard.

Question 3: Explain why a three-position enable switch stops the robot in both position 1 (released) and position 3 (fully pressed).

The three-position design accounts for two natural human reactions to danger:

\textbf{Position 1 (Released):}
- When frightened, people often release what they're holding (panic response)
- Robot stops immediately

\textbf{Position 3 (Fully Pressed):}
- When startled, people often grip tighter (startle response)
- Robot stops immediately

\textbf{Position 2 (Middle):}
- Only a deliberate, controlled hold allows operation
- Requires conscious effort to maintain
- This ensures the operator is alert and in control

Question 4: What are the four methods of collaborative robot operation according to ISO 10218?

\begin{enumerate}
\def\labelenumi{\arabic{enumi}.}
\tightlist
\item
  \textbf{Safety-Rated Monitored Stop}

  \begin{itemize}
  \tightlist
  \item
    Robot stops when human enters workspace
  \item
    Resumes when human leaves
  \item
    Uses area scanners or light curtains
  \end{itemize}
\item
  \textbf{Hand Guiding}

  \begin{itemize}
  \tightlist
  \item
    Operator physically guides robot
  \item
    Force/torque sensors enable safe teaching
  \end{itemize}
\item
  \textbf{Speed and Separation Monitoring}

  \begin{itemize}
  \tightlist
  \item
    Robot speed varies based on human proximity
  \item
    Uses external sensors to track human position
  \end{itemize}
\item
  \textbf{Power and Force Limiting (PFL)}

  \begin{itemize}
  \tightlist
  \item
    Robot limits contact force to safe levels
  \item
    Built-in joint torque sensors
  \item
    Most common cobot method
  \end{itemize}
\end{enumerate}

Question 5: What are the requirements for an E-Stop button according to IEC 60204-1?

\textbf{Physical Requirements:}
- \textbf{Color:} Red
- \textbf{Shape:} Mushroom head (palm or fist operated)
- \textbf{Background:} Yellow (high contrast)

\textbf{Functional Requirements:}
- \textbf{Self-latching:} Stays activated when pressed
- \textbf{Manual reset:} Requires deliberate action to release
- \textbf{Accessibility:} Within easy reach of all operators
- \textbf{Wiring:} Normally closed contacts (fail-safe)
- \textbf{Function:} Category 0 or Category 1 stop
- \textbf{No automatic restart:} Separate start action required after reset

\begin{center}\rule{0.5\linewidth}{0.5pt}\end{center}

\section{References}\label{references-5}

\begin{itemize}
\tightlist
\item
  ISO 12100:2010 --- Safety of machinery --- General principles for design --- Risk assessment and risk reduction
\item
  ISO 13849-1:2015 --- Safety of machinery --- Safety-related parts of control systems
\item
  ISO 13855:2010 --- Safety of machinery --- Positioning of safeguards with respect to approach speeds
\item
  ISO 10218-1/2:2011 --- Robots and robotic devices --- Safety requirements for industrial robots
\item
  ISO/TS 15066:2016 --- Robots and robotic devices --- Collaborative robots
\item
  IEC 60204-1:2016 --- Safety of machinery --- Electrical equipment of machines
\item
  IEC 62061:2021 --- Safety of machinery --- Functional safety of safety-related control systems
\item
  OSHA 29 CFR 1910.212 --- General requirements for all machines
\item
  ANSI/RIA TR R15.306 --- Task-based risk assessment methodology
\end{itemize}

\begin{center}\rule{0.5\linewidth}{0.5pt}\end{center}

\chapter{Ergonomics in Automated Manufacturing}\label{ergonomics-in-automated-manufacturing}

\begin{center}\rule{0.5\linewidth}{0.5pt}\end{center}

\section{Learning Objectives}\label{learning-objectives-6}

By the end of this chapter, you will be able to:

\begin{itemize}
\tightlist
\item
  Define ergonomics and explain its importance in manufacturing
\item
  Identify the seven major ergonomic risk factors
\item
  Recognize common musculoskeletal disorders (MSDs) and their causes
\item
  Apply ergonomic design principles to workstation design
\item
  Conduct a REBA (Rapid Entire Body Assessment) analysis
\item
  Understand the role of AI and automation in ergonomic assessment
\item
  Design workstations that optimize human performance and safety
\end{itemize}

\begin{quote}
``Ergonomics is the science of fitting the job to the worker, not forcing the worker to fit the job.''
--- International Ergonomics Association
\end{quote}

\begin{center}\rule{0.5\linewidth}{0.5pt}\end{center}

\section{What is Ergonomics?}\label{what-is-ergonomics}

\textbf{Ergonomics} (from Greek \emph{ergon} = work, \emph{nomos} = laws) is the scientific discipline focused on understanding the interactions between humans and other elements of a system, with the goal of optimizing human well-being and overall system performance.

\begin{figure}

{\centering \includegraphics{introduction_files/figure-latex/ergonomics-definition-1} 

}

\caption{The Three Domains of Ergonomics}\label{fig:ergonomics-definition}
\end{figure}

\label{tab:ergonomics-domains-table}The Three Domains of Ergonomics

Domain

Focus

Key Topics

Manufacturing Example

\textbf{Physical Ergonomics}

Human body's response to physical stress

Posture, material handling, repetitive motion, workplace layout, safety

Designing assembly workstations to reduce reaching

\textbf{Cognitive Ergonomics}

Mental processes and human-system interaction

Mental workload, decision-making, human-computer interaction, training

Designing HMI screens for quick comprehension

\textbf{Organizational Ergonomics}

Optimization of sociotechnical systems

Communication, teamwork, work schedules, quality management

Implementing job rotation to reduce fatigue

\subsection{Why Ergonomics Matters in Manufacturing}\label{why-ergonomics-matters-in-manufacturing}

\begin{figure}

{\centering \includegraphics{introduction_files/figure-latex/why-ergonomics-1} 

}

\caption{The Business Case for Ergonomics}\label{fig:why-ergonomics}
\end{figure}

The Cost of Ignoring Ergonomics

\textbf{Direct Costs:}
- Workers' compensation claims
- Medical expenses
- Legal fees
- Increased insurance premiums

\textbf{Indirect Costs (often 4-10x direct costs):}
- Lost productivity
- Training replacement workers
- Reduced quality
- Overtime to cover absent workers
- Administrative time
- Low morale

\textbf{Statistics:}
- MSDs account for \textbf{33\%} of all workplace injuries
- Average MSD claim costs \textbf{\$15,000-\$20,000}
- Lost workdays due to MSDs average \textbf{12 days} per case
- Carpal tunnel surgery costs approximately \textbf{\$30,000} per case (including lost time)

\begin{center}\rule{0.5\linewidth}{0.5pt}\end{center}

\section{The Seven Ergonomic Risk Factors}\label{the-seven-ergonomic-risk-factors}

The \textbf{Initial Ergonomic Risk Assessment (INERA)} identifies seven primary risk factors that contribute to musculoskeletal disorders. Understanding these factors is the first step in prevention.

\begin{figure}

{\centering \includegraphics{introduction_files/figure-latex/risk-factors-viz-1} 

}

\caption{The Seven Ergonomic Risk Factors}\label{fig:risk-factors-viz}
\end{figure}

\label{tab:risk-factors-table}The Seven Ergonomic Risk Factors Explained

\#

Risk Factor

Definition

Examples

Body Areas Affected

1

\textbf{Awkward Postures}

Positions that deviate from neutral body alignment

Bent wrist, raised arms, twisted back, kneeling

Back, neck, shoulders, wrists, knees

2

\textbf{Repetition}

Performing the same motion repeatedly

Assembly line tasks, typing, packaging

Hands, wrists, shoulders, neck

3

\textbf{Force}

Amount of physical effort required to do a task

Lifting heavy objects, pushing/pulling, gripping tools

Back, shoulders, hands, arms

4

\textbf{Static Positions}

Maintaining the same position for extended periods

Standing in one spot, holding arms overhead

Legs, back, shoulders, neck

5

\textbf{Contact Stress}

Pressure from hard surfaces or edges on body tissues

Resting wrists on desk edge, kneeling on hard floor

Hands, knees, forearms

6

\textbf{Vibration}

Oscillating movements transferred to the body

Power tools, driving forklifts, jackhammers

Hands, arms, spine

7

\textbf{Environmental Factors}

Temperature, lighting, noise affecting work performance

Cold environments, poor lighting, excessive noise

Eyes, ears, overall fatigue

\subsection{Risk Factor 1: Awkward Postures}\label{risk-factor-1-awkward-postures}

\textbf{Neutral posture} is the position where joints are naturally aligned, minimizing stress on muscles, tendons, and skeletal system. Any deviation from neutral increases injury risk.

\begin{figure}

{\centering \includegraphics{introduction_files/figure-latex/posture-zones-1} 

}

\caption{Posture Risk Zones}\label{fig:posture-zones}
\end{figure}

\label{tab:neutral-postures}Neutral vs.~Awkward Postures by Body Part

Body Part

Neutral Position

Awkward Position

Risk

\textbf{Head/Neck}

Balanced on spine, looking straight ahead

Bent forward \textgreater20°, tilted, twisted

Neck pain, headaches

\textbf{Shoulders}

Relaxed, arms at sides

Raised, reaching overhead, behind body

Shoulder impingement, rotator cuff

\textbf{Elbows}

Close to body, bent 90-120°

Fully extended or tightly bent

Tennis elbow, golfer's elbow

\textbf{Wrists}

Straight, in line with forearm

Bent up/down, twisted side-to-side

Carpal tunnel, tendinitis

\textbf{Back}

Natural S-curve maintained

Bent forward, twisted, arched

Low back pain, disc herniation

\textbf{Hips/Knees}

Thighs parallel to floor, feet flat

Squatting, kneeling, twisted

Knee injuries, hip pain

\subsection{Risk Factor 2: Repetition}\label{risk-factor-2-repetition}

\textbf{Repetitive motion} is one of the most significant risk factors in manufacturing. The risk increases with:

\begin{itemize}
\tightlist
\item
  \textbf{Frequency:} How often the motion occurs
\item
  \textbf{Duration:} How long the task is performed
\item
  \textbf{Recovery time:} Rest between repetitions
\end{itemize}

\label{tab:repetition-thresholds}Repetition Risk Thresholds

Cycle Time

Repetitions/Hour

Risk Level

Action Required

\textgreater{} 30 seconds

\textless{} 120

Low

Monitor

15-30 seconds

120-240

Moderate

Investigate and consider changes

\textless{} 15 seconds

\textgreater{} 240

High

Immediate intervention needed

\subsection{Risk Factor 3: Force}\label{risk-factor-3-force}

The amount of \textbf{physical effort} required significantly impacts injury risk. Force requirements depend on:

\begin{figure}

{\centering \includegraphics{introduction_files/figure-latex/force-factors-1} 

}

\caption{Factors Affecting Force Requirements}\label{fig:force-factors}
\end{figure}

Interactive Exercise: Identify the Risk Factors

\textbf{Scenario:} An assembly line worker performs the following task 400 times per shift:

\begin{enumerate}
\def\labelenumi{\arabic{enumi}.}
\tightlist
\item
  Reaches overhead to grab a part from a bin
\item
  Holds the part while inserting 4 screws with a power screwdriver
\item
  Places the completed assembly on a conveyor
\item
  The workstation has poor lighting and the floor is concrete
\end{enumerate}

\textbf{Identify all risk factors present:}

\textbf{Answer:}
1. \textbf{Awkward Posture} - Reaching overhead
2. \textbf{Repetition} - 400 cycles per shift
3. \textbf{Force} - Gripping parts, controlling screwdriver
4. \textbf{Static Position} - Holding part during assembly
5. \textbf{Vibration} - Power screwdriver
6. \textbf{Contact Stress} - Standing on concrete (feet)
7. \textbf{Environmental} - Poor lighting

\emph{This task has ALL SEVEN risk factors present!}

\begin{center}\rule{0.5\linewidth}{0.5pt}\end{center}

\section{Musculoskeletal Disorders (MSDs)}\label{musculoskeletal-disorders-msds}

\textbf{Musculoskeletal Disorders (MSDs)} are injuries and disorders affecting muscles, nerves, tendons, ligaments, joints, cartilage, and spinal discs. They are the most common and costly occupational health problem.

\begin{figure}

{\centering \includegraphics{introduction_files/figure-latex/msd-body-map-1} 

}

\caption{Common MSDs by Body Region}\label{fig:msd-body-map}
\end{figure}

\label{tab:msd-table}Common Musculoskeletal Disorders in Manufacturing

Disorder

Description

Causes

Symptoms

\textbf{Carpal Tunnel Syndrome}

Compression of median nerve in the wrist

Repetitive wrist motions, forceful gripping, vibration

Numbness, tingling, weakness in hand

\textbf{Tendinitis}

Inflammation of tendons from overuse

Repetitive motions, awkward postures, force

Pain, swelling, stiffness in affected area

\textbf{Low Back Pain}

Strain or injury to muscles/discs of lower back

Heavy lifting, bending, twisting, vibration

Pain, stiffness, limited mobility

\textbf{Rotator Cuff Injury}

Damage to shoulder tendons

Overhead work, repetitive shoulder motion

Pain, weakness, limited range of motion

\textbf{Epicondylitis}

Inflammation of elbow tendons (tennis/golfer's elbow)

Repetitive arm/wrist motions, forceful gripping

Pain on outside (tennis) or inside (golfer's) of elbow

\textbf{Trigger Finger}

Finger gets stuck in bent position

Repetitive gripping, hand tools

Clicking, locking of finger

\subsection{The MSD Development Cycle}\label{the-msd-development-cycle}

MSDs typically develop gradually through a predictable cycle:

\begin{figure}

{\centering \includegraphics{introduction_files/figure-latex/msd-cycle-1} 

}

\caption{The MSD Development Cycle}\label{fig:msd-cycle}
\end{figure}

Warning Signs: When to Take Action

\textbf{Early Warning Signs (Act Immediately):}
- Persistent fatigue in specific body parts
- Discomfort during or after work
- Tightness or stiffness
- Minor aching or soreness

\textbf{Serious Warning Signs (Seek Medical Attention):}
- Persistent pain that doesn't go away with rest
- Numbness or tingling
- Loss of strength or grip
- Swelling or inflammation
- Reduced range of motion
- Pain that wakes you at night

\begin{quote}
\textbf{Key Message:} Report symptoms early! The earlier the intervention, the better the outcome.
\end{quote}

\begin{center}\rule{0.5\linewidth}{0.5pt}\end{center}

\section{Preventing MSDs: Best Practices}\label{preventing-msds-best-practices}

Prevention is far more effective and less costly than treatment. Here are evidence-based strategies for MSD prevention:

\begin{figure}

{\centering \includegraphics{introduction_files/figure-latex/prevention-hierarchy-1} 

}

\caption{MSD Prevention Hierarchy}\label{fig:prevention-hierarchy}
\end{figure}

\label{tab:prevention-strategies}MSD Prevention Strategies

Strategy

Engineering Solutions

Worker Actions

\textbf{Maintain Neutral Posture}

Adjustable workstations, proper tool selection

Awareness of body position, stretching

\textbf{Reduce Repetition}

Automation, job rotation between stations

Vary tasks when possible

\textbf{Minimize Force}

Mechanical assists, better handles, lighter parts

Use tools properly, ask for help with heavy items

\textbf{Avoid Static Positions}

Sit-stand workstations, adjustable fixtures

Shift weight, change positions frequently

\textbf{Eliminate Contact Stress}

Padding, rounded edges, anti-fatigue mats

Use padding, wear appropriate PPE

\textbf{Control Vibration}

Vibration-dampening tools, isolation mounts

Limit exposure time, grip tools lightly

\textbf{Optimize Environment}

Proper lighting, climate control, noise reduction

Report issues, use available controls

\textbf{Allow Recovery}

Rest break areas, micro-break reminders

Take breaks, report fatigue

\subsection{Proper Lifting Technique}\label{proper-lifting-technique}

Improper lifting is a leading cause of back injuries. The NIOSH lifting equation provides guidelines, but proper technique is fundamental.

\begin{figure}

{\centering \includegraphics{introduction_files/figure-latex/lifting-technique-1} 

}

\caption{Proper Lifting Technique}\label{fig:lifting-technique}
\end{figure}

\begin{center}\rule{0.5\linewidth}{0.5pt}\end{center}

\section{Ergonomic Design Principles}\label{ergonomic-design-principles}

Effective ergonomic design follows established principles that account for human capabilities and limitations.

\begin{figure}

{\centering \includegraphics{introduction_files/figure-latex/design-principles-1} 

}

\caption{Ergonomic Design Principles}\label{fig:design-principles}
\end{figure}

\label{tab:principles-detail}Ergonomic Design Principles in Detail

Principle

Description

Implementation

\textbf{1. Neutral Posture}

Design to promote natural, balanced body positions

Adjustable workstations, proper tool angles, monitor height

\textbf{2. Accommodate Anthropometry}

Accommodate different body sizes (5th-95th percentile)

Adjustable equipment, multiple sizes available, reach considerations

\textbf{3. Reduce Excessive Force}

Minimize force requirements for all tasks

Mechanical assists, better handles, reduce weight, improve grip

\textbf{4. Reduce Excessive Motion}

Reduce extreme motions, frequencies, and durations

Bring work closer, eliminate unnecessary steps, use fixtures

\textbf{5. Optimize Work Height}

Match work surface height to task requirements

Precision work higher, heavy work lower, adjustable surfaces

\textbf{6. Minimize Static Load}

Avoid holding same position for extended periods

Support arms, provide seating options, allow movement

\textbf{7. Environmental Conditions}

Optimize lighting, temperature, noise, air quality

Task lighting, climate control, noise barriers, ventilation

\textbf{8. Cognitive Support}

Design interfaces for easy understanding and use

Clear displays, logical controls, consistent feedback

\subsection{Anthropometry: Designing for Human Dimensions}\label{anthropometry-designing-for-human-dimensions}

\textbf{Anthropometry} is the scientific study of human body measurements. Workstations must accommodate the range of workers who will use them.

\begin{figure}

{\centering \includegraphics{introduction_files/figure-latex/anthropometry-viz-1} 

}

\caption{Anthropometric Design Considerations}\label{fig:anthropometry-viz}
\end{figure}

\label{tab:anthropometry-table}Key Anthropometric Measurements for Workstation Design

Measurement

5th \% Female (cm)

50th \% Combined (cm)

95th \% Male (cm)

Design Application

Stature (Standing Height)

150

170

188

Overhead clearance, door height

Eye Height (Standing)

138

158

176

Display/sign height

Shoulder Height

121

140

158

Shelf maximum height

Elbow Height

93

104

117

Work surface height (standing)

Knuckle Height

64

74

85

Handle height for pulling

Sitting Height

79

87

97

Overhead clearance (seated)

Seated Eye Height

68

76

85

Monitor height (seated)

Seated Elbow Height

18

24

31

Armrest, work surface height

Thigh Clearance

12

15

19

Under-desk clearance

Forward Reach

64

73

84

Maximum reach distance

\subsection{Work Surface Height Guidelines}\label{work-surface-height-guidelines}

\begin{figure}

{\centering \includegraphics{introduction_files/figure-latex/work-height-1} 

}

\caption{Recommended Work Surface Heights}\label{fig:work-height}
\end{figure}

Discussion: Why Different Heights for Different Tasks?

\textbf{Precision Work (100-110 cm):}
- Requires close visual inspection
- Hands need to be at or above eye level
- Examples: Electronics assembly, inspection, watchmaking

\textbf{Light Assembly (85-95 cm):}
- General manipulation tasks
- Hands at about elbow height
- Examples: Packaging, general assembly, sorting

\textbf{Heavy Work (70-85 cm):}
- Requires pushing, pressing, or lifting
- Lower height allows use of body weight
- Examples: Packing boxes, pressing operations, heavy assembly

\textbf{Key Principle:} The work surface height should allow the worker to maintain \textbf{neutral posture} while performing the specific task.

\begin{center}\rule{0.5\linewidth}{0.5pt}\end{center}

\section{Ergonomic Assessment Tools}\label{ergonomic-assessment-tools}

Several standardized tools help assess ergonomic risk and guide interventions. Understanding these tools is essential for the process engineer.

\label{tab:assessment-tools}Common Ergonomic Assessment Tools

Tool

Full Name

Focus Area

Best Used For

\textbf{RULA}

Rapid Upper Limb Assessment

Upper body postures

Sedentary/computer work, assembly

\textbf{REBA}

Rapid Entire Body Assessment

Whole body postures

Manufacturing, healthcare, varied postures

\textbf{NIOSH Lifting Equation}

NIOSH Lifting Equation

Manual lifting tasks

Lifting, lowering, carrying tasks

\textbf{OCRA}

Occupational Repetitive Action

Repetitive upper limb tasks

Assembly lines, repetitive manufacturing

\textbf{EAWS}

Ergonomic Assessment Worksheet

Comprehensive workstation

Complete workstation analysis

\textbf{Snook Tables}

Snook Psychophysical Tables

Manual material handling

Push, pull, lift, lower, carry tasks

\subsection{REBA: Rapid Entire Body Assessment}\label{reba-rapid-entire-body-assessment}

\textbf{REBA} is one of the most widely used ergonomic assessment tools in manufacturing. It evaluates the whole body posture and provides a risk score.

\begin{figure}

{\centering \includegraphics{introduction_files/figure-latex/reba-overview-1} 

}

\caption{REBA Assessment Framework}\label{fig:reba-overview}
\end{figure}

\subsection{REBA Scoring Tables}\label{reba-scoring-tables}

\textbf{Trunk Posture Scoring:}

\label{tab:reba-trunk}REBA Trunk Posture Scores

Position

Score

Adjustment

Upright (0°)

1

---

0-20° flexion/extension

2

+1 if twisted

20-60° flexion, \textgreater20° extension

3

+1 if twisted

\textgreater60° flexion

4

+1 if twisted or side-bending

\textbf{Neck Posture Scoring:}

\label{tab:reba-neck}REBA Neck Posture Scores

Position

Score

Adjustment

0-20° flexion

1

+1 if twisted or side-bending

\textgreater20° flexion or extension

2

+1 if twisted or side-bending

\textbf{REBA Action Levels:}

\label{tab:reba-action}REBA Action Levels

REBA Score

Risk Level

Action

1

Negligible

None necessary

2-3

Low

May be necessary

4-7

Medium

Necessary

8-10

High

Necessary soon

11-15

Very High

Necessary NOW

\subsection{REBA Example Calculation}\label{reba-example-calculation}

\begin{Shaded}
\begin{Highlighting}[]
\CommentTok{\# REBA Assessment Example: Assembly Worker}

\CommentTok{\# Group A Scores (Trunk, Neck, Legs)}
\NormalTok{trunk\_score }\OtherTok{\textless{}{-}} \DecValTok{3}      \CommentTok{\# 20{-}60° forward bend}
\NormalTok{neck\_score }\OtherTok{\textless{}{-}} \DecValTok{2}       \CommentTok{\# \textgreater{}20° flexion}
\NormalTok{legs\_score }\OtherTok{\textless{}{-}} \DecValTok{1}       \CommentTok{\# Bilateral weight bearing}

\CommentTok{\# Look up Table A score (trunk + neck + legs)}
\CommentTok{\# For this combination: Score A = 4}
\NormalTok{table\_a\_score }\OtherTok{\textless{}{-}} \DecValTok{4}

\CommentTok{\# Load/Force Score}
\NormalTok{load\_force }\OtherTok{\textless{}{-}} \DecValTok{1}       \CommentTok{\# 5{-}10 kg intermittent}

\CommentTok{\# Score A with load}
\NormalTok{score\_a }\OtherTok{\textless{}{-}}\NormalTok{ table\_a\_score }\SpecialCharTok{+}\NormalTok{ load\_force}
\FunctionTok{cat}\NormalTok{(}\StringTok{"Score A (posture + load):"}\NormalTok{, score\_a, }\StringTok{"}\SpecialCharTok{\textbackslash{}n}\StringTok{"}\NormalTok{)}
\end{Highlighting}
\end{Shaded}

\begin{verbatim}
## Score A (posture + load): 5
\end{verbatim}

\begin{Shaded}
\begin{Highlighting}[]
\CommentTok{\# Group B Scores (Upper arm, Lower arm, Wrist)}
\NormalTok{upper\_arm\_score }\OtherTok{\textless{}{-}} \DecValTok{3}  \CommentTok{\# 45{-}90° flexion}
\NormalTok{lower\_arm\_score }\OtherTok{\textless{}{-}} \DecValTok{2}  \CommentTok{\# \textless{}60° or \textgreater{}100° flexion}
\NormalTok{wrist\_score }\OtherTok{\textless{}{-}} \DecValTok{2}      \CommentTok{\# \textgreater{}15° flexion/extension}

\CommentTok{\# Look up Table B score}
\CommentTok{\# For this combination: Score B = 5}
\NormalTok{table\_b\_score }\OtherTok{\textless{}{-}} \DecValTok{5}

\CommentTok{\# Coupling Score}
\NormalTok{coupling }\OtherTok{\textless{}{-}} \DecValTok{1}         \CommentTok{\# Acceptable grip}

\CommentTok{\# Score B with coupling}
\NormalTok{score\_b }\OtherTok{\textless{}{-}}\NormalTok{ table\_b\_score }\SpecialCharTok{+}\NormalTok{ coupling}
\FunctionTok{cat}\NormalTok{(}\StringTok{"Score B (posture + coupling):"}\NormalTok{, score\_b, }\StringTok{"}\SpecialCharTok{\textbackslash{}n}\StringTok{"}\NormalTok{)}
\end{Highlighting}
\end{Shaded}

\begin{verbatim}
## Score B (posture + coupling): 6
\end{verbatim}

\begin{Shaded}
\begin{Highlighting}[]
\CommentTok{\# Look up Table C (Score A × Score B)}
\CommentTok{\# For Score A = 5 and Score B = 6: Score C = 8}
\NormalTok{score\_c }\OtherTok{\textless{}{-}} \DecValTok{8}

\CommentTok{\# Activity Score}
\NormalTok{activity }\OtherTok{\textless{}{-}} \DecValTok{1}         \CommentTok{\# One or more body parts static \textgreater{}1 min}

\CommentTok{\# Final REBA Score}
\NormalTok{reba\_final }\OtherTok{\textless{}{-}}\NormalTok{ score\_c }\SpecialCharTok{+}\NormalTok{ activity}
\FunctionTok{cat}\NormalTok{(}\StringTok{"}\SpecialCharTok{\textbackslash{}n}\StringTok{=== FINAL REBA SCORE:"}\NormalTok{, reba\_final, }\StringTok{"===}\SpecialCharTok{\textbackslash{}n}\StringTok{"}\NormalTok{)}
\end{Highlighting}
\end{Shaded}

\begin{verbatim}
## 
## === FINAL REBA SCORE: 9 ===
\end{verbatim}

\begin{Shaded}
\begin{Highlighting}[]
\CommentTok{\# Interpretation}
\FunctionTok{cat}\NormalTok{(}\StringTok{"}\SpecialCharTok{\textbackslash{}n}\StringTok{Interpretation:}\SpecialCharTok{\textbackslash{}n}\StringTok{"}\NormalTok{)}
\end{Highlighting}
\end{Shaded}

\begin{verbatim}
## 
## Interpretation:
\end{verbatim}

\begin{Shaded}
\begin{Highlighting}[]
\ControlFlowTok{if}\NormalTok{ (reba\_final }\SpecialCharTok{\textless{}=} \DecValTok{1}\NormalTok{) \{}
  \FunctionTok{cat}\NormalTok{(}\StringTok{"Risk Level: Negligible {-} No action necessary}\SpecialCharTok{\textbackslash{}n}\StringTok{"}\NormalTok{)}
\NormalTok{\} }\ControlFlowTok{else} \ControlFlowTok{if}\NormalTok{ (reba\_final }\SpecialCharTok{\textless{}=} \DecValTok{3}\NormalTok{) \{}
  \FunctionTok{cat}\NormalTok{(}\StringTok{"Risk Level: Low {-} Action may be necessary}\SpecialCharTok{\textbackslash{}n}\StringTok{"}\NormalTok{)}
\NormalTok{\} }\ControlFlowTok{else} \ControlFlowTok{if}\NormalTok{ (reba\_final }\SpecialCharTok{\textless{}=} \DecValTok{7}\NormalTok{) \{}
  \FunctionTok{cat}\NormalTok{(}\StringTok{"Risk Level: Medium {-} Action necessary}\SpecialCharTok{\textbackslash{}n}\StringTok{"}\NormalTok{)}
\NormalTok{\} }\ControlFlowTok{else} \ControlFlowTok{if}\NormalTok{ (reba\_final }\SpecialCharTok{\textless{}=} \DecValTok{10}\NormalTok{) \{}
  \FunctionTok{cat}\NormalTok{(}\StringTok{"Risk Level: High {-} Action necessary soon}\SpecialCharTok{\textbackslash{}n}\StringTok{"}\NormalTok{)}
\NormalTok{\} }\ControlFlowTok{else}\NormalTok{ \{}
  \FunctionTok{cat}\NormalTok{(}\StringTok{"Risk Level: Very High {-} Action necessary NOW}\SpecialCharTok{\textbackslash{}n}\StringTok{"}\NormalTok{)}
\NormalTok{\}}
\end{Highlighting}
\end{Shaded}

\begin{verbatim}
## Risk Level: High - Action necessary soon
\end{verbatim}

Practice Problem: REBA Assessment

\textbf{Scenario:} A worker is performing a task with the following postures:

\begin{itemize}
\tightlist
\item
  \textbf{Trunk:} 30° forward flexion, twisted
\item
  \textbf{Neck:} 15° flexion
\item
  \textbf{Legs:} Weight on one leg
\item
  \textbf{Upper Arm:} 60° flexion, shoulder raised
\item
  \textbf{Lower Arm:} 80° flexion
\item
  \textbf{Wrist:} Neutral
\item
  \textbf{Load:} 8 kg carried
\item
  \textbf{Coupling:} Fair grip (handles exist but not ideal)
\item
  \textbf{Activity:} Repeated small range actions
\end{itemize}

\textbf{Calculate the REBA score and determine the action level.}

\textbf{Solution:}

\begin{verbatim}
Trunk: 3 + 1 (twisted) = 4
Neck: 1 (0-20° flexion)
Legs: 2 (weight on one leg)
Table A = 5
Load/Force = 1 (5-10kg)
Score A = 6

Upper Arm: 3 + 1 (raised) = 4
Lower Arm: 1 (60-100°)
Wrist: 1 (neutral)
Table B = 4
Coupling = 1 (fair)
Score B = 5

Table C (Score A=6, Score B=5) = 8
Activity = +1 (repeated actions)
REBA Score = 9

Risk Level: HIGH - Action necessary soon
\end{verbatim}

\subsection{NIOSH Lifting Equation}\label{niosh-lifting-equation}

The \textbf{NIOSH Lifting Equation} calculates a \textbf{Recommended Weight Limit (RWL)} for lifting tasks.

\[RWL = LC \times HM \times VM \times DM \times AM \times FM \times CM\]

Where:

\label{tab:niosh-factors}NIOSH Lifting Equation Factors

Factor

Name

Value

Description

LC

Load Constant

23 kg (51 lb)

Maximum weight under ideal conditions

HM

Horizontal Multiplier

25/H

H = horizontal distance from midpoint between ankles to hands (cm)

VM

Vertical Multiplier

1 - 0.003\textbar V-75\textbar{}

V = vertical height of hands at start (cm)

DM

Distance Multiplier

0.82 + 4.5/D

D = vertical travel distance (cm)

AM

Asymmetric Multiplier

1 - 0.0032A

A = angle of asymmetry (degrees)

FM

Frequency Multiplier

Table lookup

Based on lift frequency and duration

CM

Coupling Multiplier

Table lookup

Based on hand-load coupling quality

\textbf{Lifting Index (LI)} indicates risk level:

\[LI = \frac{\text{Actual Weight}}{\text{RWL}}\]

\label{tab:li-interpretation}Lifting Index Interpretation

Lifting Index

Risk

Action

LI ≤ 1.0

Low

Task acceptable for most workers

1.0 \textless{} LI ≤ 2.0

Moderate

Some workers may be at risk - consider changes

2.0 \textless{} LI ≤ 3.0

High

Many workers at risk - changes recommended

LI \textgreater{} 3.0

Very High

Unacceptable - immediate changes required

\begin{Shaded}
\begin{Highlighting}[]
\CommentTok{\# NIOSH Lifting Equation Example}

\CommentTok{\# Task parameters}
\NormalTok{H }\OtherTok{\textless{}{-}} \DecValTok{40}    \CommentTok{\# Horizontal distance (cm)}
\NormalTok{V }\OtherTok{\textless{}{-}} \DecValTok{30}    \CommentTok{\# Vertical height at start (cm)}
\NormalTok{D }\OtherTok{\textless{}{-}} \DecValTok{60}    \CommentTok{\# Vertical travel distance (cm)}
\NormalTok{A }\OtherTok{\textless{}{-}} \DecValTok{30}    \CommentTok{\# Asymmetry angle (degrees)}
\NormalTok{F }\OtherTok{\textless{}{-}} \DecValTok{2}     \CommentTok{\# Lifts per minute}
\NormalTok{duration }\OtherTok{\textless{}{-}} \DecValTok{2}  \CommentTok{\# Hours of lifting}

\CommentTok{\# Constants}
\NormalTok{LC }\OtherTok{\textless{}{-}} \DecValTok{23}   \CommentTok{\# Load constant (kg)}

\CommentTok{\# Calculate multipliers}
\NormalTok{HM }\OtherTok{\textless{}{-}} \DecValTok{25} \SpecialCharTok{/}\NormalTok{ H}
\NormalTok{VM }\OtherTok{\textless{}{-}} \DecValTok{1} \SpecialCharTok{{-}} \FloatTok{0.003} \SpecialCharTok{*} \FunctionTok{abs}\NormalTok{(V }\SpecialCharTok{{-}} \DecValTok{75}\NormalTok{)}
\NormalTok{DM }\OtherTok{\textless{}{-}} \FloatTok{0.82} \SpecialCharTok{+} \FloatTok{4.5} \SpecialCharTok{/}\NormalTok{ D}
\NormalTok{AM }\OtherTok{\textless{}{-}} \DecValTok{1} \SpecialCharTok{{-}} \FloatTok{0.0032} \SpecialCharTok{*}\NormalTok{ A}

\CommentTok{\# Frequency multiplier (from table, for 2 lifts/min, 2 hours, V\textless{}75cm)}
\NormalTok{FM }\OtherTok{\textless{}{-}} \FloatTok{0.84}

\CommentTok{\# Coupling multiplier (assume "fair" coupling)}
\NormalTok{CM }\OtherTok{\textless{}{-}} \FloatTok{0.95}

\FunctionTok{cat}\NormalTok{(}\StringTok{"Multipliers:}\SpecialCharTok{\textbackslash{}n}\StringTok{"}\NormalTok{)}
\end{Highlighting}
\end{Shaded}

\begin{verbatim}
## Multipliers:
\end{verbatim}

\begin{Shaded}
\begin{Highlighting}[]
\FunctionTok{cat}\NormalTok{(}\StringTok{"HM ="}\NormalTok{, }\FunctionTok{round}\NormalTok{(HM, }\DecValTok{3}\NormalTok{), }\StringTok{"}\SpecialCharTok{\textbackslash{}n}\StringTok{"}\NormalTok{)}
\end{Highlighting}
\end{Shaded}

\begin{verbatim}
## HM = 0.625
\end{verbatim}

\begin{Shaded}
\begin{Highlighting}[]
\FunctionTok{cat}\NormalTok{(}\StringTok{"VM ="}\NormalTok{, }\FunctionTok{round}\NormalTok{(VM, }\DecValTok{3}\NormalTok{), }\StringTok{"}\SpecialCharTok{\textbackslash{}n}\StringTok{"}\NormalTok{)}
\end{Highlighting}
\end{Shaded}

\begin{verbatim}
## VM = 0.865
\end{verbatim}

\begin{Shaded}
\begin{Highlighting}[]
\FunctionTok{cat}\NormalTok{(}\StringTok{"DM ="}\NormalTok{, }\FunctionTok{round}\NormalTok{(DM, }\DecValTok{3}\NormalTok{), }\StringTok{"}\SpecialCharTok{\textbackslash{}n}\StringTok{"}\NormalTok{)}
\end{Highlighting}
\end{Shaded}

\begin{verbatim}
## DM = 0.895
\end{verbatim}

\begin{Shaded}
\begin{Highlighting}[]
\FunctionTok{cat}\NormalTok{(}\StringTok{"AM ="}\NormalTok{, }\FunctionTok{round}\NormalTok{(AM, }\DecValTok{3}\NormalTok{), }\StringTok{"}\SpecialCharTok{\textbackslash{}n}\StringTok{"}\NormalTok{)}
\end{Highlighting}
\end{Shaded}

\begin{verbatim}
## AM = 0.904
\end{verbatim}

\begin{Shaded}
\begin{Highlighting}[]
\FunctionTok{cat}\NormalTok{(}\StringTok{"FM ="}\NormalTok{, FM, }\StringTok{"}\SpecialCharTok{\textbackslash{}n}\StringTok{"}\NormalTok{)}
\end{Highlighting}
\end{Shaded}

\begin{verbatim}
## FM = 0.84
\end{verbatim}

\begin{Shaded}
\begin{Highlighting}[]
\FunctionTok{cat}\NormalTok{(}\StringTok{"CM ="}\NormalTok{, CM, }\StringTok{"}\SpecialCharTok{\textbackslash{}n\textbackslash{}n}\StringTok{"}\NormalTok{)}
\end{Highlighting}
\end{Shaded}

\begin{verbatim}
## CM = 0.95
\end{verbatim}

\begin{Shaded}
\begin{Highlighting}[]
\CommentTok{\# Calculate RWL}
\NormalTok{RWL }\OtherTok{\textless{}{-}}\NormalTok{ LC }\SpecialCharTok{*}\NormalTok{ HM }\SpecialCharTok{*}\NormalTok{ VM }\SpecialCharTok{*}\NormalTok{ DM }\SpecialCharTok{*}\NormalTok{ AM }\SpecialCharTok{*}\NormalTok{ FM }\SpecialCharTok{*}\NormalTok{ CM}
\FunctionTok{cat}\NormalTok{(}\StringTok{"Recommended Weight Limit (RWL):"}\NormalTok{, }\FunctionTok{round}\NormalTok{(RWL, }\DecValTok{1}\NormalTok{), }\StringTok{"kg}\SpecialCharTok{\textbackslash{}n\textbackslash{}n}\StringTok{"}\NormalTok{)}
\end{Highlighting}
\end{Shaded}

\begin{verbatim}
## Recommended Weight Limit (RWL): 8 kg
\end{verbatim}

\begin{Shaded}
\begin{Highlighting}[]
\CommentTok{\# Calculate Lifting Index for actual load}
\NormalTok{actual\_weight }\OtherTok{\textless{}{-}} \DecValTok{15}  \CommentTok{\# kg}
\NormalTok{LI }\OtherTok{\textless{}{-}}\NormalTok{ actual\_weight }\SpecialCharTok{/}\NormalTok{ RWL}
\FunctionTok{cat}\NormalTok{(}\StringTok{"Actual Weight:"}\NormalTok{, actual\_weight, }\StringTok{"kg}\SpecialCharTok{\textbackslash{}n}\StringTok{"}\NormalTok{)}
\end{Highlighting}
\end{Shaded}

\begin{verbatim}
## Actual Weight: 15 kg
\end{verbatim}

\begin{Shaded}
\begin{Highlighting}[]
\FunctionTok{cat}\NormalTok{(}\StringTok{"Lifting Index (LI):"}\NormalTok{, }\FunctionTok{round}\NormalTok{(LI, }\DecValTok{2}\NormalTok{), }\StringTok{"}\SpecialCharTok{\textbackslash{}n\textbackslash{}n}\StringTok{"}\NormalTok{)}
\end{Highlighting}
\end{Shaded}

\begin{verbatim}
## Lifting Index (LI): 1.87
\end{verbatim}

\begin{Shaded}
\begin{Highlighting}[]
\CommentTok{\# Interpretation}
\ControlFlowTok{if}\NormalTok{ (LI }\SpecialCharTok{\textless{}=} \FloatTok{1.0}\NormalTok{) \{}
  \FunctionTok{cat}\NormalTok{(}\StringTok{"Risk Level: LOW {-} Task acceptable}\SpecialCharTok{\textbackslash{}n}\StringTok{"}\NormalTok{)}
\NormalTok{\} }\ControlFlowTok{else} \ControlFlowTok{if}\NormalTok{ (LI }\SpecialCharTok{\textless{}=} \FloatTok{2.0}\NormalTok{) \{}
  \FunctionTok{cat}\NormalTok{(}\StringTok{"Risk Level: MODERATE {-} Consider changes}\SpecialCharTok{\textbackslash{}n}\StringTok{"}\NormalTok{)}
\NormalTok{\} }\ControlFlowTok{else} \ControlFlowTok{if}\NormalTok{ (LI }\SpecialCharTok{\textless{}=} \FloatTok{3.0}\NormalTok{) \{}
  \FunctionTok{cat}\NormalTok{(}\StringTok{"Risk Level: HIGH {-} Changes recommended}\SpecialCharTok{\textbackslash{}n}\StringTok{"}\NormalTok{)}
\NormalTok{\} }\ControlFlowTok{else}\NormalTok{ \{}
  \FunctionTok{cat}\NormalTok{(}\StringTok{"Risk Level: VERY HIGH {-} Immediate changes required}\SpecialCharTok{\textbackslash{}n}\StringTok{"}\NormalTok{)}
\NormalTok{\}}
\end{Highlighting}
\end{Shaded}

\begin{verbatim}
## Risk Level: MODERATE - Consider changes
\end{verbatim}

\begin{center}\rule{0.5\linewidth}{0.5pt}\end{center}

\section{AI-Powered Ergonomic Assessment}\label{ai-powered-ergonomic-assessment}

Artificial intelligence is revolutionizing ergonomic assessment by enabling \textbf{real-time, automated analysis} of worker postures without disrupting operations.

\begin{figure}

{\centering \includegraphics{introduction_files/figure-latex/ai-tools-viz-1} 

}

\caption{AI Ergonomics Assessment Pipeline}\label{fig:ai-tools-viz}
\end{figure}

\label{tab:ai-tools-table}AI-Powered Ergonomic Assessment Tools

Tool

Technology

Key Features

Best For

\textbf{TuMeke Ergonomics}

Computer vision + wearables

Real-time REBA scoring, multi-camera support

Manufacturing, warehousing

\textbf{VelocityEHS}

AI video analysis

Automated job analysis, risk prioritization

Enterprise-wide programs

\textbf{Protex AI}

Real-time video AI

Privacy-preserving analysis, floor-wide monitoring

Large facilities, privacy-conscious

\textbf{SoterTask}

Wearable sensors + AI

Wearable clips, real-time coaching, haptic feedback

Individual worker coaching

\textbf{Viso.ai}

Custom AI models

Custom pose detection, integration APIs

Custom integrations

\textbf{ErgoPlus Platform}

Mobile app + AI

Mobile assessment, improvement tracking

Field assessments

\subsection{Benefits of AI Ergonomic Assessment}\label{benefits-of-ai-ergonomic-assessment}

\begin{figure}

{\centering \includegraphics{introduction_files/figure-latex/ai-benefits-1} 

}

\caption{Traditional vs. AI Ergonomic Assessment}\label{fig:ai-benefits}
\end{figure}

\begin{center}\rule{0.5\linewidth}{0.5pt}\end{center}

\section{Ergonomics in Automated Manufacturing}\label{ergonomics-in-automated-manufacturing-1}

Industrial automation inherently involves the mechanization of processes traditionally carried out by human operators. In the automation design phase, engineers must strike a balance between mechanization and the necessary human interaction with these systems.

\subsection{Human-Automation Interaction Points}\label{human-automation-interaction-points}

\begin{figure}

{\centering \includegraphics{introduction_files/figure-latex/automation-interaction-1} 

}

\caption{Ergonomic Considerations in Automated Systems}\label{fig:automation-interaction}
\end{figure}

\label{tab:automation-ergo}Ergonomic Considerations for Automated Systems

Interaction Point

Ergonomic Challenges

Design Solutions

\textbf{Loading/Unloading}

Repetitive motion, lifting, reaching, awkward postures

Height-adjustable conveyors, lift assists, indexing fixtures

\textbf{HMI Operation}

Static posture, visual strain, cognitive load

Adjustable screens, proper height, clear interface design

\textbf{Maintenance}

Awkward access, confined spaces, tool use

Adequate access space, tool organization, proper lighting

\textbf{Quality Inspection}

Static positions, visual strain, fine motor tasks

Adjustable fixtures, magnification, ergonomic seating

\textbf{Supervision}

Sedentary position, cognitive fatigue, vigilance

Multiple monitors at eye level, alert systems, task variety

\subsection{Collaborative Robot Ergonomics}\label{collaborative-robot-ergonomics}

Cobots present unique ergonomic opportunities and challenges:

\label{tab:cobot-ergo}Ergonomic Considerations for Collaborative Robots

Aspect

Opportunity

Challenge

\textbf{Task Allocation}

Robot handles heavy, repetitive, or precision tasks

Determining optimal task division

\textbf{Workstation Layout}

Shared workspace allows flexible positioning

Ensuring adequate space for human movement

\textbf{Pacing}

Robot adapts to human pace, not vice versa

Avoiding pressure from robot efficiency

\textbf{Force Assistance}

Robot provides powered assist for lifting

Proper force limits to prevent injury on contact

\textbf{Cognitive Load}

Robot handles routine tasks, human focuses on judgment

Maintaining situational awareness

\begin{center}\rule{0.5\linewidth}{0.5pt}\end{center}

\section{Employer Responsibilities}\label{employer-responsibilities}

Employers have both legal and ethical obligations to protect workers from ergonomic hazards.

\label{tab:employer-responsibilities}Employer Ergonomics Program Responsibilities

Responsibility

Actions

\textbf{Management Commitment}

Allocate resources, set goals, lead by example

\textbf{Worker Involvement}

Include workers in assessments, design teams, solution development

\textbf{Training}

Ergonomics awareness, proper techniques, symptom recognition

\textbf{Hazard Identification}

Regular assessments, worker surveys, injury analysis

\textbf{Early Reporting}

Non-punitive reporting system, prompt response to symptoms

\textbf{Hazard Control}

Engineering controls, work organization, PPE as last resort

\textbf{Program Evaluation}

Track injuries, measure improvements, adjust programs

OSHA Ergonomics Guidelines

While OSHA does not have a specific ergonomics standard, employers are still responsible under the \textbf{General Duty Clause} (Section 5(a)(1)) which requires employers to provide a workplace ``free from recognized hazards.''

\textbf{OSHA's Recommended Program Elements:}
1. Management leadership
2. Worker participation
3. Hazard identification and assessment
4. Hazard prevention and control
5. Education and training
6. Program evaluation

\textbf{Industry-Specific Guidelines:}
- Poultry Processing
- Nursing Homes
- Retail Grocery Stores
- Shipyards

Visit: \href{https://www.osha.gov/ergonomics}{OSHA Ergonomics eTool}

\begin{center}\rule{0.5\linewidth}{0.5pt}\end{center}

\section{Case Study: Ergonomics in Automated Assembly}\label{case-study-ergonomics-in-automated-assembly}

\textbf{Company:} Automotive parts manufacturer

\textbf{Problem:} High rate of upper extremity MSDs in assembly department
- 15 recordable injuries per year
- Lost workdays averaging 20 days per case
- Workers' comp costs exceeding \$300,000 annually

\textbf{Assessment Findings:}

\label{tab:case-study-findings}Ergonomic Assessment Findings

Workstation

Primary Issues

REBA Score

Risk Level

Sub-assembly

Overhead reaching, high repetition

9

High

Main line

Forward bending, static standing

8

High

Inspection

Awkward wrist postures, visual strain

7

Medium

Packaging

Lifting, carrying, repetition

10

Very High

\textbf{Interventions Implemented:}

\begin{enumerate}
\def\labelenumi{\arabic{enumi}.}
\tightlist
\item
  \textbf{Sub-assembly:}

  \begin{itemize}
  \tightlist
  \item
    Tilting fixtures to bring work closer
  \item
    Parts bins repositioned to eliminate overhead reaching
  \item
    Cobot added for repetitive insertion tasks
  \end{itemize}
\item
  \textbf{Main line:}

  \begin{itemize}
  \tightlist
  \item
    Height-adjustable workstations installed
  \item
    Anti-fatigue mats added
  \item
    Job rotation implemented
  \end{itemize}
\item
  \textbf{Inspection:}

  \begin{itemize}
  \tightlist
  \item
    Magnifying task lights installed
  \item
    Angled fixtures to reduce wrist bending
  \item
    Ergonomic seating provided
  \end{itemize}
\item
  \textbf{Packaging:}

  \begin{itemize}
  \tightlist
  \item
    Vacuum lift assists installed
  \item
    Conveyor heights adjusted
  \item
    Case erectors automated
  \end{itemize}
\end{enumerate}

\textbf{Results After 18 Months:}

\begin{figure}

{\centering \includegraphics{introduction_files/figure-latex/case-study-results-1} 

}

\caption{Ergonomic Intervention Results}\label{fig:case-study-results}
\end{figure}

\textbf{ROI Analysis:}

\begin{Shaded}
\begin{Highlighting}[]
\CommentTok{\# Investment}
\NormalTok{equipment\_cost }\OtherTok{\textless{}{-}} \DecValTok{150000}  \CommentTok{\# Lift assists, fixtures, cobots}
\NormalTok{training\_cost }\OtherTok{\textless{}{-}} \DecValTok{15000}
\NormalTok{total\_investment }\OtherTok{\textless{}{-}}\NormalTok{ equipment\_cost }\SpecialCharTok{+}\NormalTok{ training\_cost}

\CommentTok{\# Annual Savings}
\NormalTok{injury\_reduction }\OtherTok{\textless{}{-}}\NormalTok{ (}\DecValTok{15} \SpecialCharTok{{-}} \DecValTok{4}\NormalTok{) }\SpecialCharTok{*} \DecValTok{20000}  \CommentTok{\# Avg cost per injury}
\NormalTok{productivity\_gain }\OtherTok{\textless{}{-}} \FloatTok{0.18} \SpecialCharTok{*} \DecValTok{500000}    \CommentTok{\# 18\% improvement on labor cost}
\NormalTok{quality\_improvement }\OtherTok{\textless{}{-}} \FloatTok{0.35} \SpecialCharTok{*} \DecValTok{50000}   \CommentTok{\# 35\% defect reduction}

\NormalTok{annual\_savings }\OtherTok{\textless{}{-}}\NormalTok{ injury\_reduction }\SpecialCharTok{+}\NormalTok{ productivity\_gain }\SpecialCharTok{+}\NormalTok{ quality\_improvement}

\FunctionTok{cat}\NormalTok{(}\StringTok{"Total Investment: $"}\NormalTok{, }\FunctionTok{format}\NormalTok{(total\_investment, }\AttributeTok{big.mark =} \StringTok{","}\NormalTok{), }\StringTok{"}\SpecialCharTok{\textbackslash{}n}\StringTok{"}\NormalTok{)}
\end{Highlighting}
\end{Shaded}

\begin{verbatim}
## Total Investment: $ 165,000
\end{verbatim}

\begin{Shaded}
\begin{Highlighting}[]
\FunctionTok{cat}\NormalTok{(}\StringTok{"Annual Savings: $"}\NormalTok{, }\FunctionTok{format}\NormalTok{(annual\_savings, }\AttributeTok{big.mark =} \StringTok{","}\NormalTok{), }\StringTok{"}\SpecialCharTok{\textbackslash{}n}\StringTok{"}\NormalTok{)}
\end{Highlighting}
\end{Shaded}

\begin{verbatim}
## Annual Savings: $ 327,500
\end{verbatim}

\begin{Shaded}
\begin{Highlighting}[]
\FunctionTok{cat}\NormalTok{(}\StringTok{"Payback Period:"}\NormalTok{, }\FunctionTok{round}\NormalTok{(total\_investment }\SpecialCharTok{/}\NormalTok{ annual\_savings, }\DecValTok{1}\NormalTok{), }\StringTok{"years}\SpecialCharTok{\textbackslash{}n}\StringTok{"}\NormalTok{)}
\end{Highlighting}
\end{Shaded}

\begin{verbatim}
## Payback Period: 0.5 years
\end{verbatim}

\begin{Shaded}
\begin{Highlighting}[]
\FunctionTok{cat}\NormalTok{(}\StringTok{"5{-}Year ROI:"}\NormalTok{, }\FunctionTok{round}\NormalTok{((annual\_savings }\SpecialCharTok{*} \DecValTok{5} \SpecialCharTok{{-}}\NormalTok{ total\_investment) }\SpecialCharTok{/}\NormalTok{ total\_investment }\SpecialCharTok{*} \DecValTok{100}\NormalTok{), }\StringTok{"\%}\SpecialCharTok{\textbackslash{}n}\StringTok{"}\NormalTok{)}
\end{Highlighting}
\end{Shaded}

\begin{verbatim}
## 5-Year ROI: 892 %
\end{verbatim}

\begin{center}\rule{0.5\linewidth}{0.5pt}\end{center}

\section{Summary}\label{summary-5}

\label{tab:summary-table-ch7}Ergonomics in Manufacturing: Key Concepts Summary

Topic

Key Points

\textbf{Ergonomics Definition}

Fitting the job to the worker; physical, cognitive, organizational domains

\textbf{Risk Factors}

Posture, repetition, force, static positions, contact stress, vibration, environment

\textbf{MSDs}

Injuries to muscles, nerves, tendons; develop gradually from exposure to disability

\textbf{Prevention}

Engineering controls most effective; proper lifting, neutral postures, recovery time

\textbf{Design Principles}

Neutral posture, anthropometry, work height optimization, environmental control

\textbf{Assessment Tools}

REBA for whole body, RULA for upper limb, NIOSH for lifting tasks

\textbf{AI in Ergonomics}

Real-time monitoring, automated REBA/RULA scoring, continuous assessment

\textbf{Employer Role}

Management commitment, worker involvement, training, hazard control, evaluation

\begin{center}\rule{0.5\linewidth}{0.5pt}\end{center}

\section{Review Questions}\label{review-questions-6}

Question 1: What are the seven ergonomic risk factors, and which two are most commonly associated with carpal tunnel syndrome?

The seven ergonomic risk factors are:
1. Awkward Postures
2. Repetition
3. Force
4. Static Positions
5. Contact Stress
6. Vibration
7. Environmental Factors

\textbf{Carpal tunnel syndrome} is most commonly associated with:
- \textbf{Repetition} - Repeated wrist movements compress the median nerve
- \textbf{Force} - Forceful gripping increases pressure in the carpal tunnel
- Also contributing: awkward wrist postures and vibration

Question 2: A worker performs a task 8 times per minute for a 4-hour shift. The cycle time is 7.5 seconds. Is this considered high repetition? Why?

\textbf{Analysis:}
- Cycle time: 7.5 seconds (\textless{} 15 seconds threshold)
- Repetitions per hour: 8 × 60 = 480 (\textgreater{} 240 threshold)

\textbf{Yes, this is HIGH repetition} because:
1. Cycle time is less than 15 seconds
2. Repetitions exceed 240 per hour

\textbf{Risk Level:} High - Immediate intervention needed

\textbf{Recommended actions:}
- Job rotation with different tasks
- Automation of repetitive elements
- Micro-breaks every 20-30 minutes
- Workstation redesign to reduce cycle elements

Question 3: Explain why ergonomic design should accommodate the 5th to 95th percentile of the population rather than designing for the ``average'' person.

\textbf{Designing for the average excludes most people:}
- The ``average'' person in ALL dimensions doesn't exist
- A person may be average height but have long arms or short legs
- Designing for only the 50th percentile fits only about 50\% of users

\textbf{The 5th-95th percentile range:}
- Accommodates 90\% of the population
- Addresses both small and large users
- Provides adjustability rather than fixed dimensions

\textbf{Application rules:}
- \textbf{Clearances} (overhead, legroom): Design for 95th percentile (largest)
- \textbf{Reach distances}: Design for 5th percentile (smallest)
- \textbf{Work surfaces}: Make adjustable to accommodate range

\textbf{Example:} A fixed workstation at 95cm height:
- Too high for 5th percentile female (elbow at \textasciitilde93cm)
- Too low for 95th percentile male (elbow at \textasciitilde117cm)
- Adjustable range of 85-115cm accommodates both

Question 4: Calculate the NIOSH Lifting Index for the following task and determine if intervention is needed.

\textbf{Task Parameters:}
- Horizontal distance (H): 35 cm
- Vertical height (V): 80 cm
- Travel distance (D): 50 cm
- Asymmetry (A): 0° (symmetric lift)
- Frequency: 4 lifts/minute for 1 hour
- Coupling: Good (handles)
- Actual load: 18 kg

\textbf{Solution:}

\begin{verbatim}
LC = 23 kg
HM = 25/35 = 0.714
VM = 1 - 0.003|80-75| = 0.985
DM = 0.82 + 4.5/50 = 0.91
AM = 1 - 0.0032(0) = 1.0
FM = 0.75 (4/min, 1 hour, V>75cm, from table)
CM = 1.0 (good coupling)

RWL = 23 × 0.714 × 0.985 × 0.91 × 1.0 × 0.75 × 1.0
RWL = 11.0 kg

LI = 18 / 11.0 = 1.64

Risk Level: MODERATE (1.0 < LI < 2.0)
Action: Some workers may be at risk - consider changes
\end{verbatim}

\textbf{Recommendations:}
- Reduce horizontal distance (bring load closer)
- Reduce lifting frequency
- Use mechanical assist for loads \textgreater12 kg

Question 5: What are three advantages of AI-powered ergonomic assessment over traditional methods?

\textbf{1. Scale and Coverage:}
- Traditional: 5-10 assessments per day by trained ergonomist
- AI: Hundreds or thousands of assessments simultaneously
- Can monitor entire facility continuously

\textbf{2. Consistency and Objectivity:}
- Traditional: Inter-rater reliability issues; different assessors may score differently
- AI: Consistent scoring algorithm applied uniformly
- Removes subjective interpretation

\textbf{3. Real-time Feedback:}
- Traditional: Assessment completed, report generated days/weeks later
- AI: Immediate scoring and alerts
- Can provide real-time coaching to workers

\textbf{Additional advantages:}
- Lower cost per assessment
- Non-disruptive (workers not aware of assessment)
- Longitudinal tracking of posture patterns
- Data for predictive analytics

\begin{center}\rule{0.5\linewidth}{0.5pt}\end{center}

\section{Additional Resources}\label{additional-resources}

\subsection{Videos}\label{videos}

\textbf{Ergonomics Fundamentals:}
- \href{https://www.youtube.com/watch?v=pKJi-5wjEoI}{Introduction to Workplace Ergonomics}
- \href{https://www.youtube.com/watch?v=aNZqB30gPfE}{Safe Manual Handling Techniques}

\textbf{Assessment Tools:}
- \href{https://www.youtube.com/watch?v=wRxLNzgGSBg}{REBA Assessment Tutorial}
- \href{https://www.youtube.com/watch?v=5GjfCeBtWvE}{RULA Assessment Guide}
- \href{https://www.youtube.com/watch?v=BbNfbGRIkLQ}{NIOSH Lifting Equation Explained}

\textbf{AI Ergonomics:}
- \href{https://www.youtube.com/watch?v=N4p4NsA96Zg}{TuMeke AI Ergonomics Demo}
- \href{https://www.youtube.com/watch?v=QnR7fXD-jmc}{VelocityEHS Industrial Ergonomics}

\subsection{Online Tools}\label{online-tools}

\begin{itemize}
\tightlist
\item
  \href{https://ergo-plus.com/reba-assessment-tool-guide/}{ErgoPlus REBA Calculator}
\item
  \href{https://www.cdc.gov/niosh/topics/ergonomics/nlecalc.html}{NIOSH Lifting Equation Calculator}
\item
  \href{https://ergo.human.cornell.edu/}{Cornell University Ergonomics Web}
\end{itemize}

\begin{center}\rule{0.5\linewidth}{0.5pt}\end{center}

\section{References}\label{references-6}

\begin{itemize}
\tightlist
\item
  Berlin, C., \& Adams, C. (2017). \emph{Production Ergonomics: Designing Work Systems to Support Optimal Human Performance}. Ubiquity Press.
\item
  Hignett, S., \& McAtamney, L. (2000). Rapid Entire Body Assessment (REBA). \emph{Applied Ergonomics}, 31(2), 201-205.
\item
  McAtamney, L., \& Corlett, E.N. (1993). RULA: A Survey Method for Investigation of Work-Related Upper Limb Disorders. \emph{Applied Ergonomics}, 24(2), 91-99.
\item
  NIOSH. (1994). \emph{Applications Manual for the Revised NIOSH Lifting Equation}. DHHS (NIOSH) Publication No.~94-110.
\item
  OSHA. (2000). \emph{Ergonomics: The Study of Work}. OSHA 3125.
\item
  Waters, T.R., Putz-Anderson, V., \& Garg, A. (1993). Revised NIOSH Equation for the Design and Evaluation of Manual Lifting Tasks. \emph{Ergonomics}, 36(7), 749-776.
\item
  ISO 11228-1:2021. Ergonomics --- Manual handling --- Part 1: Lifting, lowering and carrying.
\item
  ISO 11228-3:2007. Ergonomics --- Manual handling --- Part 3: Handling of low loads at high frequency.
\end{itemize}

\begin{center}\rule{0.5\linewidth}{0.5pt}\end{center}

\chapter{Process Failure Mode and Effects Analysis (PFMEA)}\label{process-failure-mode-and-effects-analysis-pfmea}

\begin{center}\rule{0.5\linewidth}{0.5pt}\end{center}

\section{Learning Objectives}\label{learning-objectives-7}

By the end of this chapter, you will be able to:

\begin{itemize}
\tightlist
\item
  Explain the purpose and benefits of PFMEA in manufacturing
\item
  Distinguish between DFMEA (Design) and PFMEA (Process)
\item
  Apply the 10-step PFMEA methodology to a manufacturing process
\item
  Correctly rate Severity, Occurrence, and Detection using standardized scales
\item
  Calculate Risk Priority Numbers (RPN) and prioritize actions
\item
  Develop recommended actions to reduce process risks
\item
  Create and maintain PFMEA documentation as a living document
\end{itemize}

\begin{quote}
``An ounce of prevention is worth a pound of cure.''
--- Benjamin Franklin
\end{quote}

\begin{center}\rule{0.5\linewidth}{0.5pt}\end{center}

\section{Introduction to PFMEA}\label{introduction-to-pfmea}

\subsection{What is PFMEA?}\label{what-is-pfmea}

\textbf{Process Failure Mode and Effects Analysis (PFMEA)} is a systematic, proactive method for evaluating a manufacturing or business process to identify where and how it might fail, and to assess the relative impact of different failures. The goal is to identify and prioritize actions that will prevent defects before they occur.

\begin{figure}

{\centering \includegraphics{introduction_files/figure-latex/pfmea-definition-1} 

}

\caption{PFMEA: A Proactive Quality Tool}\label{fig:pfmea-definition}
\end{figure}

\subsection{Brief History of FMEA}\label{brief-history-of-fmea}

\label{tab:fmea-history}Evolution of FMEA

Year

Development

Significance

1949

US Military develops FMEA (MIL-P-1629)

First formal FMEA methodology

1960s

NASA adopts FMEA for Apollo program

Proven reliability in life-critical systems

1970s

Automotive industry begins using FMEA

Ford, GM, Chrysler adoption drives standardization

1980s

AIAG publishes first FMEA manual

Industry-wide consistency established

1990s

QS-9000 requires FMEA for automotive suppliers

FMEA becomes mandatory in supply chain

2000s

FMEA expands to healthcare, service industries

Methodology proves universal applicability

2019

AIAG-VDA FMEA Handbook published (current standard)

Harmonized global standard, introduces AP method

\subsection{DFMEA vs.~PFMEA}\label{dfmea-vs.-pfmea}

Two main types of FMEA are used in product development:

\begin{figure}

{\centering \includegraphics{introduction_files/figure-latex/dfmea-pfmea-comparison-1} 

}

\caption{DFMEA vs PFMEA: Different Focus, Same Methodology}\label{fig:dfmea-pfmea-comparison}
\end{figure}

\label{tab:dfmea-pfmea-table}DFMEA vs PFMEA Comparison

Aspect

DFMEA (Design)

PFMEA (Process)

\textbf{Primary Question}

Will the design meet requirements?

Will the process produce conforming parts?

\textbf{Failure Mode Example}

Shaft diameter too small

Shaft turned undersize

\textbf{Effect Example}

Premature bearing failure

Part fails incoming inspection

\textbf{Cause Example}

Inadequate stress analysis

Tool wear not monitored

\textbf{Control Example}

FEA simulation, prototype testing

In-process gauging, SPC

\textbf{Responsible Team}

Design engineering, R\&D

Manufacturing, quality, production

Question: When should PFMEA be conducted?

\textbf{Ideal Timing for PFMEA:}

\begin{enumerate}
\def\labelenumi{\arabic{enumi}.}
\tightlist
\item
  \textbf{New Process Development:} Before production begins, during process design
\item
  \textbf{New Product Introduction:} When a new product will use existing processes
\item
  \textbf{Process Changes:} Whenever significant changes are made to equipment, methods, or materials
\item
  \textbf{Quality Issues:} When field failures or customer complaints indicate process problems
\item
  \textbf{Regular Review:} Periodically (annually) to ensure continued relevance
\end{enumerate}

\textbf{Key Principle:} The earlier PFMEA is conducted, the lower the cost of implementing improvements. Changes during design cost 10x less than changes during production, and 100x less than changes after customer delivery.

\subsection{Why PFMEA Matters}\label{why-pfmea-matters}

\begin{figure}

{\centering \includegraphics{introduction_files/figure-latex/pfmea-benefits-1} 

}

\caption{Benefits of PFMEA Implementation}\label{fig:pfmea-benefits}
\end{figure}

\begin{center}\rule{0.5\linewidth}{0.5pt}\end{center}

\section{PFMEA Fundamentals}\label{pfmea-fundamentals}

\subsection{Core Objectives}\label{core-objectives}

The fundamental objective of PFMEA is \textbf{prevention over detection} --- identifying and eliminating potential failures before they occur, rather than finding defects after production.

\label{tab:pfmea-objectives}Core Objectives of PFMEA

Objective

Description

Outcome

\textbf{Identify Failure Modes}

Systematically identify all ways the process can fail

Comprehensive list of potential problems

\textbf{Assess Risk}

Evaluate severity, likelihood, and detectability of each failure

Quantified risk scores (RPN or AP)

\textbf{Prioritize Actions}

Focus resources on highest-risk failure modes

Action plan focused on critical issues

\textbf{Document Knowledge}

Capture institutional knowledge in a structured format

Living document for continuous improvement

\textbf{Drive Improvement}

Implement and verify effectiveness of countermeasures

Measurable reduction in process risk

\subsection{The PFMEA Team}\label{the-pfmea-team}

PFMEA is most effective when conducted by a \textbf{cross-functional team}. No single person has all the knowledge needed to identify every potential failure.

\begin{figure}

{\centering \includegraphics{introduction_files/figure-latex/pfmea-team-1} 

}

\caption{PFMEA Cross-Functional Team}\label{fig:pfmea-team}
\end{figure}

Question: Why is a cross-functional team essential?

\textbf{A cross-functional team is essential because:}

\begin{enumerate}
\def\labelenumi{\arabic{enumi}.}
\tightlist
\item
  \textbf{Diverse Knowledge:} No single person understands all aspects of the process
\item
  \textbf{Different Perspectives:} Quality sees defects, production sees constraints, maintenance sees failures
\item
  \textbf{Buy-in:} People support what they help create
\item
  \textbf{Historical Knowledge:} Operators often know about problems never formally documented
\item
  \textbf{System Thinking:} Process interactions are visible when multiple functions participate
\end{enumerate}

\textbf{Common Mistake:} Having one engineer complete PFMEA alone as a ``paperwork exercise.'' This misses critical failure modes and produces an ineffective document.

\textbf{Best Practice:} Minimum 4-6 team members, including at least one person with hands-on process experience.

\subsection{Key Terminology}\label{key-terminology}

Understanding PFMEA terminology is essential for effective analysis:

\label{tab:pfmea-terminology}PFMEA Key Terminology

Term

Definition

Example

\textbf{Process Function}

What the process step is supposed to do

`Drill hole to 10mm ± 0.1mm'

\textbf{Failure Mode}

How the process can fail to perform its function

Hole diameter undersize, oversize, off-location

\textbf{Failure Effect}

The consequence of the failure on the customer or next operation

Part rejected at assembly, field failure

\textbf{Severity (S)}

Rating (1-10) of how serious the failure effect is

10 = safety hazard, 1 = no effect

\textbf{Failure Cause}

Why the failure mode might occur

Drill bit wear, incorrect feed rate

\textbf{Occurrence (O)}

Rating (1-10) of how likely the cause is to happen

10 = very high, 1 = remote

\textbf{Current Controls}

What currently prevents the cause or detects the failure

In-process gauge, SPC chart

\textbf{Detection (D)}

Rating (1-10) of how likely current controls will detect the failure

10 = no detection, 1 = certain detection

\textbf{RPN}

Risk Priority Number = S × O × D

Range: 1 to 1000

\textbf{Action Priority (AP)}

Alternative prioritization method (AIAG-VDA)

High, Medium, Low

\subsection{The PFMEA Form}\label{the-pfmea-form}

The PFMEA is documented in a structured form. Here is a simplified template:

\label{tab:pfmea-form}PFMEA Form Template (Simplified)

Process Step

Function

Failure Mode

Effect

S

Cause

O

Current Controls

D

RPN

Action

Example:

Drill hole

Hole undersize

Assembly interference

7

Drill wear

5

Visual inspection

6

210

Add in-process gauge

\begin{center}\rule{0.5\linewidth}{0.5pt}\end{center}

\section{The PFMEA Methodology: 10 Steps}\label{the-pfmea-methodology-10-steps}

PFMEA follows a systematic 10-step methodology. Each step builds on the previous one.

\begin{figure}

{\centering \includegraphics{introduction_files/figure-latex/pfmea-steps-overview-1} 

}

\caption{The 10-Step PFMEA Methodology}\label{fig:pfmea-steps-overview}
\end{figure}

\subsection{Step 1: Define Scope and Process Boundaries}\label{step-1-define-scope-and-process-boundaries}

Before starting the analysis, clearly define what is included and excluded.

\label{tab:step1-scope}Step 1: Defining PFMEA Scope

Element

Description

Example: Injection Molding

\textbf{Process Name}

Clear name identifying the process

Plastic Part Injection Molding

\textbf{Start Boundary}

Where does our analysis begin?

Raw material loading into hopper

\textbf{End Boundary}

Where does our analysis end?

Part removed from mold, ready for packaging

\textbf{Included Operations}

All operations within scope

Material prep, injection, cooling, ejection, inspection

\textbf{Excluded Operations}

Operations analyzed separately or by others

Mold design (covered by DFMEA), packaging

\textbf{Key Assumptions}

Conditions assumed to be true

Material meets incoming spec, mold is validated

\subsection{Step 2: Process Mapping and Identifying Functions}\label{step-2-process-mapping-and-identifying-functions}

Create a detailed process flow diagram and identify the function of each step.

\begin{figure}

{\centering \includegraphics{introduction_files/figure-latex/step2-process-map-1} 

}

\caption{Example: Injection Molding Process Flow}\label{fig:step2-process-map}
\end{figure}

\subsection{Step 3: Identifying Potential Failure Modes}\label{step-3-identifying-potential-failure-modes}

For each process function, ask: \textbf{``How can this step fail to perform its intended function?''}

\label{tab:step3-failure-modes}Step 3: Identifying Failure Modes

Process Function

Potential Failure Mode

Category

Load correct material

Wrong material loaded

Wrong item

Load correct material

Material contaminated

Degraded item

Load correct material

Insufficient material

Missing item

Melt to specification temperature

Temperature too low

Below specification

Melt to specification temperature

Temperature too high

Above specification

Fill mold completely

Short shot (incomplete fill)

Incomplete operation

Fill mold completely

Flash (overfill)

Excessive operation

Fill mold completely

Air entrapment

Unintended result

\textbf{Common failure mode categories:}
- \textbf{No operation} (step doesn't happen)
- \textbf{Partial operation} (incomplete)
- \textbf{Degraded operation} (below standard)
- \textbf{Excessive operation} (too much)
- \textbf{Wrong operation} (incorrect action)
- \textbf{Intermittent operation} (inconsistent)

Interactive Exercise: Identify Failure Modes

\textbf{Process Step:} Heat treat steel parts at 850°C for 2 hours

\textbf{Function:} Achieve required hardness (58-62 HRC)

\textbf{Identify at least 5 potential failure modes:}

\textbf{Answers:}
1. Temperature too low → Parts soft (below hardness spec)
2. Temperature too high → Parts brittle/cracked
3. Time too short → Incomplete hardening
4. Time too long → Over-hardening, distortion
5. Uneven heating → Inconsistent hardness across part
6. Wrong atmosphere → Surface decarburization
7. Quench delay too long → Soft spots

\subsection{Step 4: Analyzing Failure Effects}\label{step-4-analyzing-failure-effects}

For each failure mode, determine: \textbf{``What is the consequence if this failure occurs?''}

Consider effects at multiple levels:
1. \textbf{Local effect:} Impact on immediate operation
2. \textbf{Next operation effect:} Impact on downstream process
3. \textbf{End customer effect:} Impact on final user

\label{tab:step4-effects}Step 4: Analyzing Effects at Multiple Levels

Failure Mode

Effect Level

Effect Description

Short shot

Local

Part is incomplete/missing features

Short shot

Next Operation

Part rejected at inspection, line stoppage

Short shot

End Customer

If shipped: product malfunction, safety hazard

Flash

Local

Excess material on part edges

Flash

Next Operation

Secondary trimming required, increased cycle time

Flash

End Customer

If shipped: poor fit, appearance defect

\subsection{Step 5: Determining Severity Ratings}\label{step-5-determining-severity-ratings}

\textbf{Severity (S)} rates the seriousness of the effect on a scale of 1-10.

\label{tab:severity-scale}Severity Rating Scale (S)

Rating

Effect

Criteria

Example

10

Hazardous - without warning

Safety hazard; noncompliance with regulations; no warning

Brake failure without warning

9

Hazardous - with warning

Safety hazard; noncompliance with regulations; with warning

Airbag warning light before failure

8

Very High

Product inoperable; loss of primary function

Engine won't start

7

High

Product operable but at reduced performance

Engine runs rough, reduced power

6

Moderate

Product operable; comfort/convenience affected; noticeable

Wind noise, A/C takes longer to cool

5

Low

Product operable; reduced comfort; noticed by average customer

Door handle feels loose

4

Very Low

Product operable; slightly reduced comfort; noticed by discriminating customer

Slight paint imperfection

3

Minor

Fit/finish nonconforming; noticed by discriminating customer

Panel gap slightly uneven

2

Very Minor

Fit/finish nonconforming; noticed only by trained observer

Under-hood label crooked

1

None

No discernible effect

No effect observable

\begin{quote}
\textbf{Important:} Severity is determined by the \textbf{effect}, not the failure mode. Severity can only be reduced by changing the design --- process changes typically cannot reduce severity.
\end{quote}

\subsection{Step 6: Identifying Potential Causes}\label{step-6-identifying-potential-causes}

For each failure mode, identify all possible causes: \textbf{``Why might this failure occur?''}

\begin{figure}

{\centering \includegraphics{introduction_files/figure-latex/step6-causes-1} 

}

\caption{Fishbone Diagram: Causes of Short Shot}\label{fig:step6-causes}
\end{figure}

\subsection{Step 7: Rating Occurrence}\label{step-7-rating-occurrence}

\textbf{Occurrence (O)} rates the likelihood that the cause will occur on a scale of 1-10.

\label{tab:occurrence-scale}Occurrence Rating Scale (O)

Rating

Likelihood

Criteria

Approximate Rate

Cpk Equivalent

10

Very High

Failure is almost inevitable

≥100 per 1000 (≥10\%)

\textless{} 0.33

9

Very High

Failures occur very frequently

50 per 1000 (5\%)

≥ 0.33

8

High

Failures occur frequently

20 per 1000 (2\%)

≥ 0.51

7

High

Failures occur on regular basis

10 per 1000 (1\%)

≥ 0.67

6

Moderate

Occasional failures

5 per 1000 (0.5\%)

≥ 0.83

5

Moderate

Occasional failures expected

2 per 1000 (0.2\%)

≥ 1.00

4

Low

Relatively few failures

1 per 1000 (0.1\%)

≥ 1.17

3

Low

Few failures

0.5 per 1000 (500 ppm)

≥ 1.33

2

Remote

Failure unlikely

0.1 per 1000 (100 ppm)

≥ 1.50

1

Remote

Failure nearly impossible

≤0.01 per 1000 (≤10 ppm)

≥ 1.67

\subsection{Step 8: Current Controls and Detection Ratings}\label{step-8-current-controls-and-detection-ratings}

Identify current controls and rate their ability to detect the failure or cause.

\textbf{Types of Process Controls:}

\label{tab:control-types}Types of Process Controls

Control Type

Purpose

Examples

Impact on PFMEA

\textbf{Prevention Controls}

Prevent the cause from occurring

Error-proofing (poka-yoke), interlocks, maintenance programs

Reduces Occurrence rating

\textbf{Detection Controls}

Detect the failure mode or cause after it occurs

Inspection, testing, gauging, SPC monitoring

Reduces Detection rating

\textbf{Detection Rating Scale:}

\label{tab:detection-scale}Detection Rating Scale (D)

Rating

Detection

Criteria

Example Control

10

Absolute Uncertainty

No current control; cannot detect or is not analyzed

No inspection planned

9

Very Remote

Control will probably not detect

Indirect measurement only

8

Remote

Control has poor chance of detection

Visual inspection of hidden feature

7

Very Low

Control has low chance of detection

Manual 100\% visual inspection

6

Low

Control may detect

Variable gauging after operation

5

Moderate

Control has moderate chance of detection

SPC charting

4

Moderately High

Control has moderately high chance of detection

In-station gauging with feedback

3

High

Control has high chance of detection

Automated 100\% gauging

2

Very High

Control almost certain to detect

Multiple independent automated checks

1

Almost Certain

Control will detect; error-proofed

Error-proofing prevents defect creation

\begin{quote}
\textbf{Important:} Detection ratings are often \textbf{harder to reduce} than occurrence ratings. Adding inspection doesn't prevent defects --- it only catches them. Prevention is always better than detection.
\end{quote}

Question: Why are detection ratings often harder to improve?

\textbf{Detection ratings are harder to reduce because:}

\begin{enumerate}
\def\labelenumi{\arabic{enumi}.}
\tightlist
\item
  \textbf{Detection doesn't prevent:} The defect has already been made; you're just trying to find it
\item
  \textbf{Human inspection is unreliable:} Typical effectiveness is only 80-85\% for visual inspection
\item
  \textbf{100\% inspection is expensive:} Automated inspection requires significant investment
\item
  \textbf{Some defects are hard to detect:} Internal defects, intermittent problems, latent failures
\item
  \textbf{Sampling has limitations:} Even good sampling plans miss some defects
\end{enumerate}

\textbf{Better approach:} Focus on reducing occurrence through:
- Error-proofing (poka-yoke)
- Process capability improvement
- Preventive maintenance
- Better training

\subsection{Step 9: Calculating RPN and Prioritization}\label{step-9-calculating-rpn-and-prioritization}

The \textbf{Risk Priority Number (RPN)} combines all three ratings:

\[\text{RPN} = \text{Severity (S)} \times \text{Occurrence (O)} \times \text{Detection (D)}\]

\begin{Shaded}
\begin{Highlighting}[]
\CommentTok{\# Example RPN Calculations}

\CommentTok{\# Failure Mode 1: Short shot}
\NormalTok{S1 }\OtherTok{\textless{}{-}} \DecValTok{7}    \CommentTok{\# Severity: Product inoperable (assembly)}
\NormalTok{O1 }\OtherTok{\textless{}{-}} \DecValTok{5}    \CommentTok{\# Occurrence: Occasional}
\NormalTok{D1 }\OtherTok{\textless{}{-}} \DecValTok{6}    \CommentTok{\# Detection: Moderate (visual inspection)}

\NormalTok{RPN1 }\OtherTok{\textless{}{-}}\NormalTok{ S1 }\SpecialCharTok{*}\NormalTok{ O1 }\SpecialCharTok{*}\NormalTok{ D1}
\FunctionTok{cat}\NormalTok{(}\StringTok{"Short shot RPN:"}\NormalTok{, RPN1, }\StringTok{"}\SpecialCharTok{\textbackslash{}n}\StringTok{"}\NormalTok{)}
\end{Highlighting}
\end{Shaded}

\begin{verbatim}
## Short shot RPN: 210
\end{verbatim}

\begin{Shaded}
\begin{Highlighting}[]
\CommentTok{\# Failure Mode 2: Flash}
\NormalTok{S2 }\OtherTok{\textless{}{-}} \DecValTok{4}    \CommentTok{\# Severity: Appearance issue}
\NormalTok{O2 }\OtherTok{\textless{}{-}} \DecValTok{6}    \CommentTok{\# Occurrence: Regular}
\NormalTok{D2 }\OtherTok{\textless{}{-}} \DecValTok{3}    \CommentTok{\# Detection: High (easy to see)}

\NormalTok{RPN2 }\OtherTok{\textless{}{-}}\NormalTok{ S2 }\SpecialCharTok{*}\NormalTok{ O2 }\SpecialCharTok{*}\NormalTok{ D2}
\FunctionTok{cat}\NormalTok{(}\StringTok{"Flash RPN:"}\NormalTok{, RPN2, }\StringTok{"}\SpecialCharTok{\textbackslash{}n}\StringTok{"}\NormalTok{)}
\end{Highlighting}
\end{Shaded}

\begin{verbatim}
## Flash RPN: 72
\end{verbatim}

\begin{Shaded}
\begin{Highlighting}[]
\CommentTok{\# Failure Mode 3: Contamination}
\NormalTok{S3 }\OtherTok{\textless{}{-}} \DecValTok{9}    \CommentTok{\# Severity: Safety concern (medical device)}
\NormalTok{O3 }\OtherTok{\textless{}{-}} \DecValTok{3}    \CommentTok{\# Occurrence: Low}
\NormalTok{D3 }\OtherTok{\textless{}{-}} \DecValTok{7}    \CommentTok{\# Detection: Low (internal contamination)}

\NormalTok{RPN3 }\OtherTok{\textless{}{-}}\NormalTok{ S3 }\SpecialCharTok{*}\NormalTok{ O3 }\SpecialCharTok{*}\NormalTok{ D3}
\FunctionTok{cat}\NormalTok{(}\StringTok{"Contamination RPN:"}\NormalTok{, RPN3, }\StringTok{"}\SpecialCharTok{\textbackslash{}n}\StringTok{"}\NormalTok{)}
\end{Highlighting}
\end{Shaded}

\begin{verbatim}
## Contamination RPN: 189
\end{verbatim}

\begin{Shaded}
\begin{Highlighting}[]
\CommentTok{\# Comparison}
\FunctionTok{cat}\NormalTok{(}\StringTok{"}\SpecialCharTok{\textbackslash{}n}\StringTok{{-}{-}{-} Priority Order by RPN {-}{-}{-}}\SpecialCharTok{\textbackslash{}n}\StringTok{"}\NormalTok{)}
\end{Highlighting}
\end{Shaded}

\begin{verbatim}
## 
## --- Priority Order by RPN ---
\end{verbatim}

\begin{Shaded}
\begin{Highlighting}[]
\FunctionTok{cat}\NormalTok{(}\StringTok{"1. Short shot (RPN ="}\NormalTok{, RPN1, }\StringTok{")}\SpecialCharTok{\textbackslash{}n}\StringTok{"}\NormalTok{)}
\end{Highlighting}
\end{Shaded}

\begin{verbatim}
## 1. Short shot (RPN = 210 )
\end{verbatim}

\begin{Shaded}
\begin{Highlighting}[]
\FunctionTok{cat}\NormalTok{(}\StringTok{"2. Contamination (RPN ="}\NormalTok{, RPN3, }\StringTok{")}\SpecialCharTok{\textbackslash{}n}\StringTok{"}\NormalTok{)}
\end{Highlighting}
\end{Shaded}

\begin{verbatim}
## 2. Contamination (RPN = 189 )
\end{verbatim}

\begin{Shaded}
\begin{Highlighting}[]
\FunctionTok{cat}\NormalTok{(}\StringTok{"3. Flash (RPN ="}\NormalTok{, RPN2, }\StringTok{")}\SpecialCharTok{\textbackslash{}n}\StringTok{"}\NormalTok{)}
\end{Highlighting}
\end{Shaded}

\begin{verbatim}
## 3. Flash (RPN = 72 )
\end{verbatim}

\textbf{RPN Action Thresholds (Traditional):}

\label{tab:rpn-thresholds}RPN Action Thresholds

RPN Range

Risk Level

Action Required

1-50

Low

Monitor; no immediate action

51-100

Moderate

Consider action if resources available

101-200

Significant

Action required within planning cycle

201-500

High

Immediate action required

501-1000

Critical

Stop production; immediate corrective action

Interactive Exercise: Calculate RPN

\textbf{Calculate the RPN for each failure mode and rank them:}

\begin{longtable}[]{@{}lllll@{}}
\toprule\noalign{}
Failure Mode & Severity & Occurrence & Detection & RPN \\
\midrule\noalign{}
\endhead
\bottomrule\noalign{}
\endlastfoot
A: Dimension out of spec & 6 & 4 & 5 & ? \\
B: Surface scratch & 3 & 7 & 2 & ? \\
C: Missing component & 8 & 2 & 8 & ? \\
D: Wrong material & 9 & 3 & 6 & ? \\
\end{longtable}

\textbf{Answers:}
- A: 6 × 4 × 5 = \textbf{120}
- B: 3 × 7 × 2 = \textbf{42}
- C: 8 × 2 × 8 = \textbf{128}
- D: 9 × 3 × 6 = \textbf{162}

\textbf{Priority Order:} D (162) \textgreater{} C (128) \textgreater{} A (120) \textgreater{} B (42)

\textbf{Question:} Should we always address the highest RPN first? See limitations discussion below.

\subsection{Limitations of RPN}\label{limitations-of-rpn}

RPN has known limitations that must be understood:

\begin{figure}

{\centering \includegraphics{introduction_files/figure-latex/rpn-limitations-1} 

}

\caption{RPN Limitation Example}\label{fig:rpn-limitations}
\end{figure}

\textbf{Key RPN Limitations:}

\begin{enumerate}
\def\labelenumi{\arabic{enumi}.}
\tightlist
\item
  \textbf{Same RPN, different risk:} 10×1×4 = 40, but 2×10×2 = 40
\item
  \textbf{Severity often overlooked:} High severity should always trigger action
\item
  \textbf{Arbitrary thresholds:} Why is 100 the magic number?
\item
  \textbf{Non-linear scale:} Difference between 2 and 3 is not same as 8 and 9
\end{enumerate}

\subsection{Action Priority (AP) Method --- AIAG-VDA Alternative}\label{action-priority-ap-method-aiag-vda-alternative}

The 2019 AIAG-VDA FMEA Handbook introduced the \textbf{Action Priority (AP)} method as an alternative to RPN.

\label{tab:ap-method}Action Priority (AP) Method

Priority

Criteria

Action Required

\textbf{HIGH}

Severity = 9 or 10 (any O, any D)

Must take action to improve prevention/detection

\textbf{HIGH}

Severity = 8 AND Occurrence ≥ 4

Must take action

\textbf{MEDIUM}

Severity = 7 AND Occurrence ≥ 5

Should take action

\textbf{MEDIUM}

Severity = 5-6 AND Occurrence ≥ 6 AND Detection ≥ 5

Should take action

\textbf{LOW}

All other combinations

May take action; document rationale if no action

\subsection{Step 10: Recommended Actions and Follow-Up}\label{step-10-recommended-actions-and-follow-up}

For high-priority items, develop specific recommended actions.

\label{tab:recommended-actions}Step 10: Recommended Actions with Targets

Failure Mode

Current RPN

Recommended Action

Responsible

Target Date

Target S

Target O

Target D

Target RPN

Short shot

210

Install shot size monitoring with automatic reject

Process Engineer

2024-03-15

7

3

2

42

Short shot

210

Implement preventive maintenance for nozzle cleaning

Maintenance

2024-02-28

7

2

6

84

Short shot

210

Add mold temperature monitoring with alarm

Quality Engineer

2024-03-01

7

5

4

140

\begin{center}\rule{0.5\linewidth}{0.5pt}\end{center}

\section{Complete PFMEA Example: Injection Molding}\label{complete-pfmea-example-injection-molding}

Let's work through a complete PFMEA for an injection molding process step by step.

\label{tab:complete-example}Complete PFMEA Example: Injection Molding Process

Step

Function

Failure Mode

Effect

S

Cause

O

Control

D

RPN

\begin{enumerate}
\def\labelenumi{\arabic{enumi}.}
\setcounter{enumi}{2}
\tightlist
\item
  Inject

  Fill mold completely

  Short shot

  Part scrapped, line stoppage

  7

  Insufficient shot size

  5

  Visual inspection

  6

  210

  \begin{enumerate}
  \def\labelenumii{\arabic{enumii}.}
  \setcounter{enumii}{2}
  \tightlist
  \item
    Inject

    Fill mold completely

    Flash

    Secondary trim required

    4

    Excessive pressure

    6

    Visual inspection

    3

    72

    \begin{enumerate}
    \def\labelenumiii{\arabic{enumiii}.}
    \setcounter{enumiii}{2}
    \tightlist
    \item
      Inject

      Fill mold completely

      Air entrapment

      Weak area, visual defect

      6

      Blocked vent

      4

      First article inspection

      5

      120

      \begin{enumerate}
      \def\labelenumiv{\arabic{enumiv}.}
      \setcounter{enumiv}{3}
      \tightlist
      \item
        Cool

        Solidify uniformly

        Warpage

        Dimensional rejection

        7

        Uneven cooling

        4

        CMM sample check

        4

        112

        \begin{enumerate}
        \def\labelenumv{\arabic{enumv}.}
        \tightlist
        \item
          Cool

          Solidify uniformly

          Sink marks

          Visual defect

          5

          Thick section, short pack

          5

          Visual inspection

          4

          100

          \begin{enumerate}
          \def\labelenumvi{\arabic{enumvi}.}
          \tightlist
          \item
            Eject

            Remove without damage

            Stuck in mold

            Part damage, mold damage

            8

            Insufficient draft angle

            3

            Ejector pressure monitor

            5

            120
          \end{enumerate}
        \end{enumerate}
      \end{enumerate}
    \end{enumerate}
  \end{enumerate}
\end{enumerate}

\begin{Shaded}
\begin{Highlighting}[]
\CommentTok{\# Improvement Analysis for Short Shot (Highest RPN = 210)}

\FunctionTok{cat}\NormalTok{(}\StringTok{"=== CURRENT STATE ===}\SpecialCharTok{\textbackslash{}n}\StringTok{"}\NormalTok{)}
\end{Highlighting}
\end{Shaded}

\begin{verbatim}
## === CURRENT STATE ===
\end{verbatim}

\begin{Shaded}
\begin{Highlighting}[]
\NormalTok{S\_current }\OtherTok{\textless{}{-}} \DecValTok{7}  \CommentTok{\# Severity (cannot change {-} inherent to failure)}
\NormalTok{O\_current }\OtherTok{\textless{}{-}} \DecValTok{5}  \CommentTok{\# Occurrence}
\NormalTok{D\_current }\OtherTok{\textless{}{-}} \DecValTok{6}  \CommentTok{\# Detection}
\NormalTok{RPN\_current }\OtherTok{\textless{}{-}}\NormalTok{ S\_current }\SpecialCharTok{*}\NormalTok{ O\_current }\SpecialCharTok{*}\NormalTok{ D\_current}
\FunctionTok{cat}\NormalTok{(}\StringTok{"Current RPN:"}\NormalTok{, RPN\_current, }\StringTok{"(S=7, O=5, D=6)}\SpecialCharTok{\textbackslash{}n\textbackslash{}n}\StringTok{"}\NormalTok{)}
\end{Highlighting}
\end{Shaded}

\begin{verbatim}
## Current RPN: 210 (S=7, O=5, D=6)
\end{verbatim}

\begin{Shaded}
\begin{Highlighting}[]
\FunctionTok{cat}\NormalTok{(}\StringTok{"=== IMPROVEMENT OPTIONS ===}\SpecialCharTok{\textbackslash{}n\textbackslash{}n}\StringTok{"}\NormalTok{)}
\end{Highlighting}
\end{Shaded}

\begin{verbatim}
## === IMPROVEMENT OPTIONS ===
\end{verbatim}

\begin{Shaded}
\begin{Highlighting}[]
\CommentTok{\# Option 1: Improve Detection}
\FunctionTok{cat}\NormalTok{(}\StringTok{"Option 1: Add automatic shot weight monitoring}\SpecialCharTok{\textbackslash{}n}\StringTok{"}\NormalTok{)}
\end{Highlighting}
\end{Shaded}

\begin{verbatim}
## Option 1: Add automatic shot weight monitoring
\end{verbatim}

\begin{Shaded}
\begin{Highlighting}[]
\NormalTok{D\_option1 }\OtherTok{\textless{}{-}} \DecValTok{2}  \CommentTok{\# Automated detection}
\NormalTok{RPN\_option1 }\OtherTok{\textless{}{-}}\NormalTok{ S\_current }\SpecialCharTok{*}\NormalTok{ O\_current }\SpecialCharTok{*}\NormalTok{ D\_option1}
\FunctionTok{cat}\NormalTok{(}\StringTok{"New RPN:"}\NormalTok{, RPN\_option1, }\StringTok{"(S=7, O=5, D=2)}\SpecialCharTok{\textbackslash{}n}\StringTok{"}\NormalTok{)}
\end{Highlighting}
\end{Shaded}

\begin{verbatim}
## New RPN: 70 (S=7, O=5, D=2)
\end{verbatim}

\begin{Shaded}
\begin{Highlighting}[]
\FunctionTok{cat}\NormalTok{(}\StringTok{"Reduction:"}\NormalTok{, }\FunctionTok{round}\NormalTok{((}\DecValTok{1} \SpecialCharTok{{-}}\NormalTok{ RPN\_option1}\SpecialCharTok{/}\NormalTok{RPN\_current)}\SpecialCharTok{*}\DecValTok{100}\NormalTok{), }\StringTok{"\%}\SpecialCharTok{\textbackslash{}n\textbackslash{}n}\StringTok{"}\NormalTok{)}
\end{Highlighting}
\end{Shaded}

\begin{verbatim}
## Reduction: 67 %
\end{verbatim}

\begin{Shaded}
\begin{Highlighting}[]
\CommentTok{\# Option 2: Improve Occurrence (Prevention)}
\FunctionTok{cat}\NormalTok{(}\StringTok{"Option 2: Add screw position monitoring with feedback control}\SpecialCharTok{\textbackslash{}n}\StringTok{"}\NormalTok{)}
\end{Highlighting}
\end{Shaded}

\begin{verbatim}
## Option 2: Add screw position monitoring with feedback control
\end{verbatim}

\begin{Shaded}
\begin{Highlighting}[]
\NormalTok{O\_option2 }\OtherTok{\textless{}{-}} \DecValTok{2}  \CommentTok{\# Much lower occurrence}
\NormalTok{RPN\_option2 }\OtherTok{\textless{}{-}}\NormalTok{ S\_current }\SpecialCharTok{*}\NormalTok{ O\_option2 }\SpecialCharTok{*}\NormalTok{ D\_current}
\FunctionTok{cat}\NormalTok{(}\StringTok{"New RPN:"}\NormalTok{, RPN\_option2, }\StringTok{"(S=7, O=2, D=6)}\SpecialCharTok{\textbackslash{}n}\StringTok{"}\NormalTok{)}
\end{Highlighting}
\end{Shaded}

\begin{verbatim}
## New RPN: 84 (S=7, O=2, D=6)
\end{verbatim}

\begin{Shaded}
\begin{Highlighting}[]
\FunctionTok{cat}\NormalTok{(}\StringTok{"Reduction:"}\NormalTok{, }\FunctionTok{round}\NormalTok{((}\DecValTok{1} \SpecialCharTok{{-}}\NormalTok{ RPN\_option2}\SpecialCharTok{/}\NormalTok{RPN\_current)}\SpecialCharTok{*}\DecValTok{100}\NormalTok{), }\StringTok{"\%}\SpecialCharTok{\textbackslash{}n\textbackslash{}n}\StringTok{"}\NormalTok{)}
\end{Highlighting}
\end{Shaded}

\begin{verbatim}
## Reduction: 60 %
\end{verbatim}

\begin{Shaded}
\begin{Highlighting}[]
\CommentTok{\# Option 3: Both improvements}
\FunctionTok{cat}\NormalTok{(}\StringTok{"Option 3: Both improvements combined}\SpecialCharTok{\textbackslash{}n}\StringTok{"}\NormalTok{)}
\end{Highlighting}
\end{Shaded}

\begin{verbatim}
## Option 3: Both improvements combined
\end{verbatim}

\begin{Shaded}
\begin{Highlighting}[]
\NormalTok{RPN\_option3 }\OtherTok{\textless{}{-}}\NormalTok{ S\_current }\SpecialCharTok{*}\NormalTok{ O\_option2 }\SpecialCharTok{*}\NormalTok{ D\_option1}
\FunctionTok{cat}\NormalTok{(}\StringTok{"New RPN:"}\NormalTok{, RPN\_option3, }\StringTok{"(S=7, O=2, D=2)}\SpecialCharTok{\textbackslash{}n}\StringTok{"}\NormalTok{)}
\end{Highlighting}
\end{Shaded}

\begin{verbatim}
## New RPN: 28 (S=7, O=2, D=2)
\end{verbatim}

\begin{Shaded}
\begin{Highlighting}[]
\FunctionTok{cat}\NormalTok{(}\StringTok{"Reduction:"}\NormalTok{, }\FunctionTok{round}\NormalTok{((}\DecValTok{1} \SpecialCharTok{{-}}\NormalTok{ RPN\_option3}\SpecialCharTok{/}\NormalTok{RPN\_current)}\SpecialCharTok{*}\DecValTok{100}\NormalTok{), }\StringTok{"\%}\SpecialCharTok{\textbackslash{}n}\StringTok{"}\NormalTok{)}
\end{Highlighting}
\end{Shaded}

\begin{verbatim}
## Reduction: 87 %
\end{verbatim}

\begin{figure}

{\centering \includegraphics{introduction_files/figure-latex/rpn-comparison-viz-1} 

}

\caption{RPN Improvement Comparison}\label{fig:rpn-comparison-viz}
\end{figure}

\begin{center}\rule{0.5\linewidth}{0.5pt}\end{center}

\section{Industry-Specific PFMEA Examples}\label{industry-specific-pfmea-examples}

\subsection{Example: Welding Process}\label{example-welding-process}

\label{tab:welding-pfmea}PFMEA Example: Welding Process

Failure Mode

Effect

S

Cause

O

Current Control

D

RPN

Incomplete fusion

Weak joint, potential crack

9

Low heat input, contamination

4

Visual + X-ray sample

4

144

Porosity

Leak path, strength reduction

8

Shielding gas issue, moisture

5

Visual inspection

5

200

Undercut

Stress concentration, fatigue failure

7

Excessive current, wrong angle

4

Visual inspection

4

112

Spatter

Appearance, cleanup required

3

Wrong parameters, contaminated wire

6

Visual inspection

2

36

Distortion

Dimensional non-conformance

6

Excessive heat, improper sequence

4

CMM check

5

120

\subsection{Example: Heat Treatment Process}\label{example-heat-treatment-process}

\label{tab:heat-treat-pfmea}PFMEA Example: Heat Treatment Process

Failure Mode

Effect

S

Cause

O

Current Control

D

RPN

Under-hardened

Premature wear, field failure

8

Low temp, short time

4

Hardness test sample

4

128

Over-hardened/Brittle

Part cracks under load

9

High temp, long time

3

Hardness test sample

4

108

Surface decarburization

Soft surface, wear

7

Wrong atmosphere

4

Surface hardness check

5

140

Distortion/Warpage

Dimensional rejection

6

Uneven heating, rapid quench

5

CMM sample

4

120

Quench cracking

Scrap, safety hazard

10

Thermal shock, design

2

Visual inspection

6

120

\subsection{Example: Food Processing (Pasteurization)}\label{example-food-processing-pasteurization}

\label{tab:food-pfmea}PFMEA Example: Pasteurization Process

Failure Mode

Effect

S

Cause

O

Current Control

D

RPN

Under-processing

Pathogen survival, food safety

10

Equipment malfunction, calibration

2

Continuous temp monitoring + chart recorder

2

40

Over-processing

Nutrient loss, off-flavor

5

Control system error

3

Temperature alarm

3

45

Temperature deviation

Inconsistent kill, shelf life

8

Sensor drift, flow variation

3

Calibration schedule

3

72

Time deviation

Process variation

6

Timer malfunction

3

Time verification

3

54

Cross-contamination

Product recall, illness

10

Gasket leak, CIP failure

2

CIP verification + testing

4

80

Interactive Exercise: Create a PFMEA

\textbf{Create a simple PFMEA for a coffee brewing process in a commercial setting.}

\textbf{Process Steps:}
1. Fill water reservoir
2. Load coffee grounds
3. Brew coffee
4. Dispense into cup

\textbf{Identify at least 4 failure modes with S, O, D ratings and RPN.}

\textbf{Example Answer:}

\begin{longtable}[]{@{}
  >{\raggedright\arraybackslash}p{(\linewidth - 16\tabcolsep) * \real{0.1034}}
  >{\raggedright\arraybackslash}p{(\linewidth - 16\tabcolsep) * \real{0.2414}}
  >{\raggedright\arraybackslash}p{(\linewidth - 16\tabcolsep) * \real{0.1379}}
  >{\raggedright\arraybackslash}p{(\linewidth - 16\tabcolsep) * \real{0.0517}}
  >{\raggedright\arraybackslash}p{(\linewidth - 16\tabcolsep) * \real{0.1207}}
  >{\raggedright\arraybackslash}p{(\linewidth - 16\tabcolsep) * \real{0.0517}}
  >{\raggedright\arraybackslash}p{(\linewidth - 16\tabcolsep) * \real{0.1552}}
  >{\raggedright\arraybackslash}p{(\linewidth - 16\tabcolsep) * \real{0.0517}}
  >{\raggedright\arraybackslash}p{(\linewidth - 16\tabcolsep) * \real{0.0862}}@{}}
\toprule\noalign{}
\begin{minipage}[b]{\linewidth}\raggedright
Step
\end{minipage} & \begin{minipage}[b]{\linewidth}\raggedright
Failure Mode
\end{minipage} & \begin{minipage}[b]{\linewidth}\raggedright
Effect
\end{minipage} & \begin{minipage}[b]{\linewidth}\raggedright
S
\end{minipage} & \begin{minipage}[b]{\linewidth}\raggedright
Cause
\end{minipage} & \begin{minipage}[b]{\linewidth}\raggedright
O
\end{minipage} & \begin{minipage}[b]{\linewidth}\raggedright
Control
\end{minipage} & \begin{minipage}[b]{\linewidth}\raggedright
D
\end{minipage} & \begin{minipage}[b]{\linewidth}\raggedright
RPN
\end{minipage} \\
\midrule\noalign{}
\endhead
\bottomrule\noalign{}
\endlastfoot
1. Fill water & Wrong water level & Weak/strong coffee & 4 & No level indicator & 5 & Visual check & 5 & 100 \\
2. Load grounds & Wrong amount & Weak/strong coffee & 4 & No measuring & 6 & Visual check & 6 & 144 \\
3. Brew & Temperature too low & Under-extracted & 6 & Heating element worn & 3 & Temp display & 4 & 72 \\
4. Dispense & Overflow & Burn hazard, mess & 8 & Cup sensor failure & 2 & Cup sensor & 4 & 64 \\
\end{longtable}

\textbf{Priority:} Wrong amount (144) \textgreater{} Water level (100) \textgreater{} Temperature (72) \textgreater{} Overflow (64)

\begin{center}\rule{0.5\linewidth}{0.5pt}\end{center}

\section{Integration with Other Quality Tools}\label{integration-with-other-quality-tools}

PFMEA doesn't exist in isolation --- it connects to other quality tools and systems.

\begin{figure}

{\centering \includegraphics{introduction_files/figure-latex/integration-diagram-1} 

}

\caption{PFMEA Integration with Quality Tools}\label{fig:integration-diagram}
\end{figure}

\subsection{From PFMEA to Control Plan}\label{from-pfmea-to-control-plan}

The PFMEA directly feeds the \textbf{Control Plan} --- a document that specifies what controls are needed at each process step.

\label{tab:control-plan-link}PFMEA to Control Plan Linkage

From PFMEA

To Control Plan

Process Step

Operation/Step

Failure Mode

Characteristic to Control

Severity

Classification (Critical/Significant)

Cause

Process Parameter

Current Control

Control Method

Detection

Reaction Plan

\begin{center}\rule{0.5\linewidth}{0.5pt}\end{center}

\section{Common Pitfalls and Best Practices}\label{common-pitfalls-and-best-practices}

\subsection{Common Pitfalls to Avoid}\label{common-pitfalls-to-avoid}

\label{tab:pitfalls}Common PFMEA Pitfalls and Solutions

Pitfall

Problem

Solution

\textbf{Paperwork Exercise}

PFMEA done only to satisfy customer requirement

Conduct meaningful analysis with intent to improve

\textbf{One-Person Show}

Engineer completes alone without team input

Require cross-functional team participation

\textbf{Optimistic Detection}

Overestimating effectiveness of current controls

Validate detection effectiveness with data

\textbf{Static Document}

Never updated after process changes

Review and update after any process change

\textbf{RPN Obsession}

Focusing only on RPN numbers, missing high-severity items

Also consider severity alone; use AP method

\textbf{Vague Failure Modes}

Failure modes too general to be actionable

Be specific: `hole undersize' not `bad hole'

\subsection{Best Practices}\label{best-practices}

\label{tab:best-practices}PFMEA Best Practices

Practice

Description

\textbf{Start Early}

Begin PFMEA during process design, not after problems occur

\textbf{Include Operators}

Operators know failures that never get documented

\textbf{Use Historical Data}

Use warranty data, scrap reports, customer complaints

\textbf{Be Specific}

Specific failure modes lead to specific actions

\textbf{Follow Up}

Verify that actions were implemented and effective

\textbf{Living Document}

Review and update regularly (minimum annually)

\begin{center}\rule{0.5\linewidth}{0.5pt}\end{center}

\section{Case Study: Preventing a Major Quality Issue}\label{case-study-preventing-a-major-quality-issue}

\subsection{Background}\label{background}

\textbf{Company:} Automotive brake component manufacturer

\textbf{Product:} Brake caliper mounting bracket

\textbf{Situation:} New manufacturing process being implemented for a high-volume contract

\subsection{PFMEA Analysis}\label{pfmea-analysis}

During the PFMEA session, the team identified a critical failure mode:

\label{tab:case-study-finding}Case Study: Critical Failure Mode Identified

Element

Analysis

Process Step

Drill mounting holes

Function

Create holes at 45.0 ±0.1mm spacing

Failure Mode

Hole spacing out of tolerance

Effect

Caliper misalignment, brake performance affected

Severity

9 (safety-related)

Cause

Fixture wear, thermal expansion

Occurrence

4 (occasional)

Current Control

CMM check 5 per shift

Detection

5 (moderate sampling)

RPN

180

\subsection{Actions Taken}\label{actions-taken}

Based on the PFMEA, the team implemented:

\begin{enumerate}
\def\labelenumi{\arabic{enumi}.}
\tightlist
\item
  \textbf{In-process gauging:} Automated measurement after every part (D: 5 → 2)
\item
  \textbf{Fixture monitoring:} Temperature compensation and wear tracking (O: 4 → 2)
\item
  \textbf{Enhanced fixture design:} Carbide wear surfaces (O: 4 → 2)
\end{enumerate}

\begin{Shaded}
\begin{Highlighting}[]
\CommentTok{\# Before improvements}
\NormalTok{S\_before }\OtherTok{\textless{}{-}} \DecValTok{9}
\NormalTok{O\_before }\OtherTok{\textless{}{-}} \DecValTok{4}
\NormalTok{D\_before }\OtherTok{\textless{}{-}} \DecValTok{5}
\NormalTok{RPN\_before }\OtherTok{\textless{}{-}}\NormalTok{ S\_before }\SpecialCharTok{*}\NormalTok{ O\_before }\SpecialCharTok{*}\NormalTok{ D\_before}
\FunctionTok{cat}\NormalTok{(}\StringTok{"Before RPN:"}\NormalTok{, RPN\_before, }\StringTok{"}\SpecialCharTok{\textbackslash{}n}\StringTok{"}\NormalTok{)}
\end{Highlighting}
\end{Shaded}

\begin{verbatim}
## Before RPN: 180
\end{verbatim}

\begin{Shaded}
\begin{Highlighting}[]
\CommentTok{\# After improvements}
\NormalTok{S\_after }\OtherTok{\textless{}{-}} \DecValTok{9}    \CommentTok{\# Severity unchanged (inherent)}
\NormalTok{O\_after }\OtherTok{\textless{}{-}} \DecValTok{2}    \CommentTok{\# Reduced through fixture improvements}
\NormalTok{D\_after }\OtherTok{\textless{}{-}} \DecValTok{2}    \CommentTok{\# Reduced through 100\% gauging}
\NormalTok{RPN\_after }\OtherTok{\textless{}{-}}\NormalTok{ S\_after }\SpecialCharTok{*}\NormalTok{ O\_after }\SpecialCharTok{*}\NormalTok{ D\_after}
\FunctionTok{cat}\NormalTok{(}\StringTok{"After RPN:"}\NormalTok{, RPN\_after, }\StringTok{"}\SpecialCharTok{\textbackslash{}n}\StringTok{"}\NormalTok{)}
\end{Highlighting}
\end{Shaded}

\begin{verbatim}
## After RPN: 36
\end{verbatim}

\begin{Shaded}
\begin{Highlighting}[]
\FunctionTok{cat}\NormalTok{(}\StringTok{"Reduction:"}\NormalTok{, }\FunctionTok{round}\NormalTok{((}\DecValTok{1} \SpecialCharTok{{-}}\NormalTok{ RPN\_after}\SpecialCharTok{/}\NormalTok{RPN\_before)}\SpecialCharTok{*}\DecValTok{100}\NormalTok{), }\StringTok{"\%}\SpecialCharTok{\textbackslash{}n}\StringTok{"}\NormalTok{)}
\end{Highlighting}
\end{Shaded}

\begin{verbatim}
## Reduction: 80 %
\end{verbatim}

\subsection{Results}\label{results}

\begin{itemize}
\tightlist
\item
  \textbf{Zero field failures} related to hole spacing after 18 months of production
\item
  \textbf{Estimated avoided cost:} \$2.4 million in potential recall and warranty
\item
  \textbf{Investment:} \$85,000 for gauging and fixture upgrades
\item
  \textbf{ROI:} 28:1
\end{itemize}

\begin{quote}
\textbf{Key Lesson:} The PFMEA investment of \$85,000 prevented a potential \$2.4 million problem. This is the power of proactive risk management.
\end{quote}

\begin{center}\rule{0.5\linewidth}{0.5pt}\end{center}

\section{Summary}\label{summary-6}

\label{tab:summary-table-ch8}PFMEA Key Concepts Summary

Topic

Key Points

\textbf{PFMEA Purpose}

Prevent defects before they occur; proactive vs.~reactive quality

\textbf{Key Ratings}

Severity (effect), Occurrence (likelihood), Detection (control effectiveness)

\textbf{RPN Calculation}

RPN = S × O × D; Range 1-1000; Higher = More risk

\textbf{Prioritization}

Address high RPN first, but always act on high severity (9-10)

\textbf{Action Priority}

AIAG-VDA method: High/Medium/Low based on S-O-D combination

\textbf{Best Practice}

Cross-functional team, living document, follow-up on actions

\begin{center}\rule{0.5\linewidth}{0.5pt}\end{center}

\section{Review Questions}\label{review-questions-7}

\subsection{Conceptual Questions}\label{conceptual-questions}

Question 1: What is the difference between a failure mode and a failure effect?

\textbf{Failure Mode:} How the process fails to perform its intended function
- Example: ``Hole drilled undersize''
- Example: ``Temperature too low''
- Example: ``Missing component''

\textbf{Failure Effect:} The consequence of the failure mode
- Example: ``Part rejected at assembly'' (effect of undersize hole)
- Example: ``Product not properly cured'' (effect of low temperature)
- Example: ``Product doesn't function'' (effect of missing component)

\textbf{Key Relationship:} One failure mode can have multiple effects at different levels (local, downstream, customer).

Question 2: Why should PFMEA be conducted by a cross-functional team?

\textbf{Reasons for Cross-Functional Team:}

\begin{enumerate}
\def\labelenumi{\arabic{enumi}.}
\tightlist
\item
  \textbf{Diverse expertise:} No single person understands all aspects
\item
  \textbf{Multiple perspectives:} Quality sees defects differently than production
\item
  \textbf{Historical knowledge:} Operators know undocumented problems
\item
  \textbf{Buy-in:} People support what they help create
\item
  \textbf{System thinking:} Process interactions become visible
\item
  \textbf{Better solutions:} Different functions propose different improvements
\end{enumerate}

\textbf{Typical team composition:}
- Process Engineer (owner)
- Quality Engineer
- Production Supervisor/Operator
- Maintenance Technician
- Design Engineer
- Supplier Representative (when applicable)

Question 3: Explain why detection ratings are often harder to reduce than occurrence ratings.

\textbf{Detection is harder to reduce because:}

\begin{enumerate}
\def\labelenumi{\arabic{enumi}.}
\tightlist
\item
  \textbf{Detection doesn't prevent:} Defect already exists; just trying to find it
\item
  \textbf{Human inspection is unreliable:} 80-85\% effectiveness typical
\item
  \textbf{100\% inspection is expensive:} Requires automation investment
\item
  \textbf{Some defects are hidden:} Internal defects, intermittent failures
\item
  \textbf{Sampling limitations:} Cannot catch all defects with sampling
\end{enumerate}

\textbf{Better approach:} Focus on occurrence reduction:
- Error-proofing (poka-yoke)
- Process capability improvement
- Preventive maintenance
- Better training

\textbf{Example:} Reducing drill wear (occurrence) is more effective than adding more inspection (detection).

Question 4: At what point in process development should PFMEA ideally be conducted?

\textbf{Ideal Timing:} During process design, before production begins

\textbf{Why early is better:}
- Changes cost 1x during design
- Changes cost 10x during production
- Changes cost 100x after customer delivery

\textbf{PFMEA Timing Points:}
1. \textbf{New process development:} Before equipment is ordered
2. \textbf{New product introduction:} During process planning
3. \textbf{Process changes:} Before implementing changes
4. \textbf{Quality issues:} When problems indicate process risk
5. \textbf{Regular review:} Annually at minimum

\textbf{Key Principle:} The earlier PFMEA is conducted, the more opportunity exists to design out failures rather than inspect them out.

\subsection{Calculation Questions}\label{calculation-questions}

Question 5: Given S=8, O=6, D=4, calculate RPN. If improved controls reduce detection to 2, what is the new RPN?

\textbf{Initial Calculation:}

\begin{verbatim}
RPN = S × O × D
RPN = 8 × 6 × 4 = 192
\end{verbatim}

\textbf{After Detection Improvement:}

\begin{verbatim}
New RPN = 8 × 6 × 2 = 96
\end{verbatim}

\textbf{Analysis:}
- Original RPN: 192 (Significant risk, action required)
- New RPN: 96 (Moderate risk)
- Reduction: 50\%

\textbf{Note:} Severity (8) remains unchanged because it's determined by the effect, not by process controls. Only occurrence and detection can be reduced through process improvements.

Question 6: Two failure modes have RPNs of 120 and 150. Should the higher RPN always be addressed first?

\textbf{Answer:} Not necessarily!

\textbf{Consider the following scenarios:}

\textbf{Scenario A:}
- Failure Mode 1: RPN = 150 (S=3, O=10, D=5) --- Minor cosmetic issue
- Failure Mode 2: RPN = 120 (S=10, O=3, D=4) --- Safety hazard

\textbf{Priority:} Address Failure Mode 2 first despite lower RPN because:
- Severity 10 indicates safety hazard
- High-severity items should always be addressed regardless of RPN
- The AP (Action Priority) method would classify this as HIGH priority

\textbf{When to prioritize lower RPN:}
- Higher severity (especially 9-10)
- Regulatory or safety implications
- Customer-critical characteristic
- Easier/faster to implement

\textbf{Key Lesson:} RPN is a prioritization tool, not the only decision criterion. Use engineering judgment.

\subsection{Application Questions}\label{application-questions}

Question 7: For a metal stamping operation, identify at least five potential failure modes and their likely effects.

\textbf{Metal Stamping PFMEA:}

\begin{longtable}[]{@{}
  >{\raggedright\arraybackslash}p{(\linewidth - 4\tabcolsep) * \real{0.1200}}
  >{\raggedright\arraybackslash}p{(\linewidth - 4\tabcolsep) * \real{0.5600}}
  >{\raggedright\arraybackslash}p{(\linewidth - 4\tabcolsep) * \real{0.3200}}@{}}
\toprule\noalign{}
\begin{minipage}[b]{\linewidth}\raggedright
\#
\end{minipage} & \begin{minipage}[b]{\linewidth}\raggedright
Failure Mode
\end{minipage} & \begin{minipage}[b]{\linewidth}\raggedright
Effect
\end{minipage} \\
\midrule\noalign{}
\endhead
\bottomrule\noalign{}
\endlastfoot
1 & Part not fully formed & Dimensional rejection, functional failure \\
2 & Burrs on edges & Injury hazard, assembly interference \\
3 & Cracks/splits & Structural failure, scrap \\
4 & Wrong material thickness & Dimensional/strength issues \\
5 & Mislocated features & Assembly issues, rejection \\
6 & Surface scratches & Appearance rejection, corrosion initiation \\
7 & Springback out of tolerance & Dimensional rejection \\
8 & Die mark/impression & Appearance defect \\
\end{longtable}

\textbf{Potential Causes for ``Part not fully formed'':}
- Insufficient press tonnage
- Die wear
- Material too hard
- Lubrication insufficient
- Press speed too fast

Question 8: A process has RPN of 240 (S=8, O=6, D=5). Propose three improvement strategies and calculate resulting RPN.

\textbf{Current State:} S=8, O=6, D=5, RPN=240

\textbf{Strategy 1: Improve Detection}
- Add automated inspection (100\% gauging)
- New D = 2
- New RPN = 8 × 6 × 2 = \textbf{96} (60\% reduction)

\textbf{Strategy 2: Improve Occurrence}
- Add error-proofing (poka-yoke) to prevent cause
- New O = 2
- New RPN = 8 × 2 × 5 = \textbf{80} (67\% reduction)

\textbf{Strategy 3: Combined Approach}
- Error-proofing (O → 3) + Better detection (D → 3)
- New RPN = 8 × 3 × 3 = \textbf{72} (70\% reduction)

\textbf{Recommendation:} Strategy 2 (error-proofing) is typically preferred because:
- Prevents defects from occurring
- More sustainable long-term
- Reduces inspection costs
- Better for customer confidence

\subsection{Critical Thinking Questions}\label{critical-thinking-questions}

Question 9: Discuss the limitations of using RPN as the sole prioritization method.

\textbf{RPN Limitations:}

\begin{enumerate}
\def\labelenumi{\arabic{enumi}.}
\tightlist
\item
  \textbf{Equal weighting problem:}

  \begin{itemize}
  \tightlist
  \item
    10×1×4 = 40 (safety hazard, rare)
  \item
    2×10×2 = 40 (minor, frequent)
  \item
    Same RPN, very different risk!
  \end{itemize}
\item
  \textbf{Severity often overlooked:}

  \begin{itemize}
  \tightlist
  \item
    High-severity items (9-10) should always be addressed
  \item
    RPN can hide safety-critical issues
  \end{itemize}
\item
  \textbf{Non-linear scales:}

  \begin{itemize}
  \tightlist
  \item
    Difference between S=2 and S=3 is not same as S=8 and S=9
  \item
    Multiplication assumes linear relationship
  \end{itemize}
\item
  \textbf{Arbitrary thresholds:}

  \begin{itemize}
  \tightlist
  \item
    Why is RPN=100 the ``action'' threshold?
  \item
    No scientific basis for specific cutoffs
  \end{itemize}
\item
  \textbf{Gaming the system:}

  \begin{itemize}
  \tightlist
  \item
    Teams may manipulate ratings to avoid action
  \item
    Subjectivity in ratings
  \end{itemize}
\end{enumerate}

\textbf{Better Approaches:}
- Action Priority (AP) method
- Always act on S ≥ 9 regardless of RPN
- Consider multiple criteria (cost, feasibility, timing)
- Use risk matrix visualization

Question 10: How would you handle a situation where team members disagree on severity ratings?

\textbf{Approach to Rating Disagreements:}

\textbf{1. Clarify the failure effect:}
- Ensure everyone understands the same effect
- Be specific about which customer sees which impact

\textbf{2. Reference the rating criteria:}
- Use the standardized scale consistently
- Point to specific criteria, not opinions

\textbf{3. Consider worst-case scenario:}
- PFMEA should consider worst reasonable case
- ``Could this effect happen?'' vs ``Will it happen?''

\textbf{4. Gather data:}
- Review historical data if available
- What have similar failures caused in the past?

\textbf{5. When in doubt, go higher:}
- Conservative approach is appropriate for safety
- Can always reduce later with evidence

\textbf{6. Document the discussion:}
- Record the rationale for the rating
- Note if there was disagreement and why decision was made

\textbf{7. Get external input:}
- Bring in customer perspective if needed
- Consult subject matter experts

\textbf{Final Rule:} If a safety or regulatory issue is possible, severity should be 9 or 10 regardless of probability.

\begin{center}\rule{0.5\linewidth}{0.5pt}\end{center}

\section{References}\label{references-7}

\begin{itemize}
\tightlist
\item
  AIAG \& VDA. (2019). \emph{FMEA Handbook}. Automotive Industry Action Group.
\item
  Stamatis, D.H. (2003). \emph{Failure Mode and Effect Analysis: FMEA from Theory to Execution}. ASQ Quality Press.
\item
  McDermott, R.E., Mikulak, R.J., \& Beauregard, M.R. (2009). \emph{The Basics of FMEA}. Productivity Press.
\item
  Carlson, C.S. (2012). \emph{Effective FMEAs: Achieving Safe, Reliable, and Economical Products and Processes}. Wiley.
\item
  SAE J1739. (2009). \emph{Potential Failure Mode and Effects Analysis in Design and Manufacturing}.
\item
  ISO 31000:2018. \emph{Risk management --- Guidelines}.
\item
  IATF 16949:2016. \emph{Quality management system requirements for automotive production}.
\end{itemize}

\begin{center}\rule{0.5\linewidth}{0.5pt}\end{center}

\chapter{Statistical Process Control (SPC)}\label{statistical-process-control-spc}

\begin{center}\rule{0.5\linewidth}{0.5pt}\end{center}

\section{Learning Objectives}\label{learning-objectives-8}

By the end of this chapter, you will be able to:

\begin{itemize}
\tightlist
\item
  Explain the purpose and benefits of Statistical Process Control
\item
  Distinguish between common cause and special cause variation
\item
  Select the appropriate control chart for different data types
\item
  Construct and interpret X-bar/R, X-bar/S, and Individual-MR charts
\item
  Construct and interpret attribute control charts (p, np, c, u)
\item
  Apply control chart interpretation rules (Western Electric rules)
\item
  Calculate and interpret process capability indices (Cp, Cpk, Pp, Ppk)
\item
  Respond appropriately to out-of-control signals
\end{itemize}

\begin{quote}
``In God we trust; all others must bring data.''
--- W. Edwards Deming
\end{quote}

\begin{center}\rule{0.5\linewidth}{0.5pt}\end{center}

\section{Introduction to SPC}\label{introduction-to-spc}

\subsection{What is Statistical Process Control?}\label{what-is-statistical-process-control}

\textbf{Statistical Process Control (SPC)} is a method of quality control that uses statistical methods to monitor and control a process. It helps ensure that the process operates at its full potential to produce conforming product.

\begin{figure}

{\centering \includegraphics{introduction_files/figure-latex/spc-concept-1} 

}

\caption{The SPC Philosophy: Prevention Through Monitoring}\label{fig:spc-concept}
\end{figure}

\subsection{Brief History of SPC}\label{brief-history-of-spc}

\label{tab:spc-history}Evolution of Statistical Process Control

Year

Development

Significance

1920s

Walter Shewhart develops control charts at Bell Labs

Birth of modern quality control

1930s-40s

SPC used in WWII military production

Proved effectiveness in mass production

1950s

Deming teaches SPC to Japanese industry

Foundation of Japanese quality revolution

1980s

US automotive industry adopts SPC

Response to Japanese competition

1990s

SPC becomes requirement in ISO/QS-9000

Industry-wide standardization

2000s+

Real-time SPC, automated data collection

Integration with Industry 4.0

\subsection{Why SPC Matters}\label{why-spc-matters}

\begin{figure}

{\centering \includegraphics{introduction_files/figure-latex/spc-benefits-1} 

}

\caption{Benefits of SPC Implementation}\label{fig:spc-benefits}
\end{figure}

\begin{center}\rule{0.5\linewidth}{0.5pt}\end{center}

\section{Understanding Variation}\label{understanding-variation}

\textbf{All processes exhibit variation.} No two products are ever exactly identical. The key insight of SPC is that variation comes from two fundamentally different sources, and we must respond to them differently.

\subsection{Common Cause vs.~Special Cause Variation}\label{common-cause-vs.-special-cause-variation}

\begin{figure}

{\centering \includegraphics{introduction_files/figure-latex/variation-types-1} 

}

\caption{Two Types of Variation}\label{fig:variation-types}
\end{figure}

\label{tab:variation-comparison}Common Cause vs.~Special Cause Variation

Characteristic

Common Cause Variation

Special Cause Variation

\textbf{Other Names}

Random, chance, noise, inherent

Assignable, non-random, signal

\textbf{Source}

Built into the process; many small factors

Specific, identifiable event

\textbf{Pattern}

Random, stable, predictable pattern

Non-random; shifts, trends, patterns

\textbf{Predictability}

Statistically predictable within limits

Unpredictable; something changed

\textbf{Examples}

Machine vibration, material variation, ambient temperature

Tool breakage, new operator, bad material lot, power surge

\textbf{Action Required}

Improve the system (management decision)

Find and eliminate the cause

\textbf{Who Acts}

Management --- requires system change

Operators/technicians --- local action

\subsection{The Two Mistakes in Process Control}\label{the-two-mistakes-in-process-control}

Understanding variation prevents two costly mistakes:

\begin{figure}

{\centering \includegraphics{introduction_files/figure-latex/two-mistakes-1} 

}

\caption{The Two Fundamental Mistakes}\label{fig:two-mistakes}
\end{figure}

The Funnel Experiment: Why Tampering Makes Things Worse

\textbf{Deming's Funnel Experiment} demonstrates why adjusting a stable process increases variation:

\textbf{Setup:} Drop a ball through a funnel onto a target. The ball lands near (but not exactly on) the target due to common cause variation.

\textbf{Rule 1 (Correct):} Leave the funnel alone. Variation remains constant.

\textbf{Rule 2 (Tampering):} After each drop, move the funnel to compensate for the last error.
- Result: Variation DOUBLES!

\textbf{Rule 3 (More Tampering):} Move funnel to compensate relative to the target.
- Result: Variation increases without bound!

\textbf{Lesson:} Adjusting a stable process based on individual deviations (common cause) actually makes the process worse. Only adjust when special causes are identified.

\textbf{Real-World Example:} An operator notices a part is 0.02mm below target and adjusts the machine. The next part is 0.03mm above target (the adjustment plus normal variation). They adjust again\ldots{} and the process oscillates with increasing variation.

\subsection{When is a Process ``In Control''?}\label{when-is-a-process-in-control}

A process is \textbf{in statistical control} when:

\begin{enumerate}
\def\labelenumi{\arabic{enumi}.}
\tightlist
\item
  Only common cause variation is present
\item
  All points fall within control limits
\item
  No non-random patterns exist
\item
  The process is stable and predictable
\end{enumerate}

\begin{figure}

{\centering \includegraphics{introduction_files/figure-latex/in-control-definition-1} 

}

\caption{In-Control vs. Out-of-Control Process}\label{fig:in-control-definition}
\end{figure}

\begin{center}\rule{0.5\linewidth}{0.5pt}\end{center}

\section{Control Charts: The Foundation of SPC}\label{control-charts-the-foundation-of-spc}

\subsection{What is a Control Chart?}\label{what-is-a-control-chart}

A \textbf{control chart} is a graph that shows process data over time with statistically determined control limits. It provides a visual method for distinguishing between common and special cause variation.

\begin{figure}

{\centering \includegraphics{introduction_files/figure-latex/control-chart-anatomy-1} 

}

\caption{Anatomy of a Control Chart}\label{fig:control-chart-anatomy}
\end{figure}

\subsection{Control Chart Components}\label{control-chart-components}

\label{tab:chart-components}Control Chart Components

Component

Description

Purpose

\textbf{Center Line (CL)}

The process average; target for the process

Shows where process is centered

\textbf{Upper Control Limit (UCL)}

Upper boundary of expected variation (+3σ)

Signals if process shifted too high

\textbf{Lower Control Limit (LCL)}

Lower boundary of expected variation (-3σ)

Signals if process shifted too low

\textbf{Data Points}

Individual or subgroup statistics plotted over time

Visual representation of process behavior

\textbf{Zones A, B, C}

Regions between ±1σ, ±2σ, and ±3σ for pattern analysis

Used for detecting trends and patterns

\subsection{Why ±3 Sigma?}\label{why-3-sigma}

Control limits are set at \textbf{±3 standard deviations} from the center line. This is not arbitrary:

\begin{itemize}
\tightlist
\item
  For a normal distribution, 99.73\% of data falls within ±3σ
\item
  Only 0.27\% (about 3 in 1000) will fall outside by chance
\item
  This balances the risk of both mistakes (false alarms vs.~missed signals)
\end{itemize}

\begin{figure}

{\centering \includegraphics{introduction_files/figure-latex/normal-distribution-1} 

}

\caption{Normal Distribution and Control Limits}\label{fig:normal-distribution}
\end{figure}

\subsection{Choosing the Right Control Chart}\label{choosing-the-right-control-chart}

The type of control chart depends on the \textbf{type of data} being collected:

\begin{figure}

{\centering \includegraphics{introduction_files/figure-latex/chart-selection-1} 

}

\caption{Control Chart Selection Flowchart}\label{fig:chart-selection}
\end{figure}

\label{tab:chart-summary}Control Chart Summary

Chart Type

Data Type

Sample Size

What It Charts

Typical Application

\textbf{X-bar / R}

Variable

2-9 per subgroup

Subgroup mean and range

Production sampling, multiple measurements

\textbf{X-bar / S}

Variable

≥10 per subgroup

Subgroup mean and std dev

Large subgroups, automated measurement

\textbf{I-MR}

Variable

1 (individuals)

Individual values and moving range

Continuous processes, expensive testing

\textbf{p chart}

Attribute

Variable or constant

Proportion defective

Pass/fail inspection, variable sample size

\textbf{np chart}

Attribute

Constant

Number defective

Pass/fail inspection, fixed sample size

\textbf{c chart}

Attribute

Constant

Number of defects

Defects per item (scratches, errors)

\textbf{u chart}

Attribute

Variable

Defects per unit

Defects per unit area or time

Question: Variable vs.~Attribute Data - What's the Difference?

\textbf{Variable Data (Measured):}
- Can take any value within a range
- Measured on a continuous scale
- Examples: length, weight, temperature, voltage, pressure
- More information per measurement
- Smaller sample sizes needed

\textbf{Attribute Data (Counted):}
- Discrete categories (good/bad, pass/fail)
- Counted, not measured
- Examples: number of defects, number of rejects, yes/no
- Less information per observation
- Larger sample sizes needed

\textbf{When to Use Each:}
- Use variable data when you CAN measure (more powerful)
- Use attribute data when you can only classify or count
- Sometimes you convert variable to attribute (e.g., pass/fail based on tolerance)

\begin{center}\rule{0.5\linewidth}{0.5pt}\end{center}

\section{Variable Control Charts}\label{variable-control-charts}

\subsection{X-bar and R Chart}\label{x-bar-and-r-chart}

The \textbf{X-bar and R chart} is the most common control chart for variable data. It tracks both the process average (X-bar) and the process spread (Range).

\subsubsection{Constructing an X-bar/R Chart}\label{constructing-an-x-barr-chart}

\textbf{Step 1:} Collect data in subgroups (typically 4-5 samples)

\begin{Shaded}
\begin{Highlighting}[]
\CommentTok{\# Example: Measuring shaft diameter (mm)}
\CommentTok{\# 25 subgroups of 5 measurements each}

\FunctionTok{set.seed}\NormalTok{(}\DecValTok{100}\NormalTok{)}
\NormalTok{n\_subgroups }\OtherTok{\textless{}{-}} \DecValTok{25}
\NormalTok{n\_samples }\OtherTok{\textless{}{-}} \DecValTok{5}

\CommentTok{\# Generate data}
\NormalTok{data }\OtherTok{\textless{}{-}} \FunctionTok{matrix}\NormalTok{(}\FunctionTok{rnorm}\NormalTok{(n\_subgroups }\SpecialCharTok{*}\NormalTok{ n\_samples, }\AttributeTok{mean =} \FloatTok{25.00}\NormalTok{, }\AttributeTok{sd =} \FloatTok{0.05}\NormalTok{),}
               \AttributeTok{nrow =}\NormalTok{ n\_subgroups, }\AttributeTok{ncol =}\NormalTok{ n\_samples)}

\CommentTok{\# Add a special cause (shift) at subgroup 18}
\NormalTok{data[}\DecValTok{18}\NormalTok{, ] }\OtherTok{\textless{}{-}}\NormalTok{ data[}\DecValTok{18}\NormalTok{, ] }\SpecialCharTok{+} \FloatTok{0.12}

\CommentTok{\# Calculate statistics for each subgroup}
\NormalTok{xbar }\OtherTok{\textless{}{-}} \FunctionTok{apply}\NormalTok{(data, }\DecValTok{1}\NormalTok{, mean)}
\NormalTok{R }\OtherTok{\textless{}{-}} \FunctionTok{apply}\NormalTok{(data, }\DecValTok{1}\NormalTok{, }\ControlFlowTok{function}\NormalTok{(x) }\FunctionTok{max}\NormalTok{(x) }\SpecialCharTok{{-}} \FunctionTok{min}\NormalTok{(x))}

\FunctionTok{cat}\NormalTok{(}\StringTok{"First 5 subgroups:}\SpecialCharTok{\textbackslash{}n}\StringTok{"}\NormalTok{)}
\end{Highlighting}
\end{Shaded}

\begin{verbatim}
## First 5 subgroups:
\end{verbatim}

\begin{Shaded}
\begin{Highlighting}[]
\FunctionTok{print}\NormalTok{(}\FunctionTok{round}\NormalTok{(data[}\DecValTok{1}\SpecialCharTok{:}\DecValTok{5}\NormalTok{, ], }\DecValTok{4}\NormalTok{))}
\end{Highlighting}
\end{Shaded}

\begin{verbatim}
##         [,1]    [,2]    [,3]    [,4]    [,5]
## [1,] 24.9749 24.9781 24.9776 25.0006 24.9834
## [2,] 25.0066 24.9640 24.9131 24.9456 25.0682
## [3,] 24.9961 25.0115 25.0089 25.0135 24.9765
## [4,] 25.0443 24.9421 25.0949 25.0504 25.0421
## [5,] 25.0058 25.0124 24.8864 24.8963 24.9271
\end{verbatim}

\begin{Shaded}
\begin{Highlighting}[]
\FunctionTok{cat}\NormalTok{(}\StringTok{"}\SpecialCharTok{\textbackslash{}n}\StringTok{X{-}bar values:"}\NormalTok{, }\FunctionTok{round}\NormalTok{(xbar[}\DecValTok{1}\SpecialCharTok{:}\DecValTok{5}\NormalTok{], }\DecValTok{4}\NormalTok{))}
\end{Highlighting}
\end{Shaded}

\begin{verbatim}
## 
## X-bar values: 24.9829 24.9795 25.0013 25.0348 24.9456
\end{verbatim}

\begin{Shaded}
\begin{Highlighting}[]
\FunctionTok{cat}\NormalTok{(}\StringTok{"}\SpecialCharTok{\textbackslash{}n}\StringTok{Range values:"}\NormalTok{, }\FunctionTok{round}\NormalTok{(R[}\DecValTok{1}\SpecialCharTok{:}\DecValTok{5}\NormalTok{], }\DecValTok{4}\NormalTok{))}
\end{Highlighting}
\end{Shaded}

\begin{verbatim}
## 
## Range values: 0.0257 0.1551 0.037 0.1528 0.126
\end{verbatim}

\textbf{Step 2:} Calculate control limits

\begin{Shaded}
\begin{Highlighting}[]
\CommentTok{\# Control chart constants for n=5}
\NormalTok{A2 }\OtherTok{\textless{}{-}} \FloatTok{0.577}
\NormalTok{D3 }\OtherTok{\textless{}{-}} \DecValTok{0}      \CommentTok{\# D3 = 0 for n \textless{} 7}
\NormalTok{D4 }\OtherTok{\textless{}{-}} \FloatTok{2.114}

\CommentTok{\# Calculate grand mean and average range}
\NormalTok{X\_double\_bar }\OtherTok{\textless{}{-}} \FunctionTok{mean}\NormalTok{(xbar)}
\NormalTok{R\_bar }\OtherTok{\textless{}{-}} \FunctionTok{mean}\NormalTok{(R)}

\FunctionTok{cat}\NormalTok{(}\StringTok{"Grand Mean (X{-}double{-}bar):"}\NormalTok{, }\FunctionTok{round}\NormalTok{(X\_double\_bar, }\DecValTok{4}\NormalTok{), }\StringTok{"mm}\SpecialCharTok{\textbackslash{}n}\StringTok{"}\NormalTok{)}
\end{Highlighting}
\end{Shaded}

\begin{verbatim}
## Grand Mean (X-double-bar): 25.004 mm
\end{verbatim}

\begin{Shaded}
\begin{Highlighting}[]
\FunctionTok{cat}\NormalTok{(}\StringTok{"Average Range (R{-}bar):"}\NormalTok{, }\FunctionTok{round}\NormalTok{(R\_bar, }\DecValTok{4}\NormalTok{), }\StringTok{"mm}\SpecialCharTok{\textbackslash{}n\textbackslash{}n}\StringTok{"}\NormalTok{)}
\end{Highlighting}
\end{Shaded}

\begin{verbatim}
## Average Range (R-bar): 0.1023 mm
\end{verbatim}

\begin{Shaded}
\begin{Highlighting}[]
\CommentTok{\# X{-}bar chart limits}
\NormalTok{UCL\_xbar }\OtherTok{\textless{}{-}}\NormalTok{ X\_double\_bar }\SpecialCharTok{+}\NormalTok{ A2 }\SpecialCharTok{*}\NormalTok{ R\_bar}
\NormalTok{LCL\_xbar }\OtherTok{\textless{}{-}}\NormalTok{ X\_double\_bar }\SpecialCharTok{{-}}\NormalTok{ A2 }\SpecialCharTok{*}\NormalTok{ R\_bar}

\FunctionTok{cat}\NormalTok{(}\StringTok{"X{-}bar Chart:}\SpecialCharTok{\textbackslash{}n}\StringTok{"}\NormalTok{)}
\end{Highlighting}
\end{Shaded}

\begin{verbatim}
## X-bar Chart:
\end{verbatim}

\begin{Shaded}
\begin{Highlighting}[]
\FunctionTok{cat}\NormalTok{(}\StringTok{"  UCL ="}\NormalTok{, }\FunctionTok{round}\NormalTok{(UCL\_xbar, }\DecValTok{4}\NormalTok{), }\StringTok{"mm}\SpecialCharTok{\textbackslash{}n}\StringTok{"}\NormalTok{)}
\end{Highlighting}
\end{Shaded}

\begin{verbatim}
##   UCL = 25.0631 mm
\end{verbatim}

\begin{Shaded}
\begin{Highlighting}[]
\FunctionTok{cat}\NormalTok{(}\StringTok{"  CL  ="}\NormalTok{, }\FunctionTok{round}\NormalTok{(X\_double\_bar, }\DecValTok{4}\NormalTok{), }\StringTok{"mm}\SpecialCharTok{\textbackslash{}n}\StringTok{"}\NormalTok{)}
\end{Highlighting}
\end{Shaded}

\begin{verbatim}
##   CL  = 25.004 mm
\end{verbatim}

\begin{Shaded}
\begin{Highlighting}[]
\FunctionTok{cat}\NormalTok{(}\StringTok{"  LCL ="}\NormalTok{, }\FunctionTok{round}\NormalTok{(LCL\_xbar, }\DecValTok{4}\NormalTok{), }\StringTok{"mm}\SpecialCharTok{\textbackslash{}n\textbackslash{}n}\StringTok{"}\NormalTok{)}
\end{Highlighting}
\end{Shaded}

\begin{verbatim}
##   LCL = 24.945 mm
\end{verbatim}

\begin{Shaded}
\begin{Highlighting}[]
\CommentTok{\# R chart limits}
\NormalTok{UCL\_R }\OtherTok{\textless{}{-}}\NormalTok{ D4 }\SpecialCharTok{*}\NormalTok{ R\_bar}
\NormalTok{LCL\_R }\OtherTok{\textless{}{-}}\NormalTok{ D3 }\SpecialCharTok{*}\NormalTok{ R\_bar}

\FunctionTok{cat}\NormalTok{(}\StringTok{"R Chart:}\SpecialCharTok{\textbackslash{}n}\StringTok{"}\NormalTok{)}
\end{Highlighting}
\end{Shaded}

\begin{verbatim}
## R Chart:
\end{verbatim}

\begin{Shaded}
\begin{Highlighting}[]
\FunctionTok{cat}\NormalTok{(}\StringTok{"  UCL ="}\NormalTok{, }\FunctionTok{round}\NormalTok{(UCL\_R, }\DecValTok{4}\NormalTok{), }\StringTok{"mm}\SpecialCharTok{\textbackslash{}n}\StringTok{"}\NormalTok{)}
\end{Highlighting}
\end{Shaded}

\begin{verbatim}
##   UCL = 0.2164 mm
\end{verbatim}

\begin{Shaded}
\begin{Highlighting}[]
\FunctionTok{cat}\NormalTok{(}\StringTok{"  CL  ="}\NormalTok{, }\FunctionTok{round}\NormalTok{(R\_bar, }\DecValTok{4}\NormalTok{), }\StringTok{"mm}\SpecialCharTok{\textbackslash{}n}\StringTok{"}\NormalTok{)}
\end{Highlighting}
\end{Shaded}

\begin{verbatim}
##   CL  = 0.1023 mm
\end{verbatim}

\begin{Shaded}
\begin{Highlighting}[]
\FunctionTok{cat}\NormalTok{(}\StringTok{"  LCL ="}\NormalTok{, }\FunctionTok{round}\NormalTok{(LCL\_R, }\DecValTok{4}\NormalTok{), }\StringTok{"mm}\SpecialCharTok{\textbackslash{}n}\StringTok{"}\NormalTok{)}
\end{Highlighting}
\end{Shaded}

\begin{verbatim}
##   LCL = 0 mm
\end{verbatim}

\textbf{Step 3:} Plot the charts

\begin{figure}

{\centering \includegraphics{introduction_files/figure-latex/xbar-r-plot-1} 

}

\caption{X-bar and R Control Charts}\label{fig:xbar-r-plot}
\end{figure}

\textbf{Interpretation:} Subgroup 18 shows a point above the UCL on the X-bar chart, indicating a special cause. Investigation reveals the cause and corrective action is taken.

\subsection{Control Chart Constants}\label{control-chart-constants}

\label{tab:constants-table}Control Chart Constants for X-bar/R Charts

n

A2

D3

D4

d2

2

1.880

0.000

3.267

1.128

3

1.023

0.000

2.574

1.693

4

0.729

0.000

2.282

2.059

5

0.577

0.000

2.114

2.326

6

0.483

0.000

2.004

2.534

7

0.419

0.076

1.924

2.704

8

0.373

0.136

1.864

2.847

9

0.337

0.184

1.816

2.970

10

0.308

0.223

1.777

3.078

\subsection{Individual and Moving Range (I-MR) Chart}\label{individual-and-moving-range-i-mr-chart}

When sample size is \textbf{n = 1} (one measurement per time period), use the \textbf{I-MR chart}.

\begin{figure}

{\centering \includegraphics{introduction_files/figure-latex/imr-example-1} 

}

\caption{Individual and Moving Range Chart}\label{fig:imr-example}
\end{figure}

\textbf{I-MR Chart Formulas:}

\begin{longtable}[]{@{}
  >{\raggedright\arraybackslash}p{(\linewidth - 6\tabcolsep) * \real{0.2333}}
  >{\raggedright\arraybackslash}p{(\linewidth - 6\tabcolsep) * \real{0.4333}}
  >{\raggedright\arraybackslash}p{(\linewidth - 6\tabcolsep) * \real{0.1667}}
  >{\raggedright\arraybackslash}p{(\linewidth - 6\tabcolsep) * \real{0.1667}}@{}}
\toprule\noalign{}
\begin{minipage}[b]{\linewidth}\raggedright
Chart
\end{minipage} & \begin{minipage}[b]{\linewidth}\raggedright
Center Line
\end{minipage} & \begin{minipage}[b]{\linewidth}\raggedright
UCL
\end{minipage} & \begin{minipage}[b]{\linewidth}\raggedright
LCL
\end{minipage} \\
\midrule\noalign{}
\endhead
\bottomrule\noalign{}
\endlastfoot
I Chart & \(\bar{X}\) & \(\bar{X} + 2.66\bar{MR}\) & \(\bar{X} - 2.66\bar{MR}\) \\
MR Chart & \(\bar{MR}\) & \(3.267 \times \bar{MR}\) & 0 \\
\end{longtable}

\begin{center}\rule{0.5\linewidth}{0.5pt}\end{center}

\section{Attribute Control Charts}\label{attribute-control-charts}

\subsection{p Chart (Proportion Defective)}\label{p-chart-proportion-defective}

The \textbf{p chart} tracks the proportion of defective units when sample sizes may vary.

\begin{Shaded}
\begin{Highlighting}[]
\CommentTok{\# Example: Daily inspection data}
\CommentTok{\# Varying sample sizes (different production volumes)}

\FunctionTok{set.seed}\NormalTok{(}\DecValTok{300}\NormalTok{)}
\NormalTok{days }\OtherTok{\textless{}{-}} \DecValTok{20}
\NormalTok{sample\_sizes }\OtherTok{\textless{}{-}} \FunctionTok{sample}\NormalTok{(}\DecValTok{80}\SpecialCharTok{:}\DecValTok{120}\NormalTok{, days, }\AttributeTok{replace =} \ConstantTok{TRUE}\NormalTok{)}
\NormalTok{defectives }\OtherTok{\textless{}{-}} \FunctionTok{rbinom}\NormalTok{(days, sample\_sizes, }\AttributeTok{prob =} \FloatTok{0.05}\NormalTok{)}

\CommentTok{\# Add a special cause on day 15}
\NormalTok{defectives[}\DecValTok{15}\NormalTok{] }\OtherTok{\textless{}{-}} \FunctionTok{rbinom}\NormalTok{(}\DecValTok{1}\NormalTok{, sample\_sizes[}\DecValTok{15}\NormalTok{], }\AttributeTok{prob =} \FloatTok{0.12}\NormalTok{)}

\CommentTok{\# Calculate proportion}
\NormalTok{p }\OtherTok{\textless{}{-}}\NormalTok{ defectives }\SpecialCharTok{/}\NormalTok{ sample\_sizes}

\CommentTok{\# Calculate p{-}bar (weighted average)}
\NormalTok{p\_bar }\OtherTok{\textless{}{-}} \FunctionTok{sum}\NormalTok{(defectives) }\SpecialCharTok{/} \FunctionTok{sum}\NormalTok{(sample\_sizes)}

\FunctionTok{cat}\NormalTok{(}\StringTok{"p{-}bar (average proportion defective):"}\NormalTok{, }\FunctionTok{round}\NormalTok{(p\_bar, }\DecValTok{4}\NormalTok{), }\StringTok{"}\SpecialCharTok{\textbackslash{}n\textbackslash{}n}\StringTok{"}\NormalTok{)}
\end{Highlighting}
\end{Shaded}

\begin{verbatim}
## p-bar (average proportion defective): 0.0493
\end{verbatim}

\begin{Shaded}
\begin{Highlighting}[]
\CommentTok{\# Calculate control limits (vary with sample size)}
\NormalTok{UCL\_p }\OtherTok{\textless{}{-}}\NormalTok{ p\_bar }\SpecialCharTok{+} \DecValTok{3} \SpecialCharTok{*} \FunctionTok{sqrt}\NormalTok{(p\_bar }\SpecialCharTok{*}\NormalTok{ (}\DecValTok{1} \SpecialCharTok{{-}}\NormalTok{ p\_bar) }\SpecialCharTok{/}\NormalTok{ sample\_sizes)}
\NormalTok{LCL\_p }\OtherTok{\textless{}{-}} \FunctionTok{pmax}\NormalTok{(}\DecValTok{0}\NormalTok{, p\_bar }\SpecialCharTok{{-}} \DecValTok{3} \SpecialCharTok{*} \FunctionTok{sqrt}\NormalTok{(p\_bar }\SpecialCharTok{*}\NormalTok{ (}\DecValTok{1} \SpecialCharTok{{-}}\NormalTok{ p\_bar) }\SpecialCharTok{/}\NormalTok{ sample\_sizes))}

\FunctionTok{cat}\NormalTok{(}\StringTok{"Sample of control limits (varying with n):}\SpecialCharTok{\textbackslash{}n}\StringTok{"}\NormalTok{)}
\end{Highlighting}
\end{Shaded}

\begin{verbatim}
## Sample of control limits (varying with n):
\end{verbatim}

\begin{Shaded}
\begin{Highlighting}[]
\FunctionTok{cat}\NormalTok{(}\StringTok{"Day 1: UCL ="}\NormalTok{, }\FunctionTok{round}\NormalTok{(UCL\_p[}\DecValTok{1}\NormalTok{], }\DecValTok{4}\NormalTok{), }\StringTok{", LCL ="}\NormalTok{, }\FunctionTok{round}\NormalTok{(LCL\_p[}\DecValTok{1}\NormalTok{], }\DecValTok{4}\NormalTok{), }\StringTok{"(n ="}\NormalTok{, sample\_sizes[}\DecValTok{1}\NormalTok{], }\StringTok{")}\SpecialCharTok{\textbackslash{}n}\StringTok{"}\NormalTok{)}
\end{Highlighting}
\end{Shaded}

\begin{verbatim}
## Day 1: UCL = 0.1167 , LCL = 0 (n = 93 )
\end{verbatim}

\begin{Shaded}
\begin{Highlighting}[]
\FunctionTok{cat}\NormalTok{(}\StringTok{"Day 5: UCL ="}\NormalTok{, }\FunctionTok{round}\NormalTok{(UCL\_p[}\DecValTok{5}\NormalTok{], }\DecValTok{4}\NormalTok{), }\StringTok{", LCL ="}\NormalTok{, }\FunctionTok{round}\NormalTok{(LCL\_p[}\DecValTok{5}\NormalTok{], }\DecValTok{4}\NormalTok{), }\StringTok{"(n ="}\NormalTok{, sample\_sizes[}\DecValTok{5}\NormalTok{], }\StringTok{")}\SpecialCharTok{\textbackslash{}n}\StringTok{"}\NormalTok{)}
\end{Highlighting}
\end{Shaded}

\begin{verbatim}
## Day 5: UCL = 0.1143 , LCL = 0 (n = 100 )
\end{verbatim}

\begin{figure}

{\centering \includegraphics{introduction_files/figure-latex/p-chart-plot-1} 

}

\caption{p Chart for Proportion Defective}\label{fig:p-chart-plot}
\end{figure}

\subsection{c Chart (Count of Defects)}\label{c-chart-count-of-defects}

The \textbf{c chart} tracks the number of defects when the sample size (area of opportunity) is constant.

\begin{figure}

{\centering \includegraphics{introduction_files/figure-latex/c-chart-example-1} 

}

\caption{c Chart for Defects per Unit}\label{fig:c-chart-example}
\end{figure}

\label{tab:attribute-formulas}Attribute Control Chart Formulas

Chart

Center Line

UCL

LCL

Use When

\textbf{p chart}

\(\bar{p} = \frac{\sum d}{\sum n}\)

\(\bar{p} + 3\sqrt{\frac{\bar{p}(1-\bar{p})}{n}}\)

\(\bar{p} - 3\sqrt{\frac{\bar{p}(1-\bar{p})}{n}}\)

Variable sample size, proportion needed

\textbf{np chart}

\(\bar{np} = \frac{\sum d}{k}\)

\(\bar{np} + 3\sqrt{\bar{np}(1-\bar{p})}\)

\(\bar{np} - 3\sqrt{\bar{np}(1-\bar{p})}\)

Constant sample size, count preferred

\textbf{c chart}

\(\bar{c} = \frac{\sum c}{k}\)

\(\bar{c} + 3\sqrt{\bar{c}}\)

\(\bar{c} - 3\sqrt{\bar{c}}\)

Constant opportunity, counting defects

\textbf{u chart}

\(\bar{u} = \frac{\sum c}{\sum n}\)

\(\bar{u} + 3\sqrt{\frac{\bar{u}}{n}}\)

\(\bar{u} - 3\sqrt{\frac{\bar{u}}{n}}\)

Variable opportunity, defects per unit

\begin{center}\rule{0.5\linewidth}{0.5pt}\end{center}

\section{Interpreting Control Charts}\label{interpreting-control-charts}

\subsection{Out-of-Control Signals (Western Electric Rules)}\label{out-of-control-signals-western-electric-rules}

A point doesn't have to be outside the control limits to indicate a problem. The \textbf{Western Electric Rules} identify non-random patterns:

\label{tab:we-rules}Western Electric Rules for Control Chart Interpretation

Rule

Pattern

Description

What It Indicates

Rule 1

\textbf{Point beyond control limit}

One point above UCL or below LCL

Sudden shift, extreme event

Rule 2

\textbf{Zone A: 2 of 3 consecutive points}

2 out of 3 consecutive points in Zone A (\textgreater2σ) or beyond, same side

Process shift

Rule 3

\textbf{Zone B: 4 of 5 consecutive points}

4 out of 5 consecutive points in Zone B (\textgreater1σ) or beyond, same side

Process shift

Rule 4

\textbf{8 consecutive points on one side}

8 consecutive points above or below the center line

Process shift (mean has moved)

\begin{figure}

{\centering \includegraphics{introduction_files/figure-latex/we-rules-visual-1} 

}

\caption{Visual Examples of Western Electric Rules}\label{fig:we-rules-visual}
\end{figure}

\subsection{Additional Patterns to Watch For}\label{additional-patterns-to-watch-for}

\label{tab:additional-patterns}Additional Control Chart Patterns

Pattern

Description

Possible Causes

\textbf{Trend}

6 or more consecutive points steadily increasing or decreasing

Tool wear, temperature drift, operator fatigue

\textbf{Cycle}

Repeating high-low pattern

Shift changes, batch effects, temperature cycles

\textbf{Stratification}

Points consistently near center line (too little variation)

Incorrect limits, data manipulation, mixed streams

\textbf{Mixture}

Points avoiding center line (bimodal distribution)

Two machines, two operators, two material lots

Interactive Exercise: Identify the Out-of-Control Condition

\textbf{Look at this sequence of points (standardized values):}

0.5, 0.8, 1.2, 0.9, 1.4, 1.6, 1.3, 1.8, -0.2, 0.3

\textbf{Questions:}

\begin{enumerate}
\def\labelenumi{\arabic{enumi}.}
\tightlist
\item
  Are any points beyond ±3σ?
\item
  Is there a Rule 4 violation (8 consecutive on one side)?
\item
  What pattern do you see?
\end{enumerate}

\textbf{Answers:}

\begin{enumerate}
\def\labelenumi{\arabic{enumi}.}
\tightlist
\item
  No --- all points are within ±3σ
\item
  \textbf{YES} --- Points 1-8 are all positive (above center line)
\item
  This is a \textbf{Rule 4 violation}: 8 consecutive points above the center line indicates a likely process shift upward
\end{enumerate}

\textbf{Action:} Investigate what changed --- new material lot? Different operator? Equipment adjustment?

\begin{center}\rule{0.5\linewidth}{0.5pt}\end{center}

\section{Process Capability}\label{process-capability}

\subsection{What is Process Capability?}\label{what-is-process-capability}

\textbf{Process capability} compares the process performance to the specification limits. It answers: ``Can this process consistently produce parts within tolerance?''

\begin{figure}

{\centering \includegraphics{introduction_files/figure-latex/capability-concept-1} 

}

\caption{Process Capability: Comparing Process to Specifications}\label{fig:capability-concept}
\end{figure}

\subsection{Capability Indices}\label{capability-indices}

\label{tab:capability-indices}Process Capability Indices

Index

Formula

What It Measures

σ Estimation

\textbf{Cp}

\(\frac{USL - LSL}{6\sigma}\)

Potential capability (spread only)

Within-subgroup (R-bar/d2)

\textbf{Cpk}

\(\min\left(\frac{USL - \bar{X}}{3\sigma}, \frac{\bar{X} - LSL}{3\sigma}\right)\)

Actual capability (spread + centering)

Within-subgroup (R-bar/d2)

\textbf{Pp}

\(\frac{USL - LSL}{6s}\)

Overall performance (spread only)

Overall standard deviation (s)

\textbf{Ppk}

\(\min\left(\frac{USL - \bar{X}}{3s}, \frac{\bar{X} - LSL}{3s}\right)\)

Overall performance (spread + centering)

Overall standard deviation (s)

\subsection{Cp vs.~Cpk: Why Both Matter}\label{cp-vs.-cpk-why-both-matter}

\begin{figure}

{\centering \includegraphics{introduction_files/figure-latex/cp-cpk-comparison-1} 

}

\caption{Cp vs. Cpk: The Importance of Centering}\label{fig:cp-cpk-comparison}
\end{figure}

\subsection{Capability Example Calculation}\label{capability-example-calculation}

\begin{Shaded}
\begin{Highlighting}[]
\CommentTok{\# Example: Shaft diameter specification 25.00 ± 0.15 mm}
\CommentTok{\# Process data from 25 subgroups of 5}

\NormalTok{USL }\OtherTok{\textless{}{-}} \FloatTok{25.15}
\NormalTok{LSL }\OtherTok{\textless{}{-}} \FloatTok{24.85}
\NormalTok{Target }\OtherTok{\textless{}{-}} \FloatTok{25.00}

\CommentTok{\# Using earlier X{-}bar/R data}
\NormalTok{X\_bar\_cap }\OtherTok{\textless{}{-}} \FunctionTok{mean}\NormalTok{(xbar)}
\NormalTok{R\_bar\_cap }\OtherTok{\textless{}{-}} \FunctionTok{mean}\NormalTok{(R)}

\CommentTok{\# Estimate sigma using R{-}bar/d2 (d2 = 2.326 for n=5)}
\NormalTok{d2 }\OtherTok{\textless{}{-}} \FloatTok{2.326}
\NormalTok{sigma\_within }\OtherTok{\textless{}{-}}\NormalTok{ R\_bar\_cap }\SpecialCharTok{/}\NormalTok{ d2}

\FunctionTok{cat}\NormalTok{(}\StringTok{"Process Statistics:}\SpecialCharTok{\textbackslash{}n}\StringTok{"}\NormalTok{)}
\end{Highlighting}
\end{Shaded}

\begin{verbatim}
## Process Statistics:
\end{verbatim}

\begin{Shaded}
\begin{Highlighting}[]
\FunctionTok{cat}\NormalTok{(}\StringTok{"  Mean (X{-}bar):"}\NormalTok{, }\FunctionTok{round}\NormalTok{(X\_bar\_cap, }\DecValTok{4}\NormalTok{), }\StringTok{"mm}\SpecialCharTok{\textbackslash{}n}\StringTok{"}\NormalTok{)}
\end{Highlighting}
\end{Shaded}

\begin{verbatim}
##   Mean (X-bar): 25.004 mm
\end{verbatim}

\begin{Shaded}
\begin{Highlighting}[]
\FunctionTok{cat}\NormalTok{(}\StringTok{"  Sigma (estimated):"}\NormalTok{, }\FunctionTok{round}\NormalTok{(sigma\_within, }\DecValTok{4}\NormalTok{), }\StringTok{"mm}\SpecialCharTok{\textbackslash{}n\textbackslash{}n}\StringTok{"}\NormalTok{)}
\end{Highlighting}
\end{Shaded}

\begin{verbatim}
##   Sigma (estimated): 0.044 mm
\end{verbatim}

\begin{Shaded}
\begin{Highlighting}[]
\CommentTok{\# Calculate Cp}
\NormalTok{Cp }\OtherTok{\textless{}{-}}\NormalTok{ (USL }\SpecialCharTok{{-}}\NormalTok{ LSL) }\SpecialCharTok{/}\NormalTok{ (}\DecValTok{6} \SpecialCharTok{*}\NormalTok{ sigma\_within)}
\FunctionTok{cat}\NormalTok{(}\StringTok{"Cp ="}\NormalTok{, }\FunctionTok{round}\NormalTok{(Cp, }\DecValTok{2}\NormalTok{), }\StringTok{"}\SpecialCharTok{\textbackslash{}n}\StringTok{"}\NormalTok{)}
\end{Highlighting}
\end{Shaded}

\begin{verbatim}
## Cp = 1.14
\end{verbatim}

\begin{Shaded}
\begin{Highlighting}[]
\CommentTok{\# Calculate Cpk}
\NormalTok{Cpu }\OtherTok{\textless{}{-}}\NormalTok{ (USL }\SpecialCharTok{{-}}\NormalTok{ X\_bar\_cap) }\SpecialCharTok{/}\NormalTok{ (}\DecValTok{3} \SpecialCharTok{*}\NormalTok{ sigma\_within)}
\NormalTok{Cpl }\OtherTok{\textless{}{-}}\NormalTok{ (X\_bar\_cap }\SpecialCharTok{{-}}\NormalTok{ LSL) }\SpecialCharTok{/}\NormalTok{ (}\DecValTok{3} \SpecialCharTok{*}\NormalTok{ sigma\_within)}
\NormalTok{Cpk }\OtherTok{\textless{}{-}} \FunctionTok{min}\NormalTok{(Cpu, Cpl)}

\FunctionTok{cat}\NormalTok{(}\StringTok{"Cpu ="}\NormalTok{, }\FunctionTok{round}\NormalTok{(Cpu, }\DecValTok{2}\NormalTok{), }\StringTok{"}\SpecialCharTok{\textbackslash{}n}\StringTok{"}\NormalTok{)}
\end{Highlighting}
\end{Shaded}

\begin{verbatim}
## Cpu = 1.11
\end{verbatim}

\begin{Shaded}
\begin{Highlighting}[]
\FunctionTok{cat}\NormalTok{(}\StringTok{"Cpl ="}\NormalTok{, }\FunctionTok{round}\NormalTok{(Cpl, }\DecValTok{2}\NormalTok{), }\StringTok{"}\SpecialCharTok{\textbackslash{}n}\StringTok{"}\NormalTok{)}
\end{Highlighting}
\end{Shaded}

\begin{verbatim}
## Cpl = 1.17
\end{verbatim}

\begin{Shaded}
\begin{Highlighting}[]
\FunctionTok{cat}\NormalTok{(}\StringTok{"Cpk ="}\NormalTok{, }\FunctionTok{round}\NormalTok{(Cpk, }\DecValTok{2}\NormalTok{), }\StringTok{"}\SpecialCharTok{\textbackslash{}n\textbackslash{}n}\StringTok{"}\NormalTok{)}
\end{Highlighting}
\end{Shaded}

\begin{verbatim}
## Cpk = 1.11
\end{verbatim}

\begin{Shaded}
\begin{Highlighting}[]
\CommentTok{\# Interpretation}
\FunctionTok{cat}\NormalTok{(}\StringTok{"Interpretation:}\SpecialCharTok{\textbackslash{}n}\StringTok{"}\NormalTok{)}
\end{Highlighting}
\end{Shaded}

\begin{verbatim}
## Interpretation:
\end{verbatim}

\begin{Shaded}
\begin{Highlighting}[]
\ControlFlowTok{if}\NormalTok{ (Cpk }\SpecialCharTok{\textgreater{}=} \FloatTok{1.33}\NormalTok{) \{}
  \FunctionTok{cat}\NormalTok{(}\StringTok{"Process is CAPABLE (Cpk \textgreater{}= 1.33)}\SpecialCharTok{\textbackslash{}n}\StringTok{"}\NormalTok{)}
\NormalTok{\} }\ControlFlowTok{else} \ControlFlowTok{if}\NormalTok{ (Cpk }\SpecialCharTok{\textgreater{}=} \FloatTok{1.00}\NormalTok{) \{}
  \FunctionTok{cat}\NormalTok{(}\StringTok{"Process is MARGINALLY CAPABLE (1.00 \textless{}= Cpk \textless{} 1.33)}\SpecialCharTok{\textbackslash{}n}\StringTok{"}\NormalTok{)}
\NormalTok{\} }\ControlFlowTok{else}\NormalTok{ \{}
  \FunctionTok{cat}\NormalTok{(}\StringTok{"Process is NOT CAPABLE (Cpk \textless{} 1.00)}\SpecialCharTok{\textbackslash{}n}\StringTok{"}\NormalTok{)}
\NormalTok{\}}
\end{Highlighting}
\end{Shaded}

\begin{verbatim}
## Process is MARGINALLY CAPABLE (1.00 <= Cpk < 1.33)
\end{verbatim}

\subsection{Capability Index Interpretation}\label{capability-index-interpretation}

\label{tab:capability-interpretation}Capability Index Interpretation Guide

Cpk Value

Sigma Level

PPM Defective

Interpretation

Typical Industry

\textless{} 1.00

\textless{} 3σ

\textgreater{} 2,700

Not capable --- improvement required

Requires immediate action

1.00 - 1.33

3σ - 4σ

2,700 - 64

Marginally capable --- monitor closely

Consumer products

1.33 - 1.67

4σ - 5σ

64 - 0.6

Capable --- meets typical requirements

General manufacturing

1.67 - 2.00

5σ - 6σ

0.6 - 0.002

Good capability --- automotive/aerospace

Automotive Tier 1

\textgreater{} 2.00

\textgreater{} 6σ

\textless{} 0.002

Excellent --- Six Sigma level

Aerospace, medical devices

Question: When to use Cp/Cpk vs.~Pp/Ppk?

\textbf{Cp/Cpk (Process Capability):}
- Uses within-subgroup variation (σ estimated from R-bar/d2)
- Represents the POTENTIAL capability if process is in control
- Use for ongoing process monitoring
- Requires process to be in statistical control

\textbf{Pp/Ppk (Process Performance):}
- Uses overall standard deviation (s) of all data
- Represents ACTUAL performance including all variation
- Use for initial capability studies or when process stability is unknown
- Includes both common and special cause variation

\textbf{Rule of Thumb:}
- If Pp ≈ Cp and Ppk ≈ Cpk → Process is stable
- If Pp \textless{} Cp or Ppk \textless{} Cpk → Process has special cause variation (not in control)

\begin{center}\rule{0.5\linewidth}{0.5pt}\end{center}

\section{Responding to Out-of-Control Signals}\label{responding-to-out-of-control-signals}

\subsection{The Response Process}\label{the-response-process}

\begin{figure}

{\centering \includegraphics{introduction_files/figure-latex/response-process-1} 

}

\caption{Responding to Out-of-Control Conditions}\label{fig:response-process}
\end{figure}

\subsection{Investigation Checklist (5 Ms + E)}\label{investigation-checklist-5-ms-e}

\label{tab:investigation-checklist}Investigation Checklist: 5 Ms + E

Category

Questions to Ask

\textbf{Man}

Different operator? New employee? Fatigue? Training issue?

\textbf{Machine}

Equipment malfunction? Tool wear? Maintenance due? Setup change?

\textbf{Material}

New lot? Different supplier? Out-of-spec material?

\textbf{Method}

Procedure changed? Parameter drift? Wrong program?

\textbf{Measurement}

Gauge drift? Calibration due? Different inspector?

\textbf{Environment}

Temperature change? Humidity? Vibration? Contamination?

\begin{center}\rule{0.5\linewidth}{0.5pt}\end{center}

\section{SPC Implementation}\label{spc-implementation}

\subsection{Steps to Implement SPC}\label{steps-to-implement-spc}

\label{tab:implementation-steps}SPC Implementation Steps

Step

Action

Key Considerations

1

Select the process and characteristic

Start with critical/problem processes

2

Determine measurement system adequacy (MSA)

Gauge R\&R must be acceptable (\textless{} 30\%)

3

Collect initial data (25+ subgroups)

Process must be running normally

4

Calculate trial control limits

Limits based on actual process, not specs

5

Identify and remove special causes

Investigate each out-of-control point

6

Recalculate limits with stable data

These become the standard limits

7

Implement ongoing monitoring

Train operators, establish reaction plans

8

Review and improve continuously

Update limits when process improves

\subsection{Common SPC Mistakes}\label{common-spc-mistakes}

\label{tab:spc-mistakes}Common SPC Mistakes and Corrections

Mistake

Why It's Wrong

Correct Approach

\textbf{Using specification limits as control limits}

Control limits are based on process performance, not specs

Calculate limits from process data (mean ± 3σ)

\textbf{Changing limits too frequently}

Limits should only change when process fundamentally changes

Use stable baseline; recalculate only after sustained improvement

\textbf{Ignoring the R chart}

R chart often signals problems before X-bar chart

Always analyze R chart first --- variation stability

\textbf{Over-adjusting (tampering)}

Adjusting for common cause variation increases variation

Only adjust when special cause is identified

\textbf{Not acting on signals}

Defeats the purpose of SPC

Investigate and act on every out-of-control signal

\begin{center}\rule{0.5\linewidth}{0.5pt}\end{center}

\section{Summary}\label{summary-7}

\label{tab:summary-table-ch9}SPC Key Concepts Summary

Topic

Key Points

\textbf{Variation}

Common cause (inherent) vs.~special cause (assignable); respond differently

\textbf{Control Charts}

Graph with UCL/LCL at ±3σ; distinguishes between variation types

\textbf{Variable Charts}

X-bar/R (n=2-9), X-bar/S (n≥10), I-MR (n=1)

\textbf{Attribute Charts}

p (proportion), np (count defectives), c (defects), u (defects/unit)

\textbf{Interpretation}

Western Electric rules: beyond limits, 2 of 3 in Zone A, 4 of 5 in Zone B, 8 consecutive

\textbf{Capability}

Cp (potential), Cpk (actual); ≥1.33 is capable; ≥1.67 for automotive

\begin{center}\rule{0.5\linewidth}{0.5pt}\end{center}

\section{Review Questions}\label{review-questions-8}

\subsection{Conceptual Questions}\label{conceptual-questions-1}

Question 1: What is the difference between common cause and special cause variation?

\textbf{Common Cause Variation:}
- Inherent to the process
- Always present
- Random and predictable
- Many small factors
- System-level improvement required
- Examples: machine vibration, material variation, ambient temperature

\textbf{Special Cause Variation:}
- External to normal process
- Intermittent or new
- Non-random patterns
- Specific assignable cause
- Local action to eliminate
- Examples: tool breakage, new operator, bad material lot

\textbf{Key Point:} Only act on special causes. Attempting to adjust for common cause variation (tampering) makes the process worse.

Question 2: Why are control limits set at ±3 sigma rather than ±2 sigma?

\textbf{At ±3σ:}
- 99.73\% of data falls within limits (when in control)
- Only 0.27\% false alarm rate (about 3 per 1000 points)
- Good balance between detecting real problems and avoiding false alarms

\textbf{If we used ±2σ:}
- 95.45\% of data within limits
- 4.55\% false alarm rate (about 45 per 1000 points!)
- Too many false alarms → operators ignore the chart

\textbf{If we used ±4σ:}
- 99.99\% within limits
- Almost no false alarms
- But we'd miss real problems (low sensitivity)

\textbf{The ±3σ choice balances:}
- Risk of false alarm (treating common cause as special)
- Risk of missed signal (treating special cause as common)

Question 3: When would you use an I-MR chart instead of an X-bar/R chart?

\textbf{Use I-MR Chart When:}

\begin{enumerate}
\def\labelenumi{\arabic{enumi}.}
\tightlist
\item
  \textbf{Sample size is n = 1} --- only one measurement per time period
\item
  \textbf{Destructive testing} --- can't afford multiple samples
\item
  \textbf{Expensive measurement} --- time or cost prohibitive
\item
  \textbf{Continuous process} --- each reading is independent (batch process)
\item
  \textbf{Long intervals} --- measurements far apart in time
\item
  \textbf{Homogeneous batches} --- within-batch variation is negligible
\end{enumerate}

\textbf{Examples:}
- Daily chemical batch analysis
- Weekly calibration check
- Each lot incoming inspection (one sample per lot)
- Continuous process temperature readings

\textbf{Caution:} I-MR charts are less sensitive than X-bar charts because averaging in subgroups reduces noise.

\subsection{Calculation Questions}\label{calculation-questions-1}

Question 4: Calculate control limits for an X-bar/R chart given the following data.

\textbf{Given:}
- 20 subgroups of n = 4 measurements
- Grand mean (X-double-bar) = 50.25
- Average range (R-bar) = 2.40
- Constants for n = 4: A2 = 0.729, D3 = 0, D4 = 2.282

\textbf{Calculate UCL and LCL for both charts.}

\textbf{Solution:}

\textbf{X-bar Chart:}

\begin{verbatim}
UCL = X-double-bar + A2 × R-bar
UCL = 50.25 + 0.729 × 2.40 = 50.25 + 1.75 = 52.00

LCL = X-double-bar - A2 × R-bar
LCL = 50.25 - 0.729 × 2.40 = 50.25 - 1.75 = 48.50

CL = 50.25
\end{verbatim}

\textbf{R Chart:}

\begin{verbatim}
UCL = D4 × R-bar = 2.282 × 2.40 = 5.48

LCL = D3 × R-bar = 0 × 2.40 = 0

CL = 2.40
\end{verbatim}

Question 5: Calculate Cp and Cpk for a process with the following data.

\textbf{Given:}
- Specification: 100 ± 5 (USL = 105, LSL = 95)
- Process mean = 102
- Process standard deviation (σ) = 1.5

\textbf{Solution:}

\textbf{Cp (Potential Capability):}

\begin{verbatim}
Cp = (USL - LSL) / (6σ)
Cp = (105 - 95) / (6 × 1.5)
Cp = 10 / 9 = 1.11
\end{verbatim}

\textbf{Cpk (Actual Capability):}

\begin{verbatim}
Cpu = (USL - X-bar) / (3σ) = (105 - 102) / (3 × 1.5) = 3 / 4.5 = 0.67
Cpl = (X-bar - LSL) / (3σ) = (102 - 95) / (3 × 1.5) = 7 / 4.5 = 1.56

Cpk = min(Cpu, Cpl) = min(0.67, 1.56) = 0.67
\end{verbatim}

\textbf{Interpretation:}
- Cp = 1.11 suggests process spread could fit within specs IF centered
- Cpk = 0.67 shows process is NOT capable because it's shifted toward USL
- Action: Center the process closer to 100 (target)

Question 6: A p chart has p-bar = 0.04 and sample size n = 200. Calculate control limits.

\textbf{Solution:}

\textbf{Center Line:}

\begin{verbatim}
CL = p-bar = 0.04 (4% defective)
\end{verbatim}

\textbf{Upper Control Limit:}

\begin{verbatim}
UCL = p-bar + 3 × √(p-bar × (1 - p-bar) / n)
UCL = 0.04 + 3 × √(0.04 × 0.96 / 200)
UCL = 0.04 + 3 × √(0.000192)
UCL = 0.04 + 3 × 0.01386
UCL = 0.04 + 0.0416 = 0.0816 (8.16%)
\end{verbatim}

\textbf{Lower Control Limit:}

\begin{verbatim}
LCL = p-bar - 3 × √(p-bar × (1 - p-bar) / n)
LCL = 0.04 - 0.0416 = -0.0016

Since LCL cannot be negative: LCL = 0
\end{verbatim}

\textbf{Final Limits:}
- UCL = 8.16\%
- CL = 4.00\%
- LCL = 0\%

\subsection{Application Questions}\label{application-questions-1}

Question 7: You observe 8 consecutive points above the center line but all within control limits. What does this indicate?

\textbf{This is a Rule 4 violation (Western Electric Rules).}

\textbf{What it indicates:}
- The process mean has shifted upward
- This is a SPECIAL CAUSE signal
- Even though no points are beyond limits, the pattern is not random

\textbf{Probability analysis:}
- Probability of a point above CL (random) = 0.5
- Probability of 8 consecutive above CL = 0.5\^{}8 = 0.0039 (0.39\%)
- This is very unlikely by chance alone

\textbf{Action required:}
1. Stop and mark the chart
2. Investigate what changed around the time of the first high point
3. Check: new material lot? Tool change? Operator change? Temperature?
4. Correct the cause
5. Monitor to verify process returns to normal

Question 8: A process has Cp = 2.0 but Cpk = 1.0. What does this tell you?

\textbf{Analysis:}

\textbf{Cp = 2.0 means:}
- Process spread (6σ) is half the specification width
- The process COULD produce no defects if centered
- Potential capability is excellent

\textbf{Cpk = 1.0 means:}
- Process is off-center
- Actual capability is only marginal
- Some defects are being produced

\textbf{The difference (Cp - Cpk = 1.0) indicates:}
- The process mean is shifted 3σ from the target
- One side has much less margin than the other

\textbf{Visual:}

\begin{verbatim}
LSL        Target        USL
|           |           |
      [====PROCESS====]
      ↑ shifted to one side
\end{verbatim}

\textbf{Action:}
- Center the process on target
- After centering, Cpk should approach Cp (both ≈ 2.0)
- This is an adjustment action, not a variation reduction

Question 9: Why should you always analyze the R chart before the X-bar chart?

\textbf{The R chart should be analyzed first because:}

\begin{enumerate}
\def\labelenumi{\arabic{enumi}.}
\tightlist
\item
  \textbf{Control limits depend on variation:}

  \begin{itemize}
  \tightlist
  \item
    X-bar limits are calculated using R-bar
  \item
    If R is out of control, X-bar limits are unreliable
  \end{itemize}
\item
  \textbf{R chart is more sensitive to variation changes:}

  \begin{itemize}
  \tightlist
  \item
    Tool wear shows on R chart before X-bar
  \item
    Inconsistent setup appears as increased R
  \end{itemize}
\item
  \textbf{Process must be stable in variation first:}

  \begin{itemize}
  \tightlist
  \item
    A process can't be ``in control'' if variation is changing
  \item
    Stable variation is a prerequisite for evaluating the mean
  \end{itemize}
\item
  \textbf{Different actions for each:}

  \begin{itemize}
  \tightlist
  \item
    R chart problem → variation issue (consistency)
  \item
    X-bar problem → centering issue (average)
  \end{itemize}
\end{enumerate}

\textbf{Rule:} If R chart is out of control, fix that FIRST before interpreting the X-bar chart.

\begin{center}\rule{0.5\linewidth}{0.5pt}\end{center}

\section{References}\label{references-8}

\begin{itemize}
\tightlist
\item
  Montgomery, D.C. (2019). \emph{Introduction to Statistical Quality Control} (8th ed.). Wiley.
\item
  Wheeler, D.J., \& Chambers, D.S. (1992). \emph{Understanding Statistical Process Control} (2nd ed.). SPC Press.
\item
  AIAG. (2005). \emph{Statistical Process Control (SPC) Reference Manual} (2nd ed.).
\item
  Western Electric. (1956). \emph{Statistical Quality Control Handbook}. AT\&T.
\item
  Deming, W.E. (1986). \emph{Out of the Crisis}. MIT Press.
\item
  Shewhart, W.A. (1931). \emph{Economic Control of Quality of Manufactured Product}. Van Nostrand.
\item
  ASTM E2587. Standard Practice for Use of Control Charts in Statistical Process Control.
\end{itemize}

\begin{center}\rule{0.5\linewidth}{0.5pt}\end{center}

\chapter{Total Productive Maintenance (TPM)}\label{total-productive-maintenance-tpm}

\begin{center}\rule{0.5\linewidth}{0.5pt}\end{center}

\section{Learning Objectives}\label{learning-objectives-9}

By the end of this chapter, you will be able to:

\begin{itemize}
\tightlist
\item
  Define Total Productive Maintenance and explain its objectives
\item
  Describe the 8 pillars of TPM and their interconnections
\item
  Calculate Overall Equipment Effectiveness (OEE) and its components
\item
  Identify and categorize the Six Big Losses
\item
  Implement autonomous maintenance activities
\item
  Develop preventive and predictive maintenance strategies
\item
  Apply TPM principles to improve equipment reliability and productivity
\end{itemize}

\begin{quote}
``Take care of your equipment, and it will take care of your production.''
--- Seiichi Nakajima, Father of TPM
\end{quote}

\begin{center}\rule{0.5\linewidth}{0.5pt}\end{center}

\section{Introduction to TPM}\label{introduction-to-tpm}

\subsection{What is Total Productive Maintenance?}\label{what-is-total-productive-maintenance}

\textbf{Total Productive Maintenance (TPM)} is a holistic approach to equipment maintenance that strives to achieve perfect production with no breakdowns, no defects, and no accidents. It involves everyone in the organization, from operators to top management.

\begin{figure}

{\centering \includegraphics{introduction_files/figure-latex/tpm-definition-1} 

}

\caption{The TPM Philosophy}\label{fig:tpm-definition}
\end{figure}

\subsection{History of TPM}\label{history-of-tpm}

\label{tab:tpm-history}Evolution of Maintenance Philosophy

Era

Maintenance Approach

Description

Key Change

Pre-1950s

Breakdown Maintenance

Fix it when it breaks; reactive only

No prevention

1950s

Preventive Maintenance

Scheduled maintenance to prevent failures

Time-based prevention

1960s

Productive Maintenance

Maintenance focused on reliability and maintainability

Design for maintenance

1970s

TPM Developed

Nakajima develops TPM at Nippondenso (Toyota supplier)

Operator involvement

1980s

TPM Spreads Globally

Adopted by automotive, electronics, process industries worldwide

World-class standard

1990s-Present

TPM + Industry 4.0

Integration with predictive analytics, IoT sensors, AI

Data-driven maintenance

\subsection{The Goals of TPM}\label{the-goals-of-tpm}

TPM aims to achieve \textbf{zero} targets:

\begin{figure}

{\centering \includegraphics{introduction_files/figure-latex/tpm-goals-1} 

}

\caption{TPM Zero Targets}\label{fig:tpm-goals}
\end{figure}

\subsection{Traditional vs.~TPM Approach}\label{traditional-vs.-tpm-approach}

\label{tab:traditional-vs-tpm}Traditional Maintenance vs.~TPM

Aspect

Traditional Maintenance

TPM Approach

\textbf{Equipment ownership}

`Maintenance owns the machines'

`We all own the machines'

\textbf{Maintenance responsibility}

Maintenance department only

Everyone (operators + maintenance)

\textbf{Operator role}

Operate; call maintenance when broken

Operate, clean, inspect, maintain

\textbf{Maintenance focus}

Fix breakdowns quickly

Prevent breakdowns; extend life

\textbf{Problem solving}

Reactive; fire-fighting

Proactive; root cause elimination

\textbf{Goal}

Minimize maintenance cost

Maximize equipment effectiveness

Question: Why involve operators in maintenance?

\textbf{Operators are the first line of defense because:}

\begin{enumerate}
\def\labelenumi{\arabic{enumi}.}
\tightlist
\item
  \textbf{They know the equipment:} Operators work with machines every day and notice subtle changes
\item
  \textbf{Early detection:} Small problems are detected before becoming major failures
\item
  \textbf{Ownership mindset:} People take better care of things they ``own''
\item
  \textbf{Reduced response time:} Operators can address minor issues immediately
\item
  \textbf{Freed maintenance resources:} Technicians can focus on complex tasks
\end{enumerate}

\textbf{Example:} An operator notices a slight vibration change in a motor. Under traditional maintenance, they might ignore it. Under TPM, they report it, potentially preventing a \$50,000 breakdown.

\begin{quote}
``No one knows a machine better than the person who operates it every day.''
\end{quote}

\begin{center}\rule{0.5\linewidth}{0.5pt}\end{center}

\section{The Eight Pillars of TPM}\label{the-eight-pillars-of-tpm}

TPM is built on \textbf{eight pillars}, each addressing a specific aspect of equipment management and organizational culture.

\begin{figure}

{\centering \includegraphics{introduction_files/figure-latex/eight-pillars-1} 

}

\caption{The Eight Pillars of TPM}\label{fig:eight-pillars}
\end{figure}

\label{tab:pillars-table}The Eight Pillars of TPM

\#

Pillar

Purpose

Key Activities

1

\textbf{Autonomous Maintenance}

Operators maintain their own equipment

Cleaning, inspection, lubrication, minor repairs

2

\textbf{Planned Maintenance}

Scheduled maintenance by specialists

PM schedules, predictive maintenance, spare parts

3

\textbf{Quality Maintenance}

Zero defects through equipment conditions

Identify equipment conditions causing defects

4

\textbf{Focused Improvement}

Eliminate losses through kaizen activities

Loss analysis, root cause, countermeasures

5

\textbf{Early Equipment Management}

Design equipment for maintainability

Design reviews, vertical startup, LCC analysis

6

\textbf{Training \& Education}

Develop skilled operators and technicians

Skills matrix, training programs, OJT

7

\textbf{Safety, Health, Environment}

Zero accidents, safe workplace

Hazard identification, safety audits, ergonomics

8

\textbf{TPM in Office}

Apply TPM principles to administrative processes

Streamline workflows, reduce waste in offices

\begin{center}\rule{0.5\linewidth}{0.5pt}\end{center}

\section{Overall Equipment Effectiveness (OEE)}\label{overall-equipment-effectiveness-oee}

\subsection{What is OEE?}\label{what-is-oee}

\textbf{Overall Equipment Effectiveness (OEE)} is the primary metric of TPM. It measures how effectively equipment is being used by combining three factors: Availability, Performance, and Quality.

\begin{figure}

{\centering \includegraphics{introduction_files/figure-latex/oee-formula-1} 

}

\caption{OEE Formula and Components}\label{fig:oee-formula}
\end{figure}

\subsection{OEE Component Formulas}\label{oee-component-formulas}

\label{tab:oee-formulas}OEE Component Formulas

Component

Formula

What It Measures

Losses Captured

\textbf{Availability}

\(\frac{\text{Run Time}}{\text{Planned Production Time}} \times 100\%\)

Time the equipment is running vs.~planned time

Breakdowns, changeovers, adjustments

\textbf{Performance}

\(\frac{\text{Ideal Cycle Time} \times \text{Total Count}}{\text{Run Time}} \times 100\%\)

How fast equipment runs vs.~design speed

Minor stops, reduced speed, idling

\textbf{Quality}

\(\frac{\text{Good Count}}{\text{Total Count}} \times 100\%\)

Good parts vs.~total parts produced

Defects, rework, startup rejects

\textbf{OEE}

\(\text{Availability} \times \text{Performance} \times \text{Quality}\)

Overall effectiveness combining all three

All six big losses combined

\subsection{OEE Calculation Example}\label{oee-calculation-example}

\begin{Shaded}
\begin{Highlighting}[]
\CommentTok{\# OEE Calculation Example: Injection Molding Machine}

\FunctionTok{cat}\NormalTok{(}\StringTok{"=== SHIFT DATA ===}\SpecialCharTok{\textbackslash{}n}\StringTok{"}\NormalTok{)}
\end{Highlighting}
\end{Shaded}

\begin{verbatim}
## === SHIFT DATA ===
\end{verbatim}

\begin{Shaded}
\begin{Highlighting}[]
\CommentTok{\# Time data}
\NormalTok{shift\_length }\OtherTok{\textless{}{-}} \DecValTok{480}  \CommentTok{\# minutes (8 hours)}
\NormalTok{planned\_downtime }\OtherTok{\textless{}{-}} \DecValTok{30}  \CommentTok{\# breaks, planned maintenance}
\NormalTok{planned\_production\_time }\OtherTok{\textless{}{-}}\NormalTok{ shift\_length }\SpecialCharTok{{-}}\NormalTok{ planned\_downtime}
\FunctionTok{cat}\NormalTok{(}\StringTok{"Planned Production Time:"}\NormalTok{, planned\_production\_time, }\StringTok{"minutes}\SpecialCharTok{\textbackslash{}n}\StringTok{"}\NormalTok{)}
\end{Highlighting}
\end{Shaded}

\begin{verbatim}
## Planned Production Time: 450 minutes
\end{verbatim}

\begin{Shaded}
\begin{Highlighting}[]
\CommentTok{\# Downtime losses}
\NormalTok{breakdown\_time }\OtherTok{\textless{}{-}} \DecValTok{25}  \CommentTok{\# equipment failure}
\NormalTok{changeover\_time }\OtherTok{\textless{}{-}} \DecValTok{35}  \CommentTok{\# mold changes}
\NormalTok{adjustment\_time }\OtherTok{\textless{}{-}} \DecValTok{10}  \CommentTok{\# setup adjustments}
\NormalTok{total\_downtime }\OtherTok{\textless{}{-}}\NormalTok{ breakdown\_time }\SpecialCharTok{+}\NormalTok{ changeover\_time }\SpecialCharTok{+}\NormalTok{ adjustment\_time}
\FunctionTok{cat}\NormalTok{(}\StringTok{"Unplanned Downtime:"}\NormalTok{, total\_downtime, }\StringTok{"minutes}\SpecialCharTok{\textbackslash{}n}\StringTok{"}\NormalTok{)}
\end{Highlighting}
\end{Shaded}

\begin{verbatim}
## Unplanned Downtime: 70 minutes
\end{verbatim}

\begin{Shaded}
\begin{Highlighting}[]
\NormalTok{run\_time }\OtherTok{\textless{}{-}}\NormalTok{ planned\_production\_time }\SpecialCharTok{{-}}\NormalTok{ total\_downtime}
\FunctionTok{cat}\NormalTok{(}\StringTok{"Run Time:"}\NormalTok{, run\_time, }\StringTok{"minutes}\SpecialCharTok{\textbackslash{}n\textbackslash{}n}\StringTok{"}\NormalTok{)}
\end{Highlighting}
\end{Shaded}

\begin{verbatim}
## Run Time: 380 minutes
\end{verbatim}

\begin{Shaded}
\begin{Highlighting}[]
\CommentTok{\# Production data}
\NormalTok{ideal\_cycle\_time }\OtherTok{\textless{}{-}} \FloatTok{0.5}  \CommentTok{\# minutes per part (design spec)}
\NormalTok{total\_parts }\OtherTok{\textless{}{-}} \DecValTok{700}  \CommentTok{\# parts produced}
\NormalTok{good\_parts }\OtherTok{\textless{}{-}} \DecValTok{665}   \CommentTok{\# parts passing inspection}
\NormalTok{defects }\OtherTok{\textless{}{-}}\NormalTok{ total\_parts }\SpecialCharTok{{-}}\NormalTok{ good\_parts}

\FunctionTok{cat}\NormalTok{(}\StringTok{"=== PRODUCTION DATA ===}\SpecialCharTok{\textbackslash{}n}\StringTok{"}\NormalTok{)}
\end{Highlighting}
\end{Shaded}

\begin{verbatim}
## === PRODUCTION DATA ===
\end{verbatim}

\begin{Shaded}
\begin{Highlighting}[]
\FunctionTok{cat}\NormalTok{(}\StringTok{"Total Parts Produced:"}\NormalTok{, total\_parts, }\StringTok{"}\SpecialCharTok{\textbackslash{}n}\StringTok{"}\NormalTok{)}
\end{Highlighting}
\end{Shaded}

\begin{verbatim}
## Total Parts Produced: 700
\end{verbatim}

\begin{Shaded}
\begin{Highlighting}[]
\FunctionTok{cat}\NormalTok{(}\StringTok{"Good Parts:"}\NormalTok{, good\_parts, }\StringTok{"}\SpecialCharTok{\textbackslash{}n}\StringTok{"}\NormalTok{)}
\end{Highlighting}
\end{Shaded}

\begin{verbatim}
## Good Parts: 665
\end{verbatim}

\begin{Shaded}
\begin{Highlighting}[]
\FunctionTok{cat}\NormalTok{(}\StringTok{"Defects:"}\NormalTok{, defects, }\StringTok{"}\SpecialCharTok{\textbackslash{}n\textbackslash{}n}\StringTok{"}\NormalTok{)}
\end{Highlighting}
\end{Shaded}

\begin{verbatim}
## Defects: 35
\end{verbatim}

\begin{Shaded}
\begin{Highlighting}[]
\CommentTok{\# Calculate OEE Components}
\FunctionTok{cat}\NormalTok{(}\StringTok{"=== OEE CALCULATION ===}\SpecialCharTok{\textbackslash{}n\textbackslash{}n}\StringTok{"}\NormalTok{)}
\end{Highlighting}
\end{Shaded}

\begin{verbatim}
## === OEE CALCULATION ===
\end{verbatim}

\begin{Shaded}
\begin{Highlighting}[]
\CommentTok{\# Availability}
\NormalTok{availability }\OtherTok{\textless{}{-}}\NormalTok{ (run\_time }\SpecialCharTok{/}\NormalTok{ planned\_production\_time) }\SpecialCharTok{*} \DecValTok{100}
\FunctionTok{cat}\NormalTok{(}\StringTok{"AVAILABILITY:}\SpecialCharTok{\textbackslash{}n}\StringTok{"}\NormalTok{)}
\end{Highlighting}
\end{Shaded}

\begin{verbatim}
## AVAILABILITY:
\end{verbatim}

\begin{Shaded}
\begin{Highlighting}[]
\FunctionTok{cat}\NormalTok{(}\StringTok{"  = Run Time / Planned Production Time}\SpecialCharTok{\textbackslash{}n}\StringTok{"}\NormalTok{)}
\end{Highlighting}
\end{Shaded}

\begin{verbatim}
##   = Run Time / Planned Production Time
\end{verbatim}

\begin{Shaded}
\begin{Highlighting}[]
\FunctionTok{cat}\NormalTok{(}\StringTok{"  ="}\NormalTok{, run\_time, }\StringTok{"/"}\NormalTok{, planned\_production\_time, }\StringTok{"}\SpecialCharTok{\textbackslash{}n}\StringTok{"}\NormalTok{)}
\end{Highlighting}
\end{Shaded}

\begin{verbatim}
##   = 380 / 450
\end{verbatim}

\begin{Shaded}
\begin{Highlighting}[]
\FunctionTok{cat}\NormalTok{(}\StringTok{"  ="}\NormalTok{, }\FunctionTok{round}\NormalTok{(availability, }\DecValTok{1}\NormalTok{), }\StringTok{"\%}\SpecialCharTok{\textbackslash{}n\textbackslash{}n}\StringTok{"}\NormalTok{)}
\end{Highlighting}
\end{Shaded}

\begin{verbatim}
##   = 84.4 %
\end{verbatim}

\begin{Shaded}
\begin{Highlighting}[]
\CommentTok{\# Performance}
\CommentTok{\# Actual output vs. theoretical output during run time}
\NormalTok{theoretical\_output }\OtherTok{\textless{}{-}}\NormalTok{ run\_time }\SpecialCharTok{/}\NormalTok{ ideal\_cycle\_time}
\NormalTok{performance }\OtherTok{\textless{}{-}}\NormalTok{ (total\_parts }\SpecialCharTok{/}\NormalTok{ theoretical\_output) }\SpecialCharTok{*} \DecValTok{100}
\FunctionTok{cat}\NormalTok{(}\StringTok{"PERFORMANCE:}\SpecialCharTok{\textbackslash{}n}\StringTok{"}\NormalTok{)}
\end{Highlighting}
\end{Shaded}

\begin{verbatim}
## PERFORMANCE:
\end{verbatim}

\begin{Shaded}
\begin{Highlighting}[]
\FunctionTok{cat}\NormalTok{(}\StringTok{"  Theoretical Output ="}\NormalTok{, theoretical\_output, }\StringTok{"parts}\SpecialCharTok{\textbackslash{}n}\StringTok{"}\NormalTok{)}
\end{Highlighting}
\end{Shaded}

\begin{verbatim}
##   Theoretical Output = 760 parts
\end{verbatim}

\begin{Shaded}
\begin{Highlighting}[]
\FunctionTok{cat}\NormalTok{(}\StringTok{"  Actual Output ="}\NormalTok{, total\_parts, }\StringTok{"parts}\SpecialCharTok{\textbackslash{}n}\StringTok{"}\NormalTok{)}
\end{Highlighting}
\end{Shaded}

\begin{verbatim}
##   Actual Output = 700 parts
\end{verbatim}

\begin{Shaded}
\begin{Highlighting}[]
\FunctionTok{cat}\NormalTok{(}\StringTok{"  = Actual / Theoretical}\SpecialCharTok{\textbackslash{}n}\StringTok{"}\NormalTok{)}
\end{Highlighting}
\end{Shaded}

\begin{verbatim}
##   = Actual / Theoretical
\end{verbatim}

\begin{Shaded}
\begin{Highlighting}[]
\FunctionTok{cat}\NormalTok{(}\StringTok{"  ="}\NormalTok{, }\FunctionTok{round}\NormalTok{(performance, }\DecValTok{1}\NormalTok{), }\StringTok{"\%}\SpecialCharTok{\textbackslash{}n\textbackslash{}n}\StringTok{"}\NormalTok{)}
\end{Highlighting}
\end{Shaded}

\begin{verbatim}
##   = 92.1 %
\end{verbatim}

\begin{Shaded}
\begin{Highlighting}[]
\CommentTok{\# Quality}
\NormalTok{quality }\OtherTok{\textless{}{-}}\NormalTok{ (good\_parts }\SpecialCharTok{/}\NormalTok{ total\_parts) }\SpecialCharTok{*} \DecValTok{100}
\FunctionTok{cat}\NormalTok{(}\StringTok{"QUALITY:}\SpecialCharTok{\textbackslash{}n}\StringTok{"}\NormalTok{)}
\end{Highlighting}
\end{Shaded}

\begin{verbatim}
## QUALITY:
\end{verbatim}

\begin{Shaded}
\begin{Highlighting}[]
\FunctionTok{cat}\NormalTok{(}\StringTok{"  = Good Parts / Total Parts}\SpecialCharTok{\textbackslash{}n}\StringTok{"}\NormalTok{)}
\end{Highlighting}
\end{Shaded}

\begin{verbatim}
##   = Good Parts / Total Parts
\end{verbatim}

\begin{Shaded}
\begin{Highlighting}[]
\FunctionTok{cat}\NormalTok{(}\StringTok{"  ="}\NormalTok{, good\_parts, }\StringTok{"/"}\NormalTok{, total\_parts, }\StringTok{"}\SpecialCharTok{\textbackslash{}n}\StringTok{"}\NormalTok{)}
\end{Highlighting}
\end{Shaded}

\begin{verbatim}
##   = 665 / 700
\end{verbatim}

\begin{Shaded}
\begin{Highlighting}[]
\FunctionTok{cat}\NormalTok{(}\StringTok{"  ="}\NormalTok{, }\FunctionTok{round}\NormalTok{(quality, }\DecValTok{1}\NormalTok{), }\StringTok{"\%}\SpecialCharTok{\textbackslash{}n\textbackslash{}n}\StringTok{"}\NormalTok{)}
\end{Highlighting}
\end{Shaded}

\begin{verbatim}
##   = 95 %
\end{verbatim}

\begin{Shaded}
\begin{Highlighting}[]
\CommentTok{\# Overall OEE}
\NormalTok{oee }\OtherTok{\textless{}{-}}\NormalTok{ (availability}\SpecialCharTok{/}\DecValTok{100}\NormalTok{) }\SpecialCharTok{*}\NormalTok{ (performance}\SpecialCharTok{/}\DecValTok{100}\NormalTok{) }\SpecialCharTok{*}\NormalTok{ (quality}\SpecialCharTok{/}\DecValTok{100}\NormalTok{) }\SpecialCharTok{*} \DecValTok{100}
\FunctionTok{cat}\NormalTok{(}\StringTok{"=== OVERALL OEE ===}\SpecialCharTok{\textbackslash{}n}\StringTok{"}\NormalTok{)}
\end{Highlighting}
\end{Shaded}

\begin{verbatim}
## === OVERALL OEE ===
\end{verbatim}

\begin{Shaded}
\begin{Highlighting}[]
\FunctionTok{cat}\NormalTok{(}\StringTok{"OEE ="}\NormalTok{, }\FunctionTok{round}\NormalTok{(availability, }\DecValTok{1}\NormalTok{), }\StringTok{"\% ×"}\NormalTok{, }\FunctionTok{round}\NormalTok{(performance, }\DecValTok{1}\NormalTok{), }\StringTok{"\% ×"}\NormalTok{, }\FunctionTok{round}\NormalTok{(quality, }\DecValTok{1}\NormalTok{), }\StringTok{"\%}\SpecialCharTok{\textbackslash{}n}\StringTok{"}\NormalTok{)}
\end{Highlighting}
\end{Shaded}

\begin{verbatim}
## OEE = 84.4 % × 92.1 % × 95 %
\end{verbatim}

\begin{Shaded}
\begin{Highlighting}[]
\FunctionTok{cat}\NormalTok{(}\StringTok{"OEE ="}\NormalTok{, }\FunctionTok{round}\NormalTok{(oee, }\DecValTok{1}\NormalTok{), }\StringTok{"\%}\SpecialCharTok{\textbackslash{}n}\StringTok{"}\NormalTok{)}
\end{Highlighting}
\end{Shaded}

\begin{verbatim}
## OEE = 73.9 %
\end{verbatim}

\begin{figure}

{\centering \includegraphics{introduction_files/figure-latex/oee-waterfall-1} 

}

\caption{OEE Waterfall Chart: Where Time Goes}\label{fig:oee-waterfall}
\end{figure}

\subsection{OEE Benchmarks}\label{oee-benchmarks}

\label{tab:oee-benchmarks}OEE Benchmarks

OEE Level

Classification

Interpretation

Component Targets

\textless{} 65\%

Unacceptable

Significant improvement opportunities exist

---

65\% - 75\%

Typical

Average for discrete manufacturing

A: 90\%, P: 85\%, Q: 95\%

75\% - 85\%

Good

Well-managed operations

A: 92\%, P: 90\%, Q: 97\%

\textgreater{} 85\%

World Class

Best-in-class performance

A: 95\%, P: 95\%, Q: 99\%

Interactive Exercise: Calculate OEE

\textbf{Given the following shift data, calculate OEE:}

\begin{itemize}
\tightlist
\item
  Shift length: 480 minutes
\item
  Planned breaks: 40 minutes
\item
  Equipment breakdown: 30 minutes
\item
  Setup/changeover: 20 minutes
\item
  Ideal cycle time: 1 minute/part
\item
  Total parts produced: 350
\item
  Rejected parts: 14
\end{itemize}

\textbf{Work through the calculation:}

\textbf{Solution:}

\begin{verbatim}
Planned Production Time = 480 - 40 = 440 minutes
Run Time = 440 - 30 - 20 = 390 minutes

Availability = 390 / 440 = 88.6%

Theoretical Output = 390 / 1 = 390 parts
Performance = 350 / 390 = 89.7%

Quality = (350 - 14) / 350 = 336 / 350 = 96.0%

OEE = 88.6% × 89.7% × 96.0% = 76.3%
\end{verbatim}

\textbf{Interpretation:} This is ``Good'' performance, above the typical range. The biggest opportunity is in Performance (speed losses).

\begin{center}\rule{0.5\linewidth}{0.5pt}\end{center}

\section{The Six Big Losses}\label{the-six-big-losses}

TPM categorizes all equipment losses into \textbf{six categories}, which map directly to the OEE components.

\begin{figure}

{\centering \includegraphics{introduction_files/figure-latex/six-big-losses-1} 

}

\caption{The Six Big Losses}\label{fig:six-big-losses}
\end{figure}

\label{tab:six-losses-detail}The Six Big Losses Explained

\#

Loss Category

OEE Component

Description

Examples

1

\textbf{Equipment Failure}

Availability

Unplanned equipment breakdown

Motor failure, sensor malfunction, mechanical breakdown

2

\textbf{Setup \& Adjustment}

Availability

Time for changeover, tool changes, adjustments

Die change, product changeover, calibration

3

\textbf{Idling \& Minor Stops}

Performance

Brief stoppages (jams, blockages, cleaning)

Sensor trips, material jams, overloads (\textless5 min)

4

\textbf{Reduced Speed}

Performance

Running slower than design speed

Worn tooling, operator caution, material variability

5

\textbf{Process Defects}

Quality

Defects produced during normal operation

Scrap, rework, out-of-spec product

6

\textbf{Startup Rejects}

Quality

Defects during warmup, startup, changeover

First-off rejects, warming up, stabilization

\subsection{Typical Loss Distribution}\label{typical-loss-distribution}

\begin{figure}

{\centering \includegraphics{introduction_files/figure-latex/loss-distribution-1} 

}

\caption{Typical Distribution of the Six Big Losses}\label{fig:loss-distribution}
\end{figure}

\begin{center}\rule{0.5\linewidth}{0.5pt}\end{center}

\section{Autonomous Maintenance}\label{autonomous-maintenance}

\textbf{Autonomous Maintenance (AM)} is the first and most visible pillar of TPM. It transfers basic maintenance tasks from maintenance specialists to operators.

\subsection{The Seven Steps of Autonomous Maintenance}\label{the-seven-steps-of-autonomous-maintenance}

\begin{figure}

{\centering \includegraphics{introduction_files/figure-latex/am-seven-steps-1} 

}

\caption{Seven Steps of Autonomous Maintenance}\label{fig:am-seven-steps}
\end{figure}

\label{tab:am-steps-detail}Autonomous Maintenance Steps in Detail

Step

Name

Activities

Outcome

1

\textbf{Initial Cleaning}

Deep clean equipment; identify abnormalities

Equipment baseline; 100+ abnormalities found typical

2

\textbf{Eliminate Contamination Sources}

Find and fix sources of dirt, leaks, contamination

Reduced cleaning time; contamination controlled

3

\textbf{Cleaning \& Lubrication Standards}

Create standards for cleaning, lubricating, inspection

Consistent equipment condition; clear expectations

4

\textbf{General Inspection}

Train operators on equipment function; expand skills

Skilled operators; detect abnormalities early

5

\textbf{Autonomous Inspection}

Operators perform scheduled inspections independently

Reliable self-directed maintenance program

6

\textbf{Organization \& Tidiness}

Standardize workplace; visual management

Efficient, safe, visual workplace

7

\textbf{Full Autonomous Maintenance}

Continuous improvement; operators propose improvements

Empowered operators; sustained improvement

\subsection{The AM Tag System}\label{the-am-tag-system}

During autonomous maintenance activities, operators identify abnormalities using \textbf{tags}:

\begin{figure}

{\centering \includegraphics{introduction_files/figure-latex/tag-system-1} 

}

\caption{AM Tag System}\label{fig:tag-system}
\end{figure}

\subsection{Example: AM Cleaning and Inspection Checklist}\label{example-am-cleaning-and-inspection-checklist}

\label{tab:am-checklist}Sample AM Inspection Checklist

\#

Check Item

Method

Standard

Frequency

1

General cleanliness

Visual

No dust, debris, contamination

Daily

2

Oil/lubricant levels

Visual/gauge

Between min-max marks

Daily

3

Loose bolts/fasteners

Touch/wrench

Hand tight + 1/4 turn

Weekly

4

Unusual sounds

Listen

Normal operating sound

Each startup

5

Vibration

Touch

Minimal, no unusual patterns

Each startup

6

Temperature

Touch/thermometer

\textless{} 60°C (or per spec)

Hourly

7

Leaks (oil, air, coolant)

Visual

No visible leaks

Daily

8

Safety guards in place

Visual

All guards secure

Each shift

\begin{center}\rule{0.5\linewidth}{0.5pt}\end{center}

\section{Planned Maintenance}\label{planned-maintenance}

\textbf{Planned Maintenance} is the second pillar, focusing on scheduled maintenance activities performed by maintenance specialists.

\subsection{Types of Maintenance}\label{types-of-maintenance}

\begin{figure}

{\centering \includegraphics{introduction_files/figure-latex/maintenance-types-1} 

}

\caption{Types of Maintenance Activities}\label{fig:maintenance-types}
\end{figure}

\label{tab:maintenance-comparison}Maintenance Strategy Comparison

Type

Approach

Advantages

Disadvantages

Best For

\textbf{Breakdown}

Run to failure

No upfront cost

Unplanned downtime, secondary damage

Non-critical, cheap to replace

\textbf{Preventive}

Time/cycle-based replacement

Planned downtime, parts availability

May replace good parts

Wear items with known life

\textbf{Predictive}

Condition monitoring

Optimal component life, targeted actions

Monitoring investment required

Critical equipment, expensive failures

\textbf{Proactive}

Root cause elimination

Prevents future failures

Requires analysis capability

Recurring problems

\subsection{Predictive Maintenance Technologies}\label{predictive-maintenance-technologies}

\label{tab:pdm-technologies}Predictive Maintenance Technologies

Technology

What It Detects

Equipment Applications

Typical Warning Time

\textbf{Vibration Analysis}

Imbalance, misalignment, bearing wear, looseness

Rotating equipment (motors, pumps, fans, gearboxes)

Weeks to months

\textbf{Thermography}

Hot spots, electrical issues, friction, insulation problems

Electrical panels, motors, bearings, refractory

Hours to days

\textbf{Oil Analysis}

Contamination, wear particles, lubricant degradation

Gearboxes, hydraulics, engines, compressors

Weeks to months

\textbf{Ultrasound}

Leaks, bearing defects, electrical discharge

Steam traps, valves, bearings, electrical equipment

Days to weeks

\textbf{Motor Current Analysis}

Motor winding issues, rotor bar problems, power quality

Electric motors, drives, generators

Days to weeks

\subsection{The P-F Curve}\label{the-p-f-curve}

The \textbf{P-F Curve} illustrates the concept of detecting potential failure (P) before functional failure (F).

\begin{verbatim}
## Warning in geom_point(aes(x = 4, y = 75), size = 5, color = "#F39C12"): All aesthetics have length 1, but the data has 101
## rows.
## i Please consider using `annotate()` or provide
##   this layer with data containing a single row.
\end{verbatim}

\begin{verbatim}
## Warning in geom_point(aes(x = 9, y = 10), size = 5, color = "#E74C3C"): All aesthetics have length 1, but the data has 101
## rows.
## i Please consider using `annotate()` or provide
##   this layer with data containing a single row.
\end{verbatim}

\begin{figure}

{\centering \includegraphics{introduction_files/figure-latex/pf-curve-1} 

}

\caption{The P-F Curve: Detecting Failure Early}\label{fig:pf-curve}
\end{figure}

Question: Why is predictive maintenance better than preventive maintenance?

\textbf{Preventive Maintenance (Time-Based):}
- Replaces parts at fixed intervals regardless of condition
- May replace parts still in good condition (waste)
- May miss failures that occur before scheduled replacement
- Based on average life, not individual component condition

\textbf{Predictive Maintenance (Condition-Based):}
- Monitors actual equipment condition
- Replaces parts when condition indicates need
- Maximizes component life
- Provides warning time to plan repairs

\textbf{Example:}
- Preventive: Replace bearing every 6 months (may last 9 months or fail at 4 months)
- Predictive: Monitor vibration; replace when trending indicates 2-3 weeks to failure
- Result: Optimal component life, planned downtime, no unexpected failures

\textbf{Cost Comparison:}
- Reactive repair: \$10,000 (including secondary damage, lost production)
- Preventive replacement: \$3,000 (may waste \$1,000 in parts life)
- Predictive replacement: \$2,500 (optimal timing, minimal waste)

\begin{center}\rule{0.5\linewidth}{0.5pt}\end{center}

\section{Focused Improvement (Kobetsu Kaizen)}\label{focused-improvement-kobetsu-kaizen}

\textbf{Focused Improvement} uses cross-functional teams to eliminate specific losses identified through OEE analysis.

\subsection{The Kaizen Process}\label{the-kaizen-process}

\begin{figure}

{\centering \includegraphics{introduction_files/figure-latex/kaizen-process-1} 

}

\caption{Focused Improvement Process}\label{fig:kaizen-process}
\end{figure}

\subsection{Loss Elimination Tools}\label{loss-elimination-tools}

\label{tab:loss-tools}Tools for Loss Elimination

Tool

Purpose

Application

\textbf{Pareto Analysis}

Identify biggest losses to target

Prioritize improvement efforts

\textbf{Why-Why Analysis}

Find root cause of failures

Breakdown analysis, defect investigation

\textbf{PM Analysis}

Analyze chronic losses systematically

Complex, recurring problems

\textbf{SMED}

Reduce setup/changeover time

Availability improvement (Loss \#2)

\textbf{Poka-Yoke}

Error-proof the process

Quality improvement (Losses \#5, \#6)

\begin{center}\rule{0.5\linewidth}{0.5pt}\end{center}

\section{TPM Implementation}\label{tpm-implementation}

\subsection{The 12 Steps of TPM Implementation}\label{the-12-steps-of-tpm-implementation}

\label{tab:implementation-steps-TPM}12 Steps of TPM Implementation

Phase

Step

Activity

Duration

\textbf{Preparation}

1

Announce TPM introduction by top management

1 month

\textbf{Preparation}

2

Education and campaign to introduce TPM

3-6 months

\textbf{Preparation}

3

Create TPM promotion organization

1-2 months

\textbf{Preparation}

4

Establish basic TPM policies and goals

1-2 months

\textbf{Preparation}

5

Formulate master plan for TPM development

1-2 months

\textbf{Implementation}

6

Hold TPM kick-off

1 day

\textbf{Implementation}

7

Improve effectiveness of each piece of equipment

6-12 months

\textbf{Implementation}

8

Develop autonomous maintenance program

12-24 months

\textbf{Implementation}

9

Develop planned maintenance program

12-24 months

\textbf{Implementation}

10

Conduct training to improve skills

Ongoing

\textbf{Stabilization}

11

Develop early equipment management program

12-24 months

\textbf{Stabilization}

12

Perfect TPM implementation and raise levels

Ongoing

\subsection{Success Factors}\label{success-factors}

\label{tab:success-factors}TPM Success Factors

Factor

Why Critical

\textbf{Top Management Commitment}

Resources, culture change, sustained focus

\textbf{Training and Education}

Skills to identify problems, perform maintenance

\textbf{Small Group Activities}

Operator involvement, team problem-solving

\textbf{Visual Management}

Make abnormalities obvious, track progress

\textbf{Patience and Persistence}

TPM takes 3-5 years to fully implement

\begin{center}\rule{0.5\linewidth}{0.5pt}\end{center}

\section{Case Study: TPM Implementation}\label{case-study-tpm-implementation}

\subsection{Background}\label{background-1}

\textbf{Company:} Automotive component manufacturer

\textbf{Equipment:} CNC machining center

\textbf{Initial OEE:} 52\%

\subsection{Problem Analysis}\label{problem-analysis}

\begin{Shaded}
\begin{Highlighting}[]
\CommentTok{\# Initial State}
\FunctionTok{cat}\NormalTok{(}\StringTok{"=== INITIAL OEE BREAKDOWN ===}\SpecialCharTok{\textbackslash{}n\textbackslash{}n}\StringTok{"}\NormalTok{)}
\end{Highlighting}
\end{Shaded}

\begin{verbatim}
## === INITIAL OEE BREAKDOWN ===
\end{verbatim}

\begin{Shaded}
\begin{Highlighting}[]
\NormalTok{availability\_initial }\OtherTok{\textless{}{-}} \DecValTok{78}  \CommentTok{\# \%}
\NormalTok{performance\_initial }\OtherTok{\textless{}{-}} \DecValTok{82}   \CommentTok{\# \%}
\NormalTok{quality\_initial }\OtherTok{\textless{}{-}} \DecValTok{81}       \CommentTok{\# \%}

\NormalTok{oee\_initial }\OtherTok{\textless{}{-}}\NormalTok{ (availability\_initial}\SpecialCharTok{/}\DecValTok{100}\NormalTok{) }\SpecialCharTok{*}\NormalTok{ (performance\_initial}\SpecialCharTok{/}\DecValTok{100}\NormalTok{) }\SpecialCharTok{*}\NormalTok{ (quality\_initial}\SpecialCharTok{/}\DecValTok{100}\NormalTok{) }\SpecialCharTok{*} \DecValTok{100}
\FunctionTok{cat}\NormalTok{(}\StringTok{"Availability:"}\NormalTok{, availability\_initial, }\StringTok{"\%}\SpecialCharTok{\textbackslash{}n}\StringTok{"}\NormalTok{)}
\end{Highlighting}
\end{Shaded}

\begin{verbatim}
## Availability: 78 %
\end{verbatim}

\begin{Shaded}
\begin{Highlighting}[]
\FunctionTok{cat}\NormalTok{(}\StringTok{"Performance:"}\NormalTok{, performance\_initial, }\StringTok{"\%}\SpecialCharTok{\textbackslash{}n}\StringTok{"}\NormalTok{)}
\end{Highlighting}
\end{Shaded}

\begin{verbatim}
## Performance: 82 %
\end{verbatim}

\begin{Shaded}
\begin{Highlighting}[]
\FunctionTok{cat}\NormalTok{(}\StringTok{"Quality:"}\NormalTok{, quality\_initial, }\StringTok{"\%}\SpecialCharTok{\textbackslash{}n}\StringTok{"}\NormalTok{)}
\end{Highlighting}
\end{Shaded}

\begin{verbatim}
## Quality: 81 %
\end{verbatim}

\begin{Shaded}
\begin{Highlighting}[]
\FunctionTok{cat}\NormalTok{(}\StringTok{"OEE:"}\NormalTok{, }\FunctionTok{round}\NormalTok{(oee\_initial, }\DecValTok{1}\NormalTok{), }\StringTok{"\%}\SpecialCharTok{\textbackslash{}n\textbackslash{}n}\StringTok{"}\NormalTok{)}
\end{Highlighting}
\end{Shaded}

\begin{verbatim}
## OEE: 51.8 %
\end{verbatim}

\begin{Shaded}
\begin{Highlighting}[]
\CommentTok{\# Loss breakdown}
\FunctionTok{cat}\NormalTok{(}\StringTok{"=== TOP LOSSES IDENTIFIED ===}\SpecialCharTok{\textbackslash{}n}\StringTok{"}\NormalTok{)}
\end{Highlighting}
\end{Shaded}

\begin{verbatim}
## === TOP LOSSES IDENTIFIED ===
\end{verbatim}

\begin{Shaded}
\begin{Highlighting}[]
\FunctionTok{cat}\NormalTok{(}\StringTok{"1. Equipment breakdowns: 8\% of time}\SpecialCharTok{\textbackslash{}n}\StringTok{"}\NormalTok{)}
\end{Highlighting}
\end{Shaded}

\begin{verbatim}
## 1. Equipment breakdowns: 8% of time
\end{verbatim}

\begin{Shaded}
\begin{Highlighting}[]
\FunctionTok{cat}\NormalTok{(}\StringTok{"2. Setup/changeover: 7\% of time}\SpecialCharTok{\textbackslash{}n}\StringTok{"}\NormalTok{)}
\end{Highlighting}
\end{Shaded}

\begin{verbatim}
## 2. Setup/changeover: 7% of time
\end{verbatim}

\begin{Shaded}
\begin{Highlighting}[]
\FunctionTok{cat}\NormalTok{(}\StringTok{"3. Minor stops: 12\% of time}\SpecialCharTok{\textbackslash{}n}\StringTok{"}\NormalTok{)}
\end{Highlighting}
\end{Shaded}

\begin{verbatim}
## 3. Minor stops: 12% of time
\end{verbatim}

\begin{Shaded}
\begin{Highlighting}[]
\FunctionTok{cat}\NormalTok{(}\StringTok{"4. Defects: 4\% of production}\SpecialCharTok{\textbackslash{}n}\StringTok{"}\NormalTok{)}
\end{Highlighting}
\end{Shaded}

\begin{verbatim}
## 4. Defects: 4% of production
\end{verbatim}

\subsection{Improvements Implemented}\label{improvements-implemented}

\begin{longtable}[]{@{}
  >{\raggedright\arraybackslash}p{(\linewidth - 6\tabcolsep) * \real{0.1429}}
  >{\raggedright\arraybackslash}p{(\linewidth - 6\tabcolsep) * \real{0.2857}}
  >{\raggedright\arraybackslash}p{(\linewidth - 6\tabcolsep) * \real{0.3810}}
  >{\raggedright\arraybackslash}p{(\linewidth - 6\tabcolsep) * \real{0.1905}}@{}}
\toprule\noalign{}
\begin{minipage}[b]{\linewidth}\raggedright
Loss
\end{minipage} & \begin{minipage}[b]{\linewidth}\raggedright
Root Cause
\end{minipage} & \begin{minipage}[b]{\linewidth}\raggedright
Countermeasure
\end{minipage} & \begin{minipage}[b]{\linewidth}\raggedright
Result
\end{minipage} \\
\midrule\noalign{}
\endhead
\bottomrule\noalign{}
\endlastfoot
Breakdowns & Lack of basic maintenance & Autonomous maintenance program & 8\% → 3\% \\
Setup time & No standard procedure & SMED implementation & 35 min → 12 min \\
Minor stops & Chip buildup in sensors & Improved chip management & 12\% → 4\% \\
Defects & Tool wear not detected & In-process gauging & 4\% → 1.5\% \\
\end{longtable}

\subsection{Results After 18 Months}\label{results-after-18-months}

\begin{Shaded}
\begin{Highlighting}[]
\CommentTok{\# After TPM Implementation}
\FunctionTok{cat}\NormalTok{(}\StringTok{"=== FINAL OEE BREAKDOWN ===}\SpecialCharTok{\textbackslash{}n\textbackslash{}n}\StringTok{"}\NormalTok{)}
\end{Highlighting}
\end{Shaded}

\begin{verbatim}
## === FINAL OEE BREAKDOWN ===
\end{verbatim}

\begin{Shaded}
\begin{Highlighting}[]
\NormalTok{availability\_final }\OtherTok{\textless{}{-}} \DecValTok{91}    \CommentTok{\# \%}
\NormalTok{performance\_final }\OtherTok{\textless{}{-}} \DecValTok{94}     \CommentTok{\# \%}
\NormalTok{quality\_final }\OtherTok{\textless{}{-}} \FloatTok{98.5}       \CommentTok{\# \%}

\NormalTok{oee\_final }\OtherTok{\textless{}{-}}\NormalTok{ (availability\_final}\SpecialCharTok{/}\DecValTok{100}\NormalTok{) }\SpecialCharTok{*}\NormalTok{ (performance\_final}\SpecialCharTok{/}\DecValTok{100}\NormalTok{) }\SpecialCharTok{*}\NormalTok{ (quality\_final}\SpecialCharTok{/}\DecValTok{100}\NormalTok{) }\SpecialCharTok{*} \DecValTok{100}
\FunctionTok{cat}\NormalTok{(}\StringTok{"Availability:"}\NormalTok{, availability\_final, }\StringTok{"\% (was"}\NormalTok{, availability\_initial, }\StringTok{"\%)}\SpecialCharTok{\textbackslash{}n}\StringTok{"}\NormalTok{)}
\end{Highlighting}
\end{Shaded}

\begin{verbatim}
## Availability: 91 % (was 78 %)
\end{verbatim}

\begin{Shaded}
\begin{Highlighting}[]
\FunctionTok{cat}\NormalTok{(}\StringTok{"Performance:"}\NormalTok{, performance\_final, }\StringTok{"\% (was"}\NormalTok{, performance\_initial, }\StringTok{"\%)}\SpecialCharTok{\textbackslash{}n}\StringTok{"}\NormalTok{)}
\end{Highlighting}
\end{Shaded}

\begin{verbatim}
## Performance: 94 % (was 82 %)
\end{verbatim}

\begin{Shaded}
\begin{Highlighting}[]
\FunctionTok{cat}\NormalTok{(}\StringTok{"Quality:"}\NormalTok{, quality\_final, }\StringTok{"\% (was"}\NormalTok{, quality\_initial, }\StringTok{"\%)}\SpecialCharTok{\textbackslash{}n}\StringTok{"}\NormalTok{)}
\end{Highlighting}
\end{Shaded}

\begin{verbatim}
## Quality: 98.5 % (was 81 %)
\end{verbatim}

\begin{Shaded}
\begin{Highlighting}[]
\FunctionTok{cat}\NormalTok{(}\StringTok{"OEE:"}\NormalTok{, }\FunctionTok{round}\NormalTok{(oee\_final, }\DecValTok{1}\NormalTok{), }\StringTok{"\% (was"}\NormalTok{, }\FunctionTok{round}\NormalTok{(oee\_initial, }\DecValTok{1}\NormalTok{), }\StringTok{"\%)}\SpecialCharTok{\textbackslash{}n\textbackslash{}n}\StringTok{"}\NormalTok{)}
\end{Highlighting}
\end{Shaded}

\begin{verbatim}
## OEE: 84.3 % (was 51.8 %)
\end{verbatim}

\begin{Shaded}
\begin{Highlighting}[]
\NormalTok{improvement }\OtherTok{\textless{}{-}}\NormalTok{ oee\_final }\SpecialCharTok{{-}}\NormalTok{ oee\_initial}
\FunctionTok{cat}\NormalTok{(}\StringTok{"=== IMPROVEMENT ===}\SpecialCharTok{\textbackslash{}n}\StringTok{"}\NormalTok{)}
\end{Highlighting}
\end{Shaded}

\begin{verbatim}
## === IMPROVEMENT ===
\end{verbatim}

\begin{Shaded}
\begin{Highlighting}[]
\FunctionTok{cat}\NormalTok{(}\StringTok{"OEE Improvement:"}\NormalTok{, }\FunctionTok{round}\NormalTok{(improvement, }\DecValTok{1}\NormalTok{), }\StringTok{"percentage points}\SpecialCharTok{\textbackslash{}n}\StringTok{"}\NormalTok{)}
\end{Highlighting}
\end{Shaded}

\begin{verbatim}
## OEE Improvement: 32.4 percentage points
\end{verbatim}

\begin{Shaded}
\begin{Highlighting}[]
\FunctionTok{cat}\NormalTok{(}\StringTok{"Relative Improvement:"}\NormalTok{, }\FunctionTok{round}\NormalTok{((improvement}\SpecialCharTok{/}\NormalTok{oee\_initial)}\SpecialCharTok{*}\DecValTok{100}\NormalTok{, }\DecValTok{0}\NormalTok{), }\StringTok{"\%}\SpecialCharTok{\textbackslash{}n}\StringTok{"}\NormalTok{)}
\end{Highlighting}
\end{Shaded}

\begin{verbatim}
## Relative Improvement: 63 %
\end{verbatim}

\begin{figure}

{\centering \includegraphics{introduction_files/figure-latex/case-visualization-1} 

}

\caption{OEE Improvement: Before and After TPM}\label{fig:case-visualization}
\end{figure}

\begin{center}\rule{0.5\linewidth}{0.5pt}\end{center}

\section{Summary}\label{summary-8}

\label{tab:summary-table-ch10}TPM Key Concepts Summary

Topic

Key Points

\textbf{TPM Definition}

Total involvement, productive focus, maintenance excellence; zero breakdowns/defects/accidents

\textbf{Eight Pillars}

AM, PM, Quality, Focused Improvement, Early Equipment, Training, Safety, Office --- built on 5S

\textbf{OEE}

OEE = Availability × Performance × Quality; World class \textgreater{} 85\%

\textbf{Six Big Losses}

Equipment failure, setup, minor stops, speed loss, defects, startup rejects

\textbf{Autonomous Maintenance}

Operators maintain own equipment; 7 steps from cleaning to full ownership

\textbf{Planned Maintenance}

Preventive → Predictive → Proactive; use P-F curve for optimal timing

\begin{center}\rule{0.5\linewidth}{0.5pt}\end{center}

\section{Review Questions}\label{review-questions-9}

\subsection{Conceptual Questions}\label{conceptual-questions-2}

Question 1: What does ``Total'' mean in Total Productive Maintenance?

\textbf{``Total'' has three dimensions:}

\begin{enumerate}
\def\labelenumi{\arabic{enumi}.}
\tightlist
\item
  \textbf{Total participation:} Everyone participates, from operators to top management
\item
  \textbf{Total equipment effectiveness:} Aim for maximum OEE (zero losses)
\item
  \textbf{Total system:} Covers the entire equipment lifecycle, from design to disposal
\end{enumerate}

\textbf{Key insight:} TPM is not just a maintenance program --- it's a company-wide philosophy where everyone takes responsibility for equipment care.

Question 2: Why is OEE calculated as a product rather than a sum?

\textbf{OEE uses multiplication because losses are sequential and compound:}

\begin{enumerate}
\def\labelenumi{\arabic{enumi}.}
\tightlist
\item
  \textbf{Availability} determines how much time is available
\item
  \textbf{Performance} determines how much of that time is productive
\item
  \textbf{Quality} determines how much of that production is good
\end{enumerate}

\textbf{Example:}
- If each factor is 90\%: OEE = 0.9 × 0.9 × 0.9 = 72.9\% (not 270\% or 90\%)
- This reflects reality: you can only achieve quality on parts that were produced (performance) during time the machine ran (availability)

\textbf{Why this matters:}
- Small improvements in all three factors compound
- 5\% improvement in each: 0.85 × 0.85 × 0.85 = 61\% → 0.90 × 0.90 × 0.90 = 73\%
- Focus on the weakest factor first for biggest impact

Question 3: What is the difference between autonomous maintenance and planned maintenance?

\begin{longtable}[]{@{}
  >{\raggedright\arraybackslash}p{(\linewidth - 4\tabcolsep) * \real{0.1509}}
  >{\raggedright\arraybackslash}p{(\linewidth - 4\tabcolsep) * \real{0.4528}}
  >{\raggedright\arraybackslash}p{(\linewidth - 4\tabcolsep) * \real{0.3962}}@{}}
\toprule\noalign{}
\begin{minipage}[b]{\linewidth}\raggedright
Aspect
\end{minipage} & \begin{minipage}[b]{\linewidth}\raggedright
Autonomous Maintenance
\end{minipage} & \begin{minipage}[b]{\linewidth}\raggedright
Planned Maintenance
\end{minipage} \\
\midrule\noalign{}
\endhead
\bottomrule\noalign{}
\endlastfoot
\textbf{Who} & Operators & Maintenance technicians \\
\textbf{What} & Basic care (clean, inspect, lubricate) & Complex repairs, overhauls \\
\textbf{When} & Daily, during operation & Scheduled intervals \\
\textbf{Skills} & Basic equipment knowledge & Technical expertise \\
\textbf{Goal} & Prevent deterioration & Restore/improve equipment \\
\end{longtable}

\textbf{They complement each other:}
- AM catches problems early (first line of defense)
- PM provides specialized skills when needed
- Together they achieve zero breakdowns

\subsection{Calculation Questions}\label{calculation-questions-2}

Question 4: Calculate OEE given the following data.

\textbf{Given:}
- Shift length: 480 minutes
- Breaks: 30 minutes
- Changeover: 45 minutes
- Breakdown: 20 minutes
- Ideal cycle time: 30 seconds (0.5 min)
- Total parts produced: 580
- Rejected parts: 23

\textbf{Solution:}

\begin{verbatim}
Planned Production Time = 480 - 30 = 450 minutes
Run Time = 450 - 45 - 20 = 385 minutes

Availability = 385 / 450 = 85.6%

Theoretical Output = 385 / 0.5 = 770 parts
Performance = 580 / 770 = 75.3%

Quality = (580 - 23) / 580 = 557 / 580 = 96.0%

OEE = 85.6% × 75.3% × 96.0% = 61.9%
\end{verbatim}

\textbf{Analysis:} Performance is the biggest opportunity for improvement (75.3\% vs.~targets of 95\%). Investigate minor stops and speed losses.

Question 5: A machine has OEE of 65\% with Availability 90\%, Performance 80\%, Quality 90\%. Which factor should be prioritized?

\textbf{Analysis:}

Current state:
- Availability: 90\% (Gap to world class: 5\%)
- Performance: 80\% (Gap to world class: 15\%)
- Quality: 90\% (Gap to world class: 9\%)

\textbf{Performance should be prioritized because:}

\begin{enumerate}
\def\labelenumi{\arabic{enumi}.}
\tightlist
\item
  \textbf{Biggest gap:} 15 percentage points below world class (95\%)
\item
  \textbf{Biggest OEE impact:} Improving performance 10\% → OEE = 90\% × 90\% × 90\% = 72.9\%
\item
  \textbf{Often overlooked:} Speed losses and minor stops are frequently accepted as normal
\end{enumerate}

\textbf{Investigation areas for performance:}
- Minor stops (jams, blockages, sensor trips)
- Reduced speed operation
- Idling and waiting
- Operator pace

Question 6: If current OEE is 60\% and target is 85\%, what improvement is needed in each component assuming equal improvement?

\textbf{Solution:}

Current: 60\% = A × P × Q

If we improve each equally, we need:
- Target: 85\% = A' × P' × Q'
- If A' = P' = Q' = x, then x³ = 0.85
- x = ∛0.85 = 0.947 = 94.7\%

\textbf{Each component needs to reach approximately 95\%}

\textbf{Check:} 0.947 × 0.947 × 0.947 = 0.849 ≈ 85\%

\textbf{If current factors are A=80\%, P=85\%, Q=88\%:}
- Availability needs: +15 points (80\% → 95\%)
- Performance needs: +10 points (85\% → 95\%)
- Quality needs: +7 points (88\% → 95\%)

\textbf{Prioritize by gap size:} Availability first, then Performance, then Quality.

\subsection{Application Questions}\label{application-questions-2}

Question 7: An operator notices a small oil leak on a machine. Under traditional maintenance vs.~TPM, what happens?

\textbf{Traditional Maintenance Approach:}
1. Operator continues running machine
2. May or may not report to maintenance
3. If reported, added to backlog
4. Maintenance addresses when convenient (days/weeks)
5. Leak worsens, potential bearing damage
6. Eventually leads to breakdown

\textbf{TPM Approach:}
1. Operator notices during routine inspection
2. Tags the abnormality immediately
3. Determines if operator can fix (blue tag) or needs maintenance (red tag)
4. If simple: operator tightens fitting or adds sealant
5. If complex: maintenance prioritizes based on severity
6. Root cause investigated to prevent recurrence
7. Finding shared in team meeting

\textbf{Key Differences:}
- \textbf{Ownership:} Operator takes responsibility vs.~``not my job''
- \textbf{Speed:} Immediate action vs.~delayed response
- \textbf{Prevention:} Fixed before failure vs.~reactive repair
- \textbf{Cost:} \$10 seal vs.~\$5,000 bearing replacement + downtime

Question 8: List five items an operator should check during daily autonomous maintenance.

\textbf{Five Essential Daily AM Checks:}

\begin{enumerate}
\def\labelenumi{\arabic{enumi}.}
\tightlist
\item
  \textbf{Cleanliness:} Remove chips, debris, dust; clean safety windows and sensors

  \begin{itemize}
  \tightlist
  \item
    \emph{Why:} Contamination causes wear, sensor malfunctions, quality issues
  \end{itemize}
\item
  \textbf{Lubrication:} Check oil levels, grease points, coolant level

  \begin{itemize}
  \tightlist
  \item
    \emph{Why:} Inadequate lubrication causes friction, heat, accelerated wear
  \end{itemize}
\item
  \textbf{Fasteners:} Check for loose bolts, guards, covers

  \begin{itemize}
  \tightlist
  \item
    \emph{Why:} Loose fasteners cause vibration, safety hazards, misalignment
  \end{itemize}
\item
  \textbf{Leaks:} Look for oil, coolant, air, hydraulic leaks

  \begin{itemize}
  \tightlist
  \item
    \emph{Why:} Leaks indicate seal failure, waste resources, create hazards
  \end{itemize}
\item
  \textbf{Abnormal sounds/vibration:} Listen and feel during startup

  \begin{itemize}
  \tightlist
  \item
    \emph{Why:} Changes indicate developing problems (bearings, alignment)
  \end{itemize}
\end{enumerate}

\textbf{Bonus items:}
- Safety guards in place
- Emergency stops functional
- Pressure/temperature gauges in normal range
- No unusual smells (burning, chemical)

Question 9: Explain how SMED relates to OEE improvement.

\textbf{SMED (Single Minute Exchange of Die) directly improves Availability:}

\textbf{Loss \#2: Setup and Adjustment} is an availability loss captured in OEE.

\textbf{How SMED helps:}

\begin{enumerate}
\def\labelenumi{\arabic{enumi}.}
\tightlist
\item
  \textbf{Reduces changeover time:}

  \begin{itemize}
  \tightlist
  \item
    Before SMED: 45-minute changeover
  \item
    After SMED: 10-minute changeover
  \item
    Gain: 35 minutes per changeover
  \end{itemize}
\item
  \textbf{Impact on OEE:}

  \begin{itemize}
  \tightlist
  \item
    If 2 changeovers per shift: 70 minutes saved
  \item
    On 450-minute shift: Availability improves 15\%+
  \end{itemize}
\item
  \textbf{Secondary benefits:}

  \begin{itemize}
  \tightlist
  \item
    More changeovers become practical
  \item
    Smaller batch sizes possible
  \item
    Flexibility increased
  \item
    Less WIP inventory
  \end{itemize}
\end{enumerate}

\textbf{SMED Process:}
1. Separate internal (machine stopped) from external (while running) setup
2. Convert internal to external where possible
3. Streamline remaining internal operations
4. Practice and standardize

\textbf{Example:} Die change on press
- External: Stage next die, prepare tools, preheat
- Internal: Remove old die, install new die, adjust
- Target: All possible work done while machine runs

\begin{center}\rule{0.5\linewidth}{0.5pt}\end{center}

\section{References}\label{references-9}

\begin{itemize}
\tightlist
\item
  Nakajima, S. (1988). \emph{Introduction to TPM: Total Productive Maintenance}. Productivity Press.
\item
  Nakajima, S. (1989). \emph{TPM Development Program: Implementing Total Productive Maintenance}. Productivity Press.
\item
  Shirose, K. (1992). \emph{TPM for Operators}. Productivity Press.
\item
  Willmott, P., \& McCarthy, D. (2001). \emph{TPM: A Route to World-Class Performance}. Butterworth-Heinemann.
\item
  JIPM (Japan Institute of Plant Maintenance). \emph{TPM Awards Criteria}.
\item
  Hansen, R.C. (2001). \emph{Overall Equipment Effectiveness}. Industrial Press.
\item
  Borris, S. (2006). \emph{Total Productive Maintenance}. McGraw-Hill.
\end{itemize}

\begin{center}\rule{0.5\linewidth}{0.5pt}\end{center}

\chapter{Troubleshooting and Root Cause Analysis}\label{troubleshooting-and-root-cause-analysis}

\section{Learning Objectives}\label{learning-objectives-10}

After completing this chapter, you will be able to:

\begin{enumerate}
\def\labelenumi{\arabic{enumi}.}
\tightlist
\item
  Apply systematic troubleshooting methodologies to diagnose equipment and process failures
\item
  Distinguish between symptoms, immediate causes, and root causes
\item
  Use the 5 Whys technique to drill down to fundamental causes
\item
  Construct and analyze Ishikawa (fishbone) diagrams
\item
  Apply Fault Tree Analysis (FTA) to complex failure scenarios
\item
  Use Pareto analysis to prioritize problem-solving efforts
\item
  Implement the 8D problem-solving methodology
\item
  Document findings and implement effective corrective actions
\item
  Prevent problem recurrence through systemic improvements
\end{enumerate}

\begin{center}\rule{0.5\linewidth}{0.5pt}\end{center}

\section{Introduction to Troubleshooting}\label{introduction-to-troubleshooting}

\textbf{Troubleshooting} is the systematic process of identifying, diagnosing, and resolving problems in equipment, processes, or systems. For electromechanical technicians, effective troubleshooting is perhaps the most valuable skill you can develop.

\subsection{The Cost of Downtime}\label{the-cost-of-downtime}

Equipment failures and process problems have significant financial impacts:

\pandocbounded{\includegraphics[keepaspectratio]{introduction_files/figure-latex/downtime-costs-1.pdf}}

\subsection{Reactive vs.~Proactive Problem Solving}\label{reactive-vs.-proactive-problem-solving}

\begin{table}
\centering
\caption{\label{tab:reactive-proactive}Comparison of Problem-Solving Approaches}
\centering
\begin{tabu} to \linewidth {>{\raggedright}X>{\raggedright}X>{\raggedright}X}
\hline
Aspect & Reactive Approach & Proactive Approach\\
\hline
\textbf{Timing} & \cellcolor[HTML]{ffcccc}{After failure occurs} & \cellcolor[HTML]{ccffcc}{Before or early in failure}\\
\hline
\textbf{Focus} & \cellcolor[HTML]{ffcccc}{Quick fix to restore operation} & \cellcolor[HTML]{ccffcc}{Root cause elimination}\\
\hline
\textbf{Stress Level} & \cellcolor[HTML]{ffcccc}{High - production pressure} & \cellcolor[HTML]{ccffcc}{Lower - planned approach}\\
\hline
\textbf{Cost} & \cellcolor[HTML]{ffcccc}{High - emergency repairs} & \cellcolor[HTML]{ccffcc}{Lower - scheduled work}\\
\hline
\textbf{Learning} & \cellcolor[HTML]{ffcccc}{Limited - rush to fix} & \cellcolor[HTML]{ccffcc}{High - thorough analysis}\\
\hline
\textbf{Documentation} & \cellcolor[HTML]{ffcccc}{Often skipped} & \cellcolor[HTML]{ccffcc}{Systematic and complete}\\
\hline
\end{tabu}
\end{table}

\textbf{Think About It}: Why do organizations often default to reactive troubleshooting?

Common reasons include:
- \textbf{Time pressure}: Production demands make it tempting to apply quick fixes
- \textbf{Lack of training}: Technicians may not know RCA methodologies
- \textbf{Reward systems}: Organizations may reward ``firefighters'' who quickly restore production
- \textbf{Missing data}: Without good records, patterns are hard to identify
- \textbf{Culture}: ``We've always done it this way'' mentality

The irony is that reactive troubleshooting actually costs more time in the long run due to recurring problems.

\begin{center}\rule{0.5\linewidth}{0.5pt}\end{center}

\section{Systematic Troubleshooting Methodology}\label{systematic-troubleshooting-methodology}

Effective troubleshooting follows a structured approach rather than random trial-and-error.

\subsection{The DMAIC Troubleshooting Framework}\label{the-dmaic-troubleshooting-framework}

\pandocbounded{\includegraphics[keepaspectratio]{introduction_files/figure-latex/troubleshooting-flow-1.pdf}}

\subsection{Step 1: Define the Problem}\label{step-1-define-the-problem}

A well-defined problem is half-solved. Use the \textbf{5W2H} method:

\begin{table}
\centering
\caption{\label{tab:5w2h-table}5W2H Problem Definition Framework}
\centering
\begin{tabu} to \linewidth {>{\raggedright}X>{\raggedright}X>{\raggedright}X}
\hline
Question & What to Ask & Example Response\\
\hline
\cellcolor[HTML]{2c3e50}{\textcolor{white}{\textbf{WHAT}}} & What is the problem? What is happening vs. what should happen? & \cellcolor[HTML]{f8f9fa}{Motor overheating on conveyor line 3}\\
\hline
\cellcolor[HTML]{2c3e50}{\textcolor{white}{\textbf{WHERE}}} & Where does the problem occur? Location, machine, station? & \cellcolor[HTML]{f8f9fa}{Discharge end of conveyor, motor housing}\\
\hline
\cellcolor[HTML]{2c3e50}{\textcolor{white}{\textbf{WHEN}}} & When did it start? When does it occur? Continuous or intermittent? & \cellcolor[HTML]{f8f9fa}{Started Monday, occurs after 2 hours of operation}\\
\hline
\cellcolor[HTML]{2c3e50}{\textcolor{white}{\textbf{WHO}}} & Who discovered it? Who is affected? Who was operating? & \cellcolor[HTML]{f8f9fa}{Operator noticed, maintenance called, affects packaging dept}\\
\hline
\cellcolor[HTML]{2c3e50}{\textcolor{white}{\textbf{WHY}}} & Why is this a problem? What is the impact? & \cellcolor[HTML]{f8f9fa}{Risk of motor failure, line stoppage, safety hazard}\\
\hline
\cellcolor[HTML]{2c3e50}{\textcolor{white}{\textbf{HOW}}} & How was the problem detected? How is it manifesting? & \cellcolor[HTML]{f8f9fa}{Detected by high-temp alarm at 85°C}\\
\hline
\cellcolor[HTML]{2c3e50}{\textcolor{white}{\textbf{HOW MUCH}}} & How often? How many units affected? What is the magnitude? & \cellcolor[HTML]{f8f9fa}{Occurs every day, 100\% of shifts, 15°C above normal}\\
\hline
\end{tabu}
\end{table}

\subsection{Problem Statement Template}\label{problem-statement-template}

\begin{quote}
\textbf{Good Problem Statement Format:}
``{[}WHAT{]} is occurring at {[}WHERE{]} since {[}WHEN{]}, affecting {[}WHO/WHAT{]}, with a magnitude of {[}HOW MUCH{]}.''
\end{quote}

\textbf{Example:}
\textgreater{} ``Motor overheating (85°C vs.~normal 70°C) is occurring at Conveyor Line 3 discharge motor since Monday morning, affecting packaging department production, with the condition occurring every shift after approximately 2 hours of operation.''

\textbf{Practice}: Write a problem statement for this scenario

\textbf{Scenario:} An injection molding machine is producing parts with short shots (incomplete fill). The problem was noticed by the quality inspector during the night shift. About 15\% of parts are affected. The issue seems worse when running the larger 500g parts compared to the smaller 200g parts.

\textbf{Sample Problem Statement:}
``Short shots (incomplete cavity fill) are occurring on Injection Molding Machine \#4 since night shift on Tuesday, affecting 15\% of parts produced, with the defect rate increasing significantly when running 500g parts compared to 200g parts.''

\subsection{Step 2: Gather Data and Observations}\label{step-2-gather-data-and-observations}

Before making any changes, collect information systematically:

\begin{table}
\centering
\caption{\label{tab:data-gathering}Systematic Data Gathering Checklist}
\centering
\begin{tabu} to \linewidth {>{\raggedright}X>{\raggedright}X>{\raggedright}X}
\hline
Data Source & What to Check & Tools/Methods\\
\hline
\textbf{Visual Inspection} & Physical condition, wear, contamination, alignment, damage & Flashlight, mirror, magnifier, camera\\
\hline
\textbf{Operator Interview} & What changed? When did it start? Any unusual events? & Open-ended questions, active listening\\
\hline
\textbf{Machine Parameters} & Temperatures, pressures, speeds, currents, positions & HMI screens, gauges, multimeter, thermal camera\\
\hline
\textbf{Historical Records} & Maintenance logs, previous repairs, similar issues & CMMS, logbooks, work orders\\
\hline
\textbf{Product/Output} & Defect patterns, measurements, reject rates & Calipers, gauges, CMM, visual inspection\\
\hline
\textbf{Environmental} & Temperature, humidity, vibration, power quality & Thermometer, hygrometer, vibration analyzer\\
\hline
\end{tabu}
\end{table}

\subsection{The ``Change Analysis'' Approach}\label{the-change-analysis-approach}

Many problems are caused by changes. Ask:

\begin{itemize}
\tightlist
\item
  What changed recently?
\item
  New materials, suppliers, or batches?
\item
  New operators or different shifts?
\item
  Maintenance performed?
\item
  Environmental changes (weather, season)?
\item
  Software updates or parameter changes?
\item
  New products or recipes running?
\end{itemize}

\pandocbounded{\includegraphics[keepaspectratio]{introduction_files/figure-latex/change-timeline-1.pdf}}

\begin{center}\rule{0.5\linewidth}{0.5pt}\end{center}

\section{Understanding Root Cause vs.~Symptoms}\label{understanding-root-cause-vs.-symptoms}

A critical concept in troubleshooting is distinguishing between different levels of causation.

\subsection{The Causation Hierarchy}\label{the-causation-hierarchy}

\pandocbounded{\includegraphics[keepaspectratio]{introduction_files/figure-latex/causation-levels-1.pdf}}

\subsection{Why Finding Root Cause Matters}\label{why-finding-root-cause-matters}

\begin{verbatim}
## Warning in latex_new_row_builder(target_row, table_info, bold, italic,
## monospace, : Setting full_width = TRUE will turn the table into a tabu
## environment where colors are not really easily configable with this package.
## Please consider turn off full_width.
\end{verbatim}

\begin{table}
\centering
\caption{\label{tab:fix-comparison}Comparison of Fixes at Different Causation Levels}
\centering
\begin{tabu} to \linewidth {>{\raggedright}X>{\raggedright}X>{\raggedright}X>{\raggedright}X>{\raggedright}X>{\raggedright}X}
\hline
Fix Level & Action Taken & Immediate Result & Recurrence & Short-term Cost & Long-term Cost\\
\hline
\textbf{Symptom} & Add cooling fan to motor & Motor runs cooler temporarily & Problem returns in days & \$50 & \$5000+ (repeated failures)\\
\hline
\textbf{Immediate Cause} & Replace worn bearing & Motor works for a while & Problem returns in months & \$200 & \$2400 (annual replacements)\\
\hline
\textbf{\cellcolor[HTML]{d4edda}{\textbf{Root Cause}}} & \cellcolor[HTML]{d4edda}{\textbf{Implement PM schedule with lubrication tasks}} & \cellcolor[HTML]{d4edda}{\textbf{Problem eliminated permanently}} & \cellcolor[HTML]{d4edda}{\textbf{Problem does not recur}} & \cellcolor[HTML]{d4edda}{\textbf{\$500}} & \cellcolor[HTML]{d4edda}{\textbf{\$500 (one-time implementation)}}\\
\hline
\end{tabu}
\end{table}

\begin{center}\rule{0.5\linewidth}{0.5pt}\end{center}

\section{The 5 Whys Technique}\label{the-5-whys-technique}

The \textbf{5 Whys} is a simple but powerful technique developed by Sakichi Toyoda and used within Toyota's manufacturing methods.

\subsection{How It Works}\label{how-it-works}

\begin{enumerate}
\def\labelenumi{\arabic{enumi}.}
\tightlist
\item
  State the problem
\item
  Ask ``Why?'' to identify the cause
\item
  Ask ``Why?'' again about that cause
\item
  Continue asking ``Why?'' until you reach the root cause
\item
  Typically requires 5 iterations (sometimes more, sometimes less)
\end{enumerate}

\subsection{5 Whys Example: Production Line Stoppage}\label{whys-example-production-line-stoppage}

\pandocbounded{\includegraphics[keepaspectratio]{introduction_files/figure-latex/five-whys-visual-1.pdf}}

\subsection{5 Whys Template}\label{whys-template}

\begin{verbatim}
## Warning in latex_new_row_builder(target_row, table_info, bold, italic,
## monospace, : Setting full_width = TRUE will turn the table into a tabu
## environment where colors are not really easily configable with this package.
## Please consider turn off full_width.
\end{verbatim}

\begin{table}
\centering
\caption{\label{tab:five-whys-template}5 Whys Analysis Template}
\centering
\begin{tabu} to \linewidth {>{\raggedright}X>{\raggedright}X>{\raggedright}X}
\hline
Step & Description & Typical Cause Type\\
\hline
\textbf{Problem Statement} & Clear, specific description of what happened & Symptom\\
\hline
\textbf{Why \#1} & First-level cause (usually technical/physical) & Physical\\
\hline
\textbf{Why \#2} & Why did that first cause occur? & Physical/Human\\
\hline
\textbf{Why \#3} & Why did the second cause occur? & Human/Process\\
\hline
\textbf{Why \#4} & Why did the third cause occur? & Process/System\\
\hline
\textbf{Why \#5} & Why did the fourth cause occur? & System/Culture\\
\hline
\cellcolor[HTML]{d4edda}{\textbf{\textbf{Root Cause}}} & \cellcolor[HTML]{d4edda}{\textbf{The fundamental cause that, if fixed, prevents recurrence}} & \cellcolor[HTML]{d4edda}{\textbf{Root}}\\
\hline
\end{tabu}
\end{table}

\subsection{Multiple Branches in 5 Whys}\label{multiple-branches-in-5-whys}

Sometimes a problem has multiple causes. When you encounter this, branch your analysis:

\pandocbounded{\includegraphics[keepaspectratio]{introduction_files/figure-latex/branching-whys-1.pdf}}

\textbf{Practice Exercise}: Perform a 5 Whys Analysis

\textbf{Scenario:} A CNC machine produced 50 parts that were all 0.5mm undersize on a critical diameter.

Try to complete the 5 Whys analysis before looking at the sample answer below.

\begin{center}\rule{0.5\linewidth}{0.5pt}\end{center}

\textbf{Sample Analysis:}

\begin{enumerate}
\def\labelenumi{\arabic{enumi}.}
\tightlist
\item
  \textbf{Problem:} CNC produced 50 undersize parts
\item
  \textbf{Why \#1:} The tool offset was set incorrectly
\item
  \textbf{Why \#2:} The operator entered the wrong offset value
\item
  \textbf{Why \#3:} The operator misread the setup sheet
\item
  \textbf{Why \#4:} The setup sheet had poor formatting and small print
\item
  \textbf{Why \#5:} No standard format exists for setup documentation
\end{enumerate}

\textbf{Root Cause:} Lack of standardized, clear setup documentation

\textbf{Corrective Actions:}
- Create standardized setup sheet template with large, clear fonts
- Add visual verification step (photo of correct setup)
- Implement first-piece inspection requirement before production run

\begin{center}\rule{0.5\linewidth}{0.5pt}\end{center}

\section{Ishikawa (Fishbone) Diagram}\label{ishikawa-fishbone-diagram}

The \textbf{Ishikawa diagram}, also called a \textbf{fishbone diagram} or \textbf{cause-and-effect diagram}, was developed by Kaoru Ishikawa in the 1960s. It provides a structured way to brainstorm and categorize potential causes.

\subsection{The 6M Categories}\label{the-6m-categories}

Manufacturing problems are typically organized using the \textbf{6M} categories:

\pandocbounded{\includegraphics[keepaspectratio]{introduction_files/figure-latex/6m-categories-1.pdf}}

\subsection{Constructing a Fishbone Diagram}\label{constructing-a-fishbone-diagram}

\begin{verbatim}
## Warning in geom_segment(data = spine, aes(x = x[1], y = y[1], xend = x[2], : All aesthetics have length 1, but the data has 2
## rows.
## i Please consider using `annotate()` or provide
##   this layer with data containing a single row.
\end{verbatim}

\pandocbounded{\includegraphics[keepaspectratio]{introduction_files/figure-latex/fishbone-example-1.pdf}}

\subsection{Fishbone Diagram Best Practices}\label{fishbone-diagram-best-practices}

\begin{table}
\centering
\caption{\label{tab:fishbone-tips}Best Practices for Fishbone Diagram Development}
\centering
\begin{tabu} to \linewidth {>{\raggedright}X>{\raggedright}X}
\hline
Guideline & Description\\
\hline
\textbf{Team Approach} & Include operators, engineers, maintenance, quality - diverse perspectives\\
\hline
\textbf{No Evaluation During Brainstorming} & Capture all ideas first; evaluate feasibility later\\
\hline
\textbf{Use Specific Language} & Not 'machine problem' but 'bearing wear on spindle motor'\\
\hline
\textbf{Aim for 3-5 Causes per Category} & Some categories may have more; don't force causes into categories\\
\hline
\textbf{Circle the Most Likely} & After brainstorming, identify 2-3 most probable causes to investigate\\
\hline
\textbf{Verify with Data} & Don't assume - test hypotheses with data before implementing fixes\\
\hline
\end{tabu}
\end{table}

\subsection{Video: How to Create a Fishbone Diagram}\label{video-how-to-create-a-fishbone-diagram}

\begin{center}\rule{0.5\linewidth}{0.5pt}\end{center}

\section{Fault Tree Analysis (FTA)}\label{fault-tree-analysis-fta}

\textbf{Fault Tree Analysis} is a top-down, deductive analysis method that uses Boolean logic to analyze undesired events. Originally developed for the aerospace industry, FTA is particularly useful for complex systems with multiple potential failure paths.

\subsection{FTA Symbols}\label{fta-symbols}

\begin{table}
\centering
\caption{\label{tab:fta-symbols}Fault Tree Analysis Symbols}
\centering
\begin{tabu} to \linewidth {>{\raggedright}X>{\raggedright}X>{\raggedright}X>{\raggedright}X}
\hline
Symbol & Name & Meaning & Example\\
\hline
\textbf{Rectangle} & Intermediate Event & A fault event that results from combination of other events & Pump fails to deliver flow\\
\hline
\textbf{Circle} & Basic Event & A basic failure that cannot be broken down further & Motor winding burnout\\
\hline
\textbf{Diamond} & Undeveloped Event & Event not analyzed further (outside scope or insufficient data) & Power grid failure\\
\hline
\textbf{House} & Normal Event & An event expected to occur during normal operation & Normal wear over time\\
\hline
\textbf{AND Gate} & AND Gate & Output occurs only if ALL inputs occur & Both sensor AND controller fail\\
\hline
\textbf{OR Gate} & OR Gate & Output occurs if ANY input occurs & Either fuse blows OR breaker trips\\
\hline
\end{tabu}
\end{table}

\subsection{FTA Example: Pump Fails to Deliver Flow}\label{fta-example-pump-fails-to-deliver-flow}

\pandocbounded{\includegraphics[keepaspectratio]{introduction_files/figure-latex/fta-example-1.pdf}}

\subsection{Calculating Probability with FTA}\label{calculating-probability-with-fta}

FTA allows quantitative risk analysis by calculating the probability of the top event:

\textbf{OR Gate:} P(output) = 1 - {[}(1 - P\textsubscript{A}) × (1 - P\textsubscript{B}) × \ldots{]}
\emph{Simplified:} P(output) ≈ P\textsubscript{A} + P\textsubscript{B} (when probabilities are small)

\textbf{AND Gate:} P(output) = P\textsubscript{A} × P\textsubscript{B} × \ldots{}

\begin{Shaded}
\begin{Highlighting}[]
\CommentTok{\# Example: Calculate top event probability}
\CommentTok{\# Basic event probabilities (annual failure rates)}
\NormalTok{P\_breaker\_trip }\OtherTok{\textless{}{-}} \FloatTok{0.02}
\NormalTok{P\_motor\_fail }\OtherTok{\textless{}{-}} \FloatTok{0.01}
\NormalTok{P\_impeller\_worn }\OtherTok{\textless{}{-}} \FloatTok{0.03}
\NormalTok{P\_impeller\_break }\OtherTok{\textless{}{-}} \FloatTok{0.005}
\NormalTok{P\_shaft\_seize }\OtherTok{\textless{}{-}} \FloatTok{0.008}
\NormalTok{P\_tank\_empty }\OtherTok{\textless{}{-}} \FloatTok{0.05}
\NormalTok{P\_valve\_closed }\OtherTok{\textless{}{-}} \FloatTok{0.01}
\NormalTok{P\_line\_blocked }\OtherTok{\textless{}{-}} \FloatTok{0.02}

\CommentTok{\# Impeller failure (OR gate)}
\NormalTok{P\_impeller\_fail }\OtherTok{\textless{}{-}} \DecValTok{1} \SpecialCharTok{{-}}\NormalTok{ (}\DecValTok{1} \SpecialCharTok{{-}}\NormalTok{ P\_impeller\_worn) }\SpecialCharTok{*}\NormalTok{ (}\DecValTok{1} \SpecialCharTok{{-}}\NormalTok{ P\_impeller\_break)}
\FunctionTok{cat}\NormalTok{(}\StringTok{"P(Impeller failure):"}\NormalTok{, }\FunctionTok{round}\NormalTok{(P\_impeller\_fail, }\DecValTok{4}\NormalTok{), }\StringTok{"}\SpecialCharTok{\textbackslash{}n}\StringTok{"}\NormalTok{)}
\end{Highlighting}
\end{Shaded}

\begin{verbatim}
## P(Impeller failure): 0.0349
\end{verbatim}

\begin{Shaded}
\begin{Highlighting}[]
\CommentTok{\# Mechanical failure (OR gate)}
\NormalTok{P\_mechanical }\OtherTok{\textless{}{-}} \DecValTok{1} \SpecialCharTok{{-}}\NormalTok{ (}\DecValTok{1} \SpecialCharTok{{-}}\NormalTok{ P\_motor\_fail) }\SpecialCharTok{*}\NormalTok{ (}\DecValTok{1} \SpecialCharTok{{-}}\NormalTok{ P\_impeller\_fail) }\SpecialCharTok{*}\NormalTok{ (}\DecValTok{1} \SpecialCharTok{{-}}\NormalTok{ P\_shaft\_seize)}
\FunctionTok{cat}\NormalTok{(}\StringTok{"P(Mechanical failure):"}\NormalTok{, }\FunctionTok{round}\NormalTok{(P\_mechanical, }\DecValTok{4}\NormalTok{), }\StringTok{"}\SpecialCharTok{\textbackslash{}n}\StringTok{"}\NormalTok{)}
\end{Highlighting}
\end{Shaded}

\begin{verbatim}
## P(Mechanical failure): 0.0521
\end{verbatim}

\begin{Shaded}
\begin{Highlighting}[]
\CommentTok{\# No fluid available (OR gate)}
\NormalTok{P\_no\_fluid }\OtherTok{\textless{}{-}} \DecValTok{1} \SpecialCharTok{{-}}\NormalTok{ (}\DecValTok{1} \SpecialCharTok{{-}}\NormalTok{ P\_tank\_empty) }\SpecialCharTok{*}\NormalTok{ (}\DecValTok{1} \SpecialCharTok{{-}}\NormalTok{ P\_valve\_closed) }\SpecialCharTok{*}\NormalTok{ (}\DecValTok{1} \SpecialCharTok{{-}}\NormalTok{ P\_line\_blocked)}
\FunctionTok{cat}\NormalTok{(}\StringTok{"P(No fluid available):"}\NormalTok{, }\FunctionTok{round}\NormalTok{(P\_no\_fluid, }\DecValTok{4}\NormalTok{), }\StringTok{"}\SpecialCharTok{\textbackslash{}n}\StringTok{"}\NormalTok{)}
\end{Highlighting}
\end{Shaded}

\begin{verbatim}
## P(No fluid available): 0.0783
\end{verbatim}

\begin{Shaded}
\begin{Highlighting}[]
\CommentTok{\# Top event: Pump fails (OR gate)}
\NormalTok{P\_top }\OtherTok{\textless{}{-}} \DecValTok{1} \SpecialCharTok{{-}}\NormalTok{ (}\DecValTok{1} \SpecialCharTok{{-}}\NormalTok{ P\_breaker\_trip) }\SpecialCharTok{*}\NormalTok{ (}\DecValTok{1} \SpecialCharTok{{-}}\NormalTok{ P\_mechanical) }\SpecialCharTok{*}\NormalTok{ (}\DecValTok{1} \SpecialCharTok{{-}}\NormalTok{ P\_no\_fluid)}
\FunctionTok{cat}\NormalTok{(}\StringTok{"P(Pump fails to deliver flow):"}\NormalTok{, }\FunctionTok{round}\NormalTok{(P\_top, }\DecValTok{4}\NormalTok{), }\StringTok{"}\SpecialCharTok{\textbackslash{}n}\StringTok{"}\NormalTok{)}
\end{Highlighting}
\end{Shaded}

\begin{verbatim}
## P(Pump fails to deliver flow): 0.1438
\end{verbatim}

\begin{Shaded}
\begin{Highlighting}[]
\FunctionTok{cat}\NormalTok{(}\StringTok{"This represents a"}\NormalTok{, }\FunctionTok{round}\NormalTok{(P\_top }\SpecialCharTok{*} \DecValTok{100}\NormalTok{, }\DecValTok{1}\NormalTok{), }\StringTok{"\% annual probability of failure"}\NormalTok{)}
\end{Highlighting}
\end{Shaded}

\begin{verbatim}
## This represents a 14.4 % annual probability of failure
\end{verbatim}

\begin{center}\rule{0.5\linewidth}{0.5pt}\end{center}

\section{Pareto Analysis}\label{pareto-analysis}

\textbf{Pareto Analysis} is based on the Pareto Principle (80/20 rule): roughly 80\% of effects come from 20\% of causes. This helps prioritize problem-solving efforts.

\subsection{Creating a Pareto Chart}\label{creating-a-pareto-chart}

\pandocbounded{\includegraphics[keepaspectratio]{introduction_files/figure-latex/pareto-example-1.pdf}}

\subsection{Interpreting Pareto Analysis}\label{interpreting-pareto-analysis}

\begin{Shaded}
\begin{Highlighting}[]
\CommentTok{\# Pareto analysis interpretation}
\NormalTok{defects }\OtherTok{\textless{}{-}} \FunctionTok{data.frame}\NormalTok{(}
  \AttributeTok{Defect =} \FunctionTok{c}\NormalTok{(}\StringTok{"Surface Scratch"}\NormalTok{, }\StringTok{"Dimensional Error"}\NormalTok{, }\StringTok{"Burrs"}\NormalTok{,}
             \StringTok{"Missing Feature"}\NormalTok{, }\StringTok{"Wrong Material"}\NormalTok{, }\StringTok{"Contamination"}\NormalTok{,}
             \StringTok{"Color Mismatch"}\NormalTok{, }\StringTok{"Packaging Damage"}\NormalTok{),}
  \AttributeTok{Count =} \FunctionTok{c}\NormalTok{(}\DecValTok{145}\NormalTok{, }\DecValTok{98}\NormalTok{, }\DecValTok{67}\NormalTok{, }\DecValTok{34}\NormalTok{, }\DecValTok{28}\NormalTok{, }\DecValTok{21}\NormalTok{, }\DecValTok{15}\NormalTok{, }\DecValTok{12}\NormalTok{)}
\NormalTok{)}

\NormalTok{defects }\OtherTok{\textless{}{-}}\NormalTok{ defects[}\FunctionTok{order}\NormalTok{(}\SpecialCharTok{{-}}\NormalTok{defects}\SpecialCharTok{$}\NormalTok{Count),]}
\NormalTok{defects}\SpecialCharTok{$}\NormalTok{Cumulative\_Pct }\OtherTok{\textless{}{-}} \FunctionTok{cumsum}\NormalTok{(defects}\SpecialCharTok{$}\NormalTok{Count) }\SpecialCharTok{/} \FunctionTok{sum}\NormalTok{(defects}\SpecialCharTok{$}\NormalTok{Count) }\SpecialCharTok{*} \DecValTok{100}

\CommentTok{\# Identify the "vital few" (those that contribute to 80\%)}
\NormalTok{vital\_few }\OtherTok{\textless{}{-}}\NormalTok{ defects[defects}\SpecialCharTok{$}\NormalTok{Cumulative\_Pct }\SpecialCharTok{\textless{}=} \DecValTok{80} \SpecialCharTok{|}
                      \FunctionTok{c}\NormalTok{(}\ConstantTok{TRUE}\NormalTok{, defects}\SpecialCharTok{$}\NormalTok{Cumulative\_Pct[}\SpecialCharTok{{-}}\FunctionTok{nrow}\NormalTok{(defects)] }\SpecialCharTok{\textless{}} \DecValTok{80}\NormalTok{),]}

\FunctionTok{cat}\NormalTok{(}\StringTok{"Total defects:"}\NormalTok{, }\FunctionTok{sum}\NormalTok{(defects}\SpecialCharTok{$}\NormalTok{Count), }\StringTok{"}\SpecialCharTok{\textbackslash{}n}\StringTok{"}\NormalTok{)}
\end{Highlighting}
\end{Shaded}

\begin{verbatim}
## Total defects: 420
\end{verbatim}

\begin{Shaded}
\begin{Highlighting}[]
\FunctionTok{cat}\NormalTok{(}\StringTok{"}\SpecialCharTok{\textbackslash{}n}\StringTok{Vital Few (contributing to \textasciitilde{}80\% of defects):}\SpecialCharTok{\textbackslash{}n}\StringTok{"}\NormalTok{)}
\end{Highlighting}
\end{Shaded}

\begin{verbatim}
## 
## Vital Few (contributing to ~80% of defects):
\end{verbatim}

\begin{Shaded}
\begin{Highlighting}[]
\FunctionTok{print}\NormalTok{(vital\_few[, }\FunctionTok{c}\NormalTok{(}\StringTok{"Defect"}\NormalTok{, }\StringTok{"Count"}\NormalTok{, }\StringTok{"Cumulative\_Pct"}\NormalTok{)])}
\end{Highlighting}
\end{Shaded}

\begin{verbatim}
##              Defect Count Cumulative_Pct
## 1   Surface Scratch   145       34.52381
## 2 Dimensional Error    98       57.85714
## 3             Burrs    67       73.80952
## 4   Missing Feature    34       81.90476
\end{verbatim}

\begin{Shaded}
\begin{Highlighting}[]
\FunctionTok{cat}\NormalTok{(}\StringTok{"}\SpecialCharTok{\textbackslash{}n}\StringTok{Focusing on these"}\NormalTok{, }\FunctionTok{nrow}\NormalTok{(vital\_few), }\StringTok{"defect types addresses"}\NormalTok{,}
    \FunctionTok{round}\NormalTok{(}\FunctionTok{max}\NormalTok{(vital\_few}\SpecialCharTok{$}\NormalTok{Cumulative\_Pct), }\DecValTok{1}\NormalTok{), }\StringTok{"\% of all defects"}\NormalTok{)}
\end{Highlighting}
\end{Shaded}

\begin{verbatim}
## 
## Focusing on these 4 defect types addresses 81.9 % of all defects
\end{verbatim}

\subsection{Stratified Pareto Analysis}\label{stratified-pareto-analysis}

Sometimes you need to break down categories further:

\pandocbounded{\includegraphics[keepaspectratio]{introduction_files/figure-latex/stratified-pareto-1.pdf}}

\begin{center}\rule{0.5\linewidth}{0.5pt}\end{center}

\section{Is/Is Not Analysis}\label{isis-not-analysis}

\textbf{Is/Is Not Analysis} is a structured comparison technique that helps narrow down the scope and identify distinguishing factors of a problem.

\subsection{Is/Is Not Matrix}\label{isis-not-matrix}

\begin{table}
\centering
\caption{\label{tab:is-isnot-example}Is/Is Not Analysis: Motor Overheating Problem}
\centering
\begin{tabu} to \linewidth {>{\raggedright}X>{\raggedright}X>{\raggedright}X>{\raggedright}X>{\raggedright}X}
\hline
Dimension & Question & IS & IS NOT & What's Different?\\
\hline
\textbf{WHAT} & What object has the problem? & \cellcolor[HTML]{d4edda}{Motor on Line 3} & \cellcolor[HTML]{f8d7da}{Motors on Lines 1, 2, 4} & \cellcolor[HTML]{fff3cd}{Line 3 motor is older, different model}\\
\hline
\textbf{WHAT} & What is the defect/symptom? & \cellcolor[HTML]{d4edda}{Overheating (85°C)} & \cellcolor[HTML]{f8d7da}{Noise, vibration, or failure} & \cellcolor[HTML]{fff3cd}{Thermal issue, not mechanical yet}\\
\hline
\textbf{WHERE} & Where is the problem observed? & \cellcolor[HTML]{d4edda}{Packaging area, Station 4} & \cellcolor[HTML]{f8d7da}{Any other station} & \cellcolor[HTML]{fff3cd}{Station 4 has higher load demand}\\
\hline
\textbf{WHERE} & Where on the object is the defect? & \cellcolor[HTML]{d4edda}{Motor housing, bearing end} & \cellcolor[HTML]{f8d7da}{Drive end, junction box} & \cellcolor[HTML]{fff3cd}{Bearing end runs hotter normally}\\
\hline
\textbf{WHEN} & When was it first observed? & \cellcolor[HTML]{d4edda}{Monday morning shift} & \cellcolor[HTML]{f8d7da}{Before weekend (Friday OK)} & \cellcolor[HTML]{fff3cd}{Something changed over weekend}\\
\hline
\textbf{WHEN} & When in the process/cycle? & \cellcolor[HTML]{d4edda}{After 2 hours of operation} & \cellcolor[HTML]{f8d7da}{Cold start or end of shift} & \cellcolor[HTML]{fff3cd}{Heat buildup issue}\\
\hline
\textbf{EXTENT} & How many units affected? & \cellcolor[HTML]{d4edda}{100\% - every shift} & \cellcolor[HTML]{f8d7da}{Intermittent} & \cellcolor[HTML]{fff3cd}{Consistent problem}\\
\hline
\textbf{EXTENT} & Is it trending up/down/stable? & \cellcolor[HTML]{d4edda}{Stable - same every day} & \cellcolor[HTML]{f8d7da}{Getting worse over time} & \cellcolor[HTML]{fff3cd}{Stable - not progressive failure}\\
\hline
\end{tabu}
\end{table}

\subsection{Key Insights from Is/Is Not Analysis}\label{key-insights-from-isis-not-analysis}

From the example above, we can identify:

\begin{enumerate}
\def\labelenumi{\arabic{enumi}.}
\tightlist
\item
  \textbf{Line 3 specific} - Not a general problem across all lines
\item
  \textbf{Thermal, not mechanical} - Overheating but no noise/vibration yet
\item
  \textbf{After warm-up} - Heat accumulation issue
\item
  \textbf{Started after weekend} - Something changed
\end{enumerate}

\textbf{Investigation focus:} What changed over the weekend? Check maintenance logs, any work done on Line 3, any environmental changes.

\begin{center}\rule{0.5\linewidth}{0.5pt}\end{center}

\section{The 8D Problem-Solving Methodology}\label{the-8d-problem-solving-methodology}

The \textbf{8D (Eight Disciplines)} methodology is a comprehensive problem-solving approach developed by Ford Motor Company. It's widely used in automotive and other industries.

\pandocbounded{\includegraphics[keepaspectratio]{introduction_files/figure-latex/8d-overview-1.pdf}}

\subsection{Detailed 8D Steps}\label{detailed-8d-steps}

\begin{table}
\centering
\caption{\label{tab:8d-details}8D Methodology Detailed Breakdown}
\centering
\begin{tabu} to \linewidth {>{\raggedright}X>{\raggedright}X>{\raggedright}X}
\hline
Discipline & Key Activities & Deliverables\\
\hline
\textbf{D0: Plan} & Emergency response actions, initial assessment, determine if 8D needed & 8D initiation decision, emergency actions\\
\hline
\textbf{D1: Team} & Select team leader, identify required expertise, define roles & Team charter, member list with expertise\\
\hline
\textbf{D2: Problem} & 5W2H, Is/Is Not, quantify impact, identify affected parts/lots & Problem statement, scope definition\\
\hline
\textbf{D3: Containment} & Sort suspect material, increase inspection, alert downstream & Containment actions, protected customer\\
\hline
\textbf{D4: Root Cause} & 5 Whys, Fishbone, FTA, verify cause by testing & Verified root cause(s)\\
\hline
\textbf{D5: Corrective Actions} & Brainstorm solutions, verify effectiveness before full rollout & Verified corrective action plan\\
\hline
\textbf{D6: Implementation} & Execute action plan, validate results, monitor KPIs & Implemented actions, validated results\\
\hline
\textbf{D7: Prevention} & Update FMEAs, procedures, training, share across organization & Updated procedures, similar problem prevention\\
\hline
\textbf{D8: Closure} & Final documentation, lessons learned, team recognition & 8D report, closed status\\
\hline
\end{tabu}
\end{table}

\subsection{D3: Containment Actions}\label{d3-containment-actions}

Containment is crucial to protect the customer while you investigate:

\begin{table}
\centering
\caption{\label{tab:containment-types}Types of Containment Actions}
\centering
\begin{tabu} to \linewidth {>{\raggedright}X>{\raggedright}X>{\raggedright}X}
\hline
Type & Description & Example\\
\hline
\textbf{Sort \& Inspect} & 100\% inspection of suspect inventory to separate good from bad & Inspect all parts from Lot 2024-0215\\
\hline
\textbf{Rework/Repair} & Repair nonconforming units if possible and economical & Re-machine undersize diameters to specification\\
\hline
\textbf{Scrap} & Dispose of units that cannot be repaired & Dispose of parts with cracks\\
\hline
\textbf{Hold/Quarantine} & Prevent movement of suspect material until disposition & Tag and segregate all parts from affected shift\\
\hline
\textbf{Increase Monitoring} & Add inspections, increase sampling, tighten control limits & Add 100\% visual inspection at Pack station\\
\hline
\textbf{Supplier Notification} & Notify suppliers if incoming material is suspected & Request material certification from steel supplier\\
\hline
\textbf{Customer Alert} & Alert customers to check their inventory if product shipped & Contact distributor about potentially affected batch\\
\hline
\end{tabu}
\end{table}

\begin{center}\rule{0.5\linewidth}{0.5pt}\end{center}

\section{Corrective Action Implementation}\label{corrective-action-implementation}

Effective corrective actions address the root cause and prevent recurrence.

\subsection{Types of Corrective Actions}\label{types-of-corrective-actions}

\pandocbounded{\includegraphics[keepaspectratio]{introduction_files/figure-latex/corrective-action-hierarchy-1.pdf}}

\subsection{SMART Corrective Actions}\label{smart-corrective-actions}

Corrective actions should be \textbf{SMART}:

\begin{table}
\centering
\caption{\label{tab:smart-actions}SMART Corrective Action Criteria}
\centering
\begin{tabu} to \linewidth {>{\raggedright}X>{\raggedright}X>{\raggedright}X>{\raggedright}X>{\raggedright}X}
\hline
 & Meaning & Description & Poor Example & Good Example\\
\hline
\cellcolor{steelblue}{\textcolor{white}{\textbf{S}}} & Specific & Clearly defined action, not vague & \cellcolor[HTML]{f8d7da}{Improve quality} & \cellcolor[HTML]{d4edda}{Install poka-yoke sensor on Station 3 to detect missing holes}\\
\hline
\cellcolor{steelblue}{\textcolor{white}{\textbf{M}}} & Measurable & Can verify completion and effectiveness & \cellcolor[HTML]{f8d7da}{Train operators} & \cellcolor[HTML]{d4edda}{Train all operators on new SOP-234 with signed competency verification}\\
\hline
\cellcolor{steelblue}{\textcolor{white}{\textbf{A}}} & Achievable & Realistic with available resources & \cellcolor[HTML]{f8d7da}{Redesign entire product line} & \cellcolor[HTML]{d4edda}{Add inspection step (1 min/part) with existing CMM capacity}\\
\hline
\cellcolor{steelblue}{\textcolor{white}{\textbf{R}}} & Relevant & Directly addresses the root cause & \cellcolor[HTML]{f8d7da}{Repaint the floor} & \cellcolor[HTML]{d4edda}{Add check fixture to verify hole position before assembly}\\
\hline
\cellcolor{steelblue}{\textcolor{white}{\textbf{T}}} & Time-bound & Has a target completion date & \cellcolor[HTML]{f8d7da}{Do this sometime soon} & \cellcolor[HTML]{d4edda}{Complete by March 15, 2024}\\
\hline
\end{tabu}
\end{table}

\begin{center}\rule{0.5\linewidth}{0.5pt}\end{center}

\section{Verification and Validation}\label{verification-and-validation}

Before closing out a problem, verify that your actions worked.

\subsection{Verification vs.~Validation}\label{verification-vs.-validation}

\begin{table}
\centering
\caption{\label{tab:vv-comparison}Verification vs. Validation of Corrective Actions}
\centering
\begin{tabu} to \linewidth {>{\raggedright}X>{\raggedright}X>{\raggedright}X}
\hline
Aspect & Verification & Validation\\
\hline
\textbf{Definition} & \cellcolor[HTML]{e3f2fd}{Did we implement the action correctly?} & \cellcolor[HTML]{e8f5e9}{Did the action solve the problem?}\\
\hline
\textbf{Focus} & \cellcolor[HTML]{e3f2fd}{Process compliance} & \cellcolor[HTML]{e8f5e9}{Results and effectiveness}\\
\hline
\textbf{Question} & \cellcolor[HTML]{e3f2fd}{Did we build the fix right?} & \cellcolor[HTML]{e8f5e9}{Did we build the right fix?}\\
\hline
\textbf{Timing} & \cellcolor[HTML]{e3f2fd}{During/after implementation} & \cellcolor[HTML]{e8f5e9}{After implementation, during monitoring}\\
\hline
\textbf{Methods} & \cellcolor[HTML]{e3f2fd}{Audits, inspections, checklists} & \cellcolor[HTML]{e8f5e9}{Data analysis, trend charts, capability studies}\\
\hline
\end{tabu}
\end{table}

\subsection{Effectiveness Monitoring}\label{effectiveness-monitoring}

\begin{verbatim}
## Warning in scale_x_date(): A <numeric> value was passed to a Date scale.
## i The value was converted to a <Date> object.
\end{verbatim}

\pandocbounded{\includegraphics[keepaspectratio]{introduction_files/figure-latex/effectiveness-tracking-1.pdf}}

\begin{Shaded}
\begin{Highlighting}[]
\CommentTok{\# Calculate improvement}
\NormalTok{before\_avg }\OtherTok{\textless{}{-}} \FunctionTok{mean}\NormalTok{(defect\_rate[}\DecValTok{1}\SpecialCharTok{:}\DecValTok{14}\NormalTok{])}
\NormalTok{after\_avg }\OtherTok{\textless{}{-}} \FunctionTok{mean}\NormalTok{(defect\_rate[}\DecValTok{15}\SpecialCharTok{:}\DecValTok{30}\NormalTok{])}
\NormalTok{improvement }\OtherTok{\textless{}{-}}\NormalTok{ (before\_avg }\SpecialCharTok{{-}}\NormalTok{ after\_avg) }\SpecialCharTok{/}\NormalTok{ before\_avg }\SpecialCharTok{*} \DecValTok{100}

\FunctionTok{cat}\NormalTok{(}\StringTok{"Before corrective action: "}\NormalTok{, }\FunctionTok{round}\NormalTok{(before\_avg, }\DecValTok{2}\NormalTok{), }\StringTok{"\% defect rate}\SpecialCharTok{\textbackslash{}n}\StringTok{"}\NormalTok{)}
\end{Highlighting}
\end{Shaded}

\begin{verbatim}
## Before corrective action:  5.03 % defect rate
\end{verbatim}

\begin{Shaded}
\begin{Highlighting}[]
\FunctionTok{cat}\NormalTok{(}\StringTok{"After corrective action:  "}\NormalTok{, }\FunctionTok{round}\NormalTok{(after\_avg, }\DecValTok{2}\NormalTok{), }\StringTok{"\% defect rate}\SpecialCharTok{\textbackslash{}n}\StringTok{"}\NormalTok{)}
\end{Highlighting}
\end{Shaded}

\begin{verbatim}
## After corrective action:   1.33 % defect rate
\end{verbatim}

\begin{Shaded}
\begin{Highlighting}[]
\FunctionTok{cat}\NormalTok{(}\StringTok{"Improvement:              "}\NormalTok{, }\FunctionTok{round}\NormalTok{(improvement, }\DecValTok{1}\NormalTok{), }\StringTok{"\% reduction}\SpecialCharTok{\textbackslash{}n}\StringTok{"}\NormalTok{)}
\end{Highlighting}
\end{Shaded}

\begin{verbatim}
## Improvement:               73.5 % reduction
\end{verbatim}

\begin{center}\rule{0.5\linewidth}{0.5pt}\end{center}

\section{Case Study: Automotive Welding Defects}\label{case-study-automotive-welding-defects}

Let's work through a comprehensive troubleshooting case using multiple RCA tools.

\subsection{The Problem}\label{the-problem}

A Tier-1 automotive supplier is experiencing weld porosity defects on a critical structural component. Customer complaints have increased, and the plant has received a quality alert.

\subsection{Initial Problem Definition (5W2H)}\label{initial-problem-definition-5w2h}

\begin{table}
\centering
\caption{\label{tab:case-5w2h}Case Study Problem Definition}
\centering
\begin{tabu} to \linewidth {>{\raggedright}X>{\raggedright}X}
\hline
Question & Details\\
\hline
\textbf{WHAT} & Porosity (gas pockets) in MIG welds on crossmember subassembly\\
\hline
\textbf{WHERE} & Robot welding Cell 4, occurring at weld joint W-23 (corner weld)\\
\hline
\textbf{WHEN} & Started approximately 2 weeks ago, occurs on all shifts\\
\hline
\textbf{WHO} & Detected by customer during their incoming inspection\\
\hline
\textbf{WHY} & Customer rejecting parts; risk of production line stop\\
\hline
\textbf{HOW} & Visible voids on X-ray inspection; some visible to naked eye\\
\hline
\textbf{HOW MUCH} & Estimated 8\% of parts affected (up from baseline of 0.5\%)\\
\hline
\end{tabu}
\end{table}

\subsection{5 Whys Analysis}\label{whys-analysis}

\begin{verbatim}
## Warning in latex_new_row_builder(target_row, table_info, bold, italic,
## monospace, : Setting full_width = TRUE will turn the table into a tabu
## environment where colors are not really easily configable with this package.
## Please consider turn off full_width.
\end{verbatim}

\begin{table}
\centering
\caption{\label{tab:case-5whys}5 Whys Analysis: Weld Porosity}
\centering
\begin{tabu} to \linewidth {>{\raggedright}X>{\raggedright}X}
\hline
Level & Finding\\
\hline
\textbf{Problem} & Porosity in welds at joint W-23\\
\hline
\textbf{Why \#1} & Gas entrapment during solidification\\
\hline
\textbf{Why \#2} & Inadequate shielding gas coverage\\
\hline
\textbf{Why \#3} & Gas flow rate dropped below specification\\
\hline
\textbf{Why \#4} & Flow meter was not calibrated; showing higher than actual\\
\hline
\cellcolor[HTML]{d4edda}{\textbf{\textbf{Why \#5}}} & \cellcolor[HTML]{d4edda}{\textbf{No calibration schedule existed for welding gas flow meters}}\\
\hline
\end{tabu}
\end{table}

\subsection{Fishbone Diagram Findings}\label{fishbone-diagram-findings}

After brainstorming with the cross-functional team:

\begin{verbatim}
## Warning in latex_new_row_builder(target_row, table_info, bold, italic,
## monospace, : Setting full_width = TRUE will turn the table into a tabu
## environment where colors are not really easily configable with this package.
## Please consider turn off full_width.
## Warning in latex_new_row_builder(target_row, table_info, bold, italic,
## monospace, : Setting full_width = TRUE will turn the table into a tabu
## environment where colors are not really easily configable with this package.
## Please consider turn off full_width.
\end{verbatim}

\begin{table}
\centering
\caption{\label{tab:case-fishbone-findings}Fishbone Analysis Results with Verification}
\centering
\begin{tabu} to \linewidth {>{\raggedright}X>{\raggedright}X>{\raggedright}X>{\raggedright}X}
\hline
Category & Potential Cause & Likelihood & Verification Result\\
\hline
\cellcolor[HTML]{d4edda}{\textbf{Machine}} & \cellcolor[HTML]{d4edda}{Gas flow meter out of calibration} & \cellcolor[HTML]{d4edda}{HIGH} & \cellcolor[HTML]{d4edda}{YES - Root cause}\\
\hline
\textbf{Machine} & Worn contact tip affecting arc stability & Medium & No - tips OK\\
\hline
\textbf{Material} & Different wire lot from new supplier & Medium & No - tested OK\\
\hline
\textbf{Material} & Base metal surface contamination & Low & No - clean parts\\
\hline
\textbf{Method} & Robot path too fast for corner weld & Medium & Partial contributor\\
\hline
\cellcolor[HTML]{d4edda}{\textbf{Method}} & \cellcolor[HTML]{d4edda}{Gas pre-flow time insufficient} & \cellcolor[HTML]{d4edda}{HIGH} & \cellcolor[HTML]{d4edda}{YES - Contributing}\\
\hline
\textbf{Man} & Recent operator turnover, less experience & Low & No - all shifts affected\\
\hline
\textbf{Environment} & Seasonal humidity changes & Low & No - controlled\\
\hline
\end{tabu}
\end{table}

\subsection{8D Report Summary}\label{d-report-summary}

\begin{table}
\centering
\caption{\label{tab:case-8d}8D Report: Weld Porosity Case Study}
\centering
\begin{tabu} to \linewidth {>{\raggedright}X>{\raggedright}X>{\raggedright}X}
\hline
D\# & Discipline & Summary\\
\hline
\textbf{D1} & \textbf{Team} & Team: Weld Engineer (lead), Quality, Maintenance, Production Supervisor, Supplier Quality\\
\hline
\textbf{D2} & \textbf{Problem} & 8\% porosity rate at W-23 joint since Jan 15; customer quality alert received Jan 28\\
\hline
\textbf{D3} & \textbf{Containment} & 100\% X-ray inspection of all crossmembers; sort and quarantine 3 days production\\
\hline
\textbf{D4} & \textbf{Root Cause} & Primary: Flow meter out of calibration (reading 25 CFH, actual 18 CFH). Contributing: Pre-flow time was 0.3s, spec requires 0.5s\\
\hline
\textbf{D5} & \textbf{Corrective Actions} & 1. Recalibrate flow meter immediately; 2. Increase pre-flow time to 0.5s; 3. Add flow meter to calibration schedule\\
\hline
\textbf{D6} & \textbf{Implementation} & Actions completed Feb 5; validation testing shows 0.1\% porosity rate (below 0.5\% target)\\
\hline
\textbf{D7} & \textbf{Prevention} & Updated PM schedule to include all gas flow meters quarterly; added to PFMEA with RPN recalculation\\
\hline
\textbf{D8} & \textbf{Closure} & Customer satisfied with response; 8D closed Feb 15; cost avoidance: \$45,000 (avoided line stop)\\
\hline
\end{tabu}
\end{table}

\begin{center}\rule{0.5\linewidth}{0.5pt}\end{center}

\section{Documentation and Knowledge Management}\label{documentation-and-knowledge-management}

Proper documentation ensures learnings are captured and problems don't recur.

\subsection{Key Documentation Elements}\label{key-documentation-elements}

\begin{table}
\centering
\caption{\label{tab:documentation-checklist}RCA Documentation Checklist}
\centering
\begin{tabu} to \linewidth {>{\raggedright}X>{\raggedright}X>{\raggedright}X}
\hline
Element & Contents & Benefit\\
\hline
\textbf{Problem Description} & 5W2H, problem statement, impact/cost & Clear reference for similar future problems\\
\hline
\textbf{Timeline} & When discovered, key investigation dates, implementation dates & Understanding of response time, identifying delays\\
\hline
\textbf{Data Collected} & Measurements, photos, test results, operator interviews & Evidence-based analysis, reference for future\\
\hline
\textbf{Root Cause Analysis} & 5 Whys, fishbone diagrams, FTA as applicable & Documented logical process, training material\\
\hline
\textbf{Corrective Actions} & Actions taken, responsible parties, target/actual dates & Accountability, closure verification\\
\hline
\textbf{Verification Results} & Before/after data, statistical comparison, effectiveness \% & Proof of effectiveness, audit trail\\
\hline
\textbf{Lessons Learned} & What we'd do differently, recommendations for similar situations & Continuous improvement, organizational learning\\
\hline
\textbf{System Updates} & PFMEA updates, procedure changes, training records & Prevention of recurrence, compliance\\
\hline
\end{tabu}
\end{table}

\subsection{Creating a Troubleshooting Knowledge Base}\label{creating-a-troubleshooting-knowledge-base}

Build organizational memory by:

\begin{enumerate}
\def\labelenumi{\arabic{enumi}.}
\tightlist
\item
  \textbf{Standardized templates} - Use consistent formats for all RCA documents
\item
  \textbf{Searchable database} - Store in CMMS, SharePoint, or dedicated quality system
\item
  \textbf{Categories and tags} - Equipment type, failure mode, root cause category
\item
  \textbf{Cross-references} - Link related issues and solutions
\item
  \textbf{Regular reviews} - Periodic analysis of trends and patterns
\end{enumerate}

\begin{center}\rule{0.5\linewidth}{0.5pt}\end{center}

\section{Troubleshooting Tools Quick Reference}\label{troubleshooting-tools-quick-reference}

\begin{table}
\centering
\caption{\label{tab:tool-selection}RCA Tool Selection Guide}
\centering
\begin{tabu} to \linewidth {>{\raggedright}X>{\raggedright}X>{\raggedright}X>{\raggedright}X}
\hline
Tool & Best For & When to Use & Team Size\\
\hline
\textbf{5 Whys} & Simple problems with linear cause chains & Quick investigations, starting point for analysis & 1-3\\
\hline
\textbf{Fishbone Diagram} & Brainstorming multiple potential causes & Team problem-solving sessions & 4-8\\
\hline
\textbf{Fault Tree Analysis} & Complex systems with multiple failure paths & Safety-critical systems, reliability analysis & 2-5\\
\hline
\textbf{Pareto Chart} & Prioritizing which problems to tackle first & When facing multiple problems or defect types & 1-3\\
\hline
\textbf{Is/Is Not} & Narrowing down scope and identifying distinctions & When problem is hard to define clearly & 2-4\\
\hline
\textbf{8D} & Customer complaints requiring formal response & Formal quality requirements, major issues & 5-10\\
\hline
\textbf{Change Analysis} & Problems that started suddenly & When timing of problem onset is known & 1-4\\
\hline
\end{tabu}
\end{table}

\begin{center}\rule{0.5\linewidth}{0.5pt}\end{center}

\section{Video Resources}\label{video-resources}

\subsection{Understanding 5 Whys}\label{understanding-5-whys}

\subsection{8D Problem Solving}\label{d-problem-solving}

\begin{center}\rule{0.5\linewidth}{0.5pt}\end{center}

\section{Summary}\label{summary-9}

Effective troubleshooting and root cause analysis require:

\begin{enumerate}
\def\labelenumi{\arabic{enumi}.}
\tightlist
\item
  \textbf{Systematic approach} - Don't guess; follow a structured methodology
\item
  \textbf{Clear problem definition} - Use 5W2H to clearly define what you're solving
\item
  \textbf{Data-driven analysis} - Collect facts before jumping to conclusions
\item
  \textbf{Root cause focus} - Don't stop at symptoms; dig to fundamental causes
\item
  \textbf{Multiple tools} - Use 5 Whys, Fishbone, FTA, Pareto as appropriate
\item
  \textbf{Effective corrective actions} - Aim for elimination, not just mitigation
\item
  \textbf{Verification} - Confirm your fix actually works with data
\item
  \textbf{Documentation} - Capture learnings to prevent recurrence
\end{enumerate}

Remember: The goal isn't just to fix today's problem - it's to prevent tomorrow's.

\begin{center}\rule{0.5\linewidth}{0.5pt}\end{center}

\section{Review Questions}\label{review-questions-10}

\textbf{Question 1}: What is the difference between a symptom, immediate cause, and root cause? Provide an example of each for a machine that is producing parts with excessive surface roughness.

\textbf{Answer:}

\begin{itemize}
\tightlist
\item
  \textbf{Symptom}: The observable effect of the problem

  \begin{itemize}
  \tightlist
  \item
    \emph{Example: Parts have Ra surface roughness of 3.2 μm when specification is 1.6 μm}
  \end{itemize}
\item
  \textbf{Immediate cause}: The direct, technical cause of the symptom

  \begin{itemize}
  \tightlist
  \item
    \emph{Example: Cutting tool is worn beyond its effective life}
  \end{itemize}
\item
  \textbf{Root cause}: The fundamental reason that, if eliminated, prevents recurrence

  \begin{itemize}
  \tightlist
  \item
    \emph{Example: No tool life monitoring system exists; operators judge tool changes by intuition rather than data}
  \end{itemize}
\end{itemize}

Fixing the root cause (implementing tool life tracking) prevents the problem from recurring, whereas just replacing the tool (addressing immediate cause) means it will happen again.

\textbf{Question 2}: A food processing plant is experiencing frequent jams on their packaging line. Perform a 5 Whys analysis given the following information: The jam occurs at the carton loading station, started last week, happens 4-5 times per shift, and the maintenance team found the timing of the pusher mechanism is off.

\textbf{Answer:}

\begin{enumerate}
\def\labelenumi{\arabic{enumi}.}
\tightlist
\item
  \textbf{Problem}: Carton loading station jams 4-5 times per shift
\item
  \textbf{Why \#1}: Cartons are not in position when pusher activates → Timing mismatch
\item
  \textbf{Why \#2}: Pusher timing is off by approximately 0.3 seconds
\item
  \textbf{Why \#3}: Timing adjustment was changed during last week's maintenance
\item
  \textbf{Why \#4}: Technician didn't have correct specification for timing setting
\item
  \textbf{Why \#5}: Maintenance procedure doesn't include timing specifications
\end{enumerate}

\textbf{Root Cause}: Maintenance procedures lack critical parameter specifications

\textbf{Corrective Actions}:
- Update maintenance procedure with correct timing specification (180ms ± 10ms)
- Restore timing to correct setting immediately
- Review other procedures for missing specifications
- Add parameter verification checklist to PM activities

\textbf{Question 3}: Create a fishbone diagram outline (list the potential causes under each 6M category) for the problem ``CNC machine producing parts 0.05mm oversize.''

\textbf{Answer:}

\textbf{MAN (People)}
- Operator entered wrong offset
- Incorrect setup procedure followed
- Inadequate training on new program
- Fatigue/distraction during setup

\textbf{MACHINE (Equipment)}
- Spindle thermal growth
- Ballscrew wear/backlash
- Tool holder runout
- Axis positioning error

\textbf{MATERIAL}
- Material hardness variation
- Different lot from supplier
- Material springback
- Incorrect material grade

\textbf{METHOD (Process)}
- Wrong tool offset entered
- Program error in G-code
- Incorrect cutting parameters
- Missing roughing pass

\textbf{MEASUREMENT}
- Gauge out of calibration
- Wrong measurement technique
- Temperature affecting measurement
- Incorrect datum reference

\textbf{MOTHER NATURE (Environment)}
- Shop temperature variation
- Coolant temperature change
- Vibration from nearby equipment
- Humidity affecting gauges

\textbf{Question 4}: Given the following defect data from a sheet metal stamping operation, create a Pareto analysis and identify the vital few defects to focus on.

\begin{longtable}[]{@{}ll@{}}
\toprule\noalign{}
Defect Type & Count \\
\midrule\noalign{}
\endhead
\bottomrule\noalign{}
\endlastfoot
Scratches & 89 \\
Burrs & 156 \\
Wrinkles & 45 \\
Splits/Cracks & 28 \\
Dimensional & 112 \\
Surface Dents & 67 \\
Missing Features & 18 \\
\end{longtable}

\textbf{Answer:}

\begin{Shaded}
\begin{Highlighting}[]
\CommentTok{\# Create and analyze the data}
\NormalTok{defects }\OtherTok{\textless{}{-}} \FunctionTok{data.frame}\NormalTok{(}
  \AttributeTok{Type =} \FunctionTok{c}\NormalTok{(}\StringTok{"Burrs"}\NormalTok{, }\StringTok{"Dimensional"}\NormalTok{, }\StringTok{"Scratches"}\NormalTok{, }\StringTok{"Surface Dents"}\NormalTok{,}
           \StringTok{"Wrinkles"}\NormalTok{, }\StringTok{"Splits/Cracks"}\NormalTok{, }\StringTok{"Missing Features"}\NormalTok{),}
  \AttributeTok{Count =} \FunctionTok{c}\NormalTok{(}\DecValTok{156}\NormalTok{, }\DecValTok{112}\NormalTok{, }\DecValTok{89}\NormalTok{, }\DecValTok{67}\NormalTok{, }\DecValTok{45}\NormalTok{, }\DecValTok{28}\NormalTok{, }\DecValTok{18}\NormalTok{)}
\NormalTok{)}

\CommentTok{\# Sort by count descending}
\NormalTok{defects }\OtherTok{\textless{}{-}}\NormalTok{ defects[}\FunctionTok{order}\NormalTok{(}\SpecialCharTok{{-}}\NormalTok{defects}\SpecialCharTok{$}\NormalTok{Count),]}

\CommentTok{\# Calculate percentages}
\NormalTok{defects}\SpecialCharTok{$}\NormalTok{Percent }\OtherTok{\textless{}{-}}\NormalTok{ defects}\SpecialCharTok{$}\NormalTok{Count }\SpecialCharTok{/} \FunctionTok{sum}\NormalTok{(defects}\SpecialCharTok{$}\NormalTok{Count) }\SpecialCharTok{*} \DecValTok{100}
\NormalTok{defects}\SpecialCharTok{$}\NormalTok{Cumulative }\OtherTok{\textless{}{-}} \FunctionTok{cumsum}\NormalTok{(defects}\SpecialCharTok{$}\NormalTok{Percent)}

\CommentTok{\# Display results}
\FunctionTok{print}\NormalTok{(defects)}
\end{Highlighting}
\end{Shaded}

\begin{verbatim}
##               Type Count   Percent Cumulative
## 1            Burrs   156 30.291262   30.29126
## 2      Dimensional   112 21.747573   52.03883
## 3        Scratches    89 17.281553   69.32039
## 4    Surface Dents    67 13.009709   82.33010
## 5         Wrinkles    45  8.737864   91.06796
## 6    Splits/Cracks    28  5.436893   96.50485
## 7 Missing Features    18  3.495146  100.00000
\end{verbatim}

\begin{Shaded}
\begin{Highlighting}[]
\CommentTok{\# Identify vital few}
\NormalTok{vital\_few }\OtherTok{\textless{}{-}}\NormalTok{ defects}\SpecialCharTok{$}\NormalTok{Type[defects}\SpecialCharTok{$}\NormalTok{Cumulative }\SpecialCharTok{\textless{}=} \DecValTok{80} \SpecialCharTok{|}
                           \FunctionTok{c}\NormalTok{(}\ConstantTok{TRUE}\NormalTok{, defects}\SpecialCharTok{$}\NormalTok{Cumulative[}\SpecialCharTok{{-}}\FunctionTok{nrow}\NormalTok{(defects)] }\SpecialCharTok{\textless{}} \DecValTok{80}\NormalTok{)]}
\FunctionTok{cat}\NormalTok{(}\StringTok{"}\SpecialCharTok{\textbackslash{}n}\StringTok{Vital Few (contributing to \textasciitilde{}80\%):"}\NormalTok{, }\FunctionTok{paste}\NormalTok{(vital\_few, }\AttributeTok{collapse =} \StringTok{", "}\NormalTok{))}
\end{Highlighting}
\end{Shaded}

\begin{verbatim}
## 
## Vital Few (contributing to ~80%): Burrs, Dimensional, Scratches, Surface Dents
\end{verbatim}

The \textbf{vital few} are Burrs, Dimensional errors, and Scratches - addressing these three defect types would address approximately 69\% of all defects. Adding Surface Dents would bring the total to 82\%.

\textbf{Question 5}: What are the key differences between verification and validation in the context of corrective actions?

\textbf{Answer:}

\begin{longtable}[]{@{}
  >{\raggedright\arraybackslash}p{(\linewidth - 4\tabcolsep) * \real{0.2353}}
  >{\raggedright\arraybackslash}p{(\linewidth - 4\tabcolsep) * \real{0.4118}}
  >{\raggedright\arraybackslash}p{(\linewidth - 4\tabcolsep) * \real{0.3529}}@{}}
\toprule\noalign{}
\begin{minipage}[b]{\linewidth}\raggedright
Aspect
\end{minipage} & \begin{minipage}[b]{\linewidth}\raggedright
Verification
\end{minipage} & \begin{minipage}[b]{\linewidth}\raggedright
Validation
\end{minipage} \\
\midrule\noalign{}
\endhead
\bottomrule\noalign{}
\endlastfoot
Question asked & Did we implement the action correctly? & Did the action solve the problem? \\
Focus & Process compliance & Results/outcomes \\
Timing & During/immediately after implementation & Ongoing monitoring after implementation \\
Methods & Audits, checklists, inspections & Data analysis, trend charts, SPC \\
Example & Confirm new sensor was installed per specification & Measure defect rate over 30 days to confirm reduction \\
\end{longtable}

\textbf{Verification} confirms the action was done right.
\textbf{Validation} confirms we did the right action.

Both are required - an action can be implemented perfectly (verified) but still not solve the problem (not validated).

\textbf{Question 6}: An aerospace manufacturer has a requirement to use the 8D methodology for customer complaints. What happens in D3 (Containment) and why is it critical?

\textbf{Answer:}

\textbf{D3 Containment Actions} protect the customer while the root cause investigation continues. Key activities include:

\begin{enumerate}
\def\labelenumi{\arabic{enumi}.}
\tightlist
\item
  \textbf{Immediate actions}:

  \begin{itemize}
  \tightlist
  \item
    Quarantine suspect inventory (tagged, segregated)
  \item
    100\% inspection of suspect lots
  \item
    Increase monitoring at subsequent operations
  \item
    Sort parts at customer location if already shipped
  \end{itemize}
\item
  \textbf{Documentation}:

  \begin{itemize}
  \tightlist
  \item
    Identify affected lot numbers, date codes, serial numbers
  \item
    Quantify the scope of potential exposure
  \item
    Document all containment actions taken
  \end{itemize}
\item
  \textbf{Communication}:

  \begin{itemize}
  \tightlist
  \item
    Notify customer of actions being taken
  \item
    Alert downstream operations
  \item
    Inform suppliers if incoming material involved
  \end{itemize}
\end{enumerate}

\textbf{Why critical}:
- Prevents additional defective product from reaching the customer
- Limits the scope of the problem (cost, liability)
- Demonstrates responsiveness to customer
- Buys time for proper root cause investigation
- In aerospace, safety-of-flight concerns make containment essential

Without effective containment, the problem continues while you investigate, increasing customer impact and costs exponentially.

\textbf{Question 7}: Why might a team's 5 Whys analysis stop too early? What are signs that you haven't reached the true root cause?

\textbf{Answer:}

\textbf{Reasons teams stop too early}:
1. \textbf{Blame arrives} - Analysis ends at ``operator error'' without asking why the error occurred
2. \textbf{Comfort zone} - Team stops when they reach a cause they can easily fix
3. \textbf{Time pressure} - Rush to implement a fix, skipping deeper analysis
4. \textbf{Technical limit} - Team lacks expertise to go deeper into certain areas
5. \textbf{Defensiveness} - Reaching causes that implicate management or systems creates resistance

\textbf{Signs you haven't reached root cause}:
1. \textbf{Recurrence} - If the fix is ``more training'' or ``be more careful,'' problem will return
2. \textbf{Human-based answer} - Root causes should typically be systems/processes, not individuals
3. \textbf{No prevention mechanism} - If fixing this cause doesn't prevent future occurrences, keep asking why
4. \textbf{Can still ask ``why?''} - If there's a logical next level to investigate
5. \textbf{Administrative fix only} - If the solution is just a procedure or checklist with no engineering controls

\textbf{Example of stopping too early}:
- Stopping at: ``Operator set wrong parameter''
- Should continue: Why? → No parameter sheet at workstation → Why? → No standard work documentation exists → Root cause: Missing standardized work process

The true root cause typically relates to systems, processes, or management decisions rather than individual actions.

\textbf{Question 8}: Describe how you would use Pareto analysis in a stratified manner to drill down into a quality problem.

\textbf{Answer:}

\textbf{Stratified Pareto Analysis} involves creating multiple levels of Pareto charts, progressively drilling down into the vital few categories.

\textbf{Example: High scrap rate at a machining shop}

\textbf{Level 1: Scrap by defect type}
1. Create Pareto of all defect types
2. Result: ``Dimensional errors'' is 45\% of all scrap - investigate this first

\textbf{Level 2: Dimensional errors by machine}
1. Create Pareto of dimensional errors broken down by machine
2. Result: Machine \#4 accounts for 60\% of dimensional errors

\textbf{Level 3: Machine \#4 dimensional errors by feature}
1. Create Pareto of which dimensions are failing on Machine \#4
2. Result: Bore diameter is 70\% of Machine \#4 dimensional issues

\textbf{Level 4: Bore diameter issues by shift/operator}
1. Create Pareto by shift
2. Result: Evenly distributed - not operator-dependent (machine issue confirmed)

\textbf{Final Analysis}: Investigate boring operation on Machine \#4 specifically - likely tool wear, spindle issue, or program error.

The stratified approach efficiently narrows focus from ``we have a scrap problem'' to ``boring operation on Machine \#4 has a systematic error.''

\textbf{Question 9}: A defense contractor needs to analyze a safety-critical failure. Why might Fault Tree Analysis (FTA) be preferred over simpler methods like 5 Whys?

\textbf{Answer:}

\textbf{FTA advantages for safety-critical analysis}:

\begin{enumerate}
\def\labelenumi{\arabic{enumi}.}
\tightlist
\item
  \textbf{Handles complexity}:

  \begin{itemize}
  \tightlist
  \item
    Safety-critical systems often have multiple redundancies and complex failure paths
  \item
    FTA can model combinations of events that must occur together (AND gates) or alternatively (OR gates)
  \item
    5 Whys follows single linear path; FTA captures parallel paths
  \end{itemize}
\item
  \textbf{Quantitative capability}:

  \begin{itemize}
  \tightlist
  \item
    FTA allows probability calculations for top event
  \item
    Can identify minimum cut sets (smallest combinations that cause failure)
  \item
    Supports reliability and safety integrity level (SIL) calculations
  \item
    Defense/aerospace require quantitative risk assessment
  \end{itemize}
\item
  \textbf{Documentation requirements}:

  \begin{itemize}
  \tightlist
  \item
    Regulatory bodies (FAA, DoD) often require FTA for critical systems
  \item
    Provides auditable, traceable analysis
  \item
    Standard symbols and methodology
  \end{itemize}
\item
  \textbf{Identifies hidden failures}:

  \begin{itemize}
  \tightlist
  \item
    Can reveal single points of failure
  \item
    Shows where redundancy is (or isn't) effective
  \item
    Identifies common cause failures that defeat redundancy
  \end{itemize}
\item
  \textbf{Design improvement}:

  \begin{itemize}
  \tightlist
  \item
    Visual representation helps identify where to add safeguards
  \item
    Sensitivity analysis shows which basic events most affect risk
  \item
    Supports design reviews and safety cases
  \end{itemize}
\end{enumerate}

\textbf{Example}: A missile guidance system failure
- 5 Whys might identify one cause path
- FTA would show all possible failure paths, calculate overall failure probability, and identify that two specific sensor failures AND a software error all occurring together could cause the top event, even though each alone is unlikely.

\begin{center}\rule{0.5\linewidth}{0.5pt}\end{center}

\section{References}\label{references-10}

\begin{enumerate}
\def\labelenumi{\arabic{enumi}.}
\item
  Latino, R.J., Latino, K.C., \& Latino, M.A.~(2019). \emph{Root Cause Analysis: Improving Performance for Bottom-Line Results} (5th ed.). CRC Press.
\item
  Okes, D. (2019). \emph{Root Cause Analysis: The Core of Problem Solving and Corrective Action} (2nd ed.). ASQ Quality Press.
\item
  Andersen, B., \& Fagerhaug, T. (2006). \emph{Root Cause Analysis: Simplified Tools and Techniques} (2nd ed.). ASQ Quality Press.
\item
  AIAG. (2018). \emph{CQI-20: Effective Problem Solving Guide}. Automotive Industry Action Group.
\item
  IEC 61025:2006. \emph{Fault Tree Analysis (FTA)}. International Electrotechnical Commission.
\item
  Ishikawa, K. (1990). \emph{Introduction to Quality Control}. 3A Corporation.
\item
  Juran, J.M. (1988). \emph{Juran on Planning for Quality}. Free Press.
\item
  Ford Motor Company. (2019). \emph{Global 8D Problem Solving Manual}. Ford Motor Company.
\item
  NASA. (2002). \emph{Fault Tree Handbook with Aerospace Applications}. NASA Office of Safety and Mission Assurance.
\item
  Wilson, P.F., Dell, L.D., \& Anderson, G.F. (1993). \emph{Root Cause Analysis: A Tool for Total Quality Management}. ASQ Quality Press.
\end{enumerate}

\chapter{Measurement Systems Analysis}\label{measurement-systems-analysis}

\section{Learning Objectives}\label{learning-objectives-11}

After completing this chapter, you will be able to:

\begin{enumerate}
\def\labelenumi{\arabic{enumi}.}
\tightlist
\item
  Explain the importance of measurement system quality in process control
\item
  Identify the components of measurement system variation
\item
  Define and calculate accuracy, bias, linearity, stability, repeatability, and reproducibility
\item
  Conduct and interpret a Gauge R\&R study using both the Range and ANOVA methods
\item
  Determine measurement system acceptability using \%GRR and ndc criteria
\item
  Apply attribute measurement system analysis for pass/fail decisions
\item
  Recognize common sources of measurement error and implement improvements
\item
  Understand the relationship between MSA and SPC effectiveness
\end{enumerate}

\begin{center}\rule{0.5\linewidth}{0.5pt}\end{center}

\section{Introduction to Measurement Systems Analysis}\label{introduction-to-measurement-systems-analysis}

\textbf{Measurement Systems Analysis (MSA)} is the science of evaluating the quality of measurement processes. Before we can trust our data for process control, quality decisions, or capability studies, we must first verify that our measurement system is capable of providing reliable results.

\subsection{Why MSA Matters}\label{why-msa-matters}

\begin{quote}
``You can't improve what you can't measure - but you also can't improve what you measure incorrectly.''
\end{quote}

Consider this scenario:

\pandocbounded{\includegraphics[keepaspectratio]{introduction_files/figure-latex/msa-importance-1.pdf}}

\subsection{The Cost of Poor Measurement}\label{the-cost-of-poor-measurement}

\begin{table}
\centering
\caption{\label{tab:measurement-costs}Business Impact of Poor Measurement Systems}
\centering
\begin{tabu} to \linewidth {>{\raggedright}X>{\raggedright}X>{\raggedright}X}
\hline
Impact & Description & Consequence\\
\hline
\textbf{False Accepts (Type II Error)} & Bad parts passed as good; reach customer & Warranty claims, recalls, reputation damage\\
\hline
\textbf{False Rejects (Type I Error)} & Good parts rejected as bad; scrapped or reworked & Increased scrap costs, reduced yield\\
\hline
\textbf{Process Adjustment Errors} & Adjusting process based on measurement noise, not real changes & Over-adjustment increases variation (tampering)\\
\hline
\textbf{Capability Misassessment} & Cp/Cpk calculations are wrong; true capability unknown & False confidence or unnecessary investment\\
\hline
\textbf{Customer Complaints} & Inconsistent quality due to measurement-based decisions & Loss of customer trust and business\\
\hline
\end{tabu}
\end{table}

\subsection{MSA in Industry Standards}\label{msa-in-industry-standards}

MSA is required by multiple quality standards:

\begin{itemize}
\tightlist
\item
  \textbf{IATF 16949} (Automotive): Requires MSA for all measurement systems referenced in control plans
\item
  \textbf{AS9100} (Aerospace): Requires demonstrated measurement capability
\item
  \textbf{ISO 9001}: Requires monitoring and measurement resources to be suitable
\item
  \textbf{FDA 21 CFR Part 820} (Medical Devices): Requires validated measurement equipment
\end{itemize}

\begin{center}\rule{0.5\linewidth}{0.5pt}\end{center}

\section{Components of Measurement Variation}\label{components-of-measurement-variation}

Total measurement variation comes from multiple sources. Understanding these components is essential for improvement.

\subsection{The Measurement System Model}\label{the-measurement-system-model}

\pandocbounded{\includegraphics[keepaspectratio]{introduction_files/figure-latex/variation-components-1.pdf}}

\subsection{Accuracy vs.~Precision}\label{accuracy-vs.-precision}

These are often confused but are fundamentally different concepts:

\begin{verbatim}
## Warning in geom_point(aes(x = 0, y = 0), color = "red", size = 4, shape = 3, : All aesthetics have length 1, but the data has 120
## rows.
## i Please consider using `annotate()` or provide
##   this layer with data containing a single row.
\end{verbatim}

\begin{verbatim}
## Warning in geom_circle(aes(x0 = 0, y0 = 0, r = 0.5), inherit.aes = FALSE, : All aesthetics have length 1, but the data has 120
## rows.
## i Please consider using `annotate()` or provide
##   this layer with data containing a single row.
\end{verbatim}

\begin{verbatim}
## Warning in geom_circle(aes(x0 = 0, y0 = 0, r = 1), inherit.aes = FALSE, : All aesthetics have length 1, but the data has 120
## rows.
## i Please consider using `annotate()` or provide
##   this layer with data containing a single row.
\end{verbatim}

\begin{verbatim}
## Warning in geom_circle(aes(x0 = 0, y0 = 0, r = 1.5), inherit.aes = FALSE, : All aesthetics have length 1, but the data has 120
## rows.
## i Please consider using `annotate()` or provide
##   this layer with data containing a single row.
\end{verbatim}

\pandocbounded{\includegraphics[keepaspectratio]{introduction_files/figure-latex/accuracy-precision-1.pdf}}

\begin{table}
\centering
\caption{\label{tab:accuracy-precision-table}Accuracy vs. Precision Comparison}
\centering
\begin{tabu} to \linewidth {>{\raggedright}X>{\raggedright}X>{\raggedright}X>{\raggedright}X>{\raggedright}X}
\hline
 & Definition & Affected By & Correctable? & Primary Metric\\
\hline
\textbf{Accuracy} & How close measurements are to the true value (average) & Calibration, bias, linearity & Yes - through calibration and adjustment & Bias (average error from reference)\\
\hline
\textbf{Precision} & How close measurements are to each other (spread) & Repeatability, reproducibility, resolution & Harder - often requires equipment upgrade & Standard deviation of repeated measurements\\
\hline
\end{tabu}
\end{table}

\begin{center}\rule{0.5\linewidth}{0.5pt}\end{center}

\section{Accuracy Studies}\label{accuracy-studies}

Accuracy studies evaluate how well the measurement system indicates the true value.

\subsection{Bias}\label{bias}

\textbf{Bias} is the systematic error - the difference between the average of measurements and the true (reference) value.

\[\text{Bias} = \bar{x}_{observed} - x_{reference}\]

\pandocbounded{\includegraphics[keepaspectratio]{introduction_files/figure-latex/bias-example-1.pdf}}

\begin{Shaded}
\begin{Highlighting}[]
\CommentTok{\# Bias study calculation}
\NormalTok{reference\_value }\OtherTok{\textless{}{-}} \FloatTok{50.000}  \CommentTok{\# Certified reference standard}
\NormalTok{measurements }\OtherTok{\textless{}{-}} \FunctionTok{c}\NormalTok{(}\FloatTok{50.08}\NormalTok{, }\FloatTok{50.07}\NormalTok{, }\FloatTok{50.09}\NormalTok{, }\FloatTok{50.08}\NormalTok{, }\FloatTok{50.06}\NormalTok{, }\FloatTok{50.10}\NormalTok{, }\FloatTok{50.07}\NormalTok{, }\FloatTok{50.08}\NormalTok{,}
                  \FloatTok{50.09}\NormalTok{, }\FloatTok{50.08}\NormalTok{, }\FloatTok{50.07}\NormalTok{, }\FloatTok{50.11}\NormalTok{, }\FloatTok{50.08}\NormalTok{, }\FloatTok{50.09}\NormalTok{, }\FloatTok{50.07}\NormalTok{)}

\CommentTok{\# Calculate bias}
\NormalTok{average\_measured }\OtherTok{\textless{}{-}} \FunctionTok{mean}\NormalTok{(measurements)}
\NormalTok{bias }\OtherTok{\textless{}{-}}\NormalTok{ average\_measured }\SpecialCharTok{{-}}\NormalTok{ reference\_value}

\FunctionTok{cat}\NormalTok{(}\StringTok{"Reference Value:"}\NormalTok{, reference\_value, }\StringTok{"mm}\SpecialCharTok{\textbackslash{}n}\StringTok{"}\NormalTok{)}
\end{Highlighting}
\end{Shaded}

\begin{verbatim}
## Reference Value: 50 mm
\end{verbatim}

\begin{Shaded}
\begin{Highlighting}[]
\FunctionTok{cat}\NormalTok{(}\StringTok{"Average Measured:"}\NormalTok{, }\FunctionTok{round}\NormalTok{(average\_measured, }\DecValTok{4}\NormalTok{), }\StringTok{"mm}\SpecialCharTok{\textbackslash{}n}\StringTok{"}\NormalTok{)}
\end{Highlighting}
\end{Shaded}

\begin{verbatim}
## Average Measured: 50.0813 mm
\end{verbatim}

\begin{Shaded}
\begin{Highlighting}[]
\FunctionTok{cat}\NormalTok{(}\StringTok{"Bias:"}\NormalTok{, }\FunctionTok{round}\NormalTok{(bias, }\DecValTok{4}\NormalTok{), }\StringTok{"mm}\SpecialCharTok{\textbackslash{}n}\StringTok{"}\NormalTok{)}
\end{Highlighting}
\end{Shaded}

\begin{verbatim}
## Bias: 0.0813 mm
\end{verbatim}

\begin{Shaded}
\begin{Highlighting}[]
\FunctionTok{cat}\NormalTok{(}\StringTok{"Bias as \% of Tolerance (±0.10):"}\NormalTok{, }\FunctionTok{round}\NormalTok{(}\FunctionTok{abs}\NormalTok{(bias) }\SpecialCharTok{/} \FloatTok{0.10} \SpecialCharTok{*} \DecValTok{100}\NormalTok{, }\DecValTok{1}\NormalTok{), }\StringTok{"\%}\SpecialCharTok{\textbackslash{}n}\StringTok{"}\NormalTok{)}
\end{Highlighting}
\end{Shaded}

\begin{verbatim}
## Bias as % of Tolerance (±0.10): 81.3 %
\end{verbatim}

\begin{Shaded}
\begin{Highlighting}[]
\CommentTok{\# Statistical test for significance}
\NormalTok{t\_test }\OtherTok{\textless{}{-}} \FunctionTok{t.test}\NormalTok{(measurements, }\AttributeTok{mu =}\NormalTok{ reference\_value)}
\FunctionTok{cat}\NormalTok{(}\StringTok{"}\SpecialCharTok{\textbackslash{}n}\StringTok{t{-}test p{-}value:"}\NormalTok{, }\FunctionTok{round}\NormalTok{(t\_test}\SpecialCharTok{$}\NormalTok{p.value, }\DecValTok{4}\NormalTok{))}
\end{Highlighting}
\end{Shaded}

\begin{verbatim}
## 
## t-test p-value: 0
\end{verbatim}

\begin{Shaded}
\begin{Highlighting}[]
\ControlFlowTok{if}\NormalTok{(t\_test}\SpecialCharTok{$}\NormalTok{p.value }\SpecialCharTok{\textless{}} \FloatTok{0.05}\NormalTok{) \{}
  \FunctionTok{cat}\NormalTok{(}\StringTok{" {-} Bias is statistically significant}\SpecialCharTok{\textbackslash{}n}\StringTok{"}\NormalTok{)}
\NormalTok{\} }\ControlFlowTok{else}\NormalTok{ \{}
  \FunctionTok{cat}\NormalTok{(}\StringTok{" {-} Bias is not statistically significant}\SpecialCharTok{\textbackslash{}n}\StringTok{"}\NormalTok{)}
\NormalTok{\}}
\end{Highlighting}
\end{Shaded}

\begin{verbatim}
##  - Bias is statistically significant
\end{verbatim}

\subsection{Linearity}\label{linearity}

\textbf{Linearity} measures whether bias changes across the measurement range. A gauge might be accurate at one end of its range but biased at the other.

\begin{verbatim}
## `geom_smooth()` using formula = 'y ~ x'
\end{verbatim}

\pandocbounded{\includegraphics[keepaspectratio]{introduction_files/figure-latex/linearity-study-1.pdf}}

\begin{Shaded}
\begin{Highlighting}[]
\CommentTok{\# Linearity calculation}
\NormalTok{reference }\OtherTok{\textless{}{-}} \FunctionTok{c}\NormalTok{(}\DecValTok{10}\NormalTok{, }\DecValTok{25}\NormalTok{, }\DecValTok{40}\NormalTok{, }\DecValTok{55}\NormalTok{, }\DecValTok{70}\NormalTok{)}
\NormalTok{avg\_bias }\OtherTok{\textless{}{-}} \FunctionTok{c}\NormalTok{(}\FloatTok{0.020}\NormalTok{, }\FloatTok{0.008}\NormalTok{, }\SpecialCharTok{{-}}\FloatTok{0.002}\NormalTok{, }\SpecialCharTok{{-}}\FloatTok{0.018}\NormalTok{, }\SpecialCharTok{{-}}\FloatTok{0.048}\NormalTok{)}

\CommentTok{\# Fit linear regression}
\NormalTok{model }\OtherTok{\textless{}{-}} \FunctionTok{lm}\NormalTok{(avg\_bias }\SpecialCharTok{\textasciitilde{}}\NormalTok{ reference)}

\CommentTok{\# Calculate linearity}
\NormalTok{slope }\OtherTok{\textless{}{-}} \FunctionTok{coef}\NormalTok{(model)[}\DecValTok{2}\NormalTok{]}
\NormalTok{range\_used }\OtherTok{\textless{}{-}} \FunctionTok{max}\NormalTok{(reference) }\SpecialCharTok{{-}} \FunctionTok{min}\NormalTok{(reference)}
\NormalTok{linearity }\OtherTok{\textless{}{-}} \FunctionTok{abs}\NormalTok{(slope) }\SpecialCharTok{*}\NormalTok{ range\_used}

\FunctionTok{cat}\NormalTok{(}\StringTok{"Regression equation: Bias ="}\NormalTok{, }\FunctionTok{round}\NormalTok{(}\FunctionTok{coef}\NormalTok{(model)[}\DecValTok{1}\NormalTok{], }\DecValTok{4}\NormalTok{), }\StringTok{"+"}\NormalTok{,}
    \FunctionTok{round}\NormalTok{(slope, }\DecValTok{5}\NormalTok{), }\StringTok{"× Reference}\SpecialCharTok{\textbackslash{}n}\StringTok{"}\NormalTok{)}
\end{Highlighting}
\end{Shaded}

\begin{verbatim}
## Regression equation: Bias = 0.0352 + -0.00108 × Reference
\end{verbatim}

\begin{Shaded}
\begin{Highlighting}[]
\FunctionTok{cat}\NormalTok{(}\StringTok{"Linearity (slope × range):"}\NormalTok{, }\FunctionTok{round}\NormalTok{(linearity, }\DecValTok{4}\NormalTok{), }\StringTok{"mm}\SpecialCharTok{\textbackslash{}n}\StringTok{"}\NormalTok{)}
\end{Highlighting}
\end{Shaded}

\begin{verbatim}
## Linearity (slope × range): 0.0648 mm
\end{verbatim}

\begin{Shaded}
\begin{Highlighting}[]
\FunctionTok{cat}\NormalTok{(}\StringTok{"As \% of tolerance (±0.10):"}\NormalTok{, }\FunctionTok{round}\NormalTok{(linearity }\SpecialCharTok{/} \FloatTok{0.10} \SpecialCharTok{*} \DecValTok{100}\NormalTok{, }\DecValTok{1}\NormalTok{), }\StringTok{"\%}\SpecialCharTok{\textbackslash{}n}\StringTok{"}\NormalTok{)}
\end{Highlighting}
\end{Shaded}

\begin{verbatim}
## As % of tolerance (±0.10): 64.8 %
\end{verbatim}

\begin{Shaded}
\begin{Highlighting}[]
\CommentTok{\# R{-}squared}
\FunctionTok{cat}\NormalTok{(}\StringTok{"R² ="}\NormalTok{, }\FunctionTok{round}\NormalTok{(}\FunctionTok{summary}\NormalTok{(model)}\SpecialCharTok{$}\NormalTok{r.squared, }\DecValTok{3}\NormalTok{),}
    \StringTok{"{-} indicates"}\NormalTok{, }\FunctionTok{ifelse}\NormalTok{(}\FunctionTok{summary}\NormalTok{(model)}\SpecialCharTok{$}\NormalTok{r.squared }\SpecialCharTok{\textgreater{}} \FloatTok{0.7}\NormalTok{, }\StringTok{"significant"}\NormalTok{, }\StringTok{"minor"}\NormalTok{),}
    \StringTok{"linearity issue}\SpecialCharTok{\textbackslash{}n}\StringTok{"}\NormalTok{)}
\end{Highlighting}
\end{Shaded}

\begin{verbatim}
## R² = 0.945 - indicates significant linearity issue
\end{verbatim}

\subsection{Stability}\label{stability}

\textbf{Stability} (also called drift) measures whether the measurement system's accuracy changes over time.

\begin{verbatim}
## `geom_smooth()` using formula = 'y ~ x'
\end{verbatim}

\pandocbounded{\includegraphics[keepaspectratio]{introduction_files/figure-latex/stability-study-1.pdf}}

\subsection{Stability Control Chart}\label{stability-control-chart}

For ongoing stability monitoring, use a control chart:

\begin{Shaded}
\begin{Highlighting}[]
\CommentTok{\# Stability monitoring with control chart}
\NormalTok{reference }\OtherTok{\textless{}{-}} \FloatTok{25.000}
\NormalTok{measurements }\OtherTok{\textless{}{-}} \FunctionTok{c}\NormalTok{(}\FloatTok{25.012}\NormalTok{, }\FloatTok{25.008}\NormalTok{, }\FloatTok{25.015}\NormalTok{, }\FloatTok{25.018}\NormalTok{, }\FloatTok{25.022}\NormalTok{, }\FloatTok{25.019}\NormalTok{, }\FloatTok{25.025}\NormalTok{,}
                  \FloatTok{25.028}\NormalTok{, }\FloatTok{25.031}\NormalTok{, }\FloatTok{25.026}\NormalTok{, }\FloatTok{25.033}\NormalTok{, }\FloatTok{25.038}\NormalTok{, }\FloatTok{25.035}\NormalTok{, }\FloatTok{25.041}\NormalTok{, }\FloatTok{25.039}\NormalTok{)}
\NormalTok{days }\OtherTok{\textless{}{-}} \DecValTok{1}\SpecialCharTok{:}\DecValTok{15}

\CommentTok{\# Calculate control limits (based on historical sigma)}
\NormalTok{historical\_sigma }\OtherTok{\textless{}{-}} \FloatTok{0.010}
\NormalTok{x\_bar }\OtherTok{\textless{}{-}} \FunctionTok{mean}\NormalTok{(measurements[}\DecValTok{1}\SpecialCharTok{:}\DecValTok{5}\NormalTok{])  }\CommentTok{\# Baseline from first 5 days}
\NormalTok{UCL }\OtherTok{\textless{}{-}}\NormalTok{ x\_bar }\SpecialCharTok{+} \DecValTok{3} \SpecialCharTok{*}\NormalTok{ historical\_sigma}
\NormalTok{LCL }\OtherTok{\textless{}{-}}\NormalTok{ x\_bar }\SpecialCharTok{{-}} \DecValTok{3} \SpecialCharTok{*}\NormalTok{ historical\_sigma}

\FunctionTok{cat}\NormalTok{(}\StringTok{"Baseline Average:"}\NormalTok{, }\FunctionTok{round}\NormalTok{(x\_bar, }\DecValTok{4}\NormalTok{), }\StringTok{"}\SpecialCharTok{\textbackslash{}n}\StringTok{"}\NormalTok{)}
\end{Highlighting}
\end{Shaded}

\begin{verbatim}
## Baseline Average: 25.015
\end{verbatim}

\begin{Shaded}
\begin{Highlighting}[]
\FunctionTok{cat}\NormalTok{(}\StringTok{"UCL:"}\NormalTok{, }\FunctionTok{round}\NormalTok{(UCL, }\DecValTok{4}\NormalTok{), }\StringTok{"}\SpecialCharTok{\textbackslash{}n}\StringTok{"}\NormalTok{)}
\end{Highlighting}
\end{Shaded}

\begin{verbatim}
## UCL: 25.045
\end{verbatim}

\begin{Shaded}
\begin{Highlighting}[]
\FunctionTok{cat}\NormalTok{(}\StringTok{"LCL:"}\NormalTok{, }\FunctionTok{round}\NormalTok{(LCL, }\DecValTok{4}\NormalTok{), }\StringTok{"}\SpecialCharTok{\textbackslash{}n}\StringTok{"}\NormalTok{)}
\end{Highlighting}
\end{Shaded}

\begin{verbatim}
## LCL: 24.985
\end{verbatim}

\begin{Shaded}
\begin{Highlighting}[]
\CommentTok{\# Check for out{-}of{-}control}
\NormalTok{out\_of\_control }\OtherTok{\textless{}{-}} \FunctionTok{which}\NormalTok{(measurements }\SpecialCharTok{\textgreater{}}\NormalTok{ UCL }\SpecialCharTok{|}\NormalTok{ measurements }\SpecialCharTok{\textless{}}\NormalTok{ LCL)}
\ControlFlowTok{if}\NormalTok{(}\FunctionTok{length}\NormalTok{(out\_of\_control) }\SpecialCharTok{\textgreater{}} \DecValTok{0}\NormalTok{) \{}
  \FunctionTok{cat}\NormalTok{(}\StringTok{"}\SpecialCharTok{\textbackslash{}n}\StringTok{Out{-}of{-}control points on days:"}\NormalTok{, out\_of\_control, }\StringTok{"}\SpecialCharTok{\textbackslash{}n}\StringTok{"}\NormalTok{)}
  \FunctionTok{cat}\NormalTok{(}\StringTok{"Drift detected {-} recalibration recommended}\SpecialCharTok{\textbackslash{}n}\StringTok{"}\NormalTok{)}
\NormalTok{\} }\ControlFlowTok{else}\NormalTok{ \{}
  \FunctionTok{cat}\NormalTok{(}\StringTok{"}\SpecialCharTok{\textbackslash{}n}\StringTok{No out{-}of{-}control points {-} measurement system stable}\SpecialCharTok{\textbackslash{}n}\StringTok{"}\NormalTok{)}
\NormalTok{\}}
\end{Highlighting}
\end{Shaded}

\begin{verbatim}
## 
## No out-of-control points - measurement system stable
\end{verbatim}

\begin{center}\rule{0.5\linewidth}{0.5pt}\end{center}

\section{Precision Studies: Gauge R\&R}\label{precision-studies-gauge-rr}

\textbf{Gauge R\&R} (Repeatability and Reproducibility) is the most common MSA study. It quantifies precision variation.

\subsection{Repeatability vs.~Reproducibility}\label{repeatability-vs.-reproducibility}

\pandocbounded{\includegraphics[keepaspectratio]{introduction_files/figure-latex/rr-visual-1.pdf}}

\begin{table}
\centering
\caption{\label{tab:rr-definitions}Gauge R&R Components}
\centering
\begin{tabu} to \linewidth {>{\raggedright}X>{\raggedright}X>{\raggedright}X>{\raggedright}X}
\hline
Component & Also Known As & Definition & Caused By\\
\hline
\textbf{Repeatability (EV)} & Equipment Variation, Within-operator variation & Variation when same operator measures same part multiple times with same gauge & Gauge resolution, gauge condition, fixture, technique consistency\\
\hline
\textbf{Reproducibility (AV)} & Appraiser Variation, Between-operator variation & Variation when different operators measure the same parts & Training differences, technique differences, interpretation\\
\hline
\textbf{Gauge R\&R (GRR)} & Total measurement system variation & Combined repeatability and reproducibility & All measurement system sources combined\\
\hline
\end{tabu}
\end{table}

\subsection{Gauge R\&R Study Design}\label{gauge-rr-study-design}

\begin{table}
\centering
\caption{\label{tab:grr-design}Standard Gauge R&R Study Design Parameters}
\centering
\begin{tabu} to \linewidth {>{\raggedright}X>{\raggedright}X>{\raggedright}X}
\hline
Parameter & Typical\_Value & Notes\\
\hline
\textbf{Number of Parts} & 10 & Minimum 5, 10 recommended for statistical power\\
\hline
\textbf{Number of Operators} & 3 & Minimum 2, 3 is standard for detecting operator effects\\
\hline
\textbf{Number of Trials} & 3 & Minimum 2, 3 is standard for repeatability estimation\\
\hline
\textbf{Total Measurements} & 10 × 3 × 3 = 90 & More measurements = better precision in estimates\\
\hline
\textbf{Part Selection} & Cover the full process range & Parts should represent typical production variation\\
\hline
\textbf{Randomization} & Randomize measurement order & Blind measurements when possible; prevents bias\\
\hline
\end{tabu}
\end{table}

\subsection{Range Method (Short Form)}\label{range-method-short-form}

The Range Method is a quick approximation useful for initial screening.

\begin{Shaded}
\begin{Highlighting}[]
\CommentTok{\# Range Method Gauge R\&R Calculation}
\CommentTok{\# Two operators, five parts, two trials each}

\CommentTok{\# Data: Measurements by each operator (2 trials per part)}
\NormalTok{operator\_a }\OtherTok{\textless{}{-}} \FunctionTok{matrix}\NormalTok{(}\FunctionTok{c}\NormalTok{(}
  \FloatTok{0.85}\NormalTok{, }\FloatTok{0.87}\NormalTok{,  }\CommentTok{\# Part 1}
  \FloatTok{0.92}\NormalTok{, }\FloatTok{0.91}\NormalTok{,  }\CommentTok{\# Part 2}
  \FloatTok{0.78}\NormalTok{, }\FloatTok{0.80}\NormalTok{,  }\CommentTok{\# Part 3}
  \FloatTok{0.95}\NormalTok{, }\FloatTok{0.96}\NormalTok{,  }\CommentTok{\# Part 4}
  \FloatTok{0.88}\NormalTok{, }\FloatTok{0.85}   \CommentTok{\# Part 5}
\NormalTok{), }\AttributeTok{ncol =} \DecValTok{2}\NormalTok{, }\AttributeTok{byrow =} \ConstantTok{TRUE}\NormalTok{)}

\NormalTok{operator\_b }\OtherTok{\textless{}{-}} \FunctionTok{matrix}\NormalTok{(}\FunctionTok{c}\NormalTok{(}
  \FloatTok{0.86}\NormalTok{, }\FloatTok{0.88}\NormalTok{,  }\CommentTok{\# Part 1}
  \FloatTok{0.93}\NormalTok{, }\FloatTok{0.92}\NormalTok{,  }\CommentTok{\# Part 2}
  \FloatTok{0.80}\NormalTok{, }\FloatTok{0.79}\NormalTok{,  }\CommentTok{\# Part 3}
  \FloatTok{0.94}\NormalTok{, }\FloatTok{0.97}\NormalTok{,  }\CommentTok{\# Part 4}
  \FloatTok{0.87}\NormalTok{, }\FloatTok{0.88}   \CommentTok{\# Part 5}
\NormalTok{), }\AttributeTok{ncol =} \DecValTok{2}\NormalTok{, }\AttributeTok{byrow =} \ConstantTok{TRUE}\NormalTok{)}

\CommentTok{\# Step 1: Calculate ranges for each part{-}operator combination}
\NormalTok{range\_a }\OtherTok{\textless{}{-}} \FunctionTok{apply}\NormalTok{(operator\_a, }\DecValTok{1}\NormalTok{, }\ControlFlowTok{function}\NormalTok{(x) }\FunctionTok{max}\NormalTok{(x) }\SpecialCharTok{{-}} \FunctionTok{min}\NormalTok{(x))}
\NormalTok{range\_b }\OtherTok{\textless{}{-}} \FunctionTok{apply}\NormalTok{(operator\_b, }\DecValTok{1}\NormalTok{, }\ControlFlowTok{function}\NormalTok{(x) }\FunctionTok{max}\NormalTok{(x) }\SpecialCharTok{{-}} \FunctionTok{min}\NormalTok{(x))}

\CommentTok{\# Step 2: Average range}
\NormalTok{R\_bar }\OtherTok{\textless{}{-}} \FunctionTok{mean}\NormalTok{(}\FunctionTok{c}\NormalTok{(range\_a, range\_b))}

\CommentTok{\# Step 3: Calculate Repeatability (EV)}
\CommentTok{\# d2 for 2 trials = 1.128}
\NormalTok{d2\_trials }\OtherTok{\textless{}{-}} \FloatTok{1.128}
\NormalTok{EV }\OtherTok{\textless{}{-}}\NormalTok{ R\_bar }\SpecialCharTok{/}\NormalTok{ d2\_trials}

\CommentTok{\# Step 4: Calculate part averages by operator}
\NormalTok{avg\_a }\OtherTok{\textless{}{-}} \FunctionTok{rowMeans}\NormalTok{(operator\_a)}
\NormalTok{avg\_b }\OtherTok{\textless{}{-}} \FunctionTok{rowMeans}\NormalTok{(operator\_b)}

\CommentTok{\# Step 5: Range of operator averages}
\NormalTok{X\_diff }\OtherTok{\textless{}{-}} \FunctionTok{abs}\NormalTok{(}\FunctionTok{mean}\NormalTok{(avg\_a) }\SpecialCharTok{{-}} \FunctionTok{mean}\NormalTok{(avg\_b))}

\CommentTok{\# d2 for 2 operators = 1.128}
\NormalTok{d2\_operators }\OtherTok{\textless{}{-}} \FloatTok{1.128}
\NormalTok{n\_parts }\OtherTok{\textless{}{-}} \DecValTok{5}
\NormalTok{n\_trials }\OtherTok{\textless{}{-}} \DecValTok{2}

\CommentTok{\# Step 6: Reproducibility (AV)}
\NormalTok{AV\_squared }\OtherTok{\textless{}{-}}\NormalTok{ (X\_diff }\SpecialCharTok{/}\NormalTok{ d2\_operators)}\SpecialCharTok{\^{}}\DecValTok{2} \SpecialCharTok{{-}}\NormalTok{ (EV}\SpecialCharTok{\^{}}\DecValTok{2} \SpecialCharTok{/}\NormalTok{ (n\_parts }\SpecialCharTok{*}\NormalTok{ n\_trials))}
\NormalTok{AV }\OtherTok{\textless{}{-}} \FunctionTok{sqrt}\NormalTok{(}\FunctionTok{max}\NormalTok{(}\DecValTok{0}\NormalTok{, AV\_squared))}

\CommentTok{\# Step 7: Gauge R\&R}
\NormalTok{GRR }\OtherTok{\textless{}{-}} \FunctionTok{sqrt}\NormalTok{(EV}\SpecialCharTok{\^{}}\DecValTok{2} \SpecialCharTok{+}\NormalTok{ AV}\SpecialCharTok{\^{}}\DecValTok{2}\NormalTok{)}

\CommentTok{\# Step 8: Express as \% of tolerance}
\NormalTok{tolerance }\OtherTok{\textless{}{-}} \FloatTok{0.30}  \CommentTok{\# Example tolerance of ±0.15 = 0.30 total}

\FunctionTok{cat}\NormalTok{(}\StringTok{"=== Range Method Gauge R\&R Results ===}\SpecialCharTok{\textbackslash{}n\textbackslash{}n}\StringTok{"}\NormalTok{)}
\end{Highlighting}
\end{Shaded}

\begin{verbatim}
## === Range Method Gauge R&R Results ===
\end{verbatim}

\begin{Shaded}
\begin{Highlighting}[]
\FunctionTok{cat}\NormalTok{(}\StringTok{"Average Range (R{-}bar):"}\NormalTok{, }\FunctionTok{round}\NormalTok{(R\_bar, }\DecValTok{4}\NormalTok{), }\StringTok{"}\SpecialCharTok{\textbackslash{}n}\StringTok{"}\NormalTok{)}
\end{Highlighting}
\end{Shaded}

\begin{verbatim}
## Average Range (R-bar): 0.017
\end{verbatim}

\begin{Shaded}
\begin{Highlighting}[]
\FunctionTok{cat}\NormalTok{(}\StringTok{"Repeatability (EV):"}\NormalTok{, }\FunctionTok{round}\NormalTok{(EV, }\DecValTok{4}\NormalTok{), }\StringTok{"}\SpecialCharTok{\textbackslash{}n}\StringTok{"}\NormalTok{)}
\end{Highlighting}
\end{Shaded}

\begin{verbatim}
## Repeatability (EV): 0.0151
\end{verbatim}

\begin{Shaded}
\begin{Highlighting}[]
\FunctionTok{cat}\NormalTok{(}\StringTok{"Reproducibility (AV):"}\NormalTok{, }\FunctionTok{round}\NormalTok{(AV, }\DecValTok{4}\NormalTok{), }\StringTok{"}\SpecialCharTok{\textbackslash{}n}\StringTok{"}\NormalTok{)}
\end{Highlighting}
\end{Shaded}

\begin{verbatim}
## Reproducibility (AV): 0.004
\end{verbatim}

\begin{Shaded}
\begin{Highlighting}[]
\FunctionTok{cat}\NormalTok{(}\StringTok{"Gauge R\&R (GRR):"}\NormalTok{, }\FunctionTok{round}\NormalTok{(GRR, }\DecValTok{4}\NormalTok{), }\StringTok{"}\SpecialCharTok{\textbackslash{}n\textbackslash{}n}\StringTok{"}\NormalTok{)}
\end{Highlighting}
\end{Shaded}

\begin{verbatim}
## Gauge R&R (GRR): 0.0156
\end{verbatim}

\begin{Shaded}
\begin{Highlighting}[]
\FunctionTok{cat}\NormalTok{(}\StringTok{"\%EV (of tolerance):"}\NormalTok{, }\FunctionTok{round}\NormalTok{(EV }\SpecialCharTok{/}\NormalTok{ tolerance }\SpecialCharTok{*} \DecValTok{100} \SpecialCharTok{*} \FloatTok{5.15}\NormalTok{, }\DecValTok{1}\NormalTok{), }\StringTok{"\%}\SpecialCharTok{\textbackslash{}n}\StringTok{"}\NormalTok{)}
\end{Highlighting}
\end{Shaded}

\begin{verbatim}
## %EV (of tolerance): 25.9 %
\end{verbatim}

\begin{Shaded}
\begin{Highlighting}[]
\FunctionTok{cat}\NormalTok{(}\StringTok{"\%AV (of tolerance):"}\NormalTok{, }\FunctionTok{round}\NormalTok{(AV }\SpecialCharTok{/}\NormalTok{ tolerance }\SpecialCharTok{*} \DecValTok{100} \SpecialCharTok{*} \FloatTok{5.15}\NormalTok{, }\DecValTok{1}\NormalTok{), }\StringTok{"\%}\SpecialCharTok{\textbackslash{}n}\StringTok{"}\NormalTok{)}
\end{Highlighting}
\end{Shaded}

\begin{verbatim}
## %AV (of tolerance): 6.8 %
\end{verbatim}

\begin{Shaded}
\begin{Highlighting}[]
\FunctionTok{cat}\NormalTok{(}\StringTok{"\%GRR (of tolerance):"}\NormalTok{, }\FunctionTok{round}\NormalTok{(GRR }\SpecialCharTok{/}\NormalTok{ tolerance }\SpecialCharTok{*} \DecValTok{100} \SpecialCharTok{*} \FloatTok{5.15}\NormalTok{, }\DecValTok{1}\NormalTok{), }\StringTok{"\%}\SpecialCharTok{\textbackslash{}n}\StringTok{"}\NormalTok{)}
\end{Highlighting}
\end{Shaded}

\begin{verbatim}
## %GRR (of tolerance): 26.8 %
\end{verbatim}

\subsection{ANOVA Method (Full Study)}\label{anova-method-full-study}

The ANOVA method is more accurate and provides additional information.

\begin{Shaded}
\begin{Highlighting}[]
\CommentTok{\# Full Gauge R\&R using ANOVA method}
\CommentTok{\# 10 parts, 3 operators, 3 trials}

\FunctionTok{set.seed}\NormalTok{(}\DecValTok{42}\NormalTok{)}

\CommentTok{\# Generate realistic GRR data}
\NormalTok{n\_parts }\OtherTok{\textless{}{-}} \DecValTok{10}
\NormalTok{n\_operators }\OtherTok{\textless{}{-}} \DecValTok{3}
\NormalTok{n\_trials }\OtherTok{\textless{}{-}} \DecValTok{3}

\CommentTok{\# Part true values (most of the variation)}
\NormalTok{part\_effects }\OtherTok{\textless{}{-}} \FunctionTok{rnorm}\NormalTok{(n\_parts, }\DecValTok{0}\NormalTok{, }\FloatTok{0.05}\NormalTok{)}

\CommentTok{\# Operator effects (some variation)}
\NormalTok{operator\_effects }\OtherTok{\textless{}{-}} \FunctionTok{c}\NormalTok{(}\SpecialCharTok{{-}}\FloatTok{0.008}\NormalTok{, }\FloatTok{0.003}\NormalTok{, }\FloatTok{0.005}\NormalTok{)}

\CommentTok{\# Create dataset}
\NormalTok{grr\_data }\OtherTok{\textless{}{-}} \FunctionTok{expand.grid}\NormalTok{(}
  \AttributeTok{Part =} \DecValTok{1}\SpecialCharTok{:}\NormalTok{n\_parts,}
  \AttributeTok{Operator =} \FunctionTok{paste0}\NormalTok{(}\StringTok{"Op"}\NormalTok{, }\DecValTok{1}\SpecialCharTok{:}\NormalTok{n\_operators),}
  \AttributeTok{Trial =} \DecValTok{1}\SpecialCharTok{:}\NormalTok{n\_trials}
\NormalTok{)}

\CommentTok{\# Add measurements}
\NormalTok{grr\_data}\SpecialCharTok{$}\NormalTok{Measurement }\OtherTok{\textless{}{-}} \DecValTok{10} \SpecialCharTok{+}
\NormalTok{  part\_effects[grr\_data}\SpecialCharTok{$}\NormalTok{Part] }\SpecialCharTok{+}
\NormalTok{  operator\_effects[}\FunctionTok{as.numeric}\NormalTok{(}\FunctionTok{factor}\NormalTok{(grr\_data}\SpecialCharTok{$}\NormalTok{Operator))] }\SpecialCharTok{+}
  \FunctionTok{rnorm}\NormalTok{(}\FunctionTok{nrow}\NormalTok{(grr\_data), }\DecValTok{0}\NormalTok{, }\FloatTok{0.006}\NormalTok{)  }\CommentTok{\# Repeatability error}

\CommentTok{\# Convert to factors}
\NormalTok{grr\_data}\SpecialCharTok{$}\NormalTok{Part }\OtherTok{\textless{}{-}} \FunctionTok{factor}\NormalTok{(grr\_data}\SpecialCharTok{$}\NormalTok{Part)}
\NormalTok{grr\_data}\SpecialCharTok{$}\NormalTok{Operator }\OtherTok{\textless{}{-}} \FunctionTok{factor}\NormalTok{(grr\_data}\SpecialCharTok{$}\NormalTok{Operator)}

\CommentTok{\# Fit ANOVA model}
\NormalTok{anova\_model }\OtherTok{\textless{}{-}} \FunctionTok{aov}\NormalTok{(Measurement }\SpecialCharTok{\textasciitilde{}}\NormalTok{ Part }\SpecialCharTok{+}\NormalTok{ Operator }\SpecialCharTok{+}\NormalTok{ Part}\SpecialCharTok{:}\NormalTok{Operator, }\AttributeTok{data =}\NormalTok{ grr\_data)}

\CommentTok{\# Extract variance components}
\NormalTok{anova\_table }\OtherTok{\textless{}{-}} \FunctionTok{anova}\NormalTok{(anova\_model)}
\FunctionTok{print}\NormalTok{(anova\_table)}
\end{Highlighting}
\end{Shaded}

\begin{verbatim}
## Analysis of Variance Table
## 
## Response: Measurement
##               Df   Sum Sq   Mean Sq  F value    Pr(>F)    
## Part           9 0.136635 0.0151816 343.6163 < 2.2e-16 ***
## Operator       2 0.003430 0.0017152  38.8223  1.52e-11 ***
## Part:Operator 18 0.000430 0.0000239   0.5401    0.9263    
## Residuals     60 0.002651 0.0000442                       
## ---
## Signif. codes:  0 '***' 0.001 '**' 0.01 '*' 0.05 '.' 0.1 ' ' 1
\end{verbatim}

\begin{Shaded}
\begin{Highlighting}[]
\CommentTok{\# Extract Mean Squares from ANOVA}
\NormalTok{MS\_part }\OtherTok{\textless{}{-}}\NormalTok{ anova\_table[}\StringTok{"Part"}\NormalTok{, }\StringTok{"Mean Sq"}\NormalTok{]}
\NormalTok{MS\_operator }\OtherTok{\textless{}{-}}\NormalTok{ anova\_table[}\StringTok{"Operator"}\NormalTok{, }\StringTok{"Mean Sq"}\NormalTok{]}
\NormalTok{MS\_interaction }\OtherTok{\textless{}{-}}\NormalTok{ anova\_table[}\StringTok{"Part:Operator"}\NormalTok{, }\StringTok{"Mean Sq"}\NormalTok{]}
\NormalTok{MS\_error }\OtherTok{\textless{}{-}}\NormalTok{ anova\_table[}\StringTok{"Residuals"}\NormalTok{, }\StringTok{"Mean Sq"}\NormalTok{]}

\CommentTok{\# Calculate variance components}
\NormalTok{n\_p }\OtherTok{\textless{}{-}} \DecValTok{10}  \CommentTok{\# number of parts}
\NormalTok{n\_o }\OtherTok{\textless{}{-}} \DecValTok{3}   \CommentTok{\# number of operators}
\NormalTok{n\_r }\OtherTok{\textless{}{-}} \DecValTok{3}   \CommentTok{\# number of trials}

\CommentTok{\# Repeatability variance}
\NormalTok{var\_repeatability }\OtherTok{\textless{}{-}}\NormalTok{ MS\_error}

\CommentTok{\# Interaction variance}
\NormalTok{var\_interaction }\OtherTok{\textless{}{-}} \FunctionTok{max}\NormalTok{(}\DecValTok{0}\NormalTok{, (MS\_interaction }\SpecialCharTok{{-}}\NormalTok{ MS\_error) }\SpecialCharTok{/}\NormalTok{ n\_r)}

\CommentTok{\# Operator variance}
\NormalTok{var\_operator }\OtherTok{\textless{}{-}} \FunctionTok{max}\NormalTok{(}\DecValTok{0}\NormalTok{, (MS\_operator }\SpecialCharTok{{-}}\NormalTok{ MS\_interaction) }\SpecialCharTok{/}\NormalTok{ (n\_p }\SpecialCharTok{*}\NormalTok{ n\_r))}

\CommentTok{\# Part variance}
\NormalTok{var\_part }\OtherTok{\textless{}{-}} \FunctionTok{max}\NormalTok{(}\DecValTok{0}\NormalTok{, (MS\_part }\SpecialCharTok{{-}}\NormalTok{ MS\_interaction) }\SpecialCharTok{/}\NormalTok{ (n\_o }\SpecialCharTok{*}\NormalTok{ n\_r))}

\CommentTok{\# Reproducibility = Operator + Interaction}
\NormalTok{var\_reproducibility }\OtherTok{\textless{}{-}}\NormalTok{ var\_operator }\SpecialCharTok{+}\NormalTok{ var\_interaction}

\CommentTok{\# Total Gauge R\&R}
\NormalTok{var\_GRR }\OtherTok{\textless{}{-}}\NormalTok{ var\_repeatability }\SpecialCharTok{+}\NormalTok{ var\_reproducibility}

\CommentTok{\# Total variation}
\NormalTok{var\_total }\OtherTok{\textless{}{-}}\NormalTok{ var\_part }\SpecialCharTok{+}\NormalTok{ var\_GRR}

\CommentTok{\# Convert to standard deviations}
\NormalTok{sigma\_repeatability }\OtherTok{\textless{}{-}} \FunctionTok{sqrt}\NormalTok{(var\_repeatability)}
\NormalTok{sigma\_reproducibility }\OtherTok{\textless{}{-}} \FunctionTok{sqrt}\NormalTok{(var\_reproducibility)}
\NormalTok{sigma\_GRR }\OtherTok{\textless{}{-}} \FunctionTok{sqrt}\NormalTok{(var\_GRR)}
\NormalTok{sigma\_part }\OtherTok{\textless{}{-}} \FunctionTok{sqrt}\NormalTok{(var\_part)}
\NormalTok{sigma\_total }\OtherTok{\textless{}{-}} \FunctionTok{sqrt}\NormalTok{(var\_total)}

\CommentTok{\# Calculate study variation (5.15 sigma for 99\% of distribution)}
\NormalTok{SV\_repeatability }\OtherTok{\textless{}{-}} \FloatTok{5.15} \SpecialCharTok{*}\NormalTok{ sigma\_repeatability}
\NormalTok{SV\_reproducibility }\OtherTok{\textless{}{-}} \FloatTok{5.15} \SpecialCharTok{*}\NormalTok{ sigma\_reproducibility}
\NormalTok{SV\_GRR }\OtherTok{\textless{}{-}} \FloatTok{5.15} \SpecialCharTok{*}\NormalTok{ sigma\_GRR}
\NormalTok{SV\_part }\OtherTok{\textless{}{-}} \FloatTok{5.15} \SpecialCharTok{*}\NormalTok{ sigma\_part}
\NormalTok{SV\_total }\OtherTok{\textless{}{-}} \FloatTok{5.15} \SpecialCharTok{*}\NormalTok{ sigma\_total}

\FunctionTok{cat}\NormalTok{(}\StringTok{"}\SpecialCharTok{\textbackslash{}n}\StringTok{=== ANOVA Gauge R\&R Results ===}\SpecialCharTok{\textbackslash{}n\textbackslash{}n}\StringTok{"}\NormalTok{)}
\end{Highlighting}
\end{Shaded}

\begin{verbatim}
## 
## === ANOVA Gauge R&R Results ===
\end{verbatim}

\begin{Shaded}
\begin{Highlighting}[]
\FunctionTok{cat}\NormalTok{(}\StringTok{"Variance Components:}\SpecialCharTok{\textbackslash{}n}\StringTok{"}\NormalTok{)}
\end{Highlighting}
\end{Shaded}

\begin{verbatim}
## Variance Components:
\end{verbatim}

\begin{Shaded}
\begin{Highlighting}[]
\FunctionTok{cat}\NormalTok{(}\StringTok{"  Repeatability:"}\NormalTok{, }\FunctionTok{round}\NormalTok{(var\_repeatability, }\DecValTok{8}\NormalTok{), }\StringTok{"}\SpecialCharTok{\textbackslash{}n}\StringTok{"}\NormalTok{)}
\end{Highlighting}
\end{Shaded}

\begin{verbatim}
##   Repeatability: 4.418e-05
\end{verbatim}

\begin{Shaded}
\begin{Highlighting}[]
\FunctionTok{cat}\NormalTok{(}\StringTok{"  Reproducibility:"}\NormalTok{, }\FunctionTok{round}\NormalTok{(var\_reproducibility, }\DecValTok{8}\NormalTok{), }\StringTok{"}\SpecialCharTok{\textbackslash{}n}\StringTok{"}\NormalTok{)}
\end{Highlighting}
\end{Shaded}

\begin{verbatim}
##   Reproducibility: 5.638e-05
\end{verbatim}

\begin{Shaded}
\begin{Highlighting}[]
\FunctionTok{cat}\NormalTok{(}\StringTok{"  Part{-}to{-}Part:"}\NormalTok{, }\FunctionTok{round}\NormalTok{(var\_part, }\DecValTok{8}\NormalTok{), }\StringTok{"}\SpecialCharTok{\textbackslash{}n}\StringTok{"}\NormalTok{)}
\end{Highlighting}
\end{Shaded}

\begin{verbatim}
##   Part-to-Part: 0.0016842
\end{verbatim}

\begin{Shaded}
\begin{Highlighting}[]
\FunctionTok{cat}\NormalTok{(}\StringTok{"  Total:"}\NormalTok{, }\FunctionTok{round}\NormalTok{(var\_total, }\DecValTok{8}\NormalTok{), }\StringTok{"}\SpecialCharTok{\textbackslash{}n}\StringTok{"}\NormalTok{)}
\end{Highlighting}
\end{Shaded}

\begin{verbatim}
##   Total: 0.00178476
\end{verbatim}

\begin{Shaded}
\begin{Highlighting}[]
\FunctionTok{cat}\NormalTok{(}\StringTok{"}\SpecialCharTok{\textbackslash{}n}\StringTok{Study Variation (5.15σ):}\SpecialCharTok{\textbackslash{}n}\StringTok{"}\NormalTok{)}
\end{Highlighting}
\end{Shaded}

\begin{verbatim}
## 
## Study Variation (5.15σ):
\end{verbatim}

\begin{Shaded}
\begin{Highlighting}[]
\FunctionTok{cat}\NormalTok{(}\StringTok{"  Repeatability:"}\NormalTok{, }\FunctionTok{round}\NormalTok{(SV\_repeatability, }\DecValTok{5}\NormalTok{), }\StringTok{"}\SpecialCharTok{\textbackslash{}n}\StringTok{"}\NormalTok{)}
\end{Highlighting}
\end{Shaded}

\begin{verbatim}
##   Repeatability: 0.03423
\end{verbatim}

\begin{Shaded}
\begin{Highlighting}[]
\FunctionTok{cat}\NormalTok{(}\StringTok{"  Reproducibility:"}\NormalTok{, }\FunctionTok{round}\NormalTok{(SV\_reproducibility, }\DecValTok{5}\NormalTok{), }\StringTok{"}\SpecialCharTok{\textbackslash{}n}\StringTok{"}\NormalTok{)}
\end{Highlighting}
\end{Shaded}

\begin{verbatim}
##   Reproducibility: 0.03867
\end{verbatim}

\begin{Shaded}
\begin{Highlighting}[]
\FunctionTok{cat}\NormalTok{(}\StringTok{"  Gauge R\&R:"}\NormalTok{, }\FunctionTok{round}\NormalTok{(SV\_GRR, }\DecValTok{5}\NormalTok{), }\StringTok{"}\SpecialCharTok{\textbackslash{}n}\StringTok{"}\NormalTok{)}
\end{Highlighting}
\end{Shaded}

\begin{verbatim}
##   Gauge R&R: 0.05164
\end{verbatim}

\begin{Shaded}
\begin{Highlighting}[]
\FunctionTok{cat}\NormalTok{(}\StringTok{"  Part{-}to{-}Part:"}\NormalTok{, }\FunctionTok{round}\NormalTok{(SV\_part, }\DecValTok{5}\NormalTok{), }\StringTok{"}\SpecialCharTok{\textbackslash{}n}\StringTok{"}\NormalTok{)}
\end{Highlighting}
\end{Shaded}

\begin{verbatim}
##   Part-to-Part: 0.21135
\end{verbatim}

\begin{Shaded}
\begin{Highlighting}[]
\FunctionTok{cat}\NormalTok{(}\StringTok{"  Total:"}\NormalTok{, }\FunctionTok{round}\NormalTok{(SV\_total, }\DecValTok{5}\NormalTok{), }\StringTok{"}\SpecialCharTok{\textbackslash{}n}\StringTok{"}\NormalTok{)}
\end{Highlighting}
\end{Shaded}

\begin{verbatim}
##   Total: 0.21757
\end{verbatim}

\subsection{Calculating \%GRR and Acceptance Criteria}\label{calculating-grr-and-acceptance-criteria}

\begin{Shaded}
\begin{Highlighting}[]
\CommentTok{\# Calculate \%GRR using both methods}
\NormalTok{tolerance }\OtherTok{\textless{}{-}} \FloatTok{0.30}  \CommentTok{\# Total tolerance}

\CommentTok{\# Method 1: \% of Tolerance (\%GRR\_Tolerance)}
\NormalTok{pct\_GRR\_tol }\OtherTok{\textless{}{-}}\NormalTok{ (SV\_GRR }\SpecialCharTok{/}\NormalTok{ tolerance) }\SpecialCharTok{*} \DecValTok{100}

\CommentTok{\# Method 2: \% of Total Variation (\%GRR\_TV)}
\NormalTok{pct\_GRR\_tv }\OtherTok{\textless{}{-}}\NormalTok{ (sigma\_GRR }\SpecialCharTok{/}\NormalTok{ sigma\_total) }\SpecialCharTok{*} \DecValTok{100}

\CommentTok{\# Number of Distinct Categories (ndc)}
\NormalTok{ndc }\OtherTok{\textless{}{-}} \FloatTok{1.41} \SpecialCharTok{*}\NormalTok{ (sigma\_part }\SpecialCharTok{/}\NormalTok{ sigma\_GRR)}

\FunctionTok{cat}\NormalTok{(}\StringTok{"=== Gauge R\&R Acceptance Criteria ===}\SpecialCharTok{\textbackslash{}n\textbackslash{}n}\StringTok{"}\NormalTok{)}
\end{Highlighting}
\end{Shaded}

\begin{verbatim}
## === Gauge R&R Acceptance Criteria ===
\end{verbatim}

\begin{Shaded}
\begin{Highlighting}[]
\FunctionTok{cat}\NormalTok{(}\StringTok{"\%GRR (of Tolerance):"}\NormalTok{, }\FunctionTok{round}\NormalTok{(pct\_GRR\_tol, }\DecValTok{1}\NormalTok{), }\StringTok{"\%}\SpecialCharTok{\textbackslash{}n}\StringTok{"}\NormalTok{)}
\end{Highlighting}
\end{Shaded}

\begin{verbatim}
## %GRR (of Tolerance): 17.2 %
\end{verbatim}

\begin{Shaded}
\begin{Highlighting}[]
\FunctionTok{cat}\NormalTok{(}\StringTok{"\%GRR (of Total Variation):"}\NormalTok{, }\FunctionTok{round}\NormalTok{(pct\_GRR\_tv, }\DecValTok{1}\NormalTok{), }\StringTok{"\%}\SpecialCharTok{\textbackslash{}n}\StringTok{"}\NormalTok{)}
\end{Highlighting}
\end{Shaded}

\begin{verbatim}
## %GRR (of Total Variation): 23.7 %
\end{verbatim}

\begin{Shaded}
\begin{Highlighting}[]
\FunctionTok{cat}\NormalTok{(}\StringTok{"Number of Distinct Categories (ndc):"}\NormalTok{, }\FunctionTok{round}\NormalTok{(ndc, }\DecValTok{1}\NormalTok{), }\StringTok{"}\SpecialCharTok{\textbackslash{}n\textbackslash{}n}\StringTok{"}\NormalTok{)}
\end{Highlighting}
\end{Shaded}

\begin{verbatim}
## Number of Distinct Categories (ndc): 5.8
\end{verbatim}

\begin{Shaded}
\begin{Highlighting}[]
\CommentTok{\# Acceptance criteria}
\FunctionTok{cat}\NormalTok{(}\StringTok{"AIAG Acceptance Guidelines:}\SpecialCharTok{\textbackslash{}n}\StringTok{"}\NormalTok{)}
\end{Highlighting}
\end{Shaded}

\begin{verbatim}
## AIAG Acceptance Guidelines:
\end{verbatim}

\begin{Shaded}
\begin{Highlighting}[]
\FunctionTok{cat}\NormalTok{(}\StringTok{"─────────────────────────────────────}\SpecialCharTok{\textbackslash{}n}\StringTok{"}\NormalTok{)}
\end{Highlighting}
\end{Shaded}

\begin{verbatim}
## ─────────────────────────────────────
\end{verbatim}

\begin{Shaded}
\begin{Highlighting}[]
\ControlFlowTok{if}\NormalTok{(pct\_GRR\_tol }\SpecialCharTok{\textless{}} \DecValTok{10}\NormalTok{) \{}
  \FunctionTok{cat}\NormalTok{(}\StringTok{"\%GRR \textless{} 10\%: ACCEPTABLE {-} Measurement system is acceptable}\SpecialCharTok{\textbackslash{}n}\StringTok{"}\NormalTok{)}
\NormalTok{\} }\ControlFlowTok{else} \ControlFlowTok{if}\NormalTok{(pct\_GRR\_tol }\SpecialCharTok{\textless{}} \DecValTok{30}\NormalTok{) \{}
  \FunctionTok{cat}\NormalTok{(}\StringTok{"\%GRR 10{-}30\%: MARGINAL {-} May be acceptable based on application}\SpecialCharTok{\textbackslash{}n}\StringTok{"}\NormalTok{)}
\NormalTok{\} }\ControlFlowTok{else}\NormalTok{ \{}
  \FunctionTok{cat}\NormalTok{(}\StringTok{"\%GRR \textgreater{} 30\%: UNACCEPTABLE {-} Measurement system needs improvement}\SpecialCharTok{\textbackslash{}n}\StringTok{"}\NormalTok{)}
\NormalTok{\}}
\end{Highlighting}
\end{Shaded}

\begin{verbatim}
## %GRR 10-30%: MARGINAL - May be acceptable based on application
\end{verbatim}

\begin{Shaded}
\begin{Highlighting}[]
\ControlFlowTok{if}\NormalTok{(ndc }\SpecialCharTok{\textgreater{}=} \DecValTok{5}\NormalTok{) \{}
  \FunctionTok{cat}\NormalTok{(}\StringTok{"ndc ≥ 5: ACCEPTABLE {-} Adequate discrimination}\SpecialCharTok{\textbackslash{}n}\StringTok{"}\NormalTok{)}
\NormalTok{\} }\ControlFlowTok{else} \ControlFlowTok{if}\NormalTok{(ndc }\SpecialCharTok{\textgreater{}=} \DecValTok{2}\NormalTok{) \{}
  \FunctionTok{cat}\NormalTok{(}\StringTok{"ndc 2{-}4: MARGINAL {-} Limited discrimination ability}\SpecialCharTok{\textbackslash{}n}\StringTok{"}\NormalTok{)}
\NormalTok{\} }\ControlFlowTok{else}\NormalTok{ \{}
  \FunctionTok{cat}\NormalTok{(}\StringTok{"ndc \textless{} 2: UNACCEPTABLE {-} Cannot distinguish between parts}\SpecialCharTok{\textbackslash{}n}\StringTok{"}\NormalTok{)}
\NormalTok{\}}
\end{Highlighting}
\end{Shaded}

\begin{verbatim}
## ndc ≥ 5: ACCEPTABLE - Adequate discrimination
\end{verbatim}

\subsection{Gauge R\&R Acceptance Criteria Summary}\label{gauge-rr-acceptance-criteria-summary}

\begin{verbatim}
## Warning in latex_new_row_builder(target_row, table_info, bold, italic,
## monospace, : Setting full_width = TRUE will turn the table into a tabu
## environment where colors are not really easily configable with this package.
## Please consider turn off full_width.
## Warning in latex_new_row_builder(target_row, table_info, bold, italic,
## monospace, : Setting full_width = TRUE will turn the table into a tabu
## environment where colors are not really easily configable with this package.
## Please consider turn off full_width.
## Warning in latex_new_row_builder(target_row, table_info, bold, italic,
## monospace, : Setting full_width = TRUE will turn the table into a tabu
## environment where colors are not really easily configable with this package.
## Please consider turn off full_width.
## Warning in latex_new_row_builder(target_row, table_info, bold, italic,
## monospace, : Setting full_width = TRUE will turn the table into a tabu
## environment where colors are not really easily configable with this package.
## Please consider turn off full_width.
## Warning in latex_new_row_builder(target_row, table_info, bold, italic,
## monospace, : Setting full_width = TRUE will turn the table into a tabu
## environment where colors are not really easily configable with this package.
## Please consider turn off full_width.
## Warning in latex_new_row_builder(target_row, table_info, bold, italic,
## monospace, : Setting full_width = TRUE will turn the table into a tabu
## environment where colors are not really easily configable with this package.
## Please consider turn off full_width.
\end{verbatim}

\begin{table}
\centering
\caption{\label{tab:acceptance-table}AIAG Gauge R&R Acceptance Guidelines}
\centering
\begin{tabu} to \linewidth {>{\raggedright}X>{\raggedright}X>{\raggedright}X}
\hline
Criterion & Status & Interpretation\\
\hline
\textbf{\cellcolor[HTML]{d4edda}{\%GRR < 10\%}} & \cellcolor[HTML]{d4edda}{Acceptable} & \cellcolor[HTML]{d4edda}{Measurement system is acceptable for process control and capability}\\
\hline
\textbf{\cellcolor[HTML]{fff3cd}{\%GRR 10-30\%}} & \cellcolor[HTML]{fff3cd}{Marginal} & \cellcolor[HTML]{fff3cd}{May be acceptable depending on importance, cost of gauge, repair costs}\\
\hline
\textbf{\cellcolor[HTML]{f8d7da}{\%GRR > 30\%}} & \cellcolor[HTML]{f8d7da}{Unacceptable} & \cellcolor[HTML]{f8d7da}{Measurement system is not acceptable; action required}\\
\hline
\textbf{\cellcolor[HTML]{d4edda}{ndc ≥ 5}} & \cellcolor[HTML]{d4edda}{Acceptable} & \cellcolor[HTML]{d4edda}{Gauge can adequately distinguish between parts}\\
\hline
\textbf{\cellcolor[HTML]{fff3cd}{ndc = 2-4}} & \cellcolor[HTML]{fff3cd}{Marginal} & \cellcolor[HTML]{fff3cd}{Limited ability to distinguish; may be OK for simple pass/fail}\\
\hline
\textbf{\cellcolor[HTML]{f8d7da}{ndc < 2}} & \cellcolor[HTML]{f8d7da}{Unacceptable} & \cellcolor[HTML]{f8d7da}{Gauge cannot distinguish between parts; almost useless for control}\\
\hline
\end{tabu}
\end{table}

\subsection{Visual Analysis of Gauge R\&R}\label{visual-analysis-of-gauge-rr}

\pandocbounded{\includegraphics[keepaspectratio]{introduction_files/figure-latex/grr-visuals-1.pdf}}

\begin{center}\rule{0.5\linewidth}{0.5pt}\end{center}

\section{Attribute Measurement Systems}\label{attribute-measurement-systems}

For pass/fail or categorical measurements (visual inspection, go/no-go gauges), we use \textbf{Attribute MSA} or \textbf{Attribute Agreement Analysis}.

\subsection{Kappa Statistic}\label{kappa-statistic}

The \textbf{Kappa statistic} measures agreement beyond chance:

\[\kappa = \frac{P_o - P_e}{1 - P_e}\]

Where:
- \(P_o\) = Observed proportion of agreement
- \(P_e\) = Expected proportion of agreement by chance

\begin{verbatim}
## Warning in latex_new_row_builder(target_row, table_info, bold, italic,
## monospace, : Setting full_width = TRUE will turn the table into a tabu
## environment where colors are not really easily configable with this package.
## Please consider turn off full_width.
## Warning in latex_new_row_builder(target_row, table_info, bold, italic,
## monospace, : Setting full_width = TRUE will turn the table into a tabu
## environment where colors are not really easily configable with this package.
## Please consider turn off full_width.
## Warning in latex_new_row_builder(target_row, table_info, bold, italic,
## monospace, : Setting full_width = TRUE will turn the table into a tabu
## environment where colors are not really easily configable with this package.
## Please consider turn off full_width.
## Warning in latex_new_row_builder(target_row, table_info, bold, italic,
## monospace, : Setting full_width = TRUE will turn the table into a tabu
## environment where colors are not really easily configable with this package.
## Please consider turn off full_width.
## Warning in latex_new_row_builder(target_row, table_info, bold, italic,
## monospace, : Setting full_width = TRUE will turn the table into a tabu
## environment where colors are not really easily configable with this package.
## Please consider turn off full_width.
## Warning in latex_new_row_builder(target_row, table_info, bold, italic,
## monospace, : Setting full_width = TRUE will turn the table into a tabu
## environment where colors are not really easily configable with this package.
## Please consider turn off full_width.
\end{verbatim}

\begin{table}
\centering
\caption{\label{tab:kappa-interpretation}Kappa Statistic Interpretation Guide}
\centering
\begin{tabu} to \linewidth {>{\raggedright}X>{\raggedright}X>{\raggedright}X}
\hline
Kappa & Interpretation & Action\\
\hline
\cellcolor[HTML]{f8d7da}{\textbf{< 0}} & \cellcolor[HTML]{f8d7da}{Less than chance agreement} & \cellcolor[HTML]{f8d7da}{Major issues - do not use}\\
\hline
\cellcolor[HTML]{f8d7da}{\textbf{0.00 - 0.20}} & \cellcolor[HTML]{f8d7da}{Slight agreement} & \cellcolor[HTML]{f8d7da}{Major improvement needed}\\
\hline
\cellcolor[HTML]{fff3cd}{\textbf{0.21 - 0.40}} & \cellcolor[HTML]{fff3cd}{Fair agreement} & \cellcolor[HTML]{fff3cd}{Significant improvement needed}\\
\hline
\cellcolor[HTML]{fff3cd}{\textbf{0.41 - 0.60}} & \cellcolor[HTML]{fff3cd}{Moderate agreement} & \cellcolor[HTML]{fff3cd}{Some improvement needed}\\
\hline
\cellcolor[HTML]{d4edda}{\textbf{0.61 - 0.80}} & \cellcolor[HTML]{d4edda}{Substantial agreement} & \cellcolor[HTML]{d4edda}{Acceptable for most applications}\\
\hline
\cellcolor[HTML]{d4edda}{\textbf{0.81 - 1.00}} & \cellcolor[HTML]{d4edda}{Almost perfect agreement} & \cellcolor[HTML]{d4edda}{Excellent - acceptable}\\
\hline
\end{tabu}
\end{table}

\subsection{Attribute Agreement Study Example}\label{attribute-agreement-study-example}

\begin{Shaded}
\begin{Highlighting}[]
\CommentTok{\# Attribute Agreement Analysis}
\CommentTok{\# 3 operators inspect 30 samples, 2 trials each}
\CommentTok{\# Reference standard (known good/bad) available}

\FunctionTok{set.seed}\NormalTok{(}\DecValTok{123}\NormalTok{)}
\NormalTok{n\_samples }\OtherTok{\textless{}{-}} \DecValTok{30}
\NormalTok{n\_operators }\OtherTok{\textless{}{-}} \DecValTok{3}
\NormalTok{n\_trials }\OtherTok{\textless{}{-}} \DecValTok{2}

\CommentTok{\# Reference classification (10 bad, 20 good)}
\NormalTok{reference }\OtherTok{\textless{}{-}} \FunctionTok{c}\NormalTok{(}\FunctionTok{rep}\NormalTok{(}\StringTok{"Reject"}\NormalTok{, }\DecValTok{10}\NormalTok{), }\FunctionTok{rep}\NormalTok{(}\StringTok{"Accept"}\NormalTok{, }\DecValTok{20}\NormalTok{))}

\CommentTok{\# Simulate operator decisions (some disagreement)}
\NormalTok{simulate\_decision }\OtherTok{\textless{}{-}} \ControlFlowTok{function}\NormalTok{(ref, accuracy) \{}
  \FunctionTok{sapply}\NormalTok{(ref, }\ControlFlowTok{function}\NormalTok{(r) \{}
    \ControlFlowTok{if}\NormalTok{(}\FunctionTok{runif}\NormalTok{(}\DecValTok{1}\NormalTok{) }\SpecialCharTok{\textless{}}\NormalTok{ accuracy) r }\ControlFlowTok{else} \FunctionTok{ifelse}\NormalTok{(r }\SpecialCharTok{==} \StringTok{"Accept"}\NormalTok{, }\StringTok{"Reject"}\NormalTok{, }\StringTok{"Accept"}\NormalTok{)}
\NormalTok{  \})}
\NormalTok{\}}

\CommentTok{\# Operators with different accuracy levels}
\NormalTok{op1\_t1 }\OtherTok{\textless{}{-}} \FunctionTok{simulate\_decision}\NormalTok{(reference, }\FloatTok{0.92}\NormalTok{)}
\NormalTok{op1\_t2 }\OtherTok{\textless{}{-}} \FunctionTok{simulate\_decision}\NormalTok{(reference, }\FloatTok{0.92}\NormalTok{)}
\NormalTok{op2\_t1 }\OtherTok{\textless{}{-}} \FunctionTok{simulate\_decision}\NormalTok{(reference, }\FloatTok{0.88}\NormalTok{)}
\NormalTok{op2\_t2 }\OtherTok{\textless{}{-}} \FunctionTok{simulate\_decision}\NormalTok{(reference, }\FloatTok{0.88}\NormalTok{)}
\NormalTok{op3\_t1 }\OtherTok{\textless{}{-}} \FunctionTok{simulate\_decision}\NormalTok{(reference, }\FloatTok{0.95}\NormalTok{)}
\NormalTok{op3\_t2 }\OtherTok{\textless{}{-}} \FunctionTok{simulate\_decision}\NormalTok{(reference, }\FloatTok{0.95}\NormalTok{)}

\CommentTok{\# Calculate agreement metrics}

\CommentTok{\# 1. Within{-}operator agreement (repeatability)}
\NormalTok{within\_op1 }\OtherTok{\textless{}{-}} \FunctionTok{mean}\NormalTok{(op1\_t1 }\SpecialCharTok{==}\NormalTok{ op1\_t2)}
\NormalTok{within\_op2 }\OtherTok{\textless{}{-}} \FunctionTok{mean}\NormalTok{(op2\_t1 }\SpecialCharTok{==}\NormalTok{ op2\_t2)}
\NormalTok{within\_op3 }\OtherTok{\textless{}{-}} \FunctionTok{mean}\NormalTok{(op3\_t1 }\SpecialCharTok{==}\NormalTok{ op3\_t2)}

\CommentTok{\# 2. Each operator vs. reference}
\NormalTok{op1\_vs\_ref }\OtherTok{\textless{}{-}} \FunctionTok{mean}\NormalTok{(op1\_t1 }\SpecialCharTok{==}\NormalTok{ reference }\SpecialCharTok{\&}\NormalTok{ op1\_t2 }\SpecialCharTok{==}\NormalTok{ reference)}
\NormalTok{op2\_vs\_ref }\OtherTok{\textless{}{-}} \FunctionTok{mean}\NormalTok{(op2\_t1 }\SpecialCharTok{==}\NormalTok{ reference }\SpecialCharTok{\&}\NormalTok{ op2\_t2 }\SpecialCharTok{==}\NormalTok{ reference)}
\NormalTok{op3\_vs\_ref }\OtherTok{\textless{}{-}} \FunctionTok{mean}\NormalTok{(op3\_t1 }\SpecialCharTok{==}\NormalTok{ reference }\SpecialCharTok{\&}\NormalTok{ op3\_t2 }\SpecialCharTok{==}\NormalTok{ reference)}

\CommentTok{\# 3. All operators agree (both trials)}
\NormalTok{all\_agree\_t1 }\OtherTok{\textless{}{-}} \FunctionTok{mean}\NormalTok{(op1\_t1 }\SpecialCharTok{==}\NormalTok{ op2\_t1 }\SpecialCharTok{\&}\NormalTok{ op2\_t1 }\SpecialCharTok{==}\NormalTok{ op3\_t1)}
\NormalTok{all\_agree\_both }\OtherTok{\textless{}{-}} \FunctionTok{mean}\NormalTok{(op1\_t1 }\SpecialCharTok{==}\NormalTok{ op1\_t2 }\SpecialCharTok{\&}\NormalTok{ op1\_t1 }\SpecialCharTok{==}\NormalTok{ op2\_t1 }\SpecialCharTok{\&}\NormalTok{ op2\_t1 }\SpecialCharTok{==}\NormalTok{ op2\_t2 }\SpecialCharTok{\&}
\NormalTok{                        op2\_t1 }\SpecialCharTok{==}\NormalTok{ op3\_t1 }\SpecialCharTok{\&}\NormalTok{ op3\_t1 }\SpecialCharTok{==}\NormalTok{ op3\_t2)}

\FunctionTok{cat}\NormalTok{(}\StringTok{"=== Attribute Agreement Analysis ===}\SpecialCharTok{\textbackslash{}n\textbackslash{}n}\StringTok{"}\NormalTok{)}
\end{Highlighting}
\end{Shaded}

\begin{verbatim}
## === Attribute Agreement Analysis ===
\end{verbatim}

\begin{Shaded}
\begin{Highlighting}[]
\FunctionTok{cat}\NormalTok{(}\StringTok{"Within{-}Operator Agreement (Repeatability):}\SpecialCharTok{\textbackslash{}n}\StringTok{"}\NormalTok{)}
\end{Highlighting}
\end{Shaded}

\begin{verbatim}
## Within-Operator Agreement (Repeatability):
\end{verbatim}

\begin{Shaded}
\begin{Highlighting}[]
\FunctionTok{cat}\NormalTok{(}\StringTok{"  Operator 1:"}\NormalTok{, }\FunctionTok{round}\NormalTok{(within\_op1 }\SpecialCharTok{*} \DecValTok{100}\NormalTok{, }\DecValTok{1}\NormalTok{), }\StringTok{"\%}\SpecialCharTok{\textbackslash{}n}\StringTok{"}\NormalTok{)}
\end{Highlighting}
\end{Shaded}

\begin{verbatim}
##   Operator 1: 83.3 %
\end{verbatim}

\begin{Shaded}
\begin{Highlighting}[]
\FunctionTok{cat}\NormalTok{(}\StringTok{"  Operator 2:"}\NormalTok{, }\FunctionTok{round}\NormalTok{(within\_op2 }\SpecialCharTok{*} \DecValTok{100}\NormalTok{, }\DecValTok{1}\NormalTok{), }\StringTok{"\%}\SpecialCharTok{\textbackslash{}n}\StringTok{"}\NormalTok{)}
\end{Highlighting}
\end{Shaded}

\begin{verbatim}
##   Operator 2: 76.7 %
\end{verbatim}

\begin{Shaded}
\begin{Highlighting}[]
\FunctionTok{cat}\NormalTok{(}\StringTok{"  Operator 3:"}\NormalTok{, }\FunctionTok{round}\NormalTok{(within\_op3 }\SpecialCharTok{*} \DecValTok{100}\NormalTok{, }\DecValTok{1}\NormalTok{), }\StringTok{"\%}\SpecialCharTok{\textbackslash{}n\textbackslash{}n}\StringTok{"}\NormalTok{)}
\end{Highlighting}
\end{Shaded}

\begin{verbatim}
##   Operator 3: 93.3 %
\end{verbatim}

\begin{Shaded}
\begin{Highlighting}[]
\FunctionTok{cat}\NormalTok{(}\StringTok{"Each Operator vs. Reference (both trials correct):}\SpecialCharTok{\textbackslash{}n}\StringTok{"}\NormalTok{)}
\end{Highlighting}
\end{Shaded}

\begin{verbatim}
## Each Operator vs. Reference (both trials correct):
\end{verbatim}

\begin{Shaded}
\begin{Highlighting}[]
\FunctionTok{cat}\NormalTok{(}\StringTok{"  Operator 1:"}\NormalTok{, }\FunctionTok{round}\NormalTok{(op1\_vs\_ref }\SpecialCharTok{*} \DecValTok{100}\NormalTok{, }\DecValTok{1}\NormalTok{), }\StringTok{"\%}\SpecialCharTok{\textbackslash{}n}\StringTok{"}\NormalTok{)}
\end{Highlighting}
\end{Shaded}

\begin{verbatim}
##   Operator 1: 83.3 %
\end{verbatim}

\begin{Shaded}
\begin{Highlighting}[]
\FunctionTok{cat}\NormalTok{(}\StringTok{"  Operator 2:"}\NormalTok{, }\FunctionTok{round}\NormalTok{(op2\_vs\_ref }\SpecialCharTok{*} \DecValTok{100}\NormalTok{, }\DecValTok{1}\NormalTok{), }\StringTok{"\%}\SpecialCharTok{\textbackslash{}n}\StringTok{"}\NormalTok{)}
\end{Highlighting}
\end{Shaded}

\begin{verbatim}
##   Operator 2: 73.3 %
\end{verbatim}

\begin{Shaded}
\begin{Highlighting}[]
\FunctionTok{cat}\NormalTok{(}\StringTok{"  Operator 3:"}\NormalTok{, }\FunctionTok{round}\NormalTok{(op3\_vs\_ref }\SpecialCharTok{*} \DecValTok{100}\NormalTok{, }\DecValTok{1}\NormalTok{), }\StringTok{"\%}\SpecialCharTok{\textbackslash{}n\textbackslash{}n}\StringTok{"}\NormalTok{)}
\end{Highlighting}
\end{Shaded}

\begin{verbatim}
##   Operator 3: 93.3 %
\end{verbatim}

\begin{Shaded}
\begin{Highlighting}[]
\FunctionTok{cat}\NormalTok{(}\StringTok{"All Operators Agreement:}\SpecialCharTok{\textbackslash{}n}\StringTok{"}\NormalTok{)}
\end{Highlighting}
\end{Shaded}

\begin{verbatim}
## All Operators Agreement:
\end{verbatim}

\begin{Shaded}
\begin{Highlighting}[]
\FunctionTok{cat}\NormalTok{(}\StringTok{"  All agree (one trial):"}\NormalTok{, }\FunctionTok{round}\NormalTok{(all\_agree\_t1 }\SpecialCharTok{*} \DecValTok{100}\NormalTok{, }\DecValTok{1}\NormalTok{), }\StringTok{"\%}\SpecialCharTok{\textbackslash{}n}\StringTok{"}\NormalTok{)}
\end{Highlighting}
\end{Shaded}

\begin{verbatim}
##   All agree (one trial): 70 %
\end{verbatim}

\begin{Shaded}
\begin{Highlighting}[]
\FunctionTok{cat}\NormalTok{(}\StringTok{"  All agree (both trials):"}\NormalTok{, }\FunctionTok{round}\NormalTok{(all\_agree\_both }\SpecialCharTok{*} \DecValTok{100}\NormalTok{, }\DecValTok{1}\NormalTok{), }\StringTok{"\%}\SpecialCharTok{\textbackslash{}n}\StringTok{"}\NormalTok{)}
\end{Highlighting}
\end{Shaded}

\begin{verbatim}
##   All agree (both trials): 53.3 %
\end{verbatim}

\subsection{Calculating Kappa}\label{calculating-kappa}

\begin{Shaded}
\begin{Highlighting}[]
\CommentTok{\# Calculate Kappa for Operator 1 vs Reference}
\CommentTok{\# Create confusion matrix for trial 1}
\NormalTok{actual }\OtherTok{\textless{}{-}} \FunctionTok{factor}\NormalTok{(reference, }\AttributeTok{levels =} \FunctionTok{c}\NormalTok{(}\StringTok{"Accept"}\NormalTok{, }\StringTok{"Reject"}\NormalTok{))}
\NormalTok{predicted }\OtherTok{\textless{}{-}} \FunctionTok{factor}\NormalTok{(op1\_t1, }\AttributeTok{levels =} \FunctionTok{c}\NormalTok{(}\StringTok{"Accept"}\NormalTok{, }\StringTok{"Reject"}\NormalTok{))}

\NormalTok{confusion }\OtherTok{\textless{}{-}} \FunctionTok{table}\NormalTok{(}\AttributeTok{Predicted =}\NormalTok{ predicted, }\AttributeTok{Actual =}\NormalTok{ actual)}
\FunctionTok{print}\NormalTok{(confusion)}
\end{Highlighting}
\end{Shaded}

\begin{verbatim}
##          Actual
## Predicted Accept Reject
##    Accept     17      1
##    Reject      3      9
\end{verbatim}

\begin{Shaded}
\begin{Highlighting}[]
\CommentTok{\# Calculate proportions}
\NormalTok{n }\OtherTok{\textless{}{-}} \FunctionTok{sum}\NormalTok{(confusion)}
\NormalTok{p\_o }\OtherTok{\textless{}{-}} \FunctionTok{sum}\NormalTok{(}\FunctionTok{diag}\NormalTok{(confusion)) }\SpecialCharTok{/}\NormalTok{ n  }\CommentTok{\# Observed agreement}

\CommentTok{\# Expected agreement by chance}
\NormalTok{p\_accept }\OtherTok{\textless{}{-}} \FunctionTok{sum}\NormalTok{(confusion[}\DecValTok{1}\NormalTok{,]) }\SpecialCharTok{/}\NormalTok{ n }\SpecialCharTok{*} \FunctionTok{sum}\NormalTok{(confusion[,}\DecValTok{1}\NormalTok{]) }\SpecialCharTok{/}\NormalTok{ n}
\NormalTok{p\_reject }\OtherTok{\textless{}{-}} \FunctionTok{sum}\NormalTok{(confusion[}\DecValTok{2}\NormalTok{,]) }\SpecialCharTok{/}\NormalTok{ n }\SpecialCharTok{*} \FunctionTok{sum}\NormalTok{(confusion[,}\DecValTok{2}\NormalTok{]) }\SpecialCharTok{/}\NormalTok{ n}
\NormalTok{p\_e }\OtherTok{\textless{}{-}}\NormalTok{ p\_accept }\SpecialCharTok{+}\NormalTok{ p\_reject}

\CommentTok{\# Kappa}
\NormalTok{kappa }\OtherTok{\textless{}{-}}\NormalTok{ (p\_o }\SpecialCharTok{{-}}\NormalTok{ p\_e) }\SpecialCharTok{/}\NormalTok{ (}\DecValTok{1} \SpecialCharTok{{-}}\NormalTok{ p\_e)}

\FunctionTok{cat}\NormalTok{(}\StringTok{"}\SpecialCharTok{\textbackslash{}n}\StringTok{Kappa Calculation (Operator 1 vs Reference):}\SpecialCharTok{\textbackslash{}n}\StringTok{"}\NormalTok{)}
\end{Highlighting}
\end{Shaded}

\begin{verbatim}
## 
## Kappa Calculation (Operator 1 vs Reference):
\end{verbatim}

\begin{Shaded}
\begin{Highlighting}[]
\FunctionTok{cat}\NormalTok{(}\StringTok{"Observed agreement (Po):"}\NormalTok{, }\FunctionTok{round}\NormalTok{(p\_o, }\DecValTok{3}\NormalTok{), }\StringTok{"}\SpecialCharTok{\textbackslash{}n}\StringTok{"}\NormalTok{)}
\end{Highlighting}
\end{Shaded}

\begin{verbatim}
## Observed agreement (Po): 0.867
\end{verbatim}

\begin{Shaded}
\begin{Highlighting}[]
\FunctionTok{cat}\NormalTok{(}\StringTok{"Expected agreement (Pe):"}\NormalTok{, }\FunctionTok{round}\NormalTok{(p\_e, }\DecValTok{3}\NormalTok{), }\StringTok{"}\SpecialCharTok{\textbackslash{}n}\StringTok{"}\NormalTok{)}
\end{Highlighting}
\end{Shaded}

\begin{verbatim}
## Expected agreement (Pe): 0.533
\end{verbatim}

\begin{Shaded}
\begin{Highlighting}[]
\FunctionTok{cat}\NormalTok{(}\StringTok{"Kappa:"}\NormalTok{, }\FunctionTok{round}\NormalTok{(kappa, }\DecValTok{3}\NormalTok{), }\StringTok{"}\SpecialCharTok{\textbackslash{}n}\StringTok{"}\NormalTok{)}
\end{Highlighting}
\end{Shaded}

\begin{verbatim}
## Kappa: 0.714
\end{verbatim}

\begin{Shaded}
\begin{Highlighting}[]
\CommentTok{\# Interpretation}
\ControlFlowTok{if}\NormalTok{(kappa }\SpecialCharTok{\textgreater{}=} \FloatTok{0.81}\NormalTok{) \{}
  \FunctionTok{cat}\NormalTok{(}\StringTok{"Interpretation: Almost perfect agreement}\SpecialCharTok{\textbackslash{}n}\StringTok{"}\NormalTok{)}
\NormalTok{\} }\ControlFlowTok{else} \ControlFlowTok{if}\NormalTok{(kappa }\SpecialCharTok{\textgreater{}=} \FloatTok{0.61}\NormalTok{) \{}
  \FunctionTok{cat}\NormalTok{(}\StringTok{"Interpretation: Substantial agreement}\SpecialCharTok{\textbackslash{}n}\StringTok{"}\NormalTok{)}
\NormalTok{\} }\ControlFlowTok{else} \ControlFlowTok{if}\NormalTok{(kappa }\SpecialCharTok{\textgreater{}=} \FloatTok{0.41}\NormalTok{) \{}
  \FunctionTok{cat}\NormalTok{(}\StringTok{"Interpretation: Moderate agreement}\SpecialCharTok{\textbackslash{}n}\StringTok{"}\NormalTok{)}
\NormalTok{\} }\ControlFlowTok{else}\NormalTok{ \{}
  \FunctionTok{cat}\NormalTok{(}\StringTok{"Interpretation: Fair or less agreement {-} needs improvement}\SpecialCharTok{\textbackslash{}n}\StringTok{"}\NormalTok{)}
\NormalTok{\}}
\end{Highlighting}
\end{Shaded}

\begin{verbatim}
## Interpretation: Substantial agreement
\end{verbatim}

\subsection{Attribute MSA Best Practices}\label{attribute-msa-best-practices}

\begin{table}
\centering
\caption{\label{tab:attribute-best-practices}Best Practices for Attribute MSA Studies}
\centering
\begin{tabu} to \linewidth {>{\raggedright}X>{\raggedright}X}
\hline
Practice & Description\\
\hline
\textbf{Use boundary samples} & Include samples at the accept/reject boundary where decisions are hardest\\
\hline
\textbf{Include clear accept/reject} & Include obvious accept and obvious reject samples as controls\\
\hline
\textbf{Blind testing} & Operators should not know which samples are repeated or reference\\
\hline
\textbf{Multiple trials} & Minimum 2-3 trials per operator to assess within-operator repeatability\\
\hline
\textbf{Reference standard} & Must have known correct answers to assess accuracy, not just agreement\\
\hline
\textbf{Adequate sample size} & Minimum 30-50 samples recommended for statistical validity\\
\hline
\end{tabu}
\end{table}

\begin{center}\rule{0.5\linewidth}{0.5pt}\end{center}

\section{Measurement System Resolution}\label{measurement-system-resolution}

\textbf{Resolution} (also called discrimination) is the smallest increment a measurement system can detect.

\subsection{Rule of Ten}\label{rule-of-ten}

The \textbf{Rule of Ten} states that measurement resolution should be at least 1/10 of:
- The tolerance, OR
- The process variation (6σ)

Whichever is smaller.

\begin{Shaded}
\begin{Highlighting}[]
\CommentTok{\# Resolution adequacy check}
\NormalTok{tolerance }\OtherTok{\textless{}{-}} \FloatTok{0.100}  \CommentTok{\# mm total tolerance}
\NormalTok{process\_6sigma }\OtherTok{\textless{}{-}} \FloatTok{0.080}  \CommentTok{\# 6 sigma of process}
\NormalTok{gauge\_resolution }\OtherTok{\textless{}{-}} \FloatTok{0.010}  \CommentTok{\# mm (smallest increment)}

\CommentTok{\# Check against tolerance}
\NormalTok{ratio\_tolerance }\OtherTok{\textless{}{-}}\NormalTok{ tolerance }\SpecialCharTok{/}\NormalTok{ gauge\_resolution}
\CommentTok{\# Check against process}
\NormalTok{ratio\_process }\OtherTok{\textless{}{-}}\NormalTok{ process\_6sigma }\SpecialCharTok{/}\NormalTok{ gauge\_resolution}

\FunctionTok{cat}\NormalTok{(}\StringTok{"=== Resolution Adequacy Check ===}\SpecialCharTok{\textbackslash{}n\textbackslash{}n}\StringTok{"}\NormalTok{)}
\end{Highlighting}
\end{Shaded}

\begin{verbatim}
## === Resolution Adequacy Check ===
\end{verbatim}

\begin{Shaded}
\begin{Highlighting}[]
\FunctionTok{cat}\NormalTok{(}\StringTok{"Tolerance:"}\NormalTok{, tolerance, }\StringTok{"mm}\SpecialCharTok{\textbackslash{}n}\StringTok{"}\NormalTok{)}
\end{Highlighting}
\end{Shaded}

\begin{verbatim}
## Tolerance: 0.1 mm
\end{verbatim}

\begin{Shaded}
\begin{Highlighting}[]
\FunctionTok{cat}\NormalTok{(}\StringTok{"Process 6σ:"}\NormalTok{, process\_6sigma, }\StringTok{"mm}\SpecialCharTok{\textbackslash{}n}\StringTok{"}\NormalTok{)}
\end{Highlighting}
\end{Shaded}

\begin{verbatim}
## Process 6σ: 0.08 mm
\end{verbatim}

\begin{Shaded}
\begin{Highlighting}[]
\FunctionTok{cat}\NormalTok{(}\StringTok{"Gauge Resolution:"}\NormalTok{, gauge\_resolution, }\StringTok{"mm}\SpecialCharTok{\textbackslash{}n\textbackslash{}n}\StringTok{"}\NormalTok{)}
\end{Highlighting}
\end{Shaded}

\begin{verbatim}
## Gauge Resolution: 0.01 mm
\end{verbatim}

\begin{Shaded}
\begin{Highlighting}[]
\FunctionTok{cat}\NormalTok{(}\StringTok{"Tolerance / Resolution:"}\NormalTok{, ratio\_tolerance, }\StringTok{":1"}\NormalTok{)}
\end{Highlighting}
\end{Shaded}

\begin{verbatim}
## Tolerance / Resolution: 10 :1
\end{verbatim}

\begin{Shaded}
\begin{Highlighting}[]
\ControlFlowTok{if}\NormalTok{(ratio\_tolerance }\SpecialCharTok{\textgreater{}=} \DecValTok{10}\NormalTok{) }\FunctionTok{cat}\NormalTok{(}\StringTok{" {-} ADEQUATE}\SpecialCharTok{\textbackslash{}n}\StringTok{"}\NormalTok{) }\ControlFlowTok{else} \FunctionTok{cat}\NormalTok{(}\StringTok{" {-} INADEQUATE}\SpecialCharTok{\textbackslash{}n}\StringTok{"}\NormalTok{)}
\end{Highlighting}
\end{Shaded}

\begin{verbatim}
##  - ADEQUATE
\end{verbatim}

\begin{Shaded}
\begin{Highlighting}[]
\FunctionTok{cat}\NormalTok{(}\StringTok{"Process 6σ / Resolution:"}\NormalTok{, ratio\_process, }\StringTok{":1"}\NormalTok{)}
\end{Highlighting}
\end{Shaded}

\begin{verbatim}
## Process 6σ / Resolution: 8 :1
\end{verbatim}

\begin{Shaded}
\begin{Highlighting}[]
\ControlFlowTok{if}\NormalTok{(ratio\_process }\SpecialCharTok{\textgreater{}=} \DecValTok{10}\NormalTok{) }\FunctionTok{cat}\NormalTok{(}\StringTok{" {-} ADEQUATE}\SpecialCharTok{\textbackslash{}n}\StringTok{"}\NormalTok{) }\ControlFlowTok{else} \FunctionTok{cat}\NormalTok{(}\StringTok{" {-} INADEQUATE}\SpecialCharTok{\textbackslash{}n}\StringTok{"}\NormalTok{)}
\end{Highlighting}
\end{Shaded}

\begin{verbatim}
##  - INADEQUATE
\end{verbatim}

\subsection{Resolution Impact on Gauge R\&R}\label{resolution-impact-on-gauge-rr}

\pandocbounded{\includegraphics[keepaspectratio]{introduction_files/figure-latex/resolution-grr-impact-1.pdf}}

\begin{center}\rule{0.5\linewidth}{0.5pt}\end{center}

\section{Common Sources of Measurement Error}\label{common-sources-of-measurement-error}

\pandocbounded{\includegraphics[keepaspectratio]{introduction_files/figure-latex/error-sources-1.pdf}}

\begin{table}
\centering
\caption{\label{tab:error-table}MSA Issues and Improvement Actions}
\centering
\begin{tabu} to \linewidth {>{\raggedright}X>{\raggedright}X>{\raggedright}X}
\hline
Issue Identified & Likely Causes & Improvement Actions\\
\hline
\textbf{High Repeatability (EV)} & Gauge condition, fixture, resolution, part variation during measurement & Maintain gauge, improve fixture, upgrade resolution, reduce part handling\\
\hline
\textbf{High Reproducibility (AV)} & Training, technique, procedure clarity & Standardize technique, retrain, improve written procedures\\
\hline
\textbf{Significant Bias} & Calibration, master accuracy, technique & Recalibrate, verify master, standardize technique\\
\hline
\textbf{Poor Linearity} & Gauge mechanism, calibration across range & Repair/replace gauge, multi-point calibration\\
\hline
\textbf{Drift/Stability Issues} & Environmental effects, gauge wear, master deterioration & Environmental control, maintenance, replace masters\\
\hline
\textbf{Low ndc} & Resolution, excessive gauge variation relative to part variation & Higher resolution gauge, reduce gauge variation\\
\hline
\end{tabu}
\end{table}

\begin{center}\rule{0.5\linewidth}{0.5pt}\end{center}

\section{MSA and SPC Relationship}\label{msa-and-spc-relationship}

Poor measurement systems directly impact SPC effectiveness.

\subsection{Effect on Control Charts}\label{effect-on-control-charts}

\pandocbounded{\includegraphics[keepaspectratio]{introduction_files/figure-latex/msa-spc-effect-1.pdf}}

\subsection{Effect on Capability Studies}\label{effect-on-capability-studies}

\begin{Shaded}
\begin{Highlighting}[]
\CommentTok{\# True process parameters}
\NormalTok{true\_mean }\OtherTok{\textless{}{-}} \DecValTok{100}
\NormalTok{true\_sigma }\OtherTok{\textless{}{-}} \FloatTok{2.0}
\NormalTok{USL }\OtherTok{\textless{}{-}} \DecValTok{106}
\NormalTok{LSL }\OtherTok{\textless{}{-}} \DecValTok{94}
\NormalTok{tolerance }\OtherTok{\textless{}{-}}\NormalTok{ USL }\SpecialCharTok{{-}}\NormalTok{ LSL}

\CommentTok{\# True capability}
\NormalTok{true\_Cp }\OtherTok{\textless{}{-}}\NormalTok{ tolerance }\SpecialCharTok{/}\NormalTok{ (}\DecValTok{6} \SpecialCharTok{*}\NormalTok{ true\_sigma)}
\NormalTok{true\_Cpk }\OtherTok{\textless{}{-}} \FunctionTok{min}\NormalTok{((USL }\SpecialCharTok{{-}}\NormalTok{ true\_mean), (true\_mean }\SpecialCharTok{{-}}\NormalTok{ LSL)) }\SpecialCharTok{/}\NormalTok{ (}\DecValTok{3} \SpecialCharTok{*}\NormalTok{ true\_sigma)}

\FunctionTok{cat}\NormalTok{(}\StringTok{"=== True Process Capability ===}\SpecialCharTok{\textbackslash{}n}\StringTok{"}\NormalTok{)}
\end{Highlighting}
\end{Shaded}

\begin{verbatim}
## === True Process Capability ===
\end{verbatim}

\begin{Shaded}
\begin{Highlighting}[]
\FunctionTok{cat}\NormalTok{(}\StringTok{"True σ:"}\NormalTok{, true\_sigma, }\StringTok{"}\SpecialCharTok{\textbackslash{}n}\StringTok{"}\NormalTok{)}
\end{Highlighting}
\end{Shaded}

\begin{verbatim}
## True σ: 2
\end{verbatim}

\begin{Shaded}
\begin{Highlighting}[]
\FunctionTok{cat}\NormalTok{(}\StringTok{"True Cp:"}\NormalTok{, }\FunctionTok{round}\NormalTok{(true\_Cp, }\DecValTok{2}\NormalTok{), }\StringTok{"}\SpecialCharTok{\textbackslash{}n}\StringTok{"}\NormalTok{)}
\end{Highlighting}
\end{Shaded}

\begin{verbatim}
## True Cp: 1
\end{verbatim}

\begin{Shaded}
\begin{Highlighting}[]
\FunctionTok{cat}\NormalTok{(}\StringTok{"True Cpk:"}\NormalTok{, }\FunctionTok{round}\NormalTok{(true\_Cpk, }\DecValTok{2}\NormalTok{), }\StringTok{"}\SpecialCharTok{\textbackslash{}n\textbackslash{}n}\StringTok{"}\NormalTok{)}
\end{Highlighting}
\end{Shaded}

\begin{verbatim}
## True Cpk: 1
\end{verbatim}

\begin{Shaded}
\begin{Highlighting}[]
\CommentTok{\# With good measurement system (10\% GRR)}
\NormalTok{grr\_good }\OtherTok{\textless{}{-}} \FloatTok{0.10} \SpecialCharTok{*}\NormalTok{ tolerance }\SpecialCharTok{/} \FloatTok{5.15}
\NormalTok{observed\_sigma\_good }\OtherTok{\textless{}{-}} \FunctionTok{sqrt}\NormalTok{(true\_sigma}\SpecialCharTok{\^{}}\DecValTok{2} \SpecialCharTok{+}\NormalTok{ grr\_good}\SpecialCharTok{\^{}}\DecValTok{2}\NormalTok{)}
\NormalTok{observed\_Cp\_good }\OtherTok{\textless{}{-}}\NormalTok{ tolerance }\SpecialCharTok{/}\NormalTok{ (}\DecValTok{6} \SpecialCharTok{*}\NormalTok{ observed\_sigma\_good)}
\NormalTok{observed\_Cpk\_good }\OtherTok{\textless{}{-}} \FunctionTok{min}\NormalTok{((USL }\SpecialCharTok{{-}}\NormalTok{ true\_mean), (true\_mean }\SpecialCharTok{{-}}\NormalTok{ LSL)) }\SpecialCharTok{/}\NormalTok{ (}\DecValTok{3} \SpecialCharTok{*}\NormalTok{ observed\_sigma\_good)}

\FunctionTok{cat}\NormalTok{(}\StringTok{"=== With Good MSA (10\% GRR) ===}\SpecialCharTok{\textbackslash{}n}\StringTok{"}\NormalTok{)}
\end{Highlighting}
\end{Shaded}

\begin{verbatim}
## === With Good MSA (10% GRR) ===
\end{verbatim}

\begin{Shaded}
\begin{Highlighting}[]
\FunctionTok{cat}\NormalTok{(}\StringTok{"Observed σ:"}\NormalTok{, }\FunctionTok{round}\NormalTok{(observed\_sigma\_good, }\DecValTok{3}\NormalTok{), }\StringTok{"}\SpecialCharTok{\textbackslash{}n}\StringTok{"}\NormalTok{)}
\end{Highlighting}
\end{Shaded}

\begin{verbatim}
## Observed σ: 2.014
\end{verbatim}

\begin{Shaded}
\begin{Highlighting}[]
\FunctionTok{cat}\NormalTok{(}\StringTok{"Observed Cp:"}\NormalTok{, }\FunctionTok{round}\NormalTok{(observed\_Cp\_good, }\DecValTok{2}\NormalTok{), }\StringTok{"}\SpecialCharTok{\textbackslash{}n}\StringTok{"}\NormalTok{)}
\end{Highlighting}
\end{Shaded}

\begin{verbatim}
## Observed Cp: 0.99
\end{verbatim}

\begin{Shaded}
\begin{Highlighting}[]
\FunctionTok{cat}\NormalTok{(}\StringTok{"Observed Cpk:"}\NormalTok{, }\FunctionTok{round}\NormalTok{(observed\_Cpk\_good, }\DecValTok{2}\NormalTok{), }\StringTok{"}\SpecialCharTok{\textbackslash{}n\textbackslash{}n}\StringTok{"}\NormalTok{)}
\end{Highlighting}
\end{Shaded}

\begin{verbatim}
## Observed Cpk: 0.99
\end{verbatim}

\begin{Shaded}
\begin{Highlighting}[]
\CommentTok{\# With poor measurement system (50\% GRR)}
\NormalTok{grr\_poor }\OtherTok{\textless{}{-}} \FloatTok{0.50} \SpecialCharTok{*}\NormalTok{ tolerance }\SpecialCharTok{/} \FloatTok{5.15}
\NormalTok{observed\_sigma\_poor }\OtherTok{\textless{}{-}} \FunctionTok{sqrt}\NormalTok{(true\_sigma}\SpecialCharTok{\^{}}\DecValTok{2} \SpecialCharTok{+}\NormalTok{ grr\_poor}\SpecialCharTok{\^{}}\DecValTok{2}\NormalTok{)}
\NormalTok{observed\_Cp\_poor }\OtherTok{\textless{}{-}}\NormalTok{ tolerance }\SpecialCharTok{/}\NormalTok{ (}\DecValTok{6} \SpecialCharTok{*}\NormalTok{ observed\_sigma\_poor)}
\NormalTok{observed\_Cpk\_poor }\OtherTok{\textless{}{-}} \FunctionTok{min}\NormalTok{((USL }\SpecialCharTok{{-}}\NormalTok{ true\_mean), (true\_mean }\SpecialCharTok{{-}}\NormalTok{ LSL)) }\SpecialCharTok{/}\NormalTok{ (}\DecValTok{3} \SpecialCharTok{*}\NormalTok{ observed\_sigma\_poor)}

\FunctionTok{cat}\NormalTok{(}\StringTok{"=== With Poor MSA (50\% GRR) ===}\SpecialCharTok{\textbackslash{}n}\StringTok{"}\NormalTok{)}
\end{Highlighting}
\end{Shaded}

\begin{verbatim}
## === With Poor MSA (50% GRR) ===
\end{verbatim}

\begin{Shaded}
\begin{Highlighting}[]
\FunctionTok{cat}\NormalTok{(}\StringTok{"Observed σ:"}\NormalTok{, }\FunctionTok{round}\NormalTok{(observed\_sigma\_poor, }\DecValTok{3}\NormalTok{), }\StringTok{"}\SpecialCharTok{\textbackslash{}n}\StringTok{"}\NormalTok{)}
\end{Highlighting}
\end{Shaded}

\begin{verbatim}
## Observed σ: 2.315
\end{verbatim}

\begin{Shaded}
\begin{Highlighting}[]
\FunctionTok{cat}\NormalTok{(}\StringTok{"Observed Cp:"}\NormalTok{, }\FunctionTok{round}\NormalTok{(observed\_Cp\_poor, }\DecValTok{2}\NormalTok{), }\StringTok{"}\SpecialCharTok{\textbackslash{}n}\StringTok{"}\NormalTok{)}
\end{Highlighting}
\end{Shaded}

\begin{verbatim}
## Observed Cp: 0.86
\end{verbatim}

\begin{Shaded}
\begin{Highlighting}[]
\FunctionTok{cat}\NormalTok{(}\StringTok{"Observed Cpk:"}\NormalTok{, }\FunctionTok{round}\NormalTok{(observed\_Cpk\_poor, }\DecValTok{2}\NormalTok{), }\StringTok{"}\SpecialCharTok{\textbackslash{}n\textbackslash{}n}\StringTok{"}\NormalTok{)}
\end{Highlighting}
\end{Shaded}

\begin{verbatim}
## Observed Cpk: 0.86
\end{verbatim}

\begin{Shaded}
\begin{Highlighting}[]
\FunctionTok{cat}\NormalTok{(}\StringTok{"Poor MSA makes a capable process (Cpk=1.0) appear incapable (Cpk="}\NormalTok{,}
    \FunctionTok{round}\NormalTok{(observed\_Cpk\_poor, }\DecValTok{2}\NormalTok{), }\StringTok{")}\SpecialCharTok{\textbackslash{}n}\StringTok{"}\NormalTok{)}
\end{Highlighting}
\end{Shaded}

\begin{verbatim}
## Poor MSA makes a capable process (Cpk=1.0) appear incapable (Cpk= 0.86 )
\end{verbatim}

\begin{center}\rule{0.5\linewidth}{0.5pt}\end{center}

\section{Conducting an MSA Study: Step-by-Step}\label{conducting-an-msa-study-step-by-step}

\pandocbounded{\includegraphics[keepaspectratio]{introduction_files/figure-latex/msa-flowchart-1.pdf}}

\subsection{Data Collection Sheet Template}\label{data-collection-sheet-template}

\begin{table}
\centering
\caption{\label{tab:data-collection-template}Gauge R&R Data Collection Sheet (Partial Example)}
\centering
\begin{tabular}[t]{>{\raggedleft\arraybackslash}p{20%}|>{\raggedright\arraybackslash}p{20%}|>{\raggedleft\arraybackslash}p{20%}|>{\raggedright\arraybackslash}p{20%}}
\hline
Part & Operator & Trial & Measurement\\
\hline
1 & A & 1 & \_\_\_\\
\hline
1 & A & 2 & \_\_\_\\
\hline
1 & B & 1 & \_\_\_\\
\hline
1 & B & 2 & \_\_\_\\
\hline
1 & C & 1 & \_\_\_\\
\hline
1 & C & 2 & \_\_\_\\
\hline
2 & A & 1 & \_\_\_\\
\hline
2 & A & 2 & \_\_\_\\
\hline
2 & B & 1 & \_\_\_\\
\hline
2 & B & 2 & \_\_\_\\
\hline
2 & C & 1 & \_\_\_\\
\hline
2 & C & 2 & \_\_\_\\
\hline
\end{tabular}
\end{table}

\begin{center}\rule{0.5\linewidth}{0.5pt}\end{center}

\section{Video Resources}\label{video-resources-1}

\subsection{Understanding Gauge R\&R}\label{understanding-gauge-rr}

\subsection{MSA in Practice}\label{msa-in-practice}

\begin{center}\rule{0.5\linewidth}{0.5pt}\end{center}

\section{Summary}\label{summary-10}

Measurement Systems Analysis ensures that your data can be trusted for process control and decision-making:

\begin{enumerate}
\def\labelenumi{\arabic{enumi}.}
\tightlist
\item
  \textbf{Before SPC, do MSA} - Garbage in, garbage out; verify measurement quality first
\item
  \textbf{Accuracy components} - Bias, linearity, and stability affect location (average)
\item
  \textbf{Precision components} - Repeatability (within operator) and reproducibility (between operators)
\item
  \textbf{Key metrics} - \%GRR \textless{} 30\% and ndc ≥ 5 for acceptable systems
\item
  \textbf{Resolution matters} - 10:1 ratio vs.~tolerance or process variation
\item
  \textbf{Attribute MSA} - Use Kappa for pass/fail decisions
\item
  \textbf{Continuous improvement} - Regular verification and recertification
\end{enumerate}

\begin{quote}
``The measurement system is part of the process. Fix the measurement system before trying to fix the process.''
\end{quote}

\begin{center}\rule{0.5\linewidth}{0.5pt}\end{center}

\section{Review Questions}\label{review-questions-11}

\textbf{Question 1}: A gauge R\&R study yields \%GRR = 45\% and ndc = 2. What does this mean and what should be done?

\textbf{Answer:}

This measurement system is \textbf{unacceptable}:

\begin{itemize}
\tightlist
\item
  \textbf{\%GRR = 45\%} exceeds the 30\% maximum threshold, meaning 45\% of the observed variation is due to the measurement system, not the parts
\item
  \textbf{ndc = 2} indicates the gauge can only distinguish between 2 categories of parts, which is insufficient for process control (minimum 5 required)
\end{itemize}

\textbf{Actions required:}
1. Investigate root causes of high variation:
- Check gauge calibration and condition
- Evaluate operator technique and training
- Review measurement procedure for ambiguity
- Assess fixture and part positioning
- Check gauge resolution adequacy

\begin{enumerate}
\def\labelenumi{\arabic{enumi}.}
\setcounter{enumi}{1}
\item
  Prioritize improvement efforts:

  \begin{itemize}
  \tightlist
  \item
    If repeatability (EV) is high: Focus on gauge condition, fixture, technique consistency
  \item
    If reproducibility (AV) is high: Focus on training, procedure clarity, technique standardization
  \end{itemize}
\item
  Re-run study after improvements
\item
  If improvements insufficient, consider upgrading to a higher-capability gauge
\item
  Until resolved, this gauge should not be used for process control or capability studies
\end{enumerate}

\textbf{Question 2}: What is the difference between repeatability and reproducibility? Give an example of each source of error.

\textbf{Answer:}

\textbf{Repeatability (Equipment Variation - EV)}:
- Variation when the \textbf{same operator} measures the \textbf{same part} multiple times with the \textbf{same gauge}
- Also called ``within-operator'' variation
- Represents the inherent precision of the gauge and measurement technique

\emph{Examples of repeatability errors:}
- Gauge resolution limitations
- Gauge mechanism play/backlash
- Inconsistent part positioning in fixture
- Variations in technique even by same operator
- Environmental micro-changes during measurement

\textbf{Reproducibility (Appraiser Variation - AV)}:
- Variation when \textbf{different operators} measure the \textbf{same parts}
- Also called ``between-operator'' variation
- Represents differences in how operators apply the measurement method

\emph{Examples of reproducibility errors:}
- Different technique between operators (pressure applied, angle, etc.)
- Different interpretation of measurement procedure
- Different reading of analog scales
- Training differences
- Physical differences (eyesight, hand steadiness)

\textbf{Question 3}: Calculate the bias and determine if it is acceptable. Reference value = 25.000 mm, Tolerance = ±0.050 mm, Measured values: 25.012, 25.015, 25.008, 25.011, 25.014, 25.009, 25.013, 25.010, 25.012, 25.011

\textbf{Answer:}

\begin{Shaded}
\begin{Highlighting}[]
\NormalTok{reference }\OtherTok{\textless{}{-}} \FloatTok{25.000}
\NormalTok{tolerance\_total }\OtherTok{\textless{}{-}} \FloatTok{0.100}  \CommentTok{\# ±0.050 = 0.100 total}
\NormalTok{measurements }\OtherTok{\textless{}{-}} \FunctionTok{c}\NormalTok{(}\FloatTok{25.012}\NormalTok{, }\FloatTok{25.015}\NormalTok{, }\FloatTok{25.008}\NormalTok{, }\FloatTok{25.011}\NormalTok{, }\FloatTok{25.014}\NormalTok{,}
                  \FloatTok{25.009}\NormalTok{, }\FloatTok{25.013}\NormalTok{, }\FloatTok{25.010}\NormalTok{, }\FloatTok{25.012}\NormalTok{, }\FloatTok{25.011}\NormalTok{)}

\CommentTok{\# Calculate bias}
\NormalTok{mean\_measured }\OtherTok{\textless{}{-}} \FunctionTok{mean}\NormalTok{(measurements)}
\NormalTok{bias }\OtherTok{\textless{}{-}}\NormalTok{ mean\_measured }\SpecialCharTok{{-}}\NormalTok{ reference}

\FunctionTok{cat}\NormalTok{(}\StringTok{"Reference value:"}\NormalTok{, reference, }\StringTok{"mm}\SpecialCharTok{\textbackslash{}n}\StringTok{"}\NormalTok{)}
\end{Highlighting}
\end{Shaded}

\begin{verbatim}
## Reference value: 25 mm
\end{verbatim}

\begin{Shaded}
\begin{Highlighting}[]
\FunctionTok{cat}\NormalTok{(}\StringTok{"Mean measured:"}\NormalTok{, }\FunctionTok{round}\NormalTok{(mean\_measured, }\DecValTok{4}\NormalTok{), }\StringTok{"mm}\SpecialCharTok{\textbackslash{}n}\StringTok{"}\NormalTok{)}
\end{Highlighting}
\end{Shaded}

\begin{verbatim}
## Mean measured: 25.0115 mm
\end{verbatim}

\begin{Shaded}
\begin{Highlighting}[]
\FunctionTok{cat}\NormalTok{(}\StringTok{"Bias:"}\NormalTok{, }\FunctionTok{round}\NormalTok{(bias, }\DecValTok{4}\NormalTok{), }\StringTok{"mm}\SpecialCharTok{\textbackslash{}n\textbackslash{}n}\StringTok{"}\NormalTok{)}
\end{Highlighting}
\end{Shaded}

\begin{verbatim}
## Bias: 0.0115 mm
\end{verbatim}

\begin{Shaded}
\begin{Highlighting}[]
\CommentTok{\# Express as \% of tolerance}
\NormalTok{bias\_pct }\OtherTok{\textless{}{-}} \FunctionTok{abs}\NormalTok{(bias) }\SpecialCharTok{/}\NormalTok{ tolerance\_total }\SpecialCharTok{*} \DecValTok{100}
\FunctionTok{cat}\NormalTok{(}\StringTok{"Bias as \% of tolerance:"}\NormalTok{, }\FunctionTok{round}\NormalTok{(bias\_pct, }\DecValTok{1}\NormalTok{), }\StringTok{"\%}\SpecialCharTok{\textbackslash{}n}\StringTok{"}\NormalTok{)}
\end{Highlighting}
\end{Shaded}

\begin{verbatim}
## Bias as % of tolerance: 11.5 %
\end{verbatim}

\begin{Shaded}
\begin{Highlighting}[]
\CommentTok{\# T{-}test for significance}
\NormalTok{t\_result }\OtherTok{\textless{}{-}} \FunctionTok{t.test}\NormalTok{(measurements, }\AttributeTok{mu =}\NormalTok{ reference)}
\FunctionTok{cat}\NormalTok{(}\StringTok{"T{-}test p{-}value:"}\NormalTok{, }\FunctionTok{round}\NormalTok{(t\_result}\SpecialCharTok{$}\NormalTok{p.value, }\DecValTok{6}\NormalTok{), }\StringTok{"}\SpecialCharTok{\textbackslash{}n\textbackslash{}n}\StringTok{"}\NormalTok{)}
\end{Highlighting}
\end{Shaded}

\begin{verbatim}
## T-test p-value: 0
\end{verbatim}

\begin{Shaded}
\begin{Highlighting}[]
\CommentTok{\# Acceptability}
\FunctionTok{cat}\NormalTok{(}\StringTok{"Acceptability Assessment:}\SpecialCharTok{\textbackslash{}n}\StringTok{"}\NormalTok{)}
\end{Highlighting}
\end{Shaded}

\begin{verbatim}
## Acceptability Assessment:
\end{verbatim}

\begin{Shaded}
\begin{Highlighting}[]
\ControlFlowTok{if}\NormalTok{(bias\_pct }\SpecialCharTok{\textless{}} \DecValTok{10}\NormalTok{) \{}
  \FunctionTok{cat}\NormalTok{(}\StringTok{"{-} Bias \textless{} 10\% of tolerance: ACCEPTABLE}\SpecialCharTok{\textbackslash{}n}\StringTok{"}\NormalTok{)}
\NormalTok{\} }\ControlFlowTok{else} \ControlFlowTok{if}\NormalTok{(bias\_pct }\SpecialCharTok{\textless{}} \DecValTok{25}\NormalTok{) \{}
  \FunctionTok{cat}\NormalTok{(}\StringTok{"{-} Bias 10{-}25\% of tolerance: MARGINAL}\SpecialCharTok{\textbackslash{}n}\StringTok{"}\NormalTok{)}
\NormalTok{\} }\ControlFlowTok{else}\NormalTok{ \{}
  \FunctionTok{cat}\NormalTok{(}\StringTok{"{-} Bias \textgreater{} 25\% of tolerance: UNACCEPTABLE}\SpecialCharTok{\textbackslash{}n}\StringTok{"}\NormalTok{)}
\NormalTok{\}}
\end{Highlighting}
\end{Shaded}

\begin{verbatim}
## - Bias 10-25% of tolerance: MARGINAL
\end{verbatim}

\begin{Shaded}
\begin{Highlighting}[]
\ControlFlowTok{if}\NormalTok{(t\_result}\SpecialCharTok{$}\NormalTok{p.value }\SpecialCharTok{\textless{}} \FloatTok{0.05}\NormalTok{) \{}
  \FunctionTok{cat}\NormalTok{(}\StringTok{"{-} Bias is statistically significant (p \textless{} 0.05)}\SpecialCharTok{\textbackslash{}n}\StringTok{"}\NormalTok{)}
  \FunctionTok{cat}\NormalTok{(}\StringTok{"{-} Consider recalibration or technique adjustment}\SpecialCharTok{\textbackslash{}n}\StringTok{"}\NormalTok{)}
\NormalTok{\}}
\end{Highlighting}
\end{Shaded}

\begin{verbatim}
## - Bias is statistically significant (p < 0.05)
## - Consider recalibration or technique adjustment
\end{verbatim}

The bias of 0.0115 mm (11.5\% of tolerance) is \textbf{marginally acceptable} and statistically significant. Recalibration should be considered.

\textbf{Question 4}: Why is it important to conduct MSA before capability studies (Cp, Cpk)?

\textbf{Answer:}

MSA must precede capability studies because measurement error directly inflates observed variation:

\textbf{Mathematical Relationship:}
\[\sigma^2_{observed} = \sigma^2_{process} + \sigma^2_{measurement}\]

\textbf{Impacts on Capability:}

\begin{enumerate}
\def\labelenumi{\arabic{enumi}.}
\item
  \textbf{Understated Cp/Cpk}: Observed sigma is always larger than true process sigma, so calculated capability will be lower than actual capability
\item
  \textbf{False Alarms}: A capable process may appear incapable due to measurement noise, leading to unnecessary process adjustments or investment
\item
  \textbf{Missed Issues}: High measurement variation masks actual process variation; you can't see problems through the ``fog'' of gauge error
\item
  \textbf{Bad Decisions}: Incorrect capability data leads to:

  \begin{itemize}
  \tightlist
  \item
    Wrong process acceptance decisions
  \item
    Inappropriate control limits
  \item
    Misguided improvement investments
  \item
    Customer quality issues
  \end{itemize}
\end{enumerate}

\textbf{Example:}
- True process: Cpk = 1.5 (very capable)
- With 50\% GRR: Observed Cpk ≈ 0.9 (appears incapable)
- Result: Unnecessary process ``improvement'' efforts

\textbf{Rule of Thumb}: If \%GRR \textgreater{} 30\%, any capability study is essentially meaningless because too much of the observed variation is measurement noise, not real process variation.

\textbf{Question 5}: A visual inspection has the following results: Kappa vs.~reference = 0.72, within-operator agreement = 88\%, between-operator agreement = 75\%. Evaluate this attribute measurement system.

\textbf{Answer:}

\textbf{Results Interpretation:}

\begin{enumerate}
\def\labelenumi{\arabic{enumi}.}
\tightlist
\item
  \textbf{Kappa vs.~Reference = 0.72}

  \begin{itemize}
  \tightlist
  \item
    Falls in ``Substantial agreement'' range (0.61-0.80)
  \item
    Operators are correctly classifying parts most of the time
  \item
    Acceptable for many applications, but improvement possible
  \end{itemize}
\item
  \textbf{Within-Operator Agreement = 88\%}

  \begin{itemize}
  \tightlist
  \item
    Operators are fairly consistent with themselves
  \item
    12\% of the time, same operator gives different result on same part
  \item
    Should target \textgreater90\% for critical inspections
  \end{itemize}
\item
  \textbf{Between-Operator Agreement = 75\%}

  \begin{itemize}
  \tightlist
  \item
    Operators disagree on 25\% of parts
  \item
    This is a significant reproducibility issue
  \item
    Major source of inconsistency in product quality
  \end{itemize}
\end{enumerate}

\textbf{Assessment: MARGINAL - Improvement Needed}

\textbf{Recommended Actions:}

\begin{enumerate}
\def\labelenumi{\arabic{enumi}.}
\item
  \textbf{Standardize criteria}: Create clear visual standards with boundary samples

  \begin{itemize}
  \tightlist
  \item
    Photos of accept/reject borderline cases
  \item
    Written descriptions of defect criteria
  \end{itemize}
\item
  \textbf{Training}:

  \begin{itemize}
  \tightlist
  \item
    Calibration session with all operators and reference samples
  \item
    Identify operators with lower agreement for targeted training
  \end{itemize}
\item
  \textbf{Improve conditions}:

  \begin{itemize}
  \tightlist
  \item
    Check lighting adequacy and consistency
  \item
    Ensure proper viewing distance and angle
  \item
    Reduce fatigue with appropriate break schedules
  \end{itemize}
\item
  \textbf{Consider automation}: For critical inspections, automated vision systems may provide better consistency
\item
  \textbf{Re-study after improvements} to verify effectiveness
\end{enumerate}

\textbf{Question 6}: What is the ``Rule of Ten'' for gauge resolution, and how would you apply it?

\textbf{Answer:}

\textbf{The Rule of Ten:}
The gauge resolution should be at least \textbf{1/10 of the tolerance} or \textbf{1/10 of the process variation (6σ)}, whichever is smaller.

\[\text{Resolution} \leq \frac{\min(\text{Tolerance}, 6\sigma)}{10}\]

\textbf{Application Example:}

Given:
- Part tolerance: ±0.05 mm (total = 0.10 mm)
- Process 6σ: 0.08 mm
- Available gauges: 0.01 mm resolution, 0.001 mm resolution

\textbf{Calculation:}

\begin{verbatim}
Required resolution ≤ min(0.10, 0.08) / 10
Required resolution ≤ 0.08 / 10
Required resolution ≤ 0.008 mm
\end{verbatim}

\textbf{Assessment:}
- 0.01 mm gauge: 0.01 \textgreater{} 0.008 → \textbf{INADEQUATE}
- 0.001 mm gauge: 0.001 \textless{} 0.008 → \textbf{ADEQUATE}

\textbf{Why It Matters:}
1. If resolution is too coarse, gauge cannot detect small part differences
2. Results in low ndc (number of distinct categories)
3. Control charts will show ``stair-step'' patterns
4. Capability studies will be inaccurate

\textbf{Practical Tips:}
- For SPC applications, 10:1 is minimum; 20:1 is preferred
- For capability studies, 10:1 is acceptable
- For simple pass/fail gauging, 5:1 may be acceptable

\textbf{Question 7}: Given the following ANOVA table from a Gauge R\&R study, calculate the variance components and \%GRR.

\begin{longtable}[]{@{}llll@{}}
\toprule\noalign{}
Source & DF & SS & MS \\
\midrule\noalign{}
\endhead
\bottomrule\noalign{}
\endlastfoot
Part & 9 & 0.2850 & 0.03167 \\
Operator & 2 & 0.0025 & 0.00125 \\
Part×Operator & 18 & 0.0090 & 0.00050 \\
Repeatability & 60 & 0.0180 & 0.00030 \\
Total & 89 & 0.3145 & \\
\end{longtable}

Study design: 10 parts, 3 operators, 3 trials

\textbf{Answer:}

\begin{Shaded}
\begin{Highlighting}[]
\CommentTok{\# Given values from ANOVA table}
\NormalTok{MS\_part }\OtherTok{\textless{}{-}} \FloatTok{0.03167}
\NormalTok{MS\_operator }\OtherTok{\textless{}{-}} \FloatTok{0.00125}
\NormalTok{MS\_interaction }\OtherTok{\textless{}{-}} \FloatTok{0.00050}
\NormalTok{MS\_error }\OtherTok{\textless{}{-}} \FloatTok{0.00030}

\NormalTok{n\_parts }\OtherTok{\textless{}{-}} \DecValTok{10}
\NormalTok{n\_operators }\OtherTok{\textless{}{-}} \DecValTok{3}
\NormalTok{n\_trials }\OtherTok{\textless{}{-}} \DecValTok{3}

\CommentTok{\# Variance components}
\NormalTok{var\_repeatability }\OtherTok{\textless{}{-}}\NormalTok{ MS\_error}
\FunctionTok{cat}\NormalTok{(}\StringTok{"Var(Repeatability) ="}\NormalTok{, var\_repeatability, }\StringTok{"}\SpecialCharTok{\textbackslash{}n}\StringTok{"}\NormalTok{)}
\end{Highlighting}
\end{Shaded}

\begin{verbatim}
## Var(Repeatability) = 3e-04
\end{verbatim}

\begin{Shaded}
\begin{Highlighting}[]
\NormalTok{var\_interaction }\OtherTok{\textless{}{-}} \FunctionTok{max}\NormalTok{(}\DecValTok{0}\NormalTok{, (MS\_interaction }\SpecialCharTok{{-}}\NormalTok{ MS\_error) }\SpecialCharTok{/}\NormalTok{ n\_trials)}
\FunctionTok{cat}\NormalTok{(}\StringTok{"Var(Interaction) ="}\NormalTok{, }\FunctionTok{round}\NormalTok{(var\_interaction, }\DecValTok{6}\NormalTok{), }\StringTok{"}\SpecialCharTok{\textbackslash{}n}\StringTok{"}\NormalTok{)}
\end{Highlighting}
\end{Shaded}

\begin{verbatim}
## Var(Interaction) = 6.7e-05
\end{verbatim}

\begin{Shaded}
\begin{Highlighting}[]
\NormalTok{var\_operator }\OtherTok{\textless{}{-}} \FunctionTok{max}\NormalTok{(}\DecValTok{0}\NormalTok{, (MS\_operator }\SpecialCharTok{{-}}\NormalTok{ MS\_interaction) }\SpecialCharTok{/}\NormalTok{ (n\_parts }\SpecialCharTok{*}\NormalTok{ n\_trials))}
\FunctionTok{cat}\NormalTok{(}\StringTok{"Var(Operator) ="}\NormalTok{, }\FunctionTok{round}\NormalTok{(var\_operator, }\DecValTok{6}\NormalTok{), }\StringTok{"}\SpecialCharTok{\textbackslash{}n}\StringTok{"}\NormalTok{)}
\end{Highlighting}
\end{Shaded}

\begin{verbatim}
## Var(Operator) = 2.5e-05
\end{verbatim}

\begin{Shaded}
\begin{Highlighting}[]
\NormalTok{var\_part }\OtherTok{\textless{}{-}} \FunctionTok{max}\NormalTok{(}\DecValTok{0}\NormalTok{, (MS\_part }\SpecialCharTok{{-}}\NormalTok{ MS\_interaction) }\SpecialCharTok{/}\NormalTok{ (n\_operators }\SpecialCharTok{*}\NormalTok{ n\_trials))}
\FunctionTok{cat}\NormalTok{(}\StringTok{"Var(Part) ="}\NormalTok{, }\FunctionTok{round}\NormalTok{(var\_part, }\DecValTok{6}\NormalTok{), }\StringTok{"}\SpecialCharTok{\textbackslash{}n\textbackslash{}n}\StringTok{"}\NormalTok{)}
\end{Highlighting}
\end{Shaded}

\begin{verbatim}
## Var(Part) = 0.003463
\end{verbatim}

\begin{Shaded}
\begin{Highlighting}[]
\CommentTok{\# Reproducibility = Operator + Interaction}
\NormalTok{var\_reproducibility }\OtherTok{\textless{}{-}}\NormalTok{ var\_operator }\SpecialCharTok{+}\NormalTok{ var\_interaction}
\FunctionTok{cat}\NormalTok{(}\StringTok{"Var(Reproducibility) ="}\NormalTok{, }\FunctionTok{round}\NormalTok{(var\_reproducibility, }\DecValTok{6}\NormalTok{), }\StringTok{"}\SpecialCharTok{\textbackslash{}n}\StringTok{"}\NormalTok{)}
\end{Highlighting}
\end{Shaded}

\begin{verbatim}
## Var(Reproducibility) = 9.2e-05
\end{verbatim}

\begin{Shaded}
\begin{Highlighting}[]
\CommentTok{\# Total Gauge R\&R}
\NormalTok{var\_GRR }\OtherTok{\textless{}{-}}\NormalTok{ var\_repeatability }\SpecialCharTok{+}\NormalTok{ var\_reproducibility}
\FunctionTok{cat}\NormalTok{(}\StringTok{"Var(GRR) ="}\NormalTok{, }\FunctionTok{round}\NormalTok{(var\_GRR, }\DecValTok{6}\NormalTok{), }\StringTok{"}\SpecialCharTok{\textbackslash{}n}\StringTok{"}\NormalTok{)}
\end{Highlighting}
\end{Shaded}

\begin{verbatim}
## Var(GRR) = 0.000392
\end{verbatim}

\begin{Shaded}
\begin{Highlighting}[]
\CommentTok{\# Total variation}
\NormalTok{var\_total }\OtherTok{\textless{}{-}}\NormalTok{ var\_part }\SpecialCharTok{+}\NormalTok{ var\_GRR}
\FunctionTok{cat}\NormalTok{(}\StringTok{"Var(Total) ="}\NormalTok{, }\FunctionTok{round}\NormalTok{(var\_total, }\DecValTok{6}\NormalTok{), }\StringTok{"}\SpecialCharTok{\textbackslash{}n\textbackslash{}n}\StringTok{"}\NormalTok{)}
\end{Highlighting}
\end{Shaded}

\begin{verbatim}
## Var(Total) = 0.003855
\end{verbatim}

\begin{Shaded}
\begin{Highlighting}[]
\CommentTok{\# Calculate \%GRR (of total variation)}
\NormalTok{sigma\_GRR }\OtherTok{\textless{}{-}} \FunctionTok{sqrt}\NormalTok{(var\_GRR)}
\NormalTok{sigma\_total }\OtherTok{\textless{}{-}} \FunctionTok{sqrt}\NormalTok{(var\_total)}
\NormalTok{pct\_GRR }\OtherTok{\textless{}{-}}\NormalTok{ (sigma\_GRR }\SpecialCharTok{/}\NormalTok{ sigma\_total) }\SpecialCharTok{*} \DecValTok{100}

\FunctionTok{cat}\NormalTok{(}\StringTok{"\%GRR (of Total Variation) ="}\NormalTok{, }\FunctionTok{round}\NormalTok{(pct\_GRR, }\DecValTok{1}\NormalTok{), }\StringTok{"\%}\SpecialCharTok{\textbackslash{}n\textbackslash{}n}\StringTok{"}\NormalTok{)}
\end{Highlighting}
\end{Shaded}

\begin{verbatim}
## %GRR (of Total Variation) = 31.9 %
\end{verbatim}

\begin{Shaded}
\begin{Highlighting}[]
\CommentTok{\# ndc}
\NormalTok{sigma\_part }\OtherTok{\textless{}{-}} \FunctionTok{sqrt}\NormalTok{(var\_part)}
\NormalTok{ndc }\OtherTok{\textless{}{-}} \FloatTok{1.41} \SpecialCharTok{*}\NormalTok{ (sigma\_part }\SpecialCharTok{/}\NormalTok{ sigma\_GRR)}
\FunctionTok{cat}\NormalTok{(}\StringTok{"ndc ="}\NormalTok{, }\FunctionTok{round}\NormalTok{(ndc, }\DecValTok{1}\NormalTok{), }\StringTok{"}\SpecialCharTok{\textbackslash{}n\textbackslash{}n}\StringTok{"}\NormalTok{)}
\end{Highlighting}
\end{Shaded}

\begin{verbatim}
## ndc = 4.2
\end{verbatim}

\begin{Shaded}
\begin{Highlighting}[]
\ControlFlowTok{if}\NormalTok{(pct\_GRR }\SpecialCharTok{\textless{}} \DecValTok{10}\NormalTok{) \{}
  \FunctionTok{cat}\NormalTok{(}\StringTok{"Assessment: ACCEPTABLE (\textless{} 10\%)}\SpecialCharTok{\textbackslash{}n}\StringTok{"}\NormalTok{)}
\NormalTok{\} }\ControlFlowTok{else} \ControlFlowTok{if}\NormalTok{(pct\_GRR }\SpecialCharTok{\textless{}} \DecValTok{30}\NormalTok{) \{}
  \FunctionTok{cat}\NormalTok{(}\StringTok{"Assessment: MARGINAL (10{-}30\%)}\SpecialCharTok{\textbackslash{}n}\StringTok{"}\NormalTok{)}
\NormalTok{\} }\ControlFlowTok{else}\NormalTok{ \{}
  \FunctionTok{cat}\NormalTok{(}\StringTok{"Assessment: UNACCEPTABLE (\textgreater{} 30\%)}\SpecialCharTok{\textbackslash{}n}\StringTok{"}\NormalTok{)}
\NormalTok{\}}
\end{Highlighting}
\end{Shaded}

\begin{verbatim}
## Assessment: UNACCEPTABLE (> 30%)
\end{verbatim}

\textbf{Question 8}: What actions would you take if a Gauge R\&R study showed high repeatability (EV) but low reproducibility (AV)?

\textbf{Answer:}

When \textbf{repeatability is high} (main contributor to GRR) but \textbf{reproducibility is low}, the measurement variation is coming from the gauge/equipment/technique rather than differences between operators.

\textbf{Root Cause Investigation - Focus Areas:}

\begin{enumerate}
\def\labelenumi{\arabic{enumi}.}
\tightlist
\item
  \textbf{Gauge Condition}

  \begin{itemize}
  \tightlist
  \item
    Worn measuring surfaces or contacts
  \item
    Loose components or mechanism play
  \item
    Damaged or dirty probe/sensor
  \end{itemize}
\item
  \textbf{Gauge Resolution}

  \begin{itemize}
  \tightlist
  \item
    Resolution may be inadequate for the tolerance
  \item
    Check 10:1 rule compliance
  \end{itemize}
\item
  \textbf{Fixture/Fixturing}

  \begin{itemize}
  \tightlist
  \item
    Part not held consistently
  \item
    Fixture wear or damage
  \item
    Part not seating properly
  \end{itemize}
\item
  \textbf{Measurement Technique}

  \begin{itemize}
  \tightlist
  \item
    Inconsistent contact force
  \item
    Varying measurement location on part
  \item
    Speed of measurement varies
  \end{itemize}
\item
  \textbf{Part Characteristics}

  \begin{itemize}
  \tightlist
  \item
    Part surface finish affecting readings
  \item
    Part flexibility/deflection during measurement
  \item
    Part temperature variation
  \end{itemize}
\item
  \textbf{Environment}

  \begin{itemize}
  \tightlist
  \item
    Vibration affecting readings
  \item
    Temperature instability
  \item
    Contamination
  \end{itemize}
\end{enumerate}

\textbf{Improvement Actions:}

\begin{longtable}[]{@{}ll@{}}
\toprule\noalign{}
Priority & Action \\
\midrule\noalign{}
\endhead
\bottomrule\noalign{}
\endlastfoot
1 & Verify gauge calibration and condition; repair/replace if needed \\
2 & Check and improve fixture; ensure consistent part positioning \\
3 & Evaluate and potentially upgrade gauge resolution \\
4 & Standardize technique (force, speed, location) \\
5 & Control environmental factors \\
6 & Re-run study to verify improvement \\
\end{longtable}

\textbf{Note:} Low AV in this case is actually good - it means operators are consistent with each other. The problem is the measurement equipment itself.

\begin{center}\rule{0.5\linewidth}{0.5pt}\end{center}

\section{References}\label{references-11}

\begin{enumerate}
\def\labelenumi{\arabic{enumi}.}
\item
  AIAG. (2010). \emph{Measurement Systems Analysis Reference Manual} (4th ed.). Automotive Industry Action Group.
\item
  Wheeler, D.J. (2006). \emph{EMP III: Evaluating the Measurement Process} (3rd ed.). SPC Press.
\item
  Montgomery, D.C. (2019). \emph{Introduction to Statistical Quality Control} (8th ed.). Wiley.
\item
  Burdick, R.K., Borror, C.M., \& Montgomery, D.C. (2005). \emph{Design and Analysis of Gauge R\&R Studies}. SIAM.
\item
  ASTM E2782-17. \emph{Standard Guide for Measurement Systems Analysis}.
\item
  ISO 22514-7:2021. \emph{Statistical Methods in Process Management - Capability and Performance - Part 7: Capability of Measurement Processes}.
\item
  Wheeler, D.J., \& Lyday, R.W. (1989). \emph{Evaluating the Measurement Process} (2nd ed.). SPC Press.
\item
  Minitab. (2021). \emph{Gage R\&R Study (Crossed)}. Minitab Support Documentation.
\item
  JCGM 100:2008. \emph{Evaluation of Measurement Data - Guide to the Expression of Uncertainty in Measurement} (GUM).
\item
  Dietrich, E., \& Schulze, A. (2011). \emph{Statistical Procedures for Machine and Process Qualification} (6th ed.). Hanser.
\end{enumerate}

\chapter{Industrial Automation Fundamentals}\label{industrial-automation-fundamentals}

\section{Learning Objectives}\label{learning-objectives-12}

After completing this chapter, you will be able to:

\begin{enumerate}
\def\labelenumi{\arabic{enumi}.}
\tightlist
\item
  Define industrial automation and explain its role in modern manufacturing
\item
  Distinguish between fixed, programmable, and flexible automation
\item
  Describe the architecture and function of Programmable Logic Controllers (PLCs)
\item
  Identify common industrial sensors and their applications
\item
  Explain motor control fundamentals including VFDs and servo systems
\item
  Understand industrial communication protocols and networking
\item
  Describe the role of HMI and SCADA in process monitoring
\item
  Apply safety principles to automated systems
\item
  Explain Industry 4.0 concepts and the Industrial Internet of Things (IIoT)
\end{enumerate}

\begin{center}\rule{0.5\linewidth}{0.5pt}\end{center}

\section{Introduction to Industrial Automation}\label{introduction-to-industrial-automation}

\textbf{Industrial automation} is the use of control systems, machinery, and information technologies to handle processes and machinery in an industry, replacing human intervention where possible to increase efficiency, quality, and safety.

\subsection{Why Automate?}\label{why-automate}

\pandocbounded{\includegraphics[keepaspectratio]{introduction_files/figure-latex/automation-benefits-1.pdf}}

\subsection{Automation in Key Industries}\label{automation-in-key-industries}

\begin{table}
\centering
\caption{\label{tab:industry-applications}Automation Applications by Industry}
\centering
\begin{tabu} to \linewidth {>{\raggedright}X>{\raggedright}X>{\raggedright}X>{\raggedright}X}
\hline
Industry & Application & Technology & Benefit\\
\hline
\multicolumn{4}{l}{\cellcolor[HTML]{ebf5fb}{\textbf{Automotive}}}\\
\hline
\hspace{1em}\textbf{Automotive} & Robotic welding and assembly & 6-axis robots, PLCs & Consistent weld quality, high speed\\
\hline
\hspace{1em}\textbf{Automotive} & Automated paint systems & Conveyors, spray robots & Uniform coating, reduced VOC\\
\hline
\hspace{1em}\textbf{Automotive} & Vision inspection & Machine vision, AI & 100\% inspection, defect detection\\
\hline
\multicolumn{4}{l}{\cellcolor[HTML]{eafaf1}{\textbf{Food \& Beverage}}}\\
\hline
\hspace{1em}\textbf{Food \& Beverage} & Filling and packaging lines & Servo drives, sensors & High speed, accurate fill weights\\
\hline
\hspace{1em}\textbf{Food \& Beverage} & Pasteurization control & PLCs, temperature control & Food safety, traceability\\
\hline
\hspace{1em}\textbf{Food \& Beverage} & Sorting and grading & Vision systems, conveyors & Quality sorting, reduced labor\\
\hline
\multicolumn{4}{l}{\cellcolor[HTML]{fef9e7}{\textbf{Aerospace/Defense}}}\\
\hline
\hspace{1em}\textbf{Aerospace/Defense} & CNC precision machining & CNC, CMM integration & Tight tolerances, repeatability\\
\hline
\hspace{1em}\textbf{Aerospace/Defense} & Automated testing systems & Automated test equipment & Comprehensive testing, documentation\\
\hline
\hspace{1em}\textbf{Aerospace/Defense} & Clean room automation & Robotics, HEPA systems & Contamination control, precision\\
\hline
\end{tabu}
\end{table}

\begin{center}\rule{0.5\linewidth}{0.5pt}\end{center}

\section{Types of Automation}\label{types-of-automation}

Industrial automation systems can be classified based on their flexibility and programming capability.

\subsection{Automation Classification}\label{automation-classification}

\pandocbounded{\includegraphics[keepaspectratio]{introduction_files/figure-latex/automation-types-1.pdf}}

\begin{table}
\centering
\caption{\label{tab:automation-types-table}Comparison of Automation Types}
\centering
\begin{tabu} to \linewidth {>{\raggedright}X>{\raggedright}X>{\raggedright}X>{\raggedright}X}
\hline
 & Fixed Automation & Programmable Automation & Flexible Automation\\
\hline
\textbf{Definition} & \cellcolor[HTML]{fadbd8}{Hard-wired, dedicated equipment for single product} & \cellcolor[HTML]{fcf3cf}{Equipment can be reprogrammed for different products} & \cellcolor[HTML]{d5f5e3}{Rapid changeover with minimal downtime}\\
\hline
\textbf{Product Variety} & \cellcolor[HTML]{fadbd8}{Single product or very similar variants} & \cellcolor[HTML]{fcf3cf}{Batches of different products} & \cellcolor[HTML]{d5f5e3}{High variety, even mixed on same line}\\
\hline
\textbf{Production Volume} & \cellcolor[HTML]{fadbd8}{Very high (millions of units)} & \cellcolor[HTML]{fcf3cf}{Medium to high} & \cellcolor[HTML]{d5f5e3}{Low to medium}\\
\hline
\textbf{Changeover} & \cellcolor[HTML]{fadbd8}{Difficult, expensive, time-consuming} & \cellcolor[HTML]{fcf3cf}{Requires reprogramming and setup; hours to days} & \cellcolor[HTML]{d5f5e3}{Quick, often automatic; minutes}\\
\hline
\textbf{Initial Investment} & \cellcolor[HTML]{fadbd8}{Very high} & \cellcolor[HTML]{fcf3cf}{High} & \cellcolor[HTML]{d5f5e3}{Very high}\\
\hline
\textbf{Unit Cost} & \cellcolor[HTML]{fadbd8}{Very low per unit} & \cellcolor[HTML]{fcf3cf}{Medium} & \cellcolor[HTML]{d5f5e3}{Higher per unit, but flexible}\\
\hline
\textbf{Typical Equipment} & \cellcolor[HTML]{fadbd8}{Transfer lines, dedicated assembly machines} & \cellcolor[HTML]{fcf3cf}{CNC machines, PLCs, industrial robots} & \cellcolor[HTML]{d5f5e3}{FMS, robotic cells, AGVs}\\
\hline
\textbf{Best For} & \cellcolor[HTML]{fadbd8}{Automotive components, fasteners, bottles} & \cellcolor[HTML]{fcf3cf}{Batch manufacturing, job shops} & \cellcolor[HTML]{d5f5e3}{Aerospace, custom manufacturing}\\
\hline
\end{tabu}
\end{table}

\subsection{The Automation Pyramid}\label{the-automation-pyramid}

\pandocbounded{\includegraphics[keepaspectratio]{introduction_files/figure-latex/automation-pyramid-1.pdf}}

\textbf{Understanding the Pyramid Levels}

\textbf{Level 0 - Process:} The actual physical equipment, machines, and processes being controlled. This includes conveyors, motors, valves, tanks, and the product being manufactured.

\textbf{Level 1 - Field Devices:} Sensors that measure process variables (temperature, pressure, flow, position) and actuators that affect the process (motors, valves, solenoids, drives).

\textbf{Level 2 - Control:} The ``brain'' of automation - PLCs, DCS, and motion controllers that execute control logic, process sensor inputs, and command actuators. Also includes HMI for operator interaction.

\textbf{Level 3 - Manufacturing Operations:} MES (Manufacturing Execution Systems), SCADA, production scheduling, quality management, and maintenance management. Bridges plant floor and business systems.

\textbf{Level 4 - Enterprise:} ERP systems, business intelligence, supply chain management, financial systems. Makes business decisions based on plant floor data.

\textbf{Key Principle:} Data flows up (process information), commands flow down (control directives).

\begin{center}\rule{0.5\linewidth}{0.5pt}\end{center}

\section{Programmable Logic Controllers (PLCs)}\label{programmable-logic-controllers-plcs}

The \textbf{PLC (Programmable Logic Controller)} is the workhorse of industrial automation, providing reliable, real-time control of machines and processes.

\subsection{What is a PLC?}\label{what-is-a-plc}

\pandocbounded{\includegraphics[keepaspectratio]{introduction_files/figure-latex/plc-definition-1.pdf}}

\subsection{PLC Hardware Components}\label{plc-hardware-components}

\begin{table}
\centering
\caption{\label{tab:plc-components}PLC Hardware Components}
\centering
\begin{tabu} to \linewidth {>{\raggedright}X>{\raggedright}X>{\raggedright}X}
\hline
Component & Function & Specifications\\
\hline
\textbf{CPU (Processor)} & Executes the control program, manages memory, coordinates all modules & Scan time (ms), memory size (KB/MB), I/O capacity\\
\hline
\textbf{Power Supply} & Converts AC power to DC voltages required by PLC components & Input voltage (120/240 VAC), output power (watts)\\
\hline
\textbf{Input Modules} & Interface field devices (sensors, switches) to CPU; converts signals to digital & Digital (24VDC, 120VAC) or Analog (4-20mA, 0-10V)\\
\hline
\textbf{Output Modules} & Interface CPU to field devices (motors, valves); converts digital to power signals & Digital (relay, transistor) or Analog (4-20mA, 0-10V)\\
\hline
\textbf{Communication Modules} & Enable networking: Ethernet/IP, Profinet, Modbus, DeviceNet & Protocol, speed (Mbps), ports\\
\hline
\textbf{Programming Device} & Laptop/PC with programming software for creating and downloading programs & Software: RSLogix, TIA Portal, GX Works\\
\hline
\end{tabu}
\end{table}

\subsection{The PLC Scan Cycle}\label{the-plc-scan-cycle}

PLCs operate in a continuous scan cycle:

\pandocbounded{\includegraphics[keepaspectratio]{introduction_files/figure-latex/plc-scan-cycle-1.pdf}}

\begin{Shaded}
\begin{Highlighting}[]
\CommentTok{\# Scan Time Calculation Example}
\NormalTok{input\_scan\_time }\OtherTok{\textless{}{-}} \FloatTok{1.2}    \CommentTok{\# ms}
\NormalTok{program\_execution }\OtherTok{\textless{}{-}} \FloatTok{15.5}  \CommentTok{\# ms (depends on program size)}
\NormalTok{output\_update }\OtherTok{\textless{}{-}} \FloatTok{1.0}       \CommentTok{\# ms}
\NormalTok{housekeeping }\OtherTok{\textless{}{-}} \FloatTok{2.3}        \CommentTok{\# ms}

\NormalTok{total\_scan\_time }\OtherTok{\textless{}{-}}\NormalTok{ input\_scan\_time }\SpecialCharTok{+}\NormalTok{ program\_execution }\SpecialCharTok{+}\NormalTok{ output\_update }\SpecialCharTok{+}\NormalTok{ housekeeping}

\FunctionTok{cat}\NormalTok{(}\StringTok{"PLC Scan Time Calculation:}\SpecialCharTok{\textbackslash{}n}\StringTok{"}\NormalTok{)}
\end{Highlighting}
\end{Shaded}

\begin{verbatim}
## PLC Scan Time Calculation:
\end{verbatim}

\begin{Shaded}
\begin{Highlighting}[]
\FunctionTok{cat}\NormalTok{(}\StringTok{"─────────────────────────────}\SpecialCharTok{\textbackslash{}n}\StringTok{"}\NormalTok{)}
\end{Highlighting}
\end{Shaded}

\begin{verbatim}
## ─────────────────────────────
\end{verbatim}

\begin{Shaded}
\begin{Highlighting}[]
\FunctionTok{cat}\NormalTok{(}\StringTok{"Input Scan:       "}\NormalTok{, input\_scan\_time, }\StringTok{"ms}\SpecialCharTok{\textbackslash{}n}\StringTok{"}\NormalTok{)}
\end{Highlighting}
\end{Shaded}

\begin{verbatim}
## Input Scan:        1.2 ms
\end{verbatim}

\begin{Shaded}
\begin{Highlighting}[]
\FunctionTok{cat}\NormalTok{(}\StringTok{"Program Execution:"}\NormalTok{, program\_execution, }\StringTok{"ms}\SpecialCharTok{\textbackslash{}n}\StringTok{"}\NormalTok{)}
\end{Highlighting}
\end{Shaded}

\begin{verbatim}
## Program Execution: 15.5 ms
\end{verbatim}

\begin{Shaded}
\begin{Highlighting}[]
\FunctionTok{cat}\NormalTok{(}\StringTok{"Output Update:    "}\NormalTok{, output\_update, }\StringTok{"ms}\SpecialCharTok{\textbackslash{}n}\StringTok{"}\NormalTok{)}
\end{Highlighting}
\end{Shaded}

\begin{verbatim}
## Output Update:     1 ms
\end{verbatim}

\begin{Shaded}
\begin{Highlighting}[]
\FunctionTok{cat}\NormalTok{(}\StringTok{"Housekeeping:     "}\NormalTok{, housekeeping, }\StringTok{"ms}\SpecialCharTok{\textbackslash{}n}\StringTok{"}\NormalTok{)}
\end{Highlighting}
\end{Shaded}

\begin{verbatim}
## Housekeeping:      2.3 ms
\end{verbatim}

\begin{Shaded}
\begin{Highlighting}[]
\FunctionTok{cat}\NormalTok{(}\StringTok{"─────────────────────────────}\SpecialCharTok{\textbackslash{}n}\StringTok{"}\NormalTok{)}
\end{Highlighting}
\end{Shaded}

\begin{verbatim}
## ─────────────────────────────
\end{verbatim}

\begin{Shaded}
\begin{Highlighting}[]
\FunctionTok{cat}\NormalTok{(}\StringTok{"Total Scan Time:  "}\NormalTok{, total\_scan\_time, }\StringTok{"ms}\SpecialCharTok{\textbackslash{}n}\StringTok{"}\NormalTok{)}
\end{Highlighting}
\end{Shaded}

\begin{verbatim}
## Total Scan Time:   20 ms
\end{verbatim}

\begin{Shaded}
\begin{Highlighting}[]
\FunctionTok{cat}\NormalTok{(}\StringTok{"}\SpecialCharTok{\textbackslash{}n}\StringTok{Scans per second: "}\NormalTok{, }\FunctionTok{round}\NormalTok{(}\DecValTok{1000}\SpecialCharTok{/}\NormalTok{total\_scan\_time, }\DecValTok{0}\NormalTok{), }\StringTok{"}\SpecialCharTok{\textbackslash{}n}\StringTok{"}\NormalTok{)}
\end{Highlighting}
\end{Shaded}

\begin{verbatim}
## 
## Scans per second:  50
\end{verbatim}

\subsection{PLC Programming Languages (IEC 61131-3)}\label{plc-programming-languages-iec-61131-3}

\begin{verbatim}
## Warning in latex_new_row_builder(target_row, table_info, bold, italic,
## monospace, : Setting full_width = TRUE will turn the table into a tabu
## environment where colors are not really easily configable with this package.
## Please consider turn off full_width.
## Warning in latex_new_row_builder(target_row, table_info, bold, italic,
## monospace, : Setting full_width = TRUE will turn the table into a tabu
## environment where colors are not really easily configable with this package.
## Please consider turn off full_width.
\end{verbatim}

\begin{table}
\centering
\caption{\label{tab:plc-languages}IEC 61131-3 PLC Programming Languages}
\centering
\begin{tabu} to \linewidth {>{\raggedright}X>{\raggedright}X>{\raggedright}X>{\raggedright}X}
\hline
Language & Type & Description & Best For\\
\hline
\cellcolor[HTML]{e8f6f3}{\textbf{Ladder Diagram (LD)}} & \cellcolor[HTML]{e8f6f3}{Graphical} & \cellcolor[HTML]{e8f6f3}{Resembles electrical relay circuits; most common in discrete manufacturing} & \cellcolor[HTML]{e8f6f3}{Boolean logic, interlocks, machine control}\\
\hline
\cellcolor[HTML]{ebf5fb}{\textbf{Function Block Diagram (FBD)}} & \cellcolor[HTML]{ebf5fb}{Graphical} & \cellcolor[HTML]{ebf5fb}{Uses function blocks connected by lines; good for analog/process control} & \cellcolor[HTML]{ebf5fb}{PID loops, motion control, data manipulation}\\
\hline
\textbf{Structured Text (ST)} & Textual & High-level language similar to Pascal; powerful for complex calculations & Math operations, data handling, complex algorithms\\
\hline
\textbf{Instruction List (IL)} & Textual & Low-level assembly-like language; rarely used today & Legacy systems, compact code\\
\hline
\textbf{Sequential Function Chart (SFC)} & Graphical & Sequence/state-based programming; excellent for batch processes & Sequential operations, batch control, recipes\\
\hline
\end{tabu}
\end{table}

\subsection{Ladder Logic Example}\label{ladder-logic-example}

\pandocbounded{\includegraphics[keepaspectratio]{introduction_files/figure-latex/ladder-example-1.pdf}}

\textbf{How the Start/Stop Circuit Works}

\textbf{Rung 1 - Motor Control:}
1. Power flows from L1 through the NC (Normally Closed) STOP button
2. If STOP is pressed, the circuit breaks and motor stops
3. START is NO (Normally Open) - pressing it allows power to flow to the MOTOR coil
4. When MOTOR energizes, its NO contact in the parallel branch closes
5. This ``seals in'' the circuit - motor stays running even after START is released
6. To stop, press STOP which breaks the seal-in circuit

\textbf{Rung 2 - Run Indicator:}
1. When MOTOR coil is energized, the MOTOR contact closes
2. This allows power to flow to the RUN LIGHT output
3. Light is ON whenever motor is running

\textbf{This is called a ``3-wire control'' circuit} - it provides:
- Low voltage release protection (motor won't restart after power failure)
- Maintained contact operation (don't need to hold START button)

\begin{center}\rule{0.5\linewidth}{0.5pt}\end{center}

\section{Industrial Sensors}\label{industrial-sensors}

Sensors are the ``eyes and ears'' of automation systems, providing feedback about the process to the control system.

\subsection{Sensor Classification}\label{sensor-classification}

\pandocbounded{\includegraphics[keepaspectratio]{introduction_files/figure-latex/sensor-classification-1.pdf}}

\begin{table}
\centering
\caption{\label{tab:sensor-table}Common Industrial Sensor Specifications}
\centering
\begin{tabu} to \linewidth {>{\raggedright}X>{\raggedright}X>{\raggedright}X>{\raggedright}X}
\hline
Sensor & Principle & Range & Advantages\\
\hline
\textbf{Inductive Proximity} & Eddy current change in oscillating field & 2-40mm typical & Durable, no contact, metal detection\\
\hline
\textbf{Capacitive Proximity} & Capacitance change with target approach & 2-25mm typical & Detects any material, through walls\\
\hline
\textbf{Photoelectric} & Light beam interrupted or reflected & 0.1-30m & Long range, versatile\\
\hline
\textbf{RTD (PT100)} & Resistance changes with temperature & -200 to 850°C & Accurate, stable, linear\\
\hline
\textbf{Thermocouple} & Voltage generated at junction of dissimilar metals & -200 to 2300°C & Wide range, fast response\\
\hline
\textbf{Pressure Transducer} & Diaphragm deflection converted to electrical signal & 0-10000 psi & Accurate, wide range\\
\hline
\textbf{Incremental Encoder} & Optical/magnetic pulses per revolution & 100-10000 PPR & Simple, low cost, high resolution\\
\hline
\textbf{Absolute Encoder} & Unique code for each position & 12-25 bit & Position known at power-up\\
\hline
\end{tabu}
\end{table}

\subsection{Sensor Wiring: PNP vs NPN}\label{sensor-wiring-pnp-vs-npn}

\pandocbounded{\includegraphics[keepaspectratio]{introduction_files/figure-latex/pnp-npn-1.pdf}}

\begin{center}\rule{0.5\linewidth}{0.5pt}\end{center}

\section{Actuators and Motor Control}\label{actuators-and-motor-control}

Actuators convert control signals into physical motion or action.

\subsection{Types of Industrial Actuators}\label{types-of-industrial-actuators}

\begin{table}
\centering
\caption{\label{tab:actuator-types}Industrial Actuator Types and Applications}
\centering
\begin{tabu} to \linewidth {>{\raggedright}X>{\raggedright}X>{\raggedright}X>{\raggedright}X>{\raggedright}X}
\hline
Category & Type & Application & Control & Characteristics\\
\hline
\multicolumn{5}{l}{\cellcolor[HTML]{fef9e7}{\textbf{Electric}}}\\
\hline
\hspace{1em}\textbf{Electric Motors} & AC Induction Motor & Conveyors, pumps, fans, compressors & VFD for variable speed & Robust, low cost, high power\\
\hline
\hspace{1em}\textbf{Electric Motors} & Servo Motor & Precise positioning, robotics, CNC & Servo drive with feedback & High precision, dynamic response\\
\hline
\hspace{1em}\textbf{Electric Motors} & Stepper Motor & Indexing, low-speed positioning & Stepper drive (open or closed loop) & Simple control, holds position\\
\hline
\hspace{1em}\textbf{Electric Motors} & DC Motor & Battery vehicles, legacy systems & PWM, SCR drive & Easy speed control\\
\hline
\multicolumn{5}{l}{\cellcolor[HTML]{e8f6f3}{\textbf{Pneumatic}}}\\
\hline
\hspace{1em}\textbf{Pneumatic} & Cylinder & Clamping, pushing, lifting & Solenoid valves, proportional valves & Fast, clean, moderate force\\
\hline
\hspace{1em}\textbf{Pneumatic} & Rotary Actuator & Rotating grippers, indexing & Solenoid valves & Compact, clean\\
\hline
\multicolumn{5}{l}{\cellcolor[HTML]{fdedec}{\textbf{Hydraulic}}}\\
\hline
\hspace{1em}\textbf{Hydraulic} & Cylinder & Heavy lifting, presses & Proportional/servo valves & Very high force, smooth\\
\hline
\hspace{1em}\textbf{Hydraulic} & Motor & Heavy machinery, mobile equipment & Proportional/servo valves & High power density\\
\hline
\end{tabu}
\end{table}

\subsection{Variable Frequency Drives (VFDs)}\label{variable-frequency-drives-vfds}

A \textbf{Variable Frequency Drive (VFD)} controls AC motor speed by varying the frequency and voltage of the power supplied to the motor.

\pandocbounded{\includegraphics[keepaspectratio]{introduction_files/figure-latex/vfd-diagram-1.pdf}}

\begin{Shaded}
\begin{Highlighting}[]
\CommentTok{\# VFD Speed and Torque Calculations}

\CommentTok{\# Motor nameplate data}
\NormalTok{motor\_hp }\OtherTok{\textless{}{-}} \DecValTok{10}
\NormalTok{poles }\OtherTok{\textless{}{-}} \DecValTok{4}
\NormalTok{rated\_voltage }\OtherTok{\textless{}{-}} \DecValTok{460}  \CommentTok{\# V}
\NormalTok{rated\_frequency }\OtherTok{\textless{}{-}} \DecValTok{60}  \CommentTok{\# Hz}
\NormalTok{rated\_rpm }\OtherTok{\textless{}{-}} \DecValTok{1750}
\NormalTok{slip\_rpm }\OtherTok{\textless{}{-}}\NormalTok{ (}\DecValTok{120} \SpecialCharTok{*}\NormalTok{ rated\_frequency }\SpecialCharTok{/}\NormalTok{ poles) }\SpecialCharTok{{-}}\NormalTok{ rated\_rpm}

\CommentTok{\# Synchronous speed at rated frequency}
\NormalTok{sync\_speed\_60 }\OtherTok{\textless{}{-}} \DecValTok{120} \SpecialCharTok{*}\NormalTok{ rated\_frequency }\SpecialCharTok{/}\NormalTok{ poles}
\FunctionTok{cat}\NormalTok{(}\StringTok{"Motor Data:}\SpecialCharTok{\textbackslash{}n}\StringTok{"}\NormalTok{)}
\end{Highlighting}
\end{Shaded}

\begin{verbatim}
## Motor Data:
\end{verbatim}

\begin{Shaded}
\begin{Highlighting}[]
\FunctionTok{cat}\NormalTok{(}\StringTok{"Synchronous speed at 60Hz:"}\NormalTok{, sync\_speed\_60, }\StringTok{"RPM}\SpecialCharTok{\textbackslash{}n}\StringTok{"}\NormalTok{)}
\end{Highlighting}
\end{Shaded}

\begin{verbatim}
## Synchronous speed at 60Hz: 1800 RPM
\end{verbatim}

\begin{Shaded}
\begin{Highlighting}[]
\FunctionTok{cat}\NormalTok{(}\StringTok{"Rated speed:"}\NormalTok{, rated\_rpm, }\StringTok{"RPM}\SpecialCharTok{\textbackslash{}n}\StringTok{"}\NormalTok{)}
\end{Highlighting}
\end{Shaded}

\begin{verbatim}
## Rated speed: 1750 RPM
\end{verbatim}

\begin{Shaded}
\begin{Highlighting}[]
\FunctionTok{cat}\NormalTok{(}\StringTok{"Slip:"}\NormalTok{, slip\_rpm, }\StringTok{"RPM ("}\NormalTok{, }\FunctionTok{round}\NormalTok{(slip\_rpm}\SpecialCharTok{/}\NormalTok{sync\_speed\_60}\SpecialCharTok{*}\DecValTok{100}\NormalTok{, }\DecValTok{1}\NormalTok{), }\StringTok{"\%)}\SpecialCharTok{\textbackslash{}n\textbackslash{}n}\StringTok{"}\NormalTok{)}
\end{Highlighting}
\end{Shaded}

\begin{verbatim}
## Slip: 50 RPM ( 2.8 %)
\end{verbatim}

\begin{Shaded}
\begin{Highlighting}[]
\CommentTok{\# Calculate speed at different frequencies}
\NormalTok{frequencies }\OtherTok{\textless{}{-}} \FunctionTok{c}\NormalTok{(}\DecValTok{15}\NormalTok{, }\DecValTok{30}\NormalTok{, }\DecValTok{45}\NormalTok{, }\DecValTok{60}\NormalTok{, }\DecValTok{75}\NormalTok{)}
\NormalTok{speeds }\OtherTok{\textless{}{-}} \FunctionTok{sapply}\NormalTok{(frequencies, }\ControlFlowTok{function}\NormalTok{(f) \{}
\NormalTok{  sync }\OtherTok{\textless{}{-}} \DecValTok{120} \SpecialCharTok{*}\NormalTok{ f }\SpecialCharTok{/}\NormalTok{ poles}
\NormalTok{  sync }\SpecialCharTok{{-}}\NormalTok{ slip\_rpm  }\CommentTok{\# Assuming constant slip (approximation)}
\NormalTok{\})}

\FunctionTok{cat}\NormalTok{(}\StringTok{"Speed vs Frequency (V/Hz mode):}\SpecialCharTok{\textbackslash{}n}\StringTok{"}\NormalTok{)}
\end{Highlighting}
\end{Shaded}

\begin{verbatim}
## Speed vs Frequency (V/Hz mode):
\end{verbatim}

\begin{Shaded}
\begin{Highlighting}[]
\FunctionTok{cat}\NormalTok{(}\StringTok{"─────────────────────────────}\SpecialCharTok{\textbackslash{}n}\StringTok{"}\NormalTok{)}
\end{Highlighting}
\end{Shaded}

\begin{verbatim}
## ─────────────────────────────
\end{verbatim}

\begin{Shaded}
\begin{Highlighting}[]
\ControlFlowTok{for}\NormalTok{(i }\ControlFlowTok{in} \DecValTok{1}\SpecialCharTok{:}\FunctionTok{length}\NormalTok{(frequencies)) \{}
  \FunctionTok{cat}\NormalTok{(}\FunctionTok{sprintf}\NormalTok{(}\StringTok{"\%2d Hz: \%4d RPM}\SpecialCharTok{\textbackslash{}n}\StringTok{"}\NormalTok{, frequencies[i], speeds[i]))}
\NormalTok{\}}
\end{Highlighting}
\end{Shaded}

\begin{verbatim}
## 15 Hz:  400 RPM
## 30 Hz:  850 RPM
## 45 Hz: 1300 RPM
## 60 Hz: 1750 RPM
## 75 Hz: 2200 RPM
\end{verbatim}

\begin{Shaded}
\begin{Highlighting}[]
\CommentTok{\# V/Hz ratio}
\NormalTok{vhz\_ratio }\OtherTok{\textless{}{-}}\NormalTok{ rated\_voltage }\SpecialCharTok{/}\NormalTok{ rated\_frequency}
\FunctionTok{cat}\NormalTok{(}\StringTok{"}\SpecialCharTok{\textbackslash{}n}\StringTok{V/Hz Ratio:"}\NormalTok{, }\FunctionTok{round}\NormalTok{(vhz\_ratio, }\DecValTok{2}\NormalTok{), }\StringTok{"V/Hz}\SpecialCharTok{\textbackslash{}n}\StringTok{"}\NormalTok{)}
\end{Highlighting}
\end{Shaded}

\begin{verbatim}
## 
## V/Hz Ratio: 7.67 V/Hz
\end{verbatim}

\begin{Shaded}
\begin{Highlighting}[]
\FunctionTok{cat}\NormalTok{(}\StringTok{"At 30Hz, voltage should be:"}\NormalTok{, }\DecValTok{30} \SpecialCharTok{*}\NormalTok{ vhz\_ratio, }\StringTok{"V}\SpecialCharTok{\textbackslash{}n}\StringTok{"}\NormalTok{)}
\end{Highlighting}
\end{Shaded}

\begin{verbatim}
## At 30Hz, voltage should be: 230 V
\end{verbatim}

\subsection{VFD Benefits}\label{vfd-benefits}

\begin{table}
\centering
\caption{\label{tab:vfd-benefits}Variable Frequency Drive Benefits}
\centering
\begin{tabu} to \linewidth {>{\raggedright}X>{\raggedright}X>{\raggedright}X}
\hline
Benefit & Description & Typical Impact\\
\hline
\textbf{Energy Savings} & Match motor speed to load requirements; huge savings on fans/pumps & 20-50\% energy reduction\\
\hline
\textbf{Soft Start/Stop} & Ramp up/down gradually; eliminates inrush current (6-8x normal) & Reduced electrical stress\\
\hline
\textbf{Speed Control} & Precise speed control from 0-100\%+ of base speed & N/A\\
\hline
\textbf{Process Control} & Maintain constant pressure, flow, or tension & Improved quality\\
\hline
\textbf{Reduced Mechanical Stress} & Reduced wear on belts, gears, couplings from smooth acceleration & Extended equipment life\\
\hline
\textbf{Power Factor} & VFD presents near-unity power factor to supply & Avoid PF penalties\\
\hline
\end{tabu}
\end{table}

\subsection{Servo Systems}\label{servo-systems}

\pandocbounded{\includegraphics[keepaspectratio]{introduction_files/figure-latex/servo-system-1.pdf}}

\begin{table}
\centering
\caption{\label{tab:servo-vs-vfd}VFD vs. Servo System Comparison}
\centering
\begin{tabu} to \linewidth {>{\raggedright}X>{\raggedright}X>{\raggedright}X}
\hline
Characteristic & VFD + Induction Motor & Servo System\\
\hline
\textbf{Control Type} & \cellcolor[HTML]{ebf5fb}{Open loop (usually)} & \cellcolor[HTML]{e8f8f5}{Closed loop (always)}\\
\hline
\textbf{Feedback} & \cellcolor[HTML]{ebf5fb}{Optional encoder} & \cellcolor[HTML]{e8f8f5}{High-resolution encoder required}\\
\hline
\textbf{Positioning Accuracy} & \cellcolor[HTML]{ebf5fb}{±1-5\% of speed} & \cellcolor[HTML]{e8f8f5}{±0.01° or better}\\
\hline
\textbf{Dynamic Response} & \cellcolor[HTML]{ebf5fb}{Moderate (100-500ms)} & \cellcolor[HTML]{e8f8f5}{Fast (1-10ms)}\\
\hline
\textbf{Torque at Zero Speed} & \cellcolor[HTML]{ebf5fb}{Limited (10-20\%)} & \cellcolor[HTML]{e8f8f5}{100\% continuous}\\
\hline
\textbf{Cost} & \cellcolor[HTML]{ebf5fb}{Lower (\$500-5000)} & \cellcolor[HTML]{e8f8f5}{Higher (\$2000-20000)}\\
\hline
\textbf{Typical Application} & \cellcolor[HTML]{ebf5fb}{Fans, pumps, conveyors} & \cellcolor[HTML]{e8f8f5}{Robotics, CNC, packaging}\\
\hline
\end{tabu}
\end{table}

\begin{center}\rule{0.5\linewidth}{0.5pt}\end{center}

\section{Industrial Communication Networks}\label{industrial-communication-networks}

Modern automation systems rely on communication networks to connect devices, share data, and enable coordinated control.

\subsection{Network Architecture}\label{network-architecture}

\pandocbounded{\includegraphics[keepaspectratio]{introduction_files/figure-latex/network-architecture-1.pdf}}

\subsection{Common Industrial Protocols}\label{common-industrial-protocols}

\begin{verbatim}
## Warning in latex_new_row_builder(target_row, table_info, bold, italic,
## monospace, : Setting full_width = TRUE will turn the table into a tabu
## environment where colors are not really easily configable with this package.
## Please consider turn off full_width.
## Warning in latex_new_row_builder(target_row, table_info, bold, italic,
## monospace, : Setting full_width = TRUE will turn the table into a tabu
## environment where colors are not really easily configable with this package.
## Please consider turn off full_width.
## Warning in latex_new_row_builder(target_row, table_info, bold, italic,
## monospace, : Setting full_width = TRUE will turn the table into a tabu
## environment where colors are not really easily configable with this package.
## Please consider turn off full_width.
\end{verbatim}

\begin{table}
\centering
\caption{\label{tab:protocols-table}Common Industrial Communication Protocols}
\centering
\begin{tabu} to \linewidth {>{\raggedright}X>{\raggedright}X>{\raggedright}X>{\raggedright}X>{\raggedright}X}
\hline
Protocol & Developer & Medium & Speed & Application\\
\hline
\cellcolor[HTML]{e8f6f3}{\textbf{Ethernet/IP}} & \cellcolor[HTML]{e8f6f3}{ODVA (Rockwell)} & \cellcolor[HTML]{e8f6f3}{Ethernet} & \cellcolor[HTML]{e8f6f3}{100Mbps/1Gbps} & \cellcolor[HTML]{e8f6f3}{General automation, I/O, drives}\\
\hline
\cellcolor[HTML]{e8f6f3}{\textbf{PROFINET}} & \cellcolor[HTML]{e8f6f3}{PI (Siemens)} & \cellcolor[HTML]{e8f6f3}{Ethernet} & \cellcolor[HTML]{e8f6f3}{100Mbps/1Gbps} & \cellcolor[HTML]{e8f6f3}{Siemens ecosystem, motion}\\
\hline
\textbf{Modbus TCP} & Modicon/Schneider & Ethernet & 100Mbps & Simple, open, legacy systems\\
\hline
\cellcolor[HTML]{e8f6f3}{\textbf{EtherCAT}} & \cellcolor[HTML]{e8f6f3}{Beckhoff} & \cellcolor[HTML]{e8f6f3}{Ethernet} & \cellcolor[HTML]{e8f6f3}{100Mbps} & \cellcolor[HTML]{e8f6f3}{High-speed motion, precision}\\
\hline
\textbf{DeviceNet} & ODVA & CAN-based & 500kbps & I/O, drives (legacy)\\
\hline
\textbf{PROFIBUS} & PI (Siemens) & RS-485 & 12Mbps & I/O, drives (legacy)\\
\hline
\textbf{IO-Link} & IO-Link Consortium & Point-to-point & 230.4kbps & Smart sensors\\
\hline
\textbf{OPC UA} & OPC Foundation & Ethernet/Any & Varies & IT/OT integration, IIoT\\
\hline
\end{tabu}
\end{table}

\textbf{Protocol Selection Guide}

\textbf{Choose Ethernet/IP if:}
- Using Allen-Bradley/Rockwell equipment
- Need general-purpose industrial Ethernet
- Require seamless integration with IT networks

\textbf{Choose PROFINET if:}
- Using Siemens equipment
- Need deterministic communication for motion control
- Require IRT (Isochronous Real-Time) performance

\textbf{Choose EtherCAT if:}
- Need highest speed/lowest latency
- High-precision motion control
- Many axes of coordinated motion

\textbf{Choose Modbus TCP if:}
- Simple, low-cost solution needed
- Connecting legacy equipment
- Open protocol preference (no licensing)

\textbf{Choose OPC UA if:}
- Need IT/OT convergence
- IIoT/Industry 4.0 implementation
- Secure, platform-independent communication

\begin{center}\rule{0.5\linewidth}{0.5pt}\end{center}

\section{Human-Machine Interface (HMI) and SCADA}\label{human-machine-interface-hmi-and-scada}

\subsection{HMI Overview}\label{hmi-overview}

An \textbf{HMI (Human-Machine Interface)} provides operators with a visual interface to monitor and control automated processes.

\pandocbounded{\includegraphics[keepaspectratio]{introduction_files/figure-latex/hmi-elements-1.pdf}}

\subsection{HMI Design Best Practices}\label{hmi-design-best-practices}

\begin{table}
\centering
\caption{\label{tab:hmi-practices}HMI Design Best Practices}
\centering
\begin{tabu} to \linewidth {>{\raggedright}X>{\raggedright}X>{\raggedright}X}
\hline
Principle & Best Practice & Avoid\\
\hline
\textbf{Situational Awareness} & Show abnormal conditions prominently; operator should know status at a glance & \cellcolor[HTML]{fadbd8}{Everything same color; too much detail}\\
\hline
\textbf{Alarm Management} & Prioritize alarms (critical/warning/info); avoid alarm floods; require acknowledgment & \cellcolor[HTML]{fadbd8}{Hundreds of unacknowledged alarms}\\
\hline
\textbf{Navigation} & Consistent layout; max 3 clicks to any screen; clear hierarchy & \cellcolor[HTML]{fadbd8}{Inconsistent button placement}\\
\hline
\textbf{Color Usage} & Use color sparingly for meaning; gray for normal; avoid red/green for critical info (colorblind) & \cellcolor[HTML]{fadbd8}{Rainbow colors everywhere}\\
\hline
\textbf{Data Display} & Show trends, not just values; use appropriate precision; include units & \cellcolor[HTML]{fadbd8}{Too many decimal places; no context}\\
\hline
\textbf{Controls} & Confirm destructive actions; use interlocks; provide feedback & \cellcolor[HTML]{fadbd8}{No confirmation for critical commands}\\
\hline
\end{tabu}
\end{table}

\subsection{SCADA Systems}\label{scada-systems}

\textbf{SCADA (Supervisory Control and Data Acquisition)} systems provide centralized monitoring and control across multiple locations or processes.

\begin{table}
\centering
\caption{\label{tab:scada-architecture}SCADA System Components}
\centering
\begin{tabu} to \linewidth {>{\raggedright}X>{\raggedright}X>{\raggedright}X}
\hline
Component & Function & Example\\
\hline
\textbf{MTU (Master Terminal Unit)} & Central server running SCADA software; processes data, executes logic & Wonderware, Ignition, FactoryTalk\\
\hline
\textbf{RTU (Remote Terminal Unit)} & Field device that collects data from sensors and sends to MTU & PLC, dedicated RTU hardware\\
\hline
\textbf{Communication Network} & Links MTU to RTUs; can be radio, cellular, satellite, fiber & Modbus, DNP3, IEC 61850\\
\hline
\textbf{HMI/Workstations} & Operator interface for monitoring and control & PC workstations, web clients\\
\hline
\textbf{Historian Database} & Stores historical process data for trending and analysis & SQL database, OSIsoft PI\\
\hline
\textbf{Alarm Server} & Manages alarm generation, notification, and logging & Built into SCADA or separate\\
\hline
\end{tabu}
\end{table}

\begin{center}\rule{0.5\linewidth}{0.5pt}\end{center}

\section{Safety in Automation}\label{safety-in-automation}

Safety is paramount in automated systems. Safety systems protect personnel and equipment from harm.

\subsection{Safety System Architecture}\label{safety-system-architecture}

\pandocbounded{\includegraphics[keepaspectratio]{introduction_files/figure-latex/safety-architecture-1.pdf}}

\subsection{Safety Integrity Levels (SIL)}\label{safety-integrity-levels-sil}

\begin{verbatim}
## Warning in latex_new_row_builder(target_row, table_info, bold, italic,
## monospace, : Setting full_width = TRUE will turn the table into a tabu
## environment where colors are not really easily configable with this package.
## Please consider turn off full_width.
\end{verbatim}

\begin{table}
\centering
\caption{\label{tab:sil-levels-automation}Safety Integrity Levels (IEC 61508)}
\centering
\begin{tabu} to \linewidth {>{\raggedright}X>{\raggedright}X>{\raggedright}X>{\raggedright}X>{\raggedright}X}
\hline
SIL Level & PFD (avg) & Risk Reduction Factor & Risk Level & Example Application\\
\hline
\textbf{SIL 1} & ≥10⁻² to <10⁻¹ & 10-100 & Minor injury possible; first line of defense & Warning systems, non-critical interlocks\\
\hline
\textbf{SIL 2} & ≥10⁻³ to <10⁻² & 100-1,000 & Serious injury possible; general industrial & Machine guarding, general process safety\\
\hline
\cellcolor[HTML]{fcf3cf}{\textbf{SIL 3}} & \cellcolor[HTML]{fcf3cf}{≥10⁻⁴ to <10⁻³} & \cellcolor[HTML]{fcf3cf}{1,000-10,000} & \cellcolor[HTML]{fcf3cf}{Death or severe injury; process industry standard} & \cellcolor[HTML]{fcf3cf}{Emergency shutdown, burner management}\\
\hline
\textbf{SIL 4} & ≥10⁻⁵ to <10⁻⁴ & 10,000-100,000 & Catastrophic; nuclear, aviation & Nuclear reactor protection (rarely used in manufacturing)\\
\hline
\end{tabu}
\end{table}

\begin{center}\rule{0.5\linewidth}{0.5pt}\end{center}

\section{Industry 4.0 and IIoT}\label{industry-4.0-and-iiot}

\textbf{Industry 4.0} represents the fourth industrial revolution, characterized by the integration of digital technologies into manufacturing.

\subsection{Industry 4.0 Technologies}\label{industry-4.0-technologies}

\pandocbounded{\includegraphics[keepaspectratio]{introduction_files/figure-latex/industry4-tech-1.pdf}}

\subsection{IIoT Architecture}\label{iiot-architecture}

\begin{table}
\centering
\caption{\label{tab:iiot-architecture}IIoT Architecture Layers}
\centering
\begin{tabu} to \linewidth {>{\raggedright}X>{\raggedright}X>{\raggedright}X>{\raggedright}X}
\hline
Layer & Components & Function & Examples\\
\hline
\textbf{Edge/Device Layer} & Sensors, actuators, PLCs, gateways, edge devices & Collect data, local processing, protocol conversion & Vibration sensor, smart valve, edge computer\\
\hline
\textbf{Network Layer} & Industrial Ethernet, Wi-Fi, 5G, LoRaWAN, MQTT & Secure data transport, connectivity & Cisco switches, AWS IoT Core, Kepware\\
\hline
\textbf{Platform Layer} & Cloud/on-premise servers, databases, data lakes & Store, process, and manage data at scale & Azure IoT Hub, AWS, Ignition\\
\hline
\textbf{Application Layer} & Analytics dashboards, AI/ML models, business applications & Derive insights, optimize operations, enable decisions & Power BI, TensorFlow, custom apps\\
\hline
\end{tabu}
\end{table}

\subsection{Predictive Maintenance Example}\label{predictive-maintenance-example}

\begin{Shaded}
\begin{Highlighting}[]
\CommentTok{\# Simulating predictive maintenance with vibration data}
\FunctionTok{set.seed}\NormalTok{(}\DecValTok{42}\NormalTok{)}

\CommentTok{\# Generate 30 days of vibration readings}
\NormalTok{days }\OtherTok{\textless{}{-}} \DecValTok{1}\SpecialCharTok{:}\DecValTok{30}
\NormalTok{baseline\_vibration }\OtherTok{\textless{}{-}} \FloatTok{2.5}  \CommentTok{\# mm/s RMS (normal)}

\CommentTok{\# Simulate gradual bearing degradation}
\NormalTok{vibration }\OtherTok{\textless{}{-}}\NormalTok{ baseline\_vibration }\SpecialCharTok{+}
  \FloatTok{0.05} \SpecialCharTok{*}\NormalTok{ days }\SpecialCharTok{+}           \CommentTok{\# Gradual increase}
  \FloatTok{0.5} \SpecialCharTok{*} \FunctionTok{exp}\NormalTok{((days }\SpecialCharTok{{-}} \DecValTok{25}\NormalTok{)}\SpecialCharTok{/}\DecValTok{5}\NormalTok{) }\SpecialCharTok{*}\NormalTok{ (days }\SpecialCharTok{\textgreater{}} \DecValTok{20}\NormalTok{) }\SpecialCharTok{+}  \CommentTok{\# Accelerating failure}
  \FunctionTok{rnorm}\NormalTok{(}\DecValTok{30}\NormalTok{, }\DecValTok{0}\NormalTok{, }\FloatTok{0.2}\NormalTok{)       }\CommentTok{\# Normal variation}

\CommentTok{\# Alert thresholds}
\NormalTok{warning\_threshold }\OtherTok{\textless{}{-}} \FloatTok{4.0}
\NormalTok{alarm\_threshold }\OtherTok{\textless{}{-}} \FloatTok{6.0}

\CommentTok{\# Find when thresholds crossed}
\NormalTok{warning\_day }\OtherTok{\textless{}{-}} \FunctionTok{min}\NormalTok{(}\FunctionTok{which}\NormalTok{(vibration }\SpecialCharTok{\textgreater{}}\NormalTok{ warning\_threshold))}
\NormalTok{alarm\_day }\OtherTok{\textless{}{-}} \FunctionTok{min}\NormalTok{(}\FunctionTok{which}\NormalTok{(vibration }\SpecialCharTok{\textgreater{}}\NormalTok{ alarm\_threshold))}
\end{Highlighting}
\end{Shaded}

\begin{verbatim}
## Warning in min(which(vibration > alarm_threshold)): no non-missing arguments to
## min; returning Inf
\end{verbatim}

\begin{Shaded}
\begin{Highlighting}[]
\FunctionTok{cat}\NormalTok{(}\StringTok{"Predictive Maintenance Analysis:}\SpecialCharTok{\textbackslash{}n}\StringTok{"}\NormalTok{)}
\end{Highlighting}
\end{Shaded}

\begin{verbatim}
## Predictive Maintenance Analysis:
\end{verbatim}

\begin{Shaded}
\begin{Highlighting}[]
\FunctionTok{cat}\NormalTok{(}\StringTok{"─────────────────────────────────}\SpecialCharTok{\textbackslash{}n}\StringTok{"}\NormalTok{)}
\end{Highlighting}
\end{Shaded}

\begin{verbatim}
## ─────────────────────────────────
\end{verbatim}

\begin{Shaded}
\begin{Highlighting}[]
\FunctionTok{cat}\NormalTok{(}\StringTok{"Baseline vibration:"}\NormalTok{, baseline\_vibration, }\StringTok{"mm/s RMS}\SpecialCharTok{\textbackslash{}n}\StringTok{"}\NormalTok{)}
\end{Highlighting}
\end{Shaded}

\begin{verbatim}
## Baseline vibration: 2.5 mm/s RMS
\end{verbatim}

\begin{Shaded}
\begin{Highlighting}[]
\FunctionTok{cat}\NormalTok{(}\StringTok{"Warning threshold:"}\NormalTok{, warning\_threshold, }\StringTok{"mm/s RMS}\SpecialCharTok{\textbackslash{}n}\StringTok{"}\NormalTok{)}
\end{Highlighting}
\end{Shaded}

\begin{verbatim}
## Warning threshold: 4 mm/s RMS
\end{verbatim}

\begin{Shaded}
\begin{Highlighting}[]
\FunctionTok{cat}\NormalTok{(}\StringTok{"Alarm threshold:"}\NormalTok{, alarm\_threshold, }\StringTok{"mm/s RMS}\SpecialCharTok{\textbackslash{}n\textbackslash{}n}\StringTok{"}\NormalTok{)}
\end{Highlighting}
\end{Shaded}

\begin{verbatim}
## Alarm threshold: 6 mm/s RMS
\end{verbatim}

\begin{Shaded}
\begin{Highlighting}[]
\FunctionTok{cat}\NormalTok{(}\StringTok{"Warning triggered on day:"}\NormalTok{, warning\_day, }\StringTok{"}\SpecialCharTok{\textbackslash{}n}\StringTok{"}\NormalTok{)}
\end{Highlighting}
\end{Shaded}

\begin{verbatim}
## Warning triggered on day: 24
\end{verbatim}

\begin{Shaded}
\begin{Highlighting}[]
\FunctionTok{cat}\NormalTok{(}\StringTok{"Alarm triggered on day:"}\NormalTok{, alarm\_day, }\StringTok{"}\SpecialCharTok{\textbackslash{}n}\StringTok{"}\NormalTok{)}
\end{Highlighting}
\end{Shaded}

\begin{verbatim}
## Alarm triggered on day: Inf
\end{verbatim}

\begin{Shaded}
\begin{Highlighting}[]
\FunctionTok{cat}\NormalTok{(}\StringTok{"Days of warning before alarm:"}\NormalTok{, alarm\_day }\SpecialCharTok{{-}}\NormalTok{ warning\_day, }\StringTok{"}\SpecialCharTok{\textbackslash{}n}\StringTok{"}\NormalTok{)}
\end{Highlighting}
\end{Shaded}

\begin{verbatim}
## Days of warning before alarm: Inf
\end{verbatim}

\begin{Shaded}
\begin{Highlighting}[]
\FunctionTok{cat}\NormalTok{(}\StringTok{"}\SpecialCharTok{\textbackslash{}n}\StringTok{Recommendation: Schedule bearing replacement before day"}\NormalTok{, alarm\_day, }\StringTok{"}\SpecialCharTok{\textbackslash{}n}\StringTok{"}\NormalTok{)}
\end{Highlighting}
\end{Shaded}

\begin{verbatim}
## 
## Recommendation: Schedule bearing replacement before day Inf
\end{verbatim}

\pandocbounded{\includegraphics[keepaspectratio]{introduction_files/figure-latex/predictive-plot-1.pdf}}

\begin{center}\rule{0.5\linewidth}{0.5pt}\end{center}

\section{Video Resources}\label{video-resources-2}

\subsection{Introduction to PLCs}\label{introduction-to-plcs}

\subsection{Industrial Networks Explained}\label{industrial-networks-explained}

\begin{center}\rule{0.5\linewidth}{0.5pt}\end{center}

\section{Summary}\label{summary-11}

Industrial automation integrates multiple technologies to improve manufacturing efficiency, quality, and safety:

\begin{enumerate}
\def\labelenumi{\arabic{enumi}.}
\tightlist
\item
  \textbf{Automation types} range from fixed (high-volume, single product) to flexible (low-volume, high variety)
\item
  \textbf{PLCs} are the workhorses of industrial control, using scan cycles to process inputs and control outputs
\item
  \textbf{Sensors} provide feedback on process conditions (discrete, analog, and smart)
\item
  \textbf{Motor control} through VFDs and servo systems enables precise speed and position control
\item
  \textbf{Industrial networks} connect devices using protocols like Ethernet/IP, PROFINET, and Modbus
\item
  \textbf{HMI and SCADA} provide operator interface for monitoring and control
\item
  \textbf{Safety systems} use dedicated controllers and devices to protect personnel
\item
  \textbf{Industry 4.0} brings IIoT, cloud computing, and AI to enable smart manufacturing
\end{enumerate}

\begin{center}\rule{0.5\linewidth}{0.5pt}\end{center}

\section{Review Questions}\label{review-questions-12}

\textbf{Question 1}: Compare and contrast fixed, programmable, and flexible automation. Give an example application for each.

\textbf{Answer:}

\begin{longtable}[]{@{}
  >{\raggedright\arraybackslash}p{(\linewidth - 6\tabcolsep) * \real{0.2051}}
  >{\raggedright\arraybackslash}p{(\linewidth - 6\tabcolsep) * \real{0.1795}}
  >{\raggedright\arraybackslash}p{(\linewidth - 6\tabcolsep) * \real{0.3590}}
  >{\raggedright\arraybackslash}p{(\linewidth - 6\tabcolsep) * \real{0.2564}}@{}}
\toprule\noalign{}
\begin{minipage}[b]{\linewidth}\raggedright
Aspect
\end{minipage} & \begin{minipage}[b]{\linewidth}\raggedright
Fixed
\end{minipage} & \begin{minipage}[b]{\linewidth}\raggedright
Programmable
\end{minipage} & \begin{minipage}[b]{\linewidth}\raggedright
Flexible
\end{minipage} \\
\midrule\noalign{}
\endhead
\bottomrule\noalign{}
\endlastfoot
Product variety & Single product & Batches of different products & High variety, mixed production \\
Volume & Very high & Medium-high & Low-medium \\
Changeover & Difficult, expensive & Hours to days & Minutes \\
Investment & Very high & High & Very high \\
Flexibility & None & Limited & High \\
\end{longtable}

\textbf{Examples:}

\begin{enumerate}
\def\labelenumi{\arabic{enumi}.}
\tightlist
\item
  \textbf{Fixed Automation}: Automotive transfer line producing engine blocks

  \begin{itemize}
  \tightlist
  \item
    Same machining operations repeated millions of times
  \item
    Dedicated stations optimized for specific operations
  \item
    Changing product would require complete retooling
  \end{itemize}
\item
  \textbf{Programmable Automation}: CNC machining job shop

  \begin{itemize}
  \tightlist
  \item
    Different parts programmed and run in batches
  \item
    Significant setup time between batches
  \item
    Equipment reprogrammable but not instantaneous
  \end{itemize}
\item
  \textbf{Flexible Automation}: Robotic welding cell in aerospace

  \begin{itemize}
  \tightlist
  \item
    Different assemblies run with quick changeover
  \item
    Robot program changes automatically based on part ID
  \item
    Same cell handles multiple product variants
  \end{itemize}
\end{enumerate}

\textbf{Question 2}: Explain the PLC scan cycle and why scan time is important.

\textbf{Answer:}

\textbf{The PLC Scan Cycle} consists of four main phases executed continuously:

\begin{enumerate}
\def\labelenumi{\arabic{enumi}.}
\tightlist
\item
  \textbf{Input Scan} (\textasciitilde1ms): Read all physical inputs and store values in the input image table
\item
  \textbf{Program Execution} (\textasciitilde5-20ms): Execute the user program (ladder logic, etc.) using input image data
\item
  \textbf{Output Update} (\textasciitilde1ms): Write output image table values to physical outputs
\item
  \textbf{Housekeeping} (\textasciitilde1-3ms): Communications, diagnostics, self-testing
\end{enumerate}

\textbf{Total scan time} = Sum of all phases, typically 10-50ms

\textbf{Why Scan Time Matters:}

\begin{enumerate}
\def\labelenumi{\arabic{enumi}.}
\item
  \textbf{Response Time}: An input change isn't acted upon until the next scan completes. If scan time is 50ms, worst-case response time is 50ms.
\item
  \textbf{Safety}: Fast-moving machinery requires short scan times for safety functions. A 100ms scan time with a motor at 1800 RPM means 3 shaft revolutions between scans.
\item
  \textbf{Application Limits}:

  \begin{itemize}
  \tightlist
  \item
    Motion control may need \textless1ms scan times (special motion PLCs)
  \item
    High-speed counting needs special modules
  \item
    Safety PLCs have guaranteed maximum scan times
  \end{itemize}
\item
  \textbf{Program Size Impact}: Larger programs take longer to execute, increasing scan time. Optimization may be needed for time-critical applications.
\item
  \textbf{Communication Delays}: Network updates typically happen once per scan, affecting data freshness.
\end{enumerate}

\textbf{Question 3}: What are the key differences between PNP and NPN sensor outputs? When would you use each?

\textbf{Answer:}

\textbf{PNP (Sourcing):}
- Sensor output sources current TO the load
- When activated, output connects to +V
- Load is connected between sensor output and 0V
- Current flows: +V → Sensor → Output → Load → 0V
- Common in North America and Europe

\textbf{NPN (Sinking):}
- Sensor output sinks current FROM the load
- When activated, output connects to 0V (ground)
- Load is connected between sensor output and +V
- Current flows: +V → Load → Output → Sensor → 0V
- Common in Asia (Japan)

\textbf{Selection Criteria:}

\begin{enumerate}
\def\labelenumi{\arabic{enumi}.}
\tightlist
\item
  \textbf{PLC Input Type}:

  \begin{itemize}
  \tightlist
  \item
    Sinking PLC inputs work with PNP (sourcing) sensors
  \item
    Sourcing PLC inputs work with NPN (sinking) sensors
  \end{itemize}
\item
  \textbf{Regional Standards}:

  \begin{itemize}
  \tightlist
  \item
    North America: PNP is standard
  \item
    Europe: PNP is standard
  \item
    Japan/Asia: NPN is common
  \end{itemize}
\item
  \textbf{Safety Considerations}:

  \begin{itemize}
  \tightlist
  \item
    PNP: A ground fault can cause false ON signal
  \item
    NPN: A ground fault typically causes safe-off condition
  \item
    For safety applications, consider sensor design carefully
  \end{itemize}
\item
  \textbf{Existing Infrastructure}:

  \begin{itemize}
  \tightlist
  \item
    Match new sensors to existing system type
  \item
    Converting between types requires different PLC input cards
  \end{itemize}
\end{enumerate}

\textbf{Question 4}: A VFD is controlling a 10 HP, 4-pole motor rated at 1750 RPM and 460V at 60Hz. Calculate the motor speed at 30Hz and 45Hz. What voltage should the VFD supply at these frequencies?

\textbf{Answer:}

\begin{Shaded}
\begin{Highlighting}[]
\CommentTok{\# Motor parameters}
\NormalTok{poles }\OtherTok{\textless{}{-}} \DecValTok{4}
\NormalTok{rated\_voltage }\OtherTok{\textless{}{-}} \DecValTok{460}  \CommentTok{\# V}
\NormalTok{rated\_frequency }\OtherTok{\textless{}{-}} \DecValTok{60}  \CommentTok{\# Hz}
\NormalTok{rated\_rpm }\OtherTok{\textless{}{-}} \DecValTok{1750}

\CommentTok{\# Calculate synchronous speed and slip at rated conditions}
\NormalTok{sync\_speed\_60 }\OtherTok{\textless{}{-}} \DecValTok{120} \SpecialCharTok{*}\NormalTok{ rated\_frequency }\SpecialCharTok{/}\NormalTok{ poles}
\NormalTok{slip\_rpm }\OtherTok{\textless{}{-}}\NormalTok{ sync\_speed\_60 }\SpecialCharTok{{-}}\NormalTok{ rated\_rpm}
\NormalTok{slip\_percent }\OtherTok{\textless{}{-}}\NormalTok{ (slip\_rpm }\SpecialCharTok{/}\NormalTok{ sync\_speed\_60) }\SpecialCharTok{*} \DecValTok{100}

\FunctionTok{cat}\NormalTok{(}\StringTok{"Motor Analysis:}\SpecialCharTok{\textbackslash{}n}\StringTok{"}\NormalTok{)}
\end{Highlighting}
\end{Shaded}

\begin{verbatim}
## Motor Analysis:
\end{verbatim}

\begin{Shaded}
\begin{Highlighting}[]
\FunctionTok{cat}\NormalTok{(}\StringTok{"Synchronous speed at 60Hz:"}\NormalTok{, sync\_speed\_60, }\StringTok{"RPM}\SpecialCharTok{\textbackslash{}n}\StringTok{"}\NormalTok{)}
\end{Highlighting}
\end{Shaded}

\begin{verbatim}
## Synchronous speed at 60Hz: 1800 RPM
\end{verbatim}

\begin{Shaded}
\begin{Highlighting}[]
\FunctionTok{cat}\NormalTok{(}\StringTok{"Rated speed:"}\NormalTok{, rated\_rpm, }\StringTok{"RPM}\SpecialCharTok{\textbackslash{}n}\StringTok{"}\NormalTok{)}
\end{Highlighting}
\end{Shaded}

\begin{verbatim}
## Rated speed: 1750 RPM
\end{verbatim}

\begin{Shaded}
\begin{Highlighting}[]
\FunctionTok{cat}\NormalTok{(}\StringTok{"Slip:"}\NormalTok{, slip\_rpm, }\StringTok{"RPM ("}\NormalTok{, }\FunctionTok{round}\NormalTok{(slip\_percent, }\DecValTok{1}\NormalTok{), }\StringTok{"\%)}\SpecialCharTok{\textbackslash{}n\textbackslash{}n}\StringTok{"}\NormalTok{)}
\end{Highlighting}
\end{Shaded}

\begin{verbatim}
## Slip: 50 RPM ( 2.8 %)
\end{verbatim}

\begin{Shaded}
\begin{Highlighting}[]
\CommentTok{\# V/Hz ratio}
\NormalTok{vhz\_ratio }\OtherTok{\textless{}{-}}\NormalTok{ rated\_voltage }\SpecialCharTok{/}\NormalTok{ rated\_frequency}
\FunctionTok{cat}\NormalTok{(}\StringTok{"V/Hz Ratio:"}\NormalTok{, }\FunctionTok{round}\NormalTok{(vhz\_ratio, }\DecValTok{2}\NormalTok{), }\StringTok{"V/Hz}\SpecialCharTok{\textbackslash{}n\textbackslash{}n}\StringTok{"}\NormalTok{)}
\end{Highlighting}
\end{Shaded}

\begin{verbatim}
## V/Hz Ratio: 7.67 V/Hz
\end{verbatim}

\begin{Shaded}
\begin{Highlighting}[]
\CommentTok{\# Calculate at 30Hz}
\NormalTok{freq\_30 }\OtherTok{\textless{}{-}} \DecValTok{30}
\NormalTok{sync\_30 }\OtherTok{\textless{}{-}} \DecValTok{120} \SpecialCharTok{*}\NormalTok{ freq\_30 }\SpecialCharTok{/}\NormalTok{ poles}
\NormalTok{speed\_30 }\OtherTok{\textless{}{-}}\NormalTok{ sync\_30 }\SpecialCharTok{{-}}\NormalTok{ slip\_rpm  }\CommentTok{\# Assuming constant slip in RPM}
\NormalTok{voltage\_30 }\OtherTok{\textless{}{-}}\NormalTok{ freq\_30 }\SpecialCharTok{*}\NormalTok{ vhz\_ratio}

\FunctionTok{cat}\NormalTok{(}\StringTok{"At 30Hz:}\SpecialCharTok{\textbackslash{}n}\StringTok{"}\NormalTok{)}
\end{Highlighting}
\end{Shaded}

\begin{verbatim}
## At 30Hz:
\end{verbatim}

\begin{Shaded}
\begin{Highlighting}[]
\FunctionTok{cat}\NormalTok{(}\StringTok{"  Synchronous speed:"}\NormalTok{, sync\_30, }\StringTok{"RPM}\SpecialCharTok{\textbackslash{}n}\StringTok{"}\NormalTok{)}
\end{Highlighting}
\end{Shaded}

\begin{verbatim}
##   Synchronous speed: 900 RPM
\end{verbatim}

\begin{Shaded}
\begin{Highlighting}[]
\FunctionTok{cat}\NormalTok{(}\StringTok{"  Motor speed (approx):"}\NormalTok{, speed\_30, }\StringTok{"RPM}\SpecialCharTok{\textbackslash{}n}\StringTok{"}\NormalTok{)}
\end{Highlighting}
\end{Shaded}

\begin{verbatim}
##   Motor speed (approx): 850 RPM
\end{verbatim}

\begin{Shaded}
\begin{Highlighting}[]
\FunctionTok{cat}\NormalTok{(}\StringTok{"  VFD output voltage:"}\NormalTok{, voltage\_30, }\StringTok{"V}\SpecialCharTok{\textbackslash{}n\textbackslash{}n}\StringTok{"}\NormalTok{)}
\end{Highlighting}
\end{Shaded}

\begin{verbatim}
##   VFD output voltage: 230 V
\end{verbatim}

\begin{Shaded}
\begin{Highlighting}[]
\CommentTok{\# Calculate at 45Hz}
\NormalTok{freq\_45 }\OtherTok{\textless{}{-}} \DecValTok{45}
\NormalTok{sync\_45 }\OtherTok{\textless{}{-}} \DecValTok{120} \SpecialCharTok{*}\NormalTok{ freq\_45 }\SpecialCharTok{/}\NormalTok{ poles}
\NormalTok{speed\_45 }\OtherTok{\textless{}{-}}\NormalTok{ sync\_45 }\SpecialCharTok{{-}}\NormalTok{ slip\_rpm}
\NormalTok{voltage\_45 }\OtherTok{\textless{}{-}}\NormalTok{ freq\_45 }\SpecialCharTok{*}\NormalTok{ vhz\_ratio}

\FunctionTok{cat}\NormalTok{(}\StringTok{"At 45Hz:}\SpecialCharTok{\textbackslash{}n}\StringTok{"}\NormalTok{)}
\end{Highlighting}
\end{Shaded}

\begin{verbatim}
## At 45Hz:
\end{verbatim}

\begin{Shaded}
\begin{Highlighting}[]
\FunctionTok{cat}\NormalTok{(}\StringTok{"  Synchronous speed:"}\NormalTok{, sync\_45, }\StringTok{"RPM}\SpecialCharTok{\textbackslash{}n}\StringTok{"}\NormalTok{)}
\end{Highlighting}
\end{Shaded}

\begin{verbatim}
##   Synchronous speed: 1350 RPM
\end{verbatim}

\begin{Shaded}
\begin{Highlighting}[]
\FunctionTok{cat}\NormalTok{(}\StringTok{"  Motor speed (approx):"}\NormalTok{, speed\_45, }\StringTok{"RPM}\SpecialCharTok{\textbackslash{}n}\StringTok{"}\NormalTok{)}
\end{Highlighting}
\end{Shaded}

\begin{verbatim}
##   Motor speed (approx): 1300 RPM
\end{verbatim}

\begin{Shaded}
\begin{Highlighting}[]
\FunctionTok{cat}\NormalTok{(}\StringTok{"  VFD output voltage:"}\NormalTok{, voltage\_45, }\StringTok{"V}\SpecialCharTok{\textbackslash{}n}\StringTok{"}\NormalTok{)}
\end{Highlighting}
\end{Shaded}

\begin{verbatim}
##   VFD output voltage: 345 V
\end{verbatim}

\textbf{Note:} The calculation assumes constant slip in RPM, which is an approximation. Actual slip varies somewhat with load and frequency. The V/Hz ratio is maintained constant to provide constant torque capability below base speed.

\textbf{Question 5}: Compare Ethernet/IP and PROFINET. When would you choose each?

\textbf{Answer:}

\begin{longtable}[]{@{}
  >{\raggedright\arraybackslash}p{(\linewidth - 4\tabcolsep) * \real{0.2812}}
  >{\raggedright\arraybackslash}p{(\linewidth - 4\tabcolsep) * \real{0.4062}}
  >{\raggedright\arraybackslash}p{(\linewidth - 4\tabcolsep) * \real{0.3125}}@{}}
\toprule\noalign{}
\begin{minipage}[b]{\linewidth}\raggedright
Feature
\end{minipage} & \begin{minipage}[b]{\linewidth}\raggedright
Ethernet/IP
\end{minipage} & \begin{minipage}[b]{\linewidth}\raggedright
PROFINET
\end{minipage} \\
\midrule\noalign{}
\endhead
\bottomrule\noalign{}
\endlastfoot
\textbf{Developer} & ODVA (Rockwell Automation) & PROFIBUS International (Siemens) \\
\textbf{Base Protocol} & CIP over TCP/UDP & Based on standard Ethernet \\
\textbf{Real-time} & CIP Motion for motion control & IRT (Isochronous Real-Time) \\
\textbf{Determinism} & Standard: non-deterministic; CIP Sync: deterministic & RT: soft real-time; IRT: hard real-time \\
\textbf{Typical Cycle} & 2-10ms (standard); \textless1ms (CIP Motion) & 1-10ms (RT); \textless1ms (IRT) \\
\textbf{Ecosystem} & Allen-Bradley, DeviceNet-heritage & Siemens, PROFIBUS-heritage \\
\end{longtable}

\textbf{Choose Ethernet/IP when:}
- Using Allen-Bradley/Rockwell PLCs
- Integrating with DeviceNet legacy systems
- Standard industrial Ethernet needs
- North American installations (common)
- Need CIP protocol compatibility

\textbf{Choose PROFINET when:}
- Using Siemens PLCs (S7-1200, S7-1500)
- Migrating from PROFIBUS installations
- Need IRT for high-performance motion
- European installations (common)
- Require I-Device functionality

\textbf{General Guidance:}
- Match the protocol to your PLC vendor's ecosystem
- Both are capable industrial Ethernet solutions
- Consider existing infrastructure and expertise
- For motion, evaluate specific timing requirements

\textbf{Question 6}: What is the difference between HMI and SCADA? Where would each be used?

\textbf{Answer:}

\textbf{HMI (Human-Machine Interface):}
- Local operator interface for a single machine or process
- Typically a touchscreen panel mounted on or near equipment
- Communicates directly with one or few PLCs
- Provides real-time control and monitoring
- Limited data storage (trends, alarms)
- Examples: Allen-Bradley PanelView, Siemens Comfort Panel

\textbf{SCADA (Supervisory Control and Data Acquisition):}
- Centralized system monitoring multiple processes/locations
- Server-based architecture with multiple client workstations
- Communicates with many PLCs/RTUs across large distances
- Supervisory control - high-level commands, not direct control
- Extensive historical data collection and analysis
- Examples: Wonderware, Ignition, FactoryTalk View SE

\textbf{Where Each is Used:}

\begin{longtable}[]{@{}lll@{}}
\toprule\noalign{}
Application & HMI & SCADA \\
\midrule\noalign{}
\endhead
\bottomrule\noalign{}
\endlastfoot
Single CNC machine & ✓ & \\
Packaging line & ✓ & \\
Entire factory floor & & ✓ \\
Water treatment plant & ✓ (local) & ✓ (central) \\
Oil pipeline network & & ✓ \\
Building automation & ✓ (per zone) & ✓ (campus-wide) \\
Power grid & & ✓ \\
\end{longtable}

\textbf{Key Distinction:} HMI is typically embedded/local for one process; SCADA is a system supervising many processes across the enterprise.

\textbf{Question 7}: Explain what SIL 3 means and give an example of a SIL 3 application.

\textbf{Answer:}

\textbf{SIL 3 (Safety Integrity Level 3)} is defined in IEC 61508 and represents:

\begin{itemize}
\tightlist
\item
  \textbf{PFD (Probability of Failure on Demand):} 10⁻⁴ to 10⁻³ (0.01\% to 0.1\%)
\item
  \textbf{Risk Reduction Factor:} 1,000 to 10,000
\item
  \textbf{Availability:} 99.9\% to 99.99\%
\end{itemize}

\textbf{What SIL 3 Means:}
- The safety function will fail to operate when needed less than 1 in 1,000 demands
- Requires redundant architecture (typically 2oo3 or 1oo2D)
- Requires certified safety-rated components
- Requires rigorous design, testing, and validation processes
- Requires regular proof testing and maintenance

\textbf{SIL 3 Requirements:}
- Hardware fault tolerance ≥ 1 (single fault won't cause dangerous failure)
- Safe Failure Fraction \textgreater{} 90\% (for Type A components)
- Systematic capability rating of SC3
- Extensive documentation and lifecycle management

\textbf{SIL 3 Application Examples:}

\begin{enumerate}
\def\labelenumi{\arabic{enumi}.}
\tightlist
\item
  \textbf{Emergency Shutdown System (ESD)} in oil refinery

  \begin{itemize}
  \tightlist
  \item
    Detects dangerous conditions (high pressure, fire)
  \item
    Shuts down process and isolates fuel sources
  \item
    Failure could result in explosion and fatalities
  \end{itemize}
\item
  \textbf{Burner Management System (BMS)}

  \begin{itemize}
  \tightlist
  \item
    Controls furnace/boiler ignition sequence
  \item
    Monitors flame presence and fuel/air ratio
  \item
    Prevents furnace explosions
  \end{itemize}
\item
  \textbf{High Integrity Pressure Protection System (HIPPS)}

  \begin{itemize}
  \tightlist
  \item
    Prevents overpressure in pipelines
  \item
    Isolates high-pressure source before relief valve capacity exceeded
  \item
    Protects against catastrophic pipeline rupture
  \end{itemize}
\end{enumerate}

\textbf{Question 8}: Describe three benefits of implementing IIoT in a manufacturing facility and give a specific example of each.

\textbf{Answer:}

\textbf{1. Predictive Maintenance}

\emph{Benefit:} Detect equipment degradation before failure, enabling planned maintenance during scheduled downtime rather than emergency repairs.

\emph{Example:} Vibration sensors on a conveyor gearbox transmit data to cloud analytics. Machine learning algorithms detect bearing wear signature 3 weeks before expected failure. Maintenance schedules replacement during next planned shutdown, avoiding \$50,000 in lost production from unplanned downtime.

\textbf{2. Real-Time Production Visibility}

\emph{Benefit:} Instant access to production metrics, OEE, and quality data from anywhere, enabling faster decision-making.

\emph{Example:} Dashboard shows real-time OEE for all production lines on plant manager's tablet. Sudden drop in Line 3 performance triggers alert. Investigation reveals material feed issue; corrected within 30 minutes instead of waiting for end-of-shift report.

\textbf{3. Energy Optimization}

\emph{Benefit:} Monitor energy consumption at machine level, identify waste, and optimize based on production schedule.

\emph{Example:} Power meters on each machine report to central system. Analysis reveals HVAC in unused areas runs at full capacity during weekends. Automated scheduling reduces weekend HVAC operation, saving 15\% on energy bills (\$30,000/year).

\textbf{Additional Benefits:}

\begin{enumerate}
\def\labelenumi{\arabic{enumi}.}
\setcounter{enumi}{3}
\tightlist
\item
  \textbf{Quality Traceability} - Every part traced through production with complete process data
\item
  \textbf{Remote Monitoring} - Engineers can diagnose issues without traveling to site
\item
  \textbf{Supply Chain Integration} - Real-time inventory and production data shared with suppliers
\end{enumerate}

\begin{center}\rule{0.5\linewidth}{0.5pt}\end{center}

\section{References}\label{references-12}

\begin{enumerate}
\def\labelenumi{\arabic{enumi}.}
\item
  Bolton, W. (2015). \emph{Programmable Logic Controllers} (6th ed.). Newnes.
\item
  Petruzella, F.D. (2017). \emph{Programmable Logic Controllers} (5th ed.). McGraw-Hill Education.
\item
  Rehg, J.A., \& Sartori, G.J. (2013). \emph{Industrial Electronics}. Pearson.
\item
  IEC 61131-3:2013. \emph{Programmable Controllers - Part 3: Programming Languages}.
\item
  IEC 61508:2010. \emph{Functional Safety of Electrical/Electronic/Programmable Electronic Safety-related Systems}.
\item
  ODVA. (2020). \emph{The CIP Networks Library}. ODVA, Inc.
\item
  Siemens AG. (2019). \emph{PROFINET System Description}. Siemens.
\item
  Gilchrist, A. (2016). \emph{Industry 4.0: The Industrial Internet of Things}. Apress.
\item
  ISA-95/IEC 62264. \emph{Enterprise-Control System Integration}.
\item
  Groover, M.P. (2016). \emph{Automation, Production Systems, and Computer-Integrated Manufacturing} (4th ed.). Pearson.
\end{enumerate}

\chapter{Quality Management Systems}\label{quality-management-systems}

\section{Learning Objectives}\label{learning-objectives-13}

After completing this chapter, you will be able to:

\begin{enumerate}
\def\labelenumi{\arabic{enumi}.}
\tightlist
\item
  Define quality management systems and explain their purpose in manufacturing
\item
  Describe the structure and requirements of ISO 9001:2015
\item
  Identify industry-specific quality standards (IATF 16949, AS9100, ISO 22000)
\item
  Explain the documentation hierarchy and control requirements
\item
  Conduct internal audits using systematic methodologies
\item
  Implement effective corrective and preventive action (CAPA) processes
\item
  Understand management review and continuous improvement requirements
\item
  Integrate quality tools (SPC, PFMEA, MSA) within a QMS framework
\end{enumerate}

\begin{center}\rule{0.5\linewidth}{0.5pt}\end{center}

\section{Introduction to Quality Management Systems}\label{introduction-to-quality-management-systems}

A \textbf{Quality Management System (QMS)} is a formalized system that documents processes, procedures, and responsibilities for achieving quality policies and objectives. It coordinates and directs an organization's activities to meet customer and regulatory requirements while continuously improving effectiveness.

\subsection{Why Implement a QMS?}\label{why-implement-a-qms}

\pandocbounded{\includegraphics[keepaspectratio]{introduction_files/figure-latex/qms-benefits-1.pdf}}

\subsection{The Evolution of Quality Management}\label{the-evolution-of-quality-management}

\begin{verbatim}
## Warning in latex_new_row_builder(target_row, table_info, bold, italic,
## monospace, : Setting full_width = TRUE will turn the table into a tabu
## environment where colors are not really easily configable with this package.
## Please consider turn off full_width.
\end{verbatim}

\begin{table}
\centering
\caption{\label{tab:quality-evolution}Evolution of Quality Management}
\centering
\begin{tabu} to \linewidth {>{\raggedright}X>{\raggedright}X>{\raggedright}X>{\raggedright}X}
\hline
Era & Approach & Focus & Key Figures/Standards\\
\hline
\textbf{Pre-1900s} & Craftsmanship & Individual skill; master craftsmen & Guilds\\
\hline
\textbf{1900-1940s} & Inspection & Sort good from bad; end-of-line inspection & Taylor\\
\hline
\textbf{1940-1960s} & Statistical Quality Control & Control charts; sampling; Shewhart, Deming & Shewhart, Deming\\
\hline
\textbf{1960-1980s} & Quality Assurance & Prevention; systems approach; documented procedures & Juran, Feigenbaum\\
\hline
\textbf{1980-2000s} & Total Quality Management & Company-wide quality; customer focus; continuous improvement & Crosby, ISO 9000\\
\hline
\cellcolor[HTML]{d5f5e3}{\textbf{2000s-Present}} & \cellcolor[HTML]{d5f5e3}{Integrated Management} & \cellcolor[HTML]{d5f5e3}{Risk-based thinking; integration with business strategy; digital quality} & \cellcolor[HTML]{d5f5e3}{ISO 9001:2015, Industry 4.0}\\
\hline
\end{tabu}
\end{table}

\subsection{Quality Management Principles}\label{quality-management-principles}

ISO 9000 defines seven quality management principles that form the foundation of all QMS standards:

\pandocbounded{\includegraphics[keepaspectratio]{introduction_files/figure-latex/qm-principles-1.pdf}}

\begin{center}\rule{0.5\linewidth}{0.5pt}\end{center}

\section{ISO 9001:2015 Structure}\label{iso-90012015-structure}

\textbf{ISO 9001:2015} is the international standard for quality management systems. It uses the High-Level Structure (HLS) common to all ISO management system standards.

\subsection{The Ten Clauses}\label{the-ten-clauses}

\pandocbounded{\includegraphics[keepaspectratio]{introduction_files/figure-latex/iso-clauses-1.pdf}}

\subsection{The Process Approach and PDCA}\label{the-process-approach-and-pdca}

ISO 9001:2015 emphasizes the \textbf{process approach} combined with the \textbf{Plan-Do-Check-Act (PDCA)} cycle:

\pandocbounded{\includegraphics[keepaspectratio]{introduction_files/figure-latex/pdca-cycle-1.pdf}}

\subsection{Risk-Based Thinking}\label{risk-based-thinking}

ISO 9001:2015 introduced \textbf{risk-based thinking} throughout the QMS:

\begin{table}
\centering
\caption{\label{tab:risk-based}Risk-Based Thinking in ISO 9001:2015}
\centering
\begin{tabu} to \linewidth {>{\raggedright}X>{\raggedright}X>{\raggedright}X}
\hline
Clause & Requirement & Risk\_Application\\
\hline
\textbf{4.1} & Context of the Organization & Identify internal/external issues that could affect QMS outcomes\\
\hline
\textbf{4.2} & Interested Parties & Understand needs/expectations that could impact quality\\
\hline
\textbf{6.1} & Actions to Address Risks and Opportunities & Plan actions to address risks and opportunities; integrate into processes\\
\hline
\textbf{8.1} & Operational Planning and Control & Control processes considering risks identified in planning\\
\hline
\textbf{9.1} & Monitoring and Measurement & Monitor effectiveness of actions taken to address risks\\
\hline
\textbf{10.2} & Nonconformity and Corrective Action & Analyze nonconformities; update risks/opportunities as needed\\
\hline
\end{tabu}
\end{table}

\textbf{Understanding Risk-Based Thinking}

\textbf{Risk-based thinking} doesn't require formal risk management (like FMEA for every process), but rather:

\begin{enumerate}
\def\labelenumi{\arabic{enumi}.}
\tightlist
\item
  \textbf{Consider risks} when designing and implementing the QMS
\item
  \textbf{Preventive action} is built into the system, not a separate activity
\item
  \textbf{Proportional response} - more rigorous controls for higher-risk processes
\item
  \textbf{Opportunities} are also considered (not just negative risks)
\end{enumerate}

\textbf{Example:} A food manufacturer identifies ``allergen cross-contamination'' as a significant risk. Risk-based thinking leads them to:
- Dedicated production lines for allergen-containing products
- Enhanced cleaning verification procedures
- Staff training on allergen awareness
- Increased inspection frequency after changeovers

The extent of controls is proportional to the risk level.

\begin{center}\rule{0.5\linewidth}{0.5pt}\end{center}

\section{Industry-Specific Quality Standards}\label{industry-specific-quality-standards}

Different industries have developed standards that build upon ISO 9001 with additional sector-specific requirements.

\subsection{Comparison of Major Standards}\label{comparison-of-major-standards}

\pandocbounded{\includegraphics[keepaspectratio]{introduction_files/figure-latex/standards-comparison-1.pdf}}

\subsection{IATF 16949:2016 (Automotive)}\label{iatf-169492016-automotive}

\textbf{IATF 16949} is the automotive quality standard, required by most major OEMs (Ford, GM, Toyota, VW, etc.).

\begin{table}
\centering
\caption{\label{tab:iatf-additions}IATF 16949 Additional Requirements Beyond ISO 9001}
\centering
\begin{tabu} to \linewidth {>{\raggedright}X>{\raggedright}X>{\raggedright}X}
\hline
Requirement Area & ISO 9001 & IATF 16949 Addition\\
\hline
\textbf{Product Safety} & \cellcolor[HTML]{fadbd8}{Implied} & \cellcolor[HTML]{d5f5e3}{Explicit requirements for product safety processes}\\
\hline
\textbf{APQP/PPAP} & \cellcolor[HTML]{fadbd8}{Not specified} & \cellcolor[HTML]{d5f5e3}{APQP phases required; PPAP submission mandatory}\\
\hline
\textbf{Control Plans} & \cellcolor[HTML]{fadbd8}{Not specified} & \cellcolor[HTML]{d5f5e3}{Control plans required for all parts; specific format}\\
\hline
\textbf{MSA} & \cellcolor[HTML]{fadbd8}{General requirement} & \cellcolor[HTML]{d5f5e3}{MSA studies required per AIAG MSA manual}\\
\hline
\textbf{SPC} & \cellcolor[HTML]{fadbd8}{General requirement} & \cellcolor[HTML]{d5f5e3}{SPC for all special characteristics}\\
\hline
\textbf{Supplier Quality} & \cellcolor[HTML]{fadbd8}{Basic purchasing} & \cellcolor[HTML]{d5f5e3}{Supplier development; second-party audits; IATF certification flow-down}\\
\hline
\textbf{Warranty Management} & \cellcolor[HTML]{fadbd8}{Complaint handling} & \cellcolor[HTML]{d5f5e3}{NTF analysis; warranty data analysis; field return analysis}\\
\hline
\textbf{Manufacturing Feasibility} & \cellcolor[HTML]{fadbd8}{Review of requirements} & \cellcolor[HTML]{d5f5e3}{Documented manufacturing feasibility review}\\
\hline
\end{tabu}
\end{table}

\subsection{Automotive Core Tools}\label{automotive-core-tools}

IATF 16949 requires the use of \textbf{five core tools}:

\pandocbounded{\includegraphics[keepaspectratio]{introduction_files/figure-latex/core-tools-1.pdf}}

\subsection{AS9100D (Aerospace/Defense)}\label{as9100d-aerospacedefense}

\textbf{AS9100D} adds aerospace-specific requirements to ISO 9001:

\begin{table}
\centering
\caption{\label{tab:as9100-additions}AS9100D Key Additional Requirements}
\centering
\begin{tabu} to \linewidth {>{\raggedright}X>{\raggedright}X}
\hline
Area & Requirement\\
\hline
\textbf{Configuration Management} & Control product configuration throughout lifecycle; change management\\
\hline
\textbf{First Article Inspection} & FAI per AS9102; documented verification of first production parts\\
\hline
\textbf{Counterfeit Parts Prevention} & Controls to prevent counterfeit parts entering supply chain\\
\hline
\textbf{Special Processes} & NADCAP accreditation for special processes (welding, heat treat, NDT)\\
\hline
\textbf{Product Safety} & Product safety requirements; reporting of unsafe conditions\\
\hline
\textbf{Risk Management} & Explicit risk management process (often using AS/NZS 4360 or similar)\\
\hline
\textbf{Human Factors} & Consideration of human factors in process design\\
\hline
\end{tabu}
\end{table}

\subsection{ISO 22000:2018 (Food Safety)}\label{iso-220002018-food-safety}

\textbf{ISO 22000} integrates quality management with \textbf{HACCP} (Hazard Analysis Critical Control Points):

\begin{table}
\centering
\caption{\label{tab:iso22000-structure}ISO 22000 Food Safety Management Elements}
\centering
\begin{tabu} to \linewidth {>{\raggedright}X>{\raggedright}X>{\raggedright}X}
\hline
Element & Description & Example\\
\hline
\textbf{Prerequisite Programs (PRPs)} & Basic hygiene conditions: cleaning, pest control, personnel hygiene, facility design & Sanitation schedule, hand washing stations, pest control contract\\
\hline
\textbf{Operational PRPs (OPRPs)} & PRPs essential to control specific identified hazards & Metal detector for physical hazard control\\
\hline
\textbf{HACCP Plan} & Critical Control Points (CCPs) for significant hazards; critical limits; monitoring & Pasteurization temperature ≥72°C for 15 seconds\\
\hline
\textbf{Traceability} & One-step-back, one-step-forward traceability; recall procedures & Lot coding, supplier records, distribution records\\
\hline
\textbf{Emergency Preparedness} & Procedures for food safety emergencies; recall/withdrawal & Product recall procedure, mock recall exercises\\
\hline
\textbf{Validation and Verification} & Validate control measures; verify HACCP plan effectiveness & Challenge studies, environmental monitoring, internal audits\\
\hline
\end{tabu}
\end{table}

\begin{center}\rule{0.5\linewidth}{0.5pt}\end{center}

\section{Documentation Requirements}\label{documentation-requirements}

A QMS requires documented information to ensure consistency, provide evidence, and enable improvement.

\subsection{Documentation Hierarchy}\label{documentation-hierarchy}

\pandocbounded{\includegraphics[keepaspectratio]{introduction_files/figure-latex/doc-hierarchy-1.pdf}}

\subsection{Document Control Requirements}\label{document-control-requirements}

\begin{table}
\centering
\caption{\label{tab:doc-control}Document Control Requirements (ISO 9001 Clause 7.5)}
\centering
\begin{tabu} to \linewidth {>{\raggedright}X>{\raggedright}X>{\raggedright}X}
\hline
Requirement & Description & Implementation\\
\hline
\textbf{Approval} & Documents reviewed and approved for adequacy before issue & Approval signatures or electronic workflow\\
\hline
\textbf{Review \& Update} & Reviewed, updated as necessary, and re-approved & Periodic review schedule; change request process\\
\hline
\textbf{Identification} & Identified with title, date, revision, author & Document numbering system; revision control\\
\hline
\textbf{Availability} & Relevant versions available at points of use & Controlled copies at workstations; electronic access\\
\hline
\textbf{Protection} & Protected from loss, damage, unauthorized changes & Backup systems; access controls; original storage\\
\hline
\textbf{Obsolete Control} & Obsolete documents identified and prevented from unintended use & Stamp 'OBSOLETE'; remove from use; archive\\
\hline
\textbf{External Documents} & External documents identified and distribution controlled & List of external documents; master copy control\\
\hline
\textbf{Record Retention} & Records retained for specified periods; protected; retrievable & Retention matrix; secure storage; indexing\\
\hline
\end{tabu}
\end{table}

\subsection{Documented Information Required by ISO 9001:2015}\label{documented-information-required-by-iso-90012015}

\begin{table}
\centering
\caption{\label{tab:mandatory-docs}Documented Information Required by ISO 9001:2015}
\centering
\begin{tabu} to \linewidth {>{\raggedright}X>{\raggedright}X>{\raggedright}X}
\hline
Type & Clause & Requirement\\
\hline
\multicolumn{3}{l}{\cellcolor[HTML]{ebf5fb}{\textbf{Documents (shall be maintained)}}}\\
\hline
\hspace{1em}\textbf{Documents (shall be maintained)} & 4.3 & Scope of the QMS\\
\hline
\hspace{1em}\textbf{Documents (shall be maintained)} & 5.2 & Quality policy\\
\hline
\hspace{1em}\textbf{Documents (shall be maintained)} & 6.2 & Quality objectives\\
\hline
\hspace{1em}\textbf{Documents (shall be maintained)} & 7.1.5 & Monitoring and measuring resources (calibration)\\
\hline
\hspace{1em}\textbf{Documents (shall be maintained)} & 7.2 & Evidence of competence\\
\hline
\hspace{1em}\textbf{Documents (shall be maintained)} & 8.1 & Evidence of conformity to processes\\
\hline
\hspace{1em}\textbf{Documents (shall be maintained)} & 8.5.1 & Controlled conditions for production\\
\hline
\multicolumn{3}{l}{\cellcolor[HTML]{eafaf1}{\textbf{Records (shall be retained)}}}\\
\hline
\hspace{1em}\textbf{Records (shall be retained)} & 7.1.5.1 & Calibration/verification results\\
\hline
\hspace{1em}\textbf{Records (shall be retained)} & 7.2 & Training records\\
\hline
\hspace{1em}\textbf{Records (shall be retained)} & 8.2.3.2 & Results of review of requirements\\
\hline
\hspace{1em}\textbf{Records (shall be retained)} & 8.3.3 & Design and development inputs\\
\hline
\hspace{1em}\textbf{Records (shall be retained)} & 8.3.4 & Design and development controls\\
\hline
\hspace{1em}\textbf{Records (shall be retained)} & 8.3.5 & Design and development outputs\\
\hline
\hspace{1em}\textbf{Records (shall be retained)} & 8.3.6 & Design and development changes\\
\hline
\hspace{1em}\textbf{Records (shall be retained)} & 8.5.2 & Traceability requirements\\
\hline
\hspace{1em}\textbf{Records (shall be retained)} & 8.6 & Release of products/services\\
\hline
\hspace{1em}\textbf{Records (shall be retained)} & 9.1.1 & Results of monitoring and measurement\\
\hline
\end{tabu}
\end{table}

\begin{center}\rule{0.5\linewidth}{0.5pt}\end{center}

\section{Internal Auditing}\label{internal-auditing}

\textbf{Internal audits} are systematic, independent evaluations of the QMS to determine if it conforms to requirements and is effectively implemented.

\subsection{The Audit Process}\label{the-audit-process}

\pandocbounded{\includegraphics[keepaspectratio]{introduction_files/figure-latex/audit-process-1.pdf}}

\subsection{Audit Checklist Example}\label{audit-checklist-example}

\begin{table}
\centering
\caption{\label{tab:audit-checklist}Sample Audit Checklist: Clause 7.1.5 Monitoring and Measuring Resources}
\centering
\begin{tabu} to \linewidth {>{\raggedright}X>{\raggedright}X>{\raggedright}X>{\raggedright}X}
\hline
Clause & Requirement & Question & Evidence\\
\hline
\textbf{7.1.5} & Monitoring and measuring resources suitable for the type of activities & How do you determine what measuring equipment is needed? & Gauge selection procedure\\
\hline
\textbf{7.1.5} & Resources maintained to ensure continued fitness for purpose & What maintenance is performed on measuring equipment? & PM records\\
\hline
\textbf{7.1.5} & Documented information on fitness for purpose retained & Can you show me calibration records for this micrometer? & Calibration certificate\\
\hline
\textbf{7.1.5} & Measuring equipment identified to determine status & How can I tell if this gauge is currently calibrated? & Calibration stickers/tags\\
\hline
\textbf{7.1.5.2} & Measurement traceability when required & What standards are these gauges calibrated against? & Traceability certificates\\
\hline
\end{tabu}
\end{table}

\subsection{Types of Audit Findings}\label{types-of-audit-findings}

\begin{verbatim}
## Warning in latex_new_row_builder(target_row, table_info, bold, italic,
## monospace, : Setting full_width = TRUE will turn the table into a tabu
## environment where colors are not really easily configable with this package.
## Please consider turn off full_width.
## Warning in latex_new_row_builder(target_row, table_info, bold, italic,
## monospace, : Setting full_width = TRUE will turn the table into a tabu
## environment where colors are not really easily configable with this package.
## Please consider turn off full_width.
## Warning in latex_new_row_builder(target_row, table_info, bold, italic,
## monospace, : Setting full_width = TRUE will turn the table into a tabu
## environment where colors are not really easily configable with this package.
## Please consider turn off full_width.
## Warning in latex_new_row_builder(target_row, table_info, bold, italic,
## monospace, : Setting full_width = TRUE will turn the table into a tabu
## environment where colors are not really easily configable with this package.
## Please consider turn off full_width.
\end{verbatim}

\begin{table}
\centering
\caption{\label{tab:finding-types}Types of Audit Findings}
\centering
\begin{tabu} to \linewidth {>{\raggedright}X>{\raggedright}X>{\raggedright}X>{\raggedright}X}
\hline
Finding Type & Definition & Example & Action Required\\
\hline
\cellcolor[HTML]{f8d7da}{\textbf{Major Nonconformity}} & \cellcolor[HTML]{f8d7da}{Absence or total breakdown of a system; significant risk to product/customer} & \cellcolor[HTML]{f8d7da}{No calibration program exists; all gauges are uncontrolled} & \cellcolor[HTML]{f8d7da}{Immediate corrective action; root cause analysis; may affect certification}\\
\hline
\cellcolor[HTML]{fff3cd}{\textbf{Minor Nonconformity}} & \cellcolor[HTML]{fff3cd}{Single lapse or isolated incident; system exists but not fully effective} & \cellcolor[HTML]{fff3cd}{3 of 50 gauges found past due for calibration} & \cellcolor[HTML]{fff3cd}{Corrective action required; typically 30-90 days}\\
\hline
\cellcolor[HTML]{d1ecf1}{\textbf{Observation}} & \cellcolor[HTML]{d1ecf1}{Weakness that could lead to nonconformity if not addressed} & \cellcolor[HTML]{d1ecf1}{Calibration procedure doesn't specify what to do if gauge fails} & \cellcolor[HTML]{d1ecf1}{No formal corrective action; monitor situation}\\
\hline
\cellcolor[HTML]{d4edda}{\textbf{Opportunity for Improvement}} & \cellcolor[HTML]{d4edda}{Suggestion for enhancement; not a nonconformity} & \cellcolor[HTML]{d4edda}{Consider using electronic calibration records for faster retrieval} & \cellcolor[HTML]{d4edda}{Consider for improvement; no requirement}\\
\hline
\end{tabu}
\end{table}

\subsection{Auditor Competence}\label{auditor-competence}

\begin{table}
\centering
\caption{\label{tab:auditor-competence}Internal Auditor Competency Requirements}
\centering
\begin{tabu} to \linewidth {>{\raggedright}X>{\raggedright}X>{\raggedright}X}
\hline
Competency & Description & How\_Developed\\
\hline
\textbf{Knowledge of Standards} & Understanding of ISO 9001 and applicable industry standards & Training courses; self-study; certification (CQA, Lead Auditor)\\
\hline
\textbf{Audit Techniques} & Planning, interviewing, evidence collection, reporting & Auditor training; shadowing experienced auditors\\
\hline
\textbf{Industry Knowledge} & Understanding of processes, products, and industry practices & Work experience; process knowledge; technical training\\
\hline
\textbf{Communication Skills} & Active listening, clear questioning, professional writing & Practice; feedback; soft skills training\\
\hline
\textbf{Objectivity} & Independence from area being audited; impartial & Audit program design; rotation of assignments\\
\hline
\textbf{Professional Judgment} & Ability to evaluate significance of findings & Experience; mentoring; calibration with other auditors\\
\hline
\end{tabu}
\end{table}

\begin{center}\rule{0.5\linewidth}{0.5pt}\end{center}

\section{Corrective and Preventive Action (CAPA)}\label{corrective-and-preventive-action-capa}

\textbf{Corrective action} eliminates the cause of a detected nonconformity to prevent recurrence. \textbf{Preventive action} eliminates the cause of a potential nonconformity to prevent occurrence.

\subsection{The CAPA Process}\label{the-capa-process}

\pandocbounded{\includegraphics[keepaspectratio]{introduction_files/figure-latex/capa-process-1.pdf}}

\subsection{CAPA Effectiveness Criteria}\label{capa-effectiveness-criteria}

\begin{Shaded}
\begin{Highlighting}[]
\CommentTok{\# CAPA Effectiveness Tracking}
\FunctionTok{set.seed}\NormalTok{(}\DecValTok{42}\NormalTok{)}

\CommentTok{\# Simulate CAPA data for a year}
\NormalTok{months }\OtherTok{\textless{}{-}} \FunctionTok{c}\NormalTok{(}\StringTok{"Jan"}\NormalTok{, }\StringTok{"Feb"}\NormalTok{, }\StringTok{"Mar"}\NormalTok{, }\StringTok{"Apr"}\NormalTok{, }\StringTok{"May"}\NormalTok{, }\StringTok{"Jun"}\NormalTok{,}
            \StringTok{"Jul"}\NormalTok{, }\StringTok{"Aug"}\NormalTok{, }\StringTok{"Sep"}\NormalTok{, }\StringTok{"Oct"}\NormalTok{, }\StringTok{"Nov"}\NormalTok{, }\StringTok{"Dec"}\NormalTok{)}

\NormalTok{capa\_data }\OtherTok{\textless{}{-}} \FunctionTok{data.frame}\NormalTok{(}
  \AttributeTok{Month =} \FunctionTok{factor}\NormalTok{(months, }\AttributeTok{levels =}\NormalTok{ months),}
  \AttributeTok{Total\_CAPAs =} \FunctionTok{c}\NormalTok{(}\DecValTok{12}\NormalTok{, }\DecValTok{15}\NormalTok{, }\DecValTok{18}\NormalTok{, }\DecValTok{14}\NormalTok{, }\DecValTok{11}\NormalTok{, }\DecValTok{13}\NormalTok{, }\DecValTok{10}\NormalTok{, }\DecValTok{9}\NormalTok{, }\DecValTok{8}\NormalTok{, }\DecValTok{7}\NormalTok{, }\DecValTok{9}\NormalTok{, }\DecValTok{6}\NormalTok{),}
  \AttributeTok{Closed\_On\_Time =} \FunctionTok{c}\NormalTok{(}\DecValTok{8}\NormalTok{, }\DecValTok{10}\NormalTok{, }\DecValTok{12}\NormalTok{, }\DecValTok{11}\NormalTok{, }\DecValTok{9}\NormalTok{, }\DecValTok{11}\NormalTok{, }\DecValTok{9}\NormalTok{, }\DecValTok{8}\NormalTok{, }\DecValTok{7}\NormalTok{, }\DecValTok{6}\NormalTok{, }\DecValTok{8}\NormalTok{, }\DecValTok{5}\NormalTok{),}
  \AttributeTok{Effective =} \FunctionTok{c}\NormalTok{(}\DecValTok{7}\NormalTok{, }\DecValTok{9}\NormalTok{, }\DecValTok{10}\NormalTok{, }\DecValTok{10}\NormalTok{, }\DecValTok{8}\NormalTok{, }\DecValTok{10}\NormalTok{, }\DecValTok{8}\NormalTok{, }\DecValTok{8}\NormalTok{, }\DecValTok{7}\NormalTok{, }\DecValTok{6}\NormalTok{, }\DecValTok{8}\NormalTok{, }\DecValTok{5}\NormalTok{),}
  \AttributeTok{Recurrence =} \FunctionTok{c}\NormalTok{(}\DecValTok{2}\NormalTok{, }\DecValTok{1}\NormalTok{, }\DecValTok{3}\NormalTok{, }\DecValTok{1}\NormalTok{, }\DecValTok{0}\NormalTok{, }\DecValTok{1}\NormalTok{, }\DecValTok{0}\NormalTok{, }\DecValTok{0}\NormalTok{, }\DecValTok{1}\NormalTok{, }\DecValTok{0}\NormalTok{, }\DecValTok{0}\NormalTok{, }\DecValTok{0}\NormalTok{)}
\NormalTok{)}

\CommentTok{\# Calculate metrics}
\NormalTok{capa\_data}\SpecialCharTok{$}\NormalTok{On\_Time\_Rate }\OtherTok{\textless{}{-}} \FunctionTok{round}\NormalTok{(capa\_data}\SpecialCharTok{$}\NormalTok{Closed\_On\_Time }\SpecialCharTok{/}\NormalTok{ capa\_data}\SpecialCharTok{$}\NormalTok{Total\_CAPAs }\SpecialCharTok{*} \DecValTok{100}\NormalTok{, }\DecValTok{1}\NormalTok{)}
\NormalTok{capa\_data}\SpecialCharTok{$}\NormalTok{Effectiveness\_Rate }\OtherTok{\textless{}{-}} \FunctionTok{round}\NormalTok{(capa\_data}\SpecialCharTok{$}\NormalTok{Effective }\SpecialCharTok{/}\NormalTok{ capa\_data}\SpecialCharTok{$}\NormalTok{Closed\_On\_Time }\SpecialCharTok{*} \DecValTok{100}\NormalTok{, }\DecValTok{1}\NormalTok{)}

\FunctionTok{cat}\NormalTok{(}\StringTok{"CAPA Performance Metrics Summary:}\SpecialCharTok{\textbackslash{}n}\StringTok{"}\NormalTok{)}
\end{Highlighting}
\end{Shaded}

\begin{verbatim}
## CAPA Performance Metrics Summary:
\end{verbatim}

\begin{Shaded}
\begin{Highlighting}[]
\FunctionTok{cat}\NormalTok{(}\StringTok{"─────────────────────────────────}\SpecialCharTok{\textbackslash{}n}\StringTok{"}\NormalTok{)}
\end{Highlighting}
\end{Shaded}

\begin{verbatim}
## ─────────────────────────────────
\end{verbatim}

\begin{Shaded}
\begin{Highlighting}[]
\FunctionTok{cat}\NormalTok{(}\StringTok{"Total CAPAs initiated:"}\NormalTok{, }\FunctionTok{sum}\NormalTok{(capa\_data}\SpecialCharTok{$}\NormalTok{Total\_CAPAs), }\StringTok{"}\SpecialCharTok{\textbackslash{}n}\StringTok{"}\NormalTok{)}
\end{Highlighting}
\end{Shaded}

\begin{verbatim}
## Total CAPAs initiated: 132
\end{verbatim}

\begin{Shaded}
\begin{Highlighting}[]
\FunctionTok{cat}\NormalTok{(}\StringTok{"Closed on time:"}\NormalTok{, }\FunctionTok{sum}\NormalTok{(capa\_data}\SpecialCharTok{$}\NormalTok{Closed\_On\_Time), }\StringTok{"("}\NormalTok{,}
    \FunctionTok{round}\NormalTok{(}\FunctionTok{sum}\NormalTok{(capa\_data}\SpecialCharTok{$}\NormalTok{Closed\_On\_Time)}\SpecialCharTok{/}\FunctionTok{sum}\NormalTok{(capa\_data}\SpecialCharTok{$}\NormalTok{Total\_CAPAs)}\SpecialCharTok{*}\DecValTok{100}\NormalTok{, }\DecValTok{1}\NormalTok{), }\StringTok{"\%)}\SpecialCharTok{\textbackslash{}n}\StringTok{"}\NormalTok{)}
\end{Highlighting}
\end{Shaded}

\begin{verbatim}
## Closed on time: 104 ( 78.8 %)
\end{verbatim}

\begin{Shaded}
\begin{Highlighting}[]
\FunctionTok{cat}\NormalTok{(}\StringTok{"Verified effective:"}\NormalTok{, }\FunctionTok{sum}\NormalTok{(capa\_data}\SpecialCharTok{$}\NormalTok{Effective), }\StringTok{"("}\NormalTok{,}
    \FunctionTok{round}\NormalTok{(}\FunctionTok{sum}\NormalTok{(capa\_data}\SpecialCharTok{$}\NormalTok{Effective)}\SpecialCharTok{/}\FunctionTok{sum}\NormalTok{(capa\_data}\SpecialCharTok{$}\NormalTok{Closed\_On\_Time)}\SpecialCharTok{*}\DecValTok{100}\NormalTok{, }\DecValTok{1}\NormalTok{), }\StringTok{"\%)}\SpecialCharTok{\textbackslash{}n}\StringTok{"}\NormalTok{)}
\end{Highlighting}
\end{Shaded}

\begin{verbatim}
## Verified effective: 96 ( 92.3 %)
\end{verbatim}

\begin{Shaded}
\begin{Highlighting}[]
\FunctionTok{cat}\NormalTok{(}\StringTok{"Recurrences:"}\NormalTok{, }\FunctionTok{sum}\NormalTok{(capa\_data}\SpecialCharTok{$}\NormalTok{Recurrence), }\StringTok{"}\SpecialCharTok{\textbackslash{}n}\StringTok{"}\NormalTok{)}
\end{Highlighting}
\end{Shaded}

\begin{verbatim}
## Recurrences: 9
\end{verbatim}

\pandocbounded{\includegraphics[keepaspectratio]{introduction_files/figure-latex/capa-chart-1.pdf}}

\begin{center}\rule{0.5\linewidth}{0.5pt}\end{center}

\section{Management Review}\label{management-review}

\textbf{Management review} is a formal evaluation of the QMS by top management to ensure its continuing suitability, adequacy, and effectiveness.

\subsection{Management Review Inputs}\label{management-review-inputs}

\begin{table}
\centering
\caption{\label{tab:mgmt-review-inputs}Management Review Inputs (ISO 9001 Clause 9.3.2)}
\centering
\begin{tabu} to \linewidth {>{\raggedright}X>{\raggedright}X>{\raggedright}X}
\hline
Required Input & Details to Review & Typical Data Sources\\
\hline
\textbf{Previous Review Actions} & Status of actions from previous management reviews & Previous meeting minutes; action log\\
\hline
\textbf{Changes in Issues/Context} & Changes in external/internal issues; interested party requirements & SWOT analysis; customer feedback; regulatory changes\\
\hline
\textbf{Quality Performance} & Customer satisfaction; quality objectives achievement; process performance; nonconformities; audits; supplier performance & KPI dashboards; audit reports; CAPA data; customer complaints\\
\hline
\textbf{Resource Adequacy} & Adequacy of resources for QMS maintenance and improvement & Budget status; staffing levels; equipment condition\\
\hline
\textbf{Risk/Opportunity Actions} & Effectiveness of actions taken to address risks and opportunities & Risk register updates; opportunity tracking\\
\hline
\textbf{Improvement Opportunities} & Opportunities for improvement identified from various sources & Employee suggestions; benchmarking; industry developments\\
\hline
\end{tabu}
\end{table}

\subsection{Management Review Outputs}\label{management-review-outputs}

\begin{table}
\centering
\caption{\label{tab:mgmt-review-outputs}Management Review Outputs (ISO 9001 Clause 9.3.3)}
\centering
\begin{tabu} to \linewidth {>{\raggedright}X>{\raggedright}X>{\raggedright}X}
\hline
Required Output & ISO Requirement & Examples\\
\hline
\textbf{Improvement Decisions} & Decisions and actions related to improvement opportunities & Launch kaizen project for top 3 customer complaints; implement new quality training program\\
\hline
\textbf{QMS Changes} & Any need for changes to the QMS & Revise control plan based on PFMEA update; modify inspection frequency\\
\hline
\textbf{Resource Needs} & Any resource needs & Hire additional quality technician; purchase new CMM; upgrade software\\
\hline
\end{tabu}
\end{table}

\subsection{Sample Management Review Dashboard}\label{sample-management-review-dashboard}

\pandocbounded{\includegraphics[keepaspectratio]{introduction_files/figure-latex/mgmt-dashboard-1.pdf}}

\begin{center}\rule{0.5\linewidth}{0.5pt}\end{center}

\section{Continuous Improvement}\label{continuous-improvement}

\textbf{Continuous improvement} is a fundamental principle of quality management - the ongoing effort to improve products, services, and processes.

\subsection{Improvement Methodologies}\label{improvement-methodologies}

\begin{table}
\centering
\caption{\label{tab:improvement-methods}Continuous Improvement Methodologies}
\centering
\begin{tabu} to \linewidth {>{\raggedright}X>{\raggedright}X>{\raggedright}X>{\raggedright}X}
\hline
Methodology & Origin & Best\_For & Steps\\
\hline
\textbf{PDCA} & Deming/Shewhart & General improvement cycle; process changes & Plan-Do-Check-Act\\
\hline
\textbf{DMAIC} & Six Sigma & Data-driven problem solving; reducing variation & Define-Measure-Analyze-Improve-Control\\
\hline
\textbf{Kaizen} & Toyota/Lean & Small, incremental improvements; team-based & Identify waste → Improve → Standardize\\
\hline
\textbf{A3 Problem Solving} & Toyota & Visual problem solving; root cause analysis & Background → Goals → Analysis → Countermeasures → Follow-up\\
\hline
\textbf{8D} & Ford & Customer complaints; complex problems & D0-D8 (Team, Problem, Containment, Root Cause, Actions, Verify, Prevent, Close)\\
\hline
\end{tabu}
\end{table}

\subsection{Key Performance Indicators (KPIs)}\label{key-performance-indicators-kpis}

\begin{table}
\centering
\caption{\label{tab:kpi-examples}Common Quality Management KPIs}
\centering
\begin{tabu} to \linewidth {>{\raggedright}X>{\raggedright}X>{\raggedright}X>{\raggedright}X}
\hline
Category & KPI & Formula & Target\\
\hline
\multicolumn{4}{l}{\cellcolor[HTML]{ebf5fb}{\textbf{Quality Metrics}}}\\
\hline
\hspace{1em}\textbf{Quality} & First Pass Yield (FPY) & (Good units first time / Total units) × 100\% & ≥98\%\\
\hline
\hspace{1em}\textbf{Quality} & Customer Complaints (PPM) & (Complaints / Units shipped) × 1,000,000 & <100\\
\hline
\hspace{1em}\textbf{Quality} & Internal Reject Rate & (Rejected units / Total inspected) × 100\% & <1\%\\
\hline
\hspace{1em}\textbf{Quality} & Cost of Quality & Prevention + Appraisal + Internal Failure + External Failure & <3\% of sales\\
\hline
\multicolumn{4}{l}{\cellcolor[HTML]{eafaf1}{\textbf{Delivery Metrics}}}\\
\hline
\hspace{1em}\textbf{Delivery} & On-Time Delivery (OTD) & (Orders delivered on time / Total orders) × 100\% & ≥98\%\\
\hline
\hspace{1em}\textbf{Delivery} & Lead Time & Order date to delivery date & Industry benchmark\\
\hline
\hspace{1em}\textbf{Delivery} & Order Fulfillment Rate & (Complete orders shipped / Total orders) × 100\% & ≥99\%\\
\hline
\multicolumn{4}{l}{\cellcolor[HTML]{fef9e7}{\textbf{Cost Metrics}}}\\
\hline
\hspace{1em}\textbf{Cost} & Scrap Rate & (Scrap value / Total production value) × 100\% & <2\%\\
\hline
\hspace{1em}\textbf{Cost} & Rework Cost & Labor + Material for rework & Minimize\\
\hline
\hspace{1em}\textbf{Cost} & Warranty Cost & Warranty claims cost / Revenue & <0.5\% of sales\\
\hline
\multicolumn{4}{l}{\cellcolor[HTML]{fdedec}{\textbf{Safety Metrics}}}\\
\hline
\hspace{1em}\textbf{Safety} & Recordable Incident Rate & (Recordable incidents × 200,000) / Hours worked & <1.0\\
\hline
\hspace{1em}\textbf{Safety} & Near Miss Reports & Count of near miss reports & Encourage reporting\\
\hline
\end{tabu}
\end{table}

\subsection{Cost of Quality (COQ)}\label{cost-of-quality-coq}

\begin{Shaded}
\begin{Highlighting}[]
\CommentTok{\# Cost of Quality Analysis}
\NormalTok{coq\_data }\OtherTok{\textless{}{-}} \FunctionTok{data.frame}\NormalTok{(}
  \AttributeTok{Category =} \FunctionTok{c}\NormalTok{(}\StringTok{"Prevention"}\NormalTok{, }\StringTok{"Appraisal"}\NormalTok{, }\StringTok{"Internal Failure"}\NormalTok{, }\StringTok{"External Failure"}\NormalTok{),}
  \AttributeTok{Description =} \FunctionTok{c}\NormalTok{(}\StringTok{"Quality planning, training, process control"}\NormalTok{,}
                  \StringTok{"Inspection, testing, audits, calibration"}\NormalTok{,}
                  \StringTok{"Scrap, rework, reinspection, downtime"}\NormalTok{,}
                  \StringTok{"Warranty, returns, complaints, recalls"}\NormalTok{),}
  \AttributeTok{Cost =} \FunctionTok{c}\NormalTok{(}\DecValTok{50000}\NormalTok{, }\DecValTok{75000}\NormalTok{, }\DecValTok{120000}\NormalTok{, }\DecValTok{180000}\NormalTok{)}
\NormalTok{)}

\NormalTok{total\_coq }\OtherTok{\textless{}{-}} \FunctionTok{sum}\NormalTok{(coq\_data}\SpecialCharTok{$}\NormalTok{Cost)}
\NormalTok{revenue }\OtherTok{\textless{}{-}} \DecValTok{10000000}  \CommentTok{\# $10M revenue}

\NormalTok{coq\_data}\SpecialCharTok{$}\NormalTok{Percent\_of\_COQ }\OtherTok{\textless{}{-}} \FunctionTok{round}\NormalTok{(coq\_data}\SpecialCharTok{$}\NormalTok{Cost }\SpecialCharTok{/}\NormalTok{ total\_coq }\SpecialCharTok{*} \DecValTok{100}\NormalTok{, }\DecValTok{1}\NormalTok{)}
\NormalTok{coq\_data}\SpecialCharTok{$}\NormalTok{Percent\_of\_Revenue }\OtherTok{\textless{}{-}} \FunctionTok{round}\NormalTok{(coq\_data}\SpecialCharTok{$}\NormalTok{Cost }\SpecialCharTok{/}\NormalTok{ revenue }\SpecialCharTok{*} \DecValTok{100}\NormalTok{, }\DecValTok{2}\NormalTok{)}

\FunctionTok{cat}\NormalTok{(}\StringTok{"Cost of Quality Analysis:}\SpecialCharTok{\textbackslash{}n}\StringTok{"}\NormalTok{)}
\end{Highlighting}
\end{Shaded}

\begin{verbatim}
## Cost of Quality Analysis:
\end{verbatim}

\begin{Shaded}
\begin{Highlighting}[]
\FunctionTok{cat}\NormalTok{(}\StringTok{"─────────────────────────────}\SpecialCharTok{\textbackslash{}n}\StringTok{"}\NormalTok{)}
\end{Highlighting}
\end{Shaded}

\begin{verbatim}
## ─────────────────────────────
\end{verbatim}

\begin{Shaded}
\begin{Highlighting}[]
\FunctionTok{print}\NormalTok{(coq\_data[, }\FunctionTok{c}\NormalTok{(}\StringTok{"Category"}\NormalTok{, }\StringTok{"Cost"}\NormalTok{, }\StringTok{"Percent\_of\_COQ"}\NormalTok{)])}
\end{Highlighting}
\end{Shaded}

\begin{verbatim}
##           Category   Cost Percent_of_COQ
## 1       Prevention  50000           11.8
## 2        Appraisal  75000           17.6
## 3 Internal Failure 120000           28.2
## 4 External Failure 180000           42.4
\end{verbatim}

\begin{Shaded}
\begin{Highlighting}[]
\FunctionTok{cat}\NormalTok{(}\StringTok{"─────────────────────────────}\SpecialCharTok{\textbackslash{}n}\StringTok{"}\NormalTok{)}
\end{Highlighting}
\end{Shaded}

\begin{verbatim}
## ─────────────────────────────
\end{verbatim}

\begin{Shaded}
\begin{Highlighting}[]
\FunctionTok{cat}\NormalTok{(}\StringTok{"Total Cost of Quality: $"}\NormalTok{, }\FunctionTok{format}\NormalTok{(total\_coq, }\AttributeTok{big.mark =} \StringTok{","}\NormalTok{), }\StringTok{"}\SpecialCharTok{\textbackslash{}n}\StringTok{"}\NormalTok{)}
\end{Highlighting}
\end{Shaded}

\begin{verbatim}
## Total Cost of Quality: $ 425,000
\end{verbatim}

\begin{Shaded}
\begin{Highlighting}[]
\FunctionTok{cat}\NormalTok{(}\StringTok{"COQ as \% of Revenue:"}\NormalTok{, }\FunctionTok{round}\NormalTok{(total\_coq}\SpecialCharTok{/}\NormalTok{revenue}\SpecialCharTok{*}\DecValTok{100}\NormalTok{, }\DecValTok{1}\NormalTok{), }\StringTok{"\%}\SpecialCharTok{\textbackslash{}n\textbackslash{}n}\StringTok{"}\NormalTok{)}
\end{Highlighting}
\end{Shaded}

\begin{verbatim}
## COQ as % of Revenue: 4.2 %
\end{verbatim}

\begin{Shaded}
\begin{Highlighting}[]
\CommentTok{\# Analysis}
\NormalTok{conformance }\OtherTok{\textless{}{-}}\NormalTok{ coq\_data}\SpecialCharTok{$}\NormalTok{Cost[}\DecValTok{1}\NormalTok{] }\SpecialCharTok{+}\NormalTok{ coq\_data}\SpecialCharTok{$}\NormalTok{Cost[}\DecValTok{2}\NormalTok{]}
\NormalTok{nonconformance }\OtherTok{\textless{}{-}}\NormalTok{ coq\_data}\SpecialCharTok{$}\NormalTok{Cost[}\DecValTok{3}\NormalTok{] }\SpecialCharTok{+}\NormalTok{ coq\_data}\SpecialCharTok{$}\NormalTok{Cost[}\DecValTok{4}\NormalTok{]}

\FunctionTok{cat}\NormalTok{(}\StringTok{"Conformance Costs (Prevention + Appraisal): $"}\NormalTok{, }\FunctionTok{format}\NormalTok{(conformance, }\AttributeTok{big.mark =} \StringTok{","}\NormalTok{),}
    \StringTok{"("}\NormalTok{, }\FunctionTok{round}\NormalTok{(conformance}\SpecialCharTok{/}\NormalTok{total\_coq}\SpecialCharTok{*}\DecValTok{100}\NormalTok{, }\DecValTok{1}\NormalTok{), }\StringTok{"\%)}\SpecialCharTok{\textbackslash{}n}\StringTok{"}\NormalTok{)}
\end{Highlighting}
\end{Shaded}

\begin{verbatim}
## Conformance Costs (Prevention + Appraisal): $ 125,000 ( 29.4 %)
\end{verbatim}

\begin{Shaded}
\begin{Highlighting}[]
\FunctionTok{cat}\NormalTok{(}\StringTok{"Nonconformance Costs (Failures): $"}\NormalTok{, }\FunctionTok{format}\NormalTok{(nonconformance, }\AttributeTok{big.mark =} \StringTok{","}\NormalTok{),}
    \StringTok{"("}\NormalTok{, }\FunctionTok{round}\NormalTok{(nonconformance}\SpecialCharTok{/}\NormalTok{total\_coq}\SpecialCharTok{*}\DecValTok{100}\NormalTok{, }\DecValTok{1}\NormalTok{), }\StringTok{"\%)}\SpecialCharTok{\textbackslash{}n}\StringTok{"}\NormalTok{)}
\end{Highlighting}
\end{Shaded}

\begin{verbatim}
## Nonconformance Costs (Failures): $ 3e+05 ( 70.6 %)
\end{verbatim}

\begin{Shaded}
\begin{Highlighting}[]
\FunctionTok{cat}\NormalTok{(}\StringTok{"}\SpecialCharTok{\textbackslash{}n}\StringTok{Recommendation: Increase prevention spending to reduce failure costs}\SpecialCharTok{\textbackslash{}n}\StringTok{"}\NormalTok{)}
\end{Highlighting}
\end{Shaded}

\begin{verbatim}
## 
## Recommendation: Increase prevention spending to reduce failure costs
\end{verbatim}

\begin{verbatim}
## Warning: Removed 1 row containing missing values or values
## outside the scale range (`geom_text()`).
## Removed 1 row containing missing values or values
## outside the scale range (`geom_text()`).
\end{verbatim}

\pandocbounded{\includegraphics[keepaspectratio]{introduction_files/figure-latex/coq-chart-1.pdf}}

\begin{center}\rule{0.5\linewidth}{0.5pt}\end{center}

\section{Integration with Quality Tools}\label{integration-with-quality-tools}

A mature QMS integrates various quality tools throughout the organization.

\subsection{Quality Tools Integration Map}\label{quality-tools-integration-map}

\pandocbounded{\includegraphics[keepaspectratio]{introduction_files/figure-latex/tools-integration-1.pdf}}

\subsection{Linking Tools to QMS Requirements}\label{linking-tools-to-qms-requirements}

\begin{table}
\centering
\caption{\label{tab:tools-qms-link}Quality Tools and QMS Integration}
\centering
\begin{tabu} to \linewidth {>{\raggedright}X>{\raggedright}X>{\raggedright}X>{\raggedright}X}
\hline
Quality Tool & ISO 9001 Clause & Purpose & Output Feeds Into\\
\hline
\textbf{PFMEA} & 6.1, 8.1 & Identify and mitigate process risks before production & Control Plan, Work Instructions\\
\hline
\textbf{Control Plan} & 8.1, 8.5.1 & Define controls for special characteristics & Work Instructions, SPC requirements\\
\hline
\textbf{SPC} & 9.1.1 & Monitor process performance; detect special causes & CAPA triggers, Management Review\\
\hline
\textbf{MSA} & 7.1.5 & Ensure measurement system is adequate & SPC validity, Control Plan\\
\hline
\textbf{Internal Audit} & 9.2 & Verify QMS conformance and effectiveness & CAPA triggers, Management Review\\
\hline
\textbf{CAPA} & 10.2 & Eliminate causes of nonconformities & Process improvements, Lessons learned\\
\hline
\textbf{Management Review} & 9.3 & Evaluate QMS suitability and effectiveness & Strategic decisions, Resource allocation\\
\hline
\end{tabu}
\end{table}

\begin{center}\rule{0.5\linewidth}{0.5pt}\end{center}

\section{Certification Process}\label{certification-process}

Organizations can seek \textbf{third-party certification} to demonstrate QMS conformance to customers and stakeholders.

\subsection{The Certification Journey}\label{the-certification-journey}

\begin{verbatim}
## Warning in geom_segment(aes(x = 1, xend = 8, y = 1, yend = 1), color = "gray70", : All aesthetics have length 1, but the data has 8
## rows.
## i Please consider using `annotate()` or provide
##   this layer with data containing a single row.
\end{verbatim}

\pandocbounded{\includegraphics[keepaspectratio]{introduction_files/figure-latex/certification-process-1.pdf}}

\subsection{Certification Audit Types}\label{certification-audit-types}

\begin{table}
\centering
\caption{\label{tab:audit-types}Third-Party Certification Audit Types}
\centering
\begin{tabu} to \linewidth {>{\raggedright}X>{\raggedright}X>{\raggedright}X>{\raggedright}X}
\hline
Audit Type & Purpose & Focus Areas & Duration\\
\hline
\textbf{Stage 1 (Documentation)} & Review QMS documentation; verify readiness for Stage 2 & Quality manual, procedures, scope, internal audits, management review & 0.5-1 day (remote or on-site)\\
\hline
\textbf{Stage 2 (Certification)} & Evaluate implementation and effectiveness of QMS & All clauses; process effectiveness; records; interviews & 2-5 days depending on size\\
\hline
\textbf{Surveillance} & Verify continued conformance (annual) & Selected processes; previous findings; changes & 1-2 days annually\\
\hline
\textbf{Recertification} & Full re-audit at end of 3-year cycle & Complete review similar to Stage 2 & 2-4 days\\
\hline
\end{tabu}
\end{table}

\begin{center}\rule{0.5\linewidth}{0.5pt}\end{center}

\section{Video Resources}\label{video-resources-3}

\subsection{ISO 9001:2015 Overview}\label{iso-90012015-overview}

\subsection{Internal Auditing Best Practices}\label{internal-auditing-best-practices}

\begin{center}\rule{0.5\linewidth}{0.5pt}\end{center}

\section{Summary}\label{summary-12}

A Quality Management System provides the framework for consistently meeting customer requirements and improving organizational performance:

\begin{enumerate}
\def\labelenumi{\arabic{enumi}.}
\tightlist
\item
  \textbf{ISO 9001:2015} provides the foundational requirements using risk-based thinking and PDCA
\item
  \textbf{Industry standards} (IATF 16949, AS9100, ISO 22000) add sector-specific requirements
\item
  \textbf{Documentation} establishes the hierarchy from policy to procedures to records
\item
  \textbf{Internal audits} verify conformance and identify improvement opportunities
\item
  \textbf{CAPA} eliminates root causes of nonconformities and prevents recurrence
\item
  \textbf{Management review} ensures QMS suitability and drives strategic improvement
\item
  \textbf{Continuous improvement} is embedded through KPIs, COQ analysis, and improvement methodologies
\item
  \textbf{Quality tools} (PFMEA, SPC, MSA) integrate throughout the product lifecycle
\end{enumerate}

\begin{quote}
``Quality is not an act, it is a habit.'' --- Aristotle
\end{quote}

\begin{center}\rule{0.5\linewidth}{0.5pt}\end{center}

\section{Review Questions}\label{review-questions-13}

\textbf{Question 1}: Explain the seven quality management principles and how they relate to each other.

\textbf{Answer:}

The seven quality management principles form an interconnected foundation:

\begin{enumerate}
\def\labelenumi{\arabic{enumi}.}
\tightlist
\item
  \textbf{Customer Focus} (Central principle)

  \begin{itemize}
  \tightlist
  \item
    Understanding and meeting customer needs is the primary purpose
  \item
    All other principles support this goal
  \end{itemize}
\item
  \textbf{Leadership}

  \begin{itemize}
  \tightlist
  \item
    Leaders establish unity of purpose and direction
  \item
    Create conditions for people to achieve quality objectives
  \item
    Enables all other principles
  \end{itemize}
\item
  \textbf{Engagement of People}

  \begin{itemize}
  \tightlist
  \item
    Competent, empowered people at all levels
  \item
    Essential for effective processes and customer focus
  \end{itemize}
\item
  \textbf{Process Approach}

  \begin{itemize}
  \tightlist
  \item
    Activities managed as interrelated processes
  \item
    Enables consistent, predictable results
  \end{itemize}
\item
  \textbf{Improvement}

  \begin{itemize}
  \tightlist
  \item
    Ongoing focus on enhancing performance
  \item
    Essential for maintaining customer satisfaction
  \end{itemize}
\item
  \textbf{Evidence-based Decision Making}

  \begin{itemize}
  \tightlist
  \item
    Decisions based on analysis of data
  \item
    Reduces uncertainty; improves decision quality
  \end{itemize}
\item
  \textbf{Relationship Management}

  \begin{itemize}
  \tightlist
  \item
    Managing relationships with interested parties (suppliers, partners)
  \item
    Optimizes their impact on performance
  \end{itemize}
\end{enumerate}

\textbf{Relationships:}
- Customer Focus drives the need for all others
- Leadership enables and supports all principles
- Engagement of People is essential for Process Approach and Improvement
- Evidence-based Decision Making supports Improvement
- Relationship Management extends quality beyond the organization

\textbf{Question 2}: Compare ISO 9001, IATF 16949, and AS9100. When would each be required?

\textbf{Answer:}

\begin{longtable}[]{@{}
  >{\raggedright\arraybackslash}p{(\linewidth - 6\tabcolsep) * \real{0.2273}}
  >{\raggedright\arraybackslash}p{(\linewidth - 6\tabcolsep) * \real{0.1364}}
  >{\raggedright\arraybackslash}p{(\linewidth - 6\tabcolsep) * \real{0.2273}}
  >{\raggedright\arraybackslash}p{(\linewidth - 6\tabcolsep) * \real{0.4091}}@{}}
\toprule\noalign{}
\begin{minipage}[b]{\linewidth}\raggedright
Standard
\end{minipage} & \begin{minipage}[b]{\linewidth}\raggedright
Base
\end{minipage} & \begin{minipage}[b]{\linewidth}\raggedright
Industry
\end{minipage} & \begin{minipage}[b]{\linewidth}\raggedright
Additional Focus
\end{minipage} \\
\midrule\noalign{}
\endhead
\bottomrule\noalign{}
\endlastfoot
\textbf{ISO 9001:2015} & Standalone & General & Quality fundamentals \\
\textbf{IATF 16949:2016} & ISO 9001 + automotive & Automotive & PPAP, APQP, Core Tools \\
\textbf{AS9100D} & ISO 9001 + aerospace & Aerospace/Defense & Configuration mgmt, FAI, counterfeit prevention \\
\end{longtable}

\textbf{When Required:}

\textbf{ISO 9001:}
- Any organization wanting to demonstrate quality capability
- Customer or contract requirement for QMS certification
- Basis for all other standards
- Suitable for service, manufacturing, any industry

\textbf{IATF 16949:}
- Suppliers to automotive OEMs (Ford, GM, Toyota, VW, etc.)
- Required for production parts/materials
- Required throughout automotive supply chain
- Includes Customer-Specific Requirements (CSRs)

\textbf{AS9100:}
- Suppliers to aerospace/defense contractors
- Boeing, Airbus, Lockheed Martin, military programs
- Required for flight-critical components
- OASIS database registration required

\textbf{Key Differences:}
- IATF 16949: Emphasizes PPAP, APQP, manufacturing process control
- AS9100: Emphasizes traceability, configuration control, first article inspection
- Both build on ISO 9001 with sector-specific additions

\textbf{Question 3}: Describe the documentation hierarchy in a QMS. Give an example of each level.

\textbf{Answer:}

\textbf{Level 1: Quality Policy and Objectives}
- Strategic direction for quality
- Signed by top management
- Communicated to all employees
- \emph{Example:} ``ABC Company is committed to delivering defect-free products that meet customer requirements through continuous improvement of our processes and people.''

\textbf{Level 2: Quality Manual and Procedures}
- Describes WHAT is done and WHO is responsible
- System-level documents
- \emph{Example:} Document Control Procedure (QP-001)
- Purpose: Control creation, approval, distribution of documents
- Scope: All QMS documents
- Responsibilities: Document Control Coordinator, Department Managers
- Process flow for document changes

\textbf{Level 3: Work Instructions and Specifications}
- Describes HOW tasks are performed
- Detailed step-by-step instructions
- \emph{Example:} Work Instruction for CNC Lathe Setup (WI-MAC-015)
- Step 1: Verify program number matches work order
- Step 2: Install correct tooling per setup sheet
- Step 3: Set tool offsets using touch-off procedure
- Step 4: Run first piece and verify dimensions

\textbf{Level 4: Records and Forms}
- Evidence that activities were performed
- Completed forms, inspection results, logs
- \emph{Example:} First Article Inspection Report (FM-QC-008)
- Part number, revision, date
- Dimensional measurements vs.~specifications
- Inspector signature
- Disposition (Accept/Reject)

\textbf{Question 4}: What are the key elements of an effective internal audit program?

\textbf{Answer:}

\textbf{1. Audit Program Planning}
- Annual audit schedule covering all QMS processes
- Risk-based frequency (higher risk = more frequent)
- Documented audit program procedure
- Resource allocation (trained auditors, time)

\textbf{2. Auditor Competence}
- Trained in audit techniques
- Knowledge of standards (ISO 9001, industry-specific)
- Understanding of processes being audited
- Independence from area being audited

\textbf{3. Audit Preparation}
- Review of relevant documents (procedures, previous audits)
- Prepared checklist linked to requirements
- Notification to auditee
- Defined scope and objectives

\textbf{4. Audit Execution}
- Opening meeting (purpose, scope, schedule)
- Evidence collection (documents, records, interviews, observation)
- Objective evaluation against criteria
- Note taking and evidence documentation

\textbf{5. Reporting}
- Clear finding statements (nonconformities, observations)
- Reference to requirement violated
- Objective evidence cited
- Closing meeting to present findings

\textbf{6. Follow-up}
- Corrective action requests issued
- Root cause analysis required
- Verification of corrective action implementation
- Verification of effectiveness

\textbf{7. Continuous Improvement of Audit Program}
- Evaluate audit program effectiveness
- Update based on organizational changes
- Calibrate auditors periodically
- Incorporate lessons learned

\textbf{Question 5}: Explain the difference between corrective action and preventive action. Provide an example of each.

\textbf{Answer:}

\textbf{Corrective Action:}
- Eliminates the cause of a \emph{detected} nonconformity
- Reactive - responds to something that has already happened
- Goal: Prevent recurrence of the specific problem

\emph{Example:}
- \textbf{Nonconformity:} Customer received shipment with wrong part numbers (3 incidents in past month)
- \textbf{Root Cause:} Shipping labels printed from outdated part number list
- \textbf{Corrective Action:}
1. Update master part number list
2. Link shipping system directly to ERP for current data
3. Add verification step comparing label to packing list
- \textbf{Verification:} No recurrence after 3 months

\textbf{Preventive Action:}
- Eliminates the cause of a \emph{potential} nonconformity
- Proactive - addresses something that hasn't happened yet
- Goal: Prevent occurrence before it happens

\emph{Example:}
- \textbf{Potential Nonconformity:} New CNC machine may produce out-of-spec parts due to operator unfamiliarity
- \textbf{Risk Identified Through:} PFMEA analysis during new equipment installation
- \textbf{Preventive Action:}
1. Develop comprehensive training program before machine goes live
2. Create detailed setup procedures with photos
3. Implement first-article inspection for first 30 days
4. Assign experienced mentor to new operators
- \textbf{Verification:} Track first-pass yield during initial production period

\textbf{Key Difference:}
- Corrective = Fix what went wrong
- Preventive = Stop what might go wrong

\textbf{Note:} ISO 9001:2015 doesn't explicitly require ``preventive action'' as a separate process. Instead, risk-based thinking throughout the QMS serves the preventive function.

\textbf{Question 6}: Calculate and analyze the Cost of Quality for the following data. What recommendations would you make?

\begin{longtable}[]{@{}ll@{}}
\toprule\noalign{}
Category & Cost \\
\midrule\noalign{}
\endhead
\bottomrule\noalign{}
\endlastfoot
Quality planning & \$35,000 \\
Training & \$25,000 \\
Incoming inspection & \$40,000 \\
In-process inspection & \$55,000 \\
Final inspection & \$30,000 \\
Scrap & \$85,000 \\
Rework & \$65,000 \\
Warranty claims & \$120,000 \\
Customer returns & \$45,000 \\
\end{longtable}

Annual revenue: \$8,000,000

\textbf{Answer:}

\begin{Shaded}
\begin{Highlighting}[]
\CommentTok{\# Cost of Quality Analysis}
\NormalTok{prevention }\OtherTok{\textless{}{-}} \DecValTok{35000} \SpecialCharTok{+} \DecValTok{25000}  \CommentTok{\# Planning + Training}
\NormalTok{appraisal }\OtherTok{\textless{}{-}} \DecValTok{40000} \SpecialCharTok{+} \DecValTok{55000} \SpecialCharTok{+} \DecValTok{30000}  \CommentTok{\# Inspections}
\NormalTok{internal\_failure }\OtherTok{\textless{}{-}} \DecValTok{85000} \SpecialCharTok{+} \DecValTok{65000}  \CommentTok{\# Scrap + Rework}
\NormalTok{external\_failure }\OtherTok{\textless{}{-}} \DecValTok{120000} \SpecialCharTok{+} \DecValTok{45000}  \CommentTok{\# Warranty + Returns}

\NormalTok{total\_coq }\OtherTok{\textless{}{-}}\NormalTok{ prevention }\SpecialCharTok{+}\NormalTok{ appraisal }\SpecialCharTok{+}\NormalTok{ internal\_failure }\SpecialCharTok{+}\NormalTok{ external\_failure}
\NormalTok{revenue }\OtherTok{\textless{}{-}} \DecValTok{8000000}

\FunctionTok{cat}\NormalTok{(}\StringTok{"Cost of Quality Breakdown:}\SpecialCharTok{\textbackslash{}n}\StringTok{"}\NormalTok{)}
\end{Highlighting}
\end{Shaded}

\begin{verbatim}
## Cost of Quality Breakdown:
\end{verbatim}

\begin{Shaded}
\begin{Highlighting}[]
\FunctionTok{cat}\NormalTok{(}\StringTok{"─────────────────────────────}\SpecialCharTok{\textbackslash{}n}\StringTok{"}\NormalTok{)}
\end{Highlighting}
\end{Shaded}

\begin{verbatim}
## ─────────────────────────────
\end{verbatim}

\begin{Shaded}
\begin{Highlighting}[]
\FunctionTok{cat}\NormalTok{(}\StringTok{"Prevention:       $"}\NormalTok{, }\FunctionTok{format}\NormalTok{(prevention, }\AttributeTok{big.mark =} \StringTok{","}\NormalTok{),}
    \StringTok{"("}\NormalTok{, }\FunctionTok{round}\NormalTok{(prevention}\SpecialCharTok{/}\NormalTok{total\_coq}\SpecialCharTok{*}\DecValTok{100}\NormalTok{, }\DecValTok{1}\NormalTok{), }\StringTok{"\%)}\SpecialCharTok{\textbackslash{}n}\StringTok{"}\NormalTok{)}
\end{Highlighting}
\end{Shaded}

\begin{verbatim}
## Prevention:       $ 60,000 ( 12 %)
\end{verbatim}

\begin{Shaded}
\begin{Highlighting}[]
\FunctionTok{cat}\NormalTok{(}\StringTok{"Appraisal:        $"}\NormalTok{, }\FunctionTok{format}\NormalTok{(appraisal, }\AttributeTok{big.mark =} \StringTok{","}\NormalTok{),}
    \StringTok{"("}\NormalTok{, }\FunctionTok{round}\NormalTok{(appraisal}\SpecialCharTok{/}\NormalTok{total\_coq}\SpecialCharTok{*}\DecValTok{100}\NormalTok{, }\DecValTok{1}\NormalTok{), }\StringTok{"\%)}\SpecialCharTok{\textbackslash{}n}\StringTok{"}\NormalTok{)}
\end{Highlighting}
\end{Shaded}

\begin{verbatim}
## Appraisal:        $ 125,000 ( 25 %)
\end{verbatim}

\begin{Shaded}
\begin{Highlighting}[]
\FunctionTok{cat}\NormalTok{(}\StringTok{"Internal Failure: $"}\NormalTok{, }\FunctionTok{format}\NormalTok{(internal\_failure, }\AttributeTok{big.mark =} \StringTok{","}\NormalTok{),}
    \StringTok{"("}\NormalTok{, }\FunctionTok{round}\NormalTok{(internal\_failure}\SpecialCharTok{/}\NormalTok{total\_coq}\SpecialCharTok{*}\DecValTok{100}\NormalTok{, }\DecValTok{1}\NormalTok{), }\StringTok{"\%)}\SpecialCharTok{\textbackslash{}n}\StringTok{"}\NormalTok{)}
\end{Highlighting}
\end{Shaded}

\begin{verbatim}
## Internal Failure: $ 150,000 ( 30 %)
\end{verbatim}

\begin{Shaded}
\begin{Highlighting}[]
\FunctionTok{cat}\NormalTok{(}\StringTok{"External Failure: $"}\NormalTok{, }\FunctionTok{format}\NormalTok{(external\_failure, }\AttributeTok{big.mark =} \StringTok{","}\NormalTok{),}
    \StringTok{"("}\NormalTok{, }\FunctionTok{round}\NormalTok{(external\_failure}\SpecialCharTok{/}\NormalTok{total\_coq}\SpecialCharTok{*}\DecValTok{100}\NormalTok{, }\DecValTok{1}\NormalTok{), }\StringTok{"\%)}\SpecialCharTok{\textbackslash{}n}\StringTok{"}\NormalTok{)}
\end{Highlighting}
\end{Shaded}

\begin{verbatim}
## External Failure: $ 165,000 ( 33 %)
\end{verbatim}

\begin{Shaded}
\begin{Highlighting}[]
\FunctionTok{cat}\NormalTok{(}\StringTok{"─────────────────────────────}\SpecialCharTok{\textbackslash{}n}\StringTok{"}\NormalTok{)}
\end{Highlighting}
\end{Shaded}

\begin{verbatim}
## ─────────────────────────────
\end{verbatim}

\begin{Shaded}
\begin{Highlighting}[]
\FunctionTok{cat}\NormalTok{(}\StringTok{"Total COQ:        $"}\NormalTok{, }\FunctionTok{format}\NormalTok{(total\_coq, }\AttributeTok{big.mark =} \StringTok{","}\NormalTok{), }\StringTok{"}\SpecialCharTok{\textbackslash{}n}\StringTok{"}\NormalTok{)}
\end{Highlighting}
\end{Shaded}

\begin{verbatim}
## Total COQ:        $ 5e+05
\end{verbatim}

\begin{Shaded}
\begin{Highlighting}[]
\FunctionTok{cat}\NormalTok{(}\StringTok{"COQ \% of Revenue:"}\NormalTok{, }\FunctionTok{round}\NormalTok{(total\_coq}\SpecialCharTok{/}\NormalTok{revenue}\SpecialCharTok{*}\DecValTok{100}\NormalTok{, }\DecValTok{1}\NormalTok{), }\StringTok{"\%}\SpecialCharTok{\textbackslash{}n\textbackslash{}n}\StringTok{"}\NormalTok{)}
\end{Highlighting}
\end{Shaded}

\begin{verbatim}
## COQ % of Revenue: 6.2 %
\end{verbatim}

\begin{Shaded}
\begin{Highlighting}[]
\NormalTok{conformance }\OtherTok{\textless{}{-}}\NormalTok{ prevention }\SpecialCharTok{+}\NormalTok{ appraisal}
\NormalTok{nonconformance }\OtherTok{\textless{}{-}}\NormalTok{ internal\_failure }\SpecialCharTok{+}\NormalTok{ external\_failure}

\FunctionTok{cat}\NormalTok{(}\StringTok{"Conformance Costs:    $"}\NormalTok{, }\FunctionTok{format}\NormalTok{(conformance, }\AttributeTok{big.mark =} \StringTok{","}\NormalTok{),}
    \StringTok{"("}\NormalTok{, }\FunctionTok{round}\NormalTok{(conformance}\SpecialCharTok{/}\NormalTok{total\_coq}\SpecialCharTok{*}\DecValTok{100}\NormalTok{, }\DecValTok{1}\NormalTok{), }\StringTok{"\%)}\SpecialCharTok{\textbackslash{}n}\StringTok{"}\NormalTok{)}
\end{Highlighting}
\end{Shaded}

\begin{verbatim}
## Conformance Costs:    $ 185,000 ( 37 %)
\end{verbatim}

\begin{Shaded}
\begin{Highlighting}[]
\FunctionTok{cat}\NormalTok{(}\StringTok{"Nonconformance Costs: $"}\NormalTok{, }\FunctionTok{format}\NormalTok{(nonconformance, }\AttributeTok{big.mark =} \StringTok{","}\NormalTok{),}
    \StringTok{"("}\NormalTok{, }\FunctionTok{round}\NormalTok{(nonconformance}\SpecialCharTok{/}\NormalTok{total\_coq}\SpecialCharTok{*}\DecValTok{100}\NormalTok{, }\DecValTok{1}\NormalTok{), }\StringTok{"\%)}\SpecialCharTok{\textbackslash{}n}\StringTok{"}\NormalTok{)}
\end{Highlighting}
\end{Shaded}

\begin{verbatim}
## Nonconformance Costs: $ 315,000 ( 63 %)
\end{verbatim}

\textbf{Analysis:}
- COQ at 7.5\% of revenue is HIGH (target \textless3\%)
- External failure costs are the largest category (27.5\%)
- Nonconformance costs (60.8\%) greatly exceed conformance (39.2\%)
- Prevention is only 10\% of COQ - very low

\textbf{Recommendations:}
1. \textbf{Increase Prevention Investment:}
- Implement SPC on critical processes to reduce defects
- Enhance operator training on quality standards
- Apply PFMEA to identify and control failure modes

\begin{enumerate}
\def\labelenumi{\arabic{enumi}.}
\setcounter{enumi}{1}
\tightlist
\item
  \textbf{Reduce External Failures:}

  \begin{itemize}
  \tightlist
  \item
    Analyze warranty data for root causes
  \item
    Improve final inspection effectiveness
  \item
    Implement containment for known issues
  \end{itemize}
\item
  \textbf{Optimize Appraisal:}

  \begin{itemize}
  \tightlist
  \item
    Use SPC to reduce inspection where processes are capable
  \item
    Implement risk-based inspection (more for critical, less for proven)
  \end{itemize}
\item
  \textbf{Target COQ Reduction:}

  \begin{itemize}
  \tightlist
  \item
    Short-term: Reduce external failures by 50\% (\$82,500 savings)
  \item
    Medium-term: Reduce internal failures by 30\% (\$45,000 savings)
  \item
    Long-term: Achieve COQ \textless3\% of revenue (\$240,000)
  \end{itemize}
\end{enumerate}

\textbf{Question 7}: What is risk-based thinking in ISO 9001:2015? How is it different from formal risk management?

\textbf{Answer:}

\textbf{Risk-Based Thinking in ISO 9001:2015:}

Risk-based thinking is a way of considering risks and opportunities when designing, implementing, and operating the QMS. It:

\begin{itemize}
\tightlist
\item
  Applies throughout the standard (not a separate clause)
\item
  Is proportional to potential impact on conformity
\item
  Makes preventive action part of routine planning
\item
  Considers both risks (negative) and opportunities (positive)
\end{itemize}

\textbf{Where Risk-Based Thinking Applies:}

\begin{longtable}[]{@{}
  >{\raggedright\arraybackslash}p{(\linewidth - 2\tabcolsep) * \real{0.3810}}
  >{\raggedright\arraybackslash}p{(\linewidth - 2\tabcolsep) * \real{0.6190}}@{}}
\toprule\noalign{}
\begin{minipage}[b]{\linewidth}\raggedright
Clause
\end{minipage} & \begin{minipage}[b]{\linewidth}\raggedright
Application
\end{minipage} \\
\midrule\noalign{}
\endhead
\bottomrule\noalign{}
\endlastfoot
4.1 Context & Identify external/internal issues affecting QMS \\
4.2 Interested Parties & Understand requirements that could impact quality \\
6.1 Planning & Plan actions to address risks and opportunities \\
8.1 Operations & Consider risks in operational planning \\
9.1 Performance & Monitor effectiveness of risk actions \\
10.2 Corrective Action & Update risks/opportunities based on nonconformities \\
\end{longtable}

\textbf{How It Differs from Formal Risk Management:}

\begin{longtable}[]{@{}
  >{\raggedright\arraybackslash}p{(\linewidth - 4\tabcolsep) * \real{0.1509}}
  >{\raggedright\arraybackslash}p{(\linewidth - 4\tabcolsep) * \real{0.3962}}
  >{\raggedright\arraybackslash}p{(\linewidth - 4\tabcolsep) * \real{0.4528}}@{}}
\toprule\noalign{}
\begin{minipage}[b]{\linewidth}\raggedright
Aspect
\end{minipage} & \begin{minipage}[b]{\linewidth}\raggedright
Risk-Based Thinking
\end{minipage} & \begin{minipage}[b]{\linewidth}\raggedright
Formal Risk Management
\end{minipage} \\
\midrule\noalign{}
\endhead
\bottomrule\noalign{}
\endlastfoot
Requirement & Implicit throughout ISO 9001 & Explicit (ISO 31000) \\
Documentation & No formal risk register required & Documented risk register \\
Methodology & No specific method required & Defined methodology (FMEA, etc.) \\
Quantification & Not required & Often quantitative \\
Scope & QMS processes & Enterprise-wide \\
\end{longtable}

\textbf{Practical Implementation:}

Organizations should:
1. Consider risks when designing processes
2. Build controls proportional to risk level
3. Monitor and review effectiveness
4. Update as context changes

ISO 9001 does NOT require:
- Formal risk management process
- Risk register
- Specific risk assessment methodology
- Documented risk assessments (unless organization decides to)

However, many organizations do use tools like PFMEA to fulfill risk-based thinking requirements, especially in automotive (IATF 16949 requires it).

\textbf{Question 8}: Describe what management review is and what should be included in the inputs and outputs.

\textbf{Answer:}

\textbf{What is Management Review?}

Management review is a formal, periodic evaluation of the QMS by top management to:
- Ensure continuing suitability (appropriate for the organization)
- Ensure adequacy (sufficient to meet requirements)
- Ensure effectiveness (achieving intended results)
- Ensure alignment with strategic direction
- Identify improvement opportunities

\textbf{Frequency:} At least annually; many organizations conduct quarterly reviews

\textbf{Required Inputs (ISO 9001 Clause 9.3.2):}

\begin{longtable}[]{@{}
  >{\raggedright\arraybackslash}p{(\linewidth - 2\tabcolsep) * \real{0.3043}}
  >{\raggedright\arraybackslash}p{(\linewidth - 2\tabcolsep) * \real{0.6957}}@{}}
\toprule\noalign{}
\begin{minipage}[b]{\linewidth}\raggedright
Input
\end{minipage} & \begin{minipage}[b]{\linewidth}\raggedright
What to Review
\end{minipage} \\
\midrule\noalign{}
\endhead
\bottomrule\noalign{}
\endlastfoot
Status of previous actions & Actions from prior reviews - completed? effective? \\
Changes in external/internal issues & Market changes, regulations, technology, organizational changes \\
QMS performance and effectiveness & Customer satisfaction, quality objectives, process performance, nonconformities, audit results, supplier performance \\
Resource adequacy & People, infrastructure, environment, knowledge \\
Effectiveness of risk actions & Did actions address risks/opportunities? \\
Improvement opportunities & Suggestions, benchmarking, new technologies \\
\end{longtable}

\textbf{Required Outputs (ISO 9001 Clause 9.3.3):}

\begin{longtable}[]{@{}
  >{\raggedright\arraybackslash}p{(\linewidth - 2\tabcolsep) * \real{0.4444}}
  >{\raggedright\arraybackslash}p{(\linewidth - 2\tabcolsep) * \real{0.5556}}@{}}
\toprule\noalign{}
\begin{minipage}[b]{\linewidth}\raggedright
Output
\end{minipage} & \begin{minipage}[b]{\linewidth}\raggedright
Examples
\end{minipage} \\
\midrule\noalign{}
\endhead
\bottomrule\noalign{}
\endlastfoot
Improvement decisions & Launch specific improvement projects, address recurring issues \\
QMS changes & Revise quality objectives, update processes, modify documentation \\
Resource needs & Budget approvals, hiring, equipment purchases, training \\
\end{longtable}

\textbf{Best Practices:}
- Use a standard agenda aligned with required inputs
- Prepare data in advance (dashboards, trend charts)
- Document decisions and action items with owners and due dates
- Follow up on action completion
- Keep minutes as quality records

\textbf{Evidence Required:}
- Meeting minutes or management review report
- Attendance record (top management participation)
- Action items with status tracking
- Data/reports presented

\begin{center}\rule{0.5\linewidth}{0.5pt}\end{center}

\section{References}\label{references-13}

\begin{enumerate}
\def\labelenumi{\arabic{enumi}.}
\item
  ISO 9001:2015. \emph{Quality Management Systems - Requirements}. International Organization for Standardization.
\item
  ISO 9000:2015. \emph{Quality Management Systems - Fundamentals and Vocabulary}. International Organization for Standardization.
\item
  IATF 16949:2016. \emph{Quality Management System Requirements for Automotive Production and Relevant Service Parts Organizations}. International Automotive Task Force.
\item
  AS9100D:2016. \emph{Quality Management Systems - Requirements for Aviation, Space and Defense Organizations}. SAE International.
\item
  ISO 22000:2018. \emph{Food Safety Management Systems - Requirements for Any Organization in the Food Chain}. International Organization for Standardization.
\item
  ISO 19011:2018. \emph{Guidelines for Auditing Management Systems}. International Organization for Standardization.
\item
  Hoyle, D. (2017). \emph{ISO 9000 Quality Systems Handbook} (7th ed.). Routledge.
\item
  AIAG. (2010). \emph{Advanced Product Quality Planning and Control Plan (APQP)} (2nd ed.). Automotive Industry Action Group.
\item
  ASQ. (2019). \emph{The Certified Quality Auditor Handbook} (4th ed.). ASQ Quality Press.
\item
  Juran, J.M., \& De Feo, J.A. (2010). \emph{Juran's Quality Handbook} (6th ed.). McGraw-Hill.
\end{enumerate}

\chapter{Industrial Robotics}\label{industrial-robotics}

\section{Learning Objectives}\label{learning-objectives-14}

After completing this chapter, you will be able to:

\begin{enumerate}
\def\labelenumi{\arabic{enumi}.}
\tightlist
\item
  Define industrial robots and explain their role in modern manufacturing
\item
  Identify the main types of robot configurations and their applications
\item
  Describe robot specifications including payload, reach, repeatability, and speed
\item
  Explain the components of a robotic system (manipulator, controller, teach pendant, end effector)
\item
  Understand robot coordinate systems and motion types
\item
  Identify common robot applications in automotive, food, and defense industries
\item
  Apply robot safety standards and safeguarding methods
\item
  Describe collaborative robot (cobot) technology and applications
\item
  Understand basic robot programming concepts
\end{enumerate}

\begin{center}\rule{0.5\linewidth}{0.5pt}\end{center}

\section{Introduction to Industrial Robotics}\label{introduction-to-industrial-robotics}

An \textbf{industrial robot} is defined by ISO 8373 as ``an automatically controlled, reprogrammable, multipurpose manipulator, programmable in three or more axes, which can be either fixed in place or mobile for use in industrial automation applications.''

\subsection{Why Use Robots?}\label{why-use-robots}

\pandocbounded{\includegraphics[keepaspectratio]{introduction_files/figure-latex/robot-benefits-1.pdf}}

\subsection{Robot Market Growth}\label{robot-market-growth}

\begin{verbatim}
## `geom_smooth()` using formula = 'y ~ x'
\end{verbatim}

\pandocbounded{\includegraphics[keepaspectratio]{introduction_files/figure-latex/robot-market-1.pdf}}

\subsection{Industries Using Robots}\label{industries-using-robots}

\begin{table}
\centering
\caption{\label{tab:robot-industries}Industrial Robot Usage by Industry Sector}
\centering
\begin{tabu} to \linewidth {>{\raggedright}X>{\raggedleft}X>{\raggedright}X>{\raggedright}X}
\hline
Industry & Market Share (\%) & Key Applications & Growth Trend\\
\hline
\textbf{Automotive} & 30 & Welding, painting, assembly, material handling & Stable\\
\hline
\textbf{Electrical/Electronics} & 25 & Assembly, testing, packaging, dispensing & High\\
\hline
\textbf{Metal \& Machinery} & 12 & Machine tending, welding, cutting & Moderate\\
\hline
\textbf{Food \& Beverage} & 8 & Palletizing, packaging, pick-and-place & High\\
\hline
\textbf{Plastics \& Chemicals} & 5 & Injection molding, packaging, assembly & Moderate\\
\hline
\textbf{Aerospace/Defense} & 4 & Drilling, fastening, inspection, composite layup & High\\
\hline
\end{tabu}
\end{table}

\begin{center}\rule{0.5\linewidth}{0.5pt}\end{center}

\section{Robot Configurations}\label{robot-configurations}

Industrial robots come in various mechanical configurations, each suited to different applications.

\subsection{Common Robot Types}\label{common-robot-types}

\pandocbounded{\includegraphics[keepaspectratio]{introduction_files/figure-latex/robot-types-1.pdf}}

\subsection{Detailed Configuration Comparison}\label{detailed-configuration-comparison}

\begin{table}
\centering
\caption{\label{tab:config-comparison}Robot Configuration Comparison}
\centering
\begin{tabu} to \linewidth {>{\raggedright}X>{\raggedright}X>{\raggedright}X>{\raggedright}X>{\raggedright}X>{\raggedright}X>{\raggedright}X>{\raggedright}X}
\hline
Configuration & Axes & Payload Range & Reach & Repeatability & Speed & Cost & Best Applications\\
\hline
\textbf{6-Axis Articulated} & 6 & 3-2300 kg & 0.5-4 m & ±0.02-0.1 mm & Moderate-Fast & \$\$-\$\$\$\$ & Complex 3D tasks, maximum flexibility\\
\hline
\textbf{SCARA} & 4 & 1-20 kg & 0.2-1.2 m & ±0.01-0.05 mm & Very Fast & \$\$ & Fast assembly, pick-place, screw driving\\
\hline
\textbf{Delta/Parallel} & 3-4 & 0.5-12 kg & 0.5-1.5 m & ±0.05-0.1 mm & Extremely Fast & \$\$\$ & High-speed picking, packaging lines\\
\hline
\textbf{Cartesian/Gantry} & 3-6 & 5-1000+ kg & Custom (meters) & ±0.05-0.5 mm & Moderate & \$-\$\$\$\$ & Large work areas, heavy loads, simple paths\\
\hline
\end{tabu}
\end{table}

\subsection{Degrees of Freedom}\label{degrees-of-freedom}

\pandocbounded{\includegraphics[keepaspectratio]{introduction_files/figure-latex/degrees-freedom-1.pdf}}

\begin{center}\rule{0.5\linewidth}{0.5pt}\end{center}

\section{Robot Specifications}\label{robot-specifications}

Understanding robot specifications is essential for proper selection and application.

\subsection{Key Specifications}\label{key-specifications}

\begin{table}
\centering
\caption{\label{tab:robot-specs}Key Robot Specifications Explained}
\centering
\begin{tabu} to \linewidth {>{\raggedright}X>{\raggedright}X>{\raggedright}X>{\raggedright}X}
\hline
Specification & Definition & Typical\_Values & Importance\\
\hline
\textbf{Payload} & Maximum mass the robot can handle at full speed & 3-2300 kg (varies widely by type) & Must exceed part + gripper + safety factor\\
\hline
\textbf{Reach} & Maximum distance from base to tool center point & 500-4000 mm for articulated robots & Must cover all required work positions\\
\hline
\textbf{Repeatability} & Ability to return to same position repeatedly & ±0.01 to ±0.1 mm & Critical for precision assembly; welding\\
\hline
\textbf{Accuracy} & Ability to reach a commanded position exactly & ±0.1 to ±1.0 mm & Important for offline programming\\
\hline
\textbf{Maximum Speed} & Maximum velocity at tool center point or joint speeds & 1-12 m/s TCP; 100-500°/s joint & Affects cycle time; may be derated with payload\\
\hline
\textbf{Degrees of Freedom} & Number of independent axes of motion & 4-7 axes & More axes = more flexibility\\
\hline
\textbf{Mounting} & How robot is installed (floor, ceiling, wall, angle) & Floor (most common), ceiling, wall, shelf & Affects work envelope and floor space\\
\hline
\textbf{IP Rating} & Ingress Protection rating for dust/water resistance & IP40 (standard) to IP67 (harsh environments) & Required for foundry, food, wash-down applications\\
\hline
\end{tabu}
\end{table}

\subsection{Work Envelope}\label{work-envelope}

The \textbf{work envelope} (or workspace) is the three-dimensional space the robot can reach.

\pandocbounded{\includegraphics[keepaspectratio]{introduction_files/figure-latex/work-envelope-1.pdf}}

\subsection{Payload Considerations}\label{payload-considerations}

\begin{Shaded}
\begin{Highlighting}[]
\CommentTok{\# Payload Calculation Example}
\NormalTok{robot\_max\_payload }\OtherTok{\textless{}{-}} \DecValTok{10}  \CommentTok{\# kg}
\NormalTok{gripper\_weight }\OtherTok{\textless{}{-}} \FloatTok{2.5}    \CommentTok{\# kg}
\NormalTok{part\_weight }\OtherTok{\textless{}{-}} \FloatTok{5.0}       \CommentTok{\# kg}
\NormalTok{safety\_factor }\OtherTok{\textless{}{-}} \FloatTok{1.2}     \CommentTok{\# 20\% safety margin}

\NormalTok{total\_load }\OtherTok{\textless{}{-}}\NormalTok{ gripper\_weight }\SpecialCharTok{+}\NormalTok{ part\_weight}
\NormalTok{required\_capacity }\OtherTok{\textless{}{-}}\NormalTok{ total\_load }\SpecialCharTok{*}\NormalTok{ safety\_factor}
\NormalTok{available\_capacity }\OtherTok{\textless{}{-}}\NormalTok{ robot\_max\_payload }\SpecialCharTok{{-}}\NormalTok{ gripper\_weight}

\FunctionTok{cat}\NormalTok{(}\StringTok{"Payload Analysis:}\SpecialCharTok{\textbackslash{}n}\StringTok{"}\NormalTok{)}
\end{Highlighting}
\end{Shaded}

\begin{verbatim}
## Payload Analysis:
\end{verbatim}

\begin{Shaded}
\begin{Highlighting}[]
\FunctionTok{cat}\NormalTok{(}\StringTok{"─────────────────────────────}\SpecialCharTok{\textbackslash{}n}\StringTok{"}\NormalTok{)}
\end{Highlighting}
\end{Shaded}

\begin{verbatim}
## ─────────────────────────────
\end{verbatim}

\begin{Shaded}
\begin{Highlighting}[]
\FunctionTok{cat}\NormalTok{(}\StringTok{"Robot maximum payload:"}\NormalTok{, robot\_max\_payload, }\StringTok{"kg}\SpecialCharTok{\textbackslash{}n}\StringTok{"}\NormalTok{)}
\end{Highlighting}
\end{Shaded}

\begin{verbatim}
## Robot maximum payload: 10 kg
\end{verbatim}

\begin{Shaded}
\begin{Highlighting}[]
\FunctionTok{cat}\NormalTok{(}\StringTok{"Gripper weight:"}\NormalTok{, gripper\_weight, }\StringTok{"kg}\SpecialCharTok{\textbackslash{}n}\StringTok{"}\NormalTok{)}
\end{Highlighting}
\end{Shaded}

\begin{verbatim}
## Gripper weight: 2.5 kg
\end{verbatim}

\begin{Shaded}
\begin{Highlighting}[]
\FunctionTok{cat}\NormalTok{(}\StringTok{"Part weight:"}\NormalTok{, part\_weight, }\StringTok{"kg}\SpecialCharTok{\textbackslash{}n}\StringTok{"}\NormalTok{)}
\end{Highlighting}
\end{Shaded}

\begin{verbatim}
## Part weight: 5 kg
\end{verbatim}

\begin{Shaded}
\begin{Highlighting}[]
\FunctionTok{cat}\NormalTok{(}\StringTok{"Total load:"}\NormalTok{, total\_load, }\StringTok{"kg}\SpecialCharTok{\textbackslash{}n}\StringTok{"}\NormalTok{)}
\end{Highlighting}
\end{Shaded}

\begin{verbatim}
## Total load: 7.5 kg
\end{verbatim}

\begin{Shaded}
\begin{Highlighting}[]
\FunctionTok{cat}\NormalTok{(}\StringTok{"With safety factor ("}\NormalTok{, safety\_factor, }\StringTok{"x):"}\NormalTok{, required\_capacity, }\StringTok{"kg}\SpecialCharTok{\textbackslash{}n}\StringTok{"}\NormalTok{)}
\end{Highlighting}
\end{Shaded}

\begin{verbatim}
## With safety factor ( 1.2 x): 9 kg
\end{verbatim}

\begin{Shaded}
\begin{Highlighting}[]
\FunctionTok{cat}\NormalTok{(}\StringTok{"─────────────────────────────}\SpecialCharTok{\textbackslash{}n}\StringTok{"}\NormalTok{)}
\end{Highlighting}
\end{Shaded}

\begin{verbatim}
## ─────────────────────────────
\end{verbatim}

\begin{Shaded}
\begin{Highlighting}[]
\FunctionTok{cat}\NormalTok{(}\StringTok{"Available capacity for part:"}\NormalTok{, available\_capacity, }\StringTok{"kg}\SpecialCharTok{\textbackslash{}n}\StringTok{"}\NormalTok{)}
\end{Highlighting}
\end{Shaded}

\begin{verbatim}
## Available capacity for part: 7.5 kg
\end{verbatim}

\begin{Shaded}
\begin{Highlighting}[]
\ControlFlowTok{if}\NormalTok{(required\_capacity }\SpecialCharTok{\textless{}=}\NormalTok{ robot\_max\_payload) \{}
  \FunctionTok{cat}\NormalTok{(}\StringTok{"Result: ACCEPTABLE {-} Robot can handle this load}\SpecialCharTok{\textbackslash{}n}\StringTok{"}\NormalTok{)}
\NormalTok{\} }\ControlFlowTok{else}\NormalTok{ \{}
  \FunctionTok{cat}\NormalTok{(}\StringTok{"Result: EXCEEDED {-} Select larger robot or lighter gripper}\SpecialCharTok{\textbackslash{}n}\StringTok{"}\NormalTok{)}
\NormalTok{\}}
\end{Highlighting}
\end{Shaded}

\begin{verbatim}
## Result: ACCEPTABLE - Robot can handle this load
\end{verbatim}

\begin{Shaded}
\begin{Highlighting}[]
\CommentTok{\# Note about moment load}
\FunctionTok{cat}\NormalTok{(}\StringTok{"}\SpecialCharTok{\textbackslash{}n}\StringTok{Note: Also verify moment load (payload × distance from flange)}\SpecialCharTok{\textbackslash{}n}\StringTok{"}\NormalTok{)}
\end{Highlighting}
\end{Shaded}

\begin{verbatim}
## 
## Note: Also verify moment load (payload × distance from flange)
\end{verbatim}

\begin{Shaded}
\begin{Highlighting}[]
\FunctionTok{cat}\NormalTok{(}\StringTok{"Moment at wrist:"}\NormalTok{, part\_weight, }\StringTok{"kg ×"}\NormalTok{, }\FloatTok{0.3}\NormalTok{, }\StringTok{"m ="}\NormalTok{, part\_weight }\SpecialCharTok{*} \FloatTok{0.3}\NormalTok{, }\StringTok{"kg·m}\SpecialCharTok{\textbackslash{}n}\StringTok{"}\NormalTok{)}
\end{Highlighting}
\end{Shaded}

\begin{verbatim}
## Moment at wrist: 5 kg × 0.3 m = 1.5 kg·m
\end{verbatim}

\subsection{Repeatability vs Accuracy}\label{repeatability-vs-accuracy}

\begin{verbatim}
## Warning in geom_point(aes(x = 0, y = 0), color = "red", size = 4, shape = 3, : All aesthetics have length 1, but the data has 90
## rows.
## i Please consider using `annotate()` or provide
##   this layer with data containing a single row.
\end{verbatim}

\begin{verbatim}
## Warning in geom_circle(aes(x0 = 0, y0 = 0, r = 0.05), inherit.aes = FALSE, : All aesthetics have length 1, but the data has 90
## rows.
## i Please consider using `annotate()` or provide
##   this layer with data containing a single row.
\end{verbatim}

\pandocbounded{\includegraphics[keepaspectratio]{introduction_files/figure-latex/repeat-vs-accuracy-1.pdf}}

\textbf{Understanding the Difference}

\textbf{Repeatability:}
- Ability to return to the SAME taught position
- More important for most applications
- Quoted spec (e.g., ±0.05mm) is typically 3σ
- Achieved through consistent mechanical design and servo control

\textbf{Accuracy:}
- Ability to reach a COMMANDED position
- Important for offline programming
- Affected by mechanical tolerances, deflection, calibration
- Can be improved through calibration

\textbf{Why Repeatability Matters More:}
In most applications, the robot is taught positions by jogging to them. As long as it returns to those positions consistently (repeatability), accuracy doesn't matter. Accuracy only matters when positions are calculated (offline programming) rather than taught.

\begin{center}\rule{0.5\linewidth}{0.5pt}\end{center}

\section{Robot System Components}\label{robot-system-components}

A complete robotic system consists of several integrated components.

\subsection{System Architecture}\label{system-architecture}

\pandocbounded{\includegraphics[keepaspectratio]{introduction_files/figure-latex/robot-system-1.pdf}}

\subsection{End Effectors (EOAT)}\label{end-effectors-eoat}

\textbf{End of Arm Tooling (EOAT)} is the device attached to the robot's wrist to interact with parts.

\begin{table}
\centering
\caption{\label{tab:end-effectors}Common End Effector Types}
\centering
\begin{tabu} to \linewidth {>{\raggedright}X>{\raggedright}X>{\raggedright}X>{\raggedright}X}
\hline
EOAT Type & Operating Principle & Applications & Key Considerations\\
\hline
\textbf{Mechanical Gripper} & Fingers actuated by pneumatic, electric, or hydraulic & General part handling; assembly & Part geometry; grip force; cycle time\\
\hline
\textbf{Vacuum Gripper} & Suction cups powered by venturi or vacuum pump & Flat surfaces; sheet material; boxes & Surface porosity; part weight; seal\\
\hline
\textbf{Magnetic Gripper} & Electromagnetic or permanent magnet & Ferrous metal parts; steel sheets & Part material; residual magnetism\\
\hline
\textbf{Welding Torch} & MIG, TIG, spot welding equipment & Automotive body, frames, components & Process parameters; wire feed; shielding\\
\hline
\textbf{Spray Gun} & Paint, adhesive, sealant dispensing & Automotive painting; sealing & Pattern; coverage; waste\\
\hline
\textbf{Deburring Tool} & Rotary tool for edge finishing & Casting, machining finishing & Speed; pressure; consistency\\
\hline
\textbf{Force/Torque Sensor} & Measures forces and torques at tool & Assembly; polishing; insertion & Sensitivity; overload protection\\
\hline
\textbf{Vision Camera} & 2D or 3D imaging for guidance & Part location; inspection; guidance & Resolution; lighting; processing speed\\
\hline
\end{tabu}
\end{table}

\subsection{Gripper Selection Example}\label{gripper-selection-example}

\begin{Shaded}
\begin{Highlighting}[]
\CommentTok{\# Gripper Force Calculation}
\NormalTok{part\_mass }\OtherTok{\textless{}{-}} \FloatTok{3.0}        \CommentTok{\# kg}
\NormalTok{acceleration }\OtherTok{\textless{}{-}} \DecValTok{15}      \CommentTok{\# m/s² (robot acceleration)}
\NormalTok{gravity }\OtherTok{\textless{}{-}} \FloatTok{9.81}         \CommentTok{\# m/s²}
\NormalTok{safety\_factor }\OtherTok{\textless{}{-}} \FloatTok{2.0}    \CommentTok{\# Minimum safety factor}
\NormalTok{friction\_coefficient }\OtherTok{\textless{}{-}} \FloatTok{0.3}  \CommentTok{\# Steel on rubber}

\CommentTok{\# Calculate forces}
\NormalTok{weight\_force }\OtherTok{\textless{}{-}}\NormalTok{ part\_mass }\SpecialCharTok{*}\NormalTok{ gravity}
\NormalTok{inertia\_force }\OtherTok{\textless{}{-}}\NormalTok{ part\_mass }\SpecialCharTok{*}\NormalTok{ acceleration}
\NormalTok{total\_force }\OtherTok{\textless{}{-}} \FunctionTok{sqrt}\NormalTok{(weight\_force}\SpecialCharTok{\^{}}\DecValTok{2} \SpecialCharTok{+}\NormalTok{ inertia\_force}\SpecialCharTok{\^{}}\DecValTok{2}\NormalTok{)}

\CommentTok{\# Required grip force (friction gripper)}
\NormalTok{required\_grip }\OtherTok{\textless{}{-}}\NormalTok{ total\_force }\SpecialCharTok{/}\NormalTok{ friction\_coefficient }\SpecialCharTok{*}\NormalTok{ safety\_factor}

\FunctionTok{cat}\NormalTok{(}\StringTok{"Gripper Force Calculation:}\SpecialCharTok{\textbackslash{}n}\StringTok{"}\NormalTok{)}
\end{Highlighting}
\end{Shaded}

\begin{verbatim}
## Gripper Force Calculation:
\end{verbatim}

\begin{Shaded}
\begin{Highlighting}[]
\FunctionTok{cat}\NormalTok{(}\StringTok{"─────────────────────────────}\SpecialCharTok{\textbackslash{}n}\StringTok{"}\NormalTok{)}
\end{Highlighting}
\end{Shaded}

\begin{verbatim}
## ─────────────────────────────
\end{verbatim}

\begin{Shaded}
\begin{Highlighting}[]
\FunctionTok{cat}\NormalTok{(}\StringTok{"Part mass:"}\NormalTok{, part\_mass, }\StringTok{"kg}\SpecialCharTok{\textbackslash{}n}\StringTok{"}\NormalTok{)}
\end{Highlighting}
\end{Shaded}

\begin{verbatim}
## Part mass: 3 kg
\end{verbatim}

\begin{Shaded}
\begin{Highlighting}[]
\FunctionTok{cat}\NormalTok{(}\StringTok{"Weight force:"}\NormalTok{, }\FunctionTok{round}\NormalTok{(weight\_force, }\DecValTok{1}\NormalTok{), }\StringTok{"N}\SpecialCharTok{\textbackslash{}n}\StringTok{"}\NormalTok{)}
\end{Highlighting}
\end{Shaded}

\begin{verbatim}
## Weight force: 29.4 N
\end{verbatim}

\begin{Shaded}
\begin{Highlighting}[]
\FunctionTok{cat}\NormalTok{(}\StringTok{"Inertia force:"}\NormalTok{, }\FunctionTok{round}\NormalTok{(inertia\_force, }\DecValTok{1}\NormalTok{), }\StringTok{"N}\SpecialCharTok{\textbackslash{}n}\StringTok{"}\NormalTok{)}
\end{Highlighting}
\end{Shaded}

\begin{verbatim}
## Inertia force: 45 N
\end{verbatim}

\begin{Shaded}
\begin{Highlighting}[]
\FunctionTok{cat}\NormalTok{(}\StringTok{"Combined force:"}\NormalTok{, }\FunctionTok{round}\NormalTok{(total\_force, }\DecValTok{1}\NormalTok{), }\StringTok{"N}\SpecialCharTok{\textbackslash{}n}\StringTok{"}\NormalTok{)}
\end{Highlighting}
\end{Shaded}

\begin{verbatim}
## Combined force: 53.8 N
\end{verbatim}

\begin{Shaded}
\begin{Highlighting}[]
\FunctionTok{cat}\NormalTok{(}\StringTok{"Friction coefficient:"}\NormalTok{, friction\_coefficient, }\StringTok{"}\SpecialCharTok{\textbackslash{}n}\StringTok{"}\NormalTok{)}
\end{Highlighting}
\end{Shaded}

\begin{verbatim}
## Friction coefficient: 0.3
\end{verbatim}

\begin{Shaded}
\begin{Highlighting}[]
\FunctionTok{cat}\NormalTok{(}\StringTok{"Safety factor:"}\NormalTok{, safety\_factor, }\StringTok{"}\SpecialCharTok{\textbackslash{}n}\StringTok{"}\NormalTok{)}
\end{Highlighting}
\end{Shaded}

\begin{verbatim}
## Safety factor: 2
\end{verbatim}

\begin{Shaded}
\begin{Highlighting}[]
\FunctionTok{cat}\NormalTok{(}\StringTok{"─────────────────────────────}\SpecialCharTok{\textbackslash{}n}\StringTok{"}\NormalTok{)}
\end{Highlighting}
\end{Shaded}

\begin{verbatim}
## ─────────────────────────────
\end{verbatim}

\begin{Shaded}
\begin{Highlighting}[]
\FunctionTok{cat}\NormalTok{(}\StringTok{"Required grip force:"}\NormalTok{, }\FunctionTok{round}\NormalTok{(required\_grip, }\DecValTok{1}\NormalTok{), }\StringTok{"N}\SpecialCharTok{\textbackslash{}n}\StringTok{"}\NormalTok{)}
\end{Highlighting}
\end{Shaded}

\begin{verbatim}
## Required grip force: 358.5 N
\end{verbatim}

\begin{Shaded}
\begin{Highlighting}[]
\FunctionTok{cat}\NormalTok{(}\StringTok{"}\SpecialCharTok{\textbackslash{}n}\StringTok{Select gripper with grip force ≥"}\NormalTok{, }\FunctionTok{ceiling}\NormalTok{(required\_grip}\SpecialCharTok{/}\DecValTok{10}\NormalTok{)}\SpecialCharTok{*}\DecValTok{10}\NormalTok{, }\StringTok{"N}\SpecialCharTok{\textbackslash{}n}\StringTok{"}\NormalTok{)}
\end{Highlighting}
\end{Shaded}

\begin{verbatim}
## 
## Select gripper with grip force ≥ 360 N
\end{verbatim}

\begin{center}\rule{0.5\linewidth}{0.5pt}\end{center}

\section{Coordinate Systems and Motion}\label{coordinate-systems-and-motion}

Understanding coordinate systems is essential for robot programming.

\subsection{Coordinate Frames}\label{coordinate-frames}

\pandocbounded{\includegraphics[keepaspectratio]{introduction_files/figure-latex/coordinate-frames-1.pdf}}

\subsection{Motion Types}\label{motion-types}

\begin{table}
\centering
\caption{\label{tab:motion-types}Robot Motion Types}
\centering
\begin{tabu} to \linewidth {>{\raggedright}X>{\raggedright}X>{\raggedright}X>{\raggedright}X>{\raggedright}X}
\hline
Motion Type & Path Characteristic & Speed Control & When to Use & Cycle Time\\
\hline
\textbf{Joint Move (MoveJ)} & Unpredictable path; each joint moves at constant speed & Joint velocity (\% or deg/s) & Fast point-to-point; collision-free space; approach moves & Fastest\\
\hline
\textbf{Linear Move (MoveL)} & Straight line from start to end; TCP follows linear path & TCP velocity (mm/s) & Process paths (welding, sealing); precise positioning & Moderate\\
\hline
\textbf{Circular Move (MoveC)} & Arc or circle; defined by start, via, and end points & TCP velocity (mm/s) & Arc welding; edge following; contoured surfaces & Slowest\\
\hline
\end{tabu}
\end{table}

\pandocbounded{\includegraphics[keepaspectratio]{introduction_files/figure-latex/motion-visualization-1.pdf}}

\begin{center}\rule{0.5\linewidth}{0.5pt}\end{center}

\section{Robot Applications}\label{robot-applications}

\subsection{Automotive Applications}\label{automotive-applications}

\pandocbounded{\includegraphics[keepaspectratio]{introduction_files/figure-latex/auto-applications-1.pdf}}

\subsection{Food \& Beverage Applications}\label{food-beverage-applications}

\begin{table}
\centering
\caption{\label{tab:food-applications}Food & Beverage Robot Applications}
\centering
\begin{tabu} to \linewidth {>{\raggedright}X>{\raggedright}X>{\raggedright}X>{\raggedright}X}
\hline
Application & Description & Robot Type & Special Requirements\\
\hline
\textbf{Palletizing} & Stack cases/bags onto pallets in patterns & 4-axis palletizer, 6-axis & IP65+, washdown, food-safe lubricants\\
\hline
\textbf{Case Packing} & Load products into cases/cartons & Delta, SCARA, 6-axis & High speed, gentle handling\\
\hline
\textbf{Pick and Place} & Transfer products between conveyors/stations & Delta (high speed), SCARA & Extremely fast cycles, hygienic design\\
\hline
\textbf{Cutting/Portioning} & Protein portioning, cake cutting & 6-axis with vision & Precise cutting, washdown capable\\
\hline
\textbf{Decorating} & Applying icing, toppings, decorations & 6-axis, delta & Food-grade materials, precise dispensing\\
\hline
\textbf{Quality Inspection} & Vision-based defect detection, grading & 6-axis with vision & Hygienic design, reliable detection\\
\hline
\end{tabu}
\end{table}

\subsection{Aerospace/Defense Applications}\label{aerospacedefense-applications}

\begin{table}
\centering
\caption{\label{tab:aerospace-applications}Aerospace/Defense Robot Applications}
\centering
\begin{tabu} to \linewidth {>{\raggedright}X>{\raggedright}X>{\raggedright}X>{\raggedright}X}
\hline
Application & Description & Accuracy Required & Challenges\\
\hline
\textbf{Drilling \& Fastening} & Precision hole drilling; rivet/bolt installation & ±0.05mm position & Drill normal to curved surfaces; chip management\\
\hline
\textbf{Composite Layup} & Automated fiber placement (AFP); tape laying & ±0.25mm fiber placement & Complex contours; material handling\\
\hline
\textbf{NDT/Inspection} & Ultrasonic, X-ray, visual inspection of structures & Full coverage & Large part access; data management\\
\hline
\textbf{Surface Treatment} & Painting, coating, surface prep for bonding & Consistent thickness & Large envelopes; environmental control\\
\hline
\textbf{Assembly} & Large structure assembly; wing-to-fuselage & ±0.5mm & Heavy payloads; coordination of multiple robots\\
\hline
\end{tabu}
\end{table}

\begin{center}\rule{0.5\linewidth}{0.5pt}\end{center}

\section{Robot Safety}\label{robot-safety}

Robot safety is critical to protect workers from potential hazards.

\subsection{Hazard Categories}\label{hazard-categories}

\begin{table}
\centering
\caption{\label{tab:robot-hazards}Robot Hazard Categories and Mitigations}
\centering
\begin{tabu} to \linewidth {>{\raggedright}X>{\raggedright}X>{\raggedright}X>{\raggedright}X}
\hline
Hazard & Description & Risk\_Factors & Mitigation\\
\hline
\textbf{Impact} & Robot strikes person with arm or tooling & Speed, mass, sharp edges, unexpected motion & Guarding, reduced speed zones, sensors\\
\hline
\textbf{Crushing} & Person caught between robot and fixed object & Limited escape space, high forces & Minimum clearances, presence detection\\
\hline
\textbf{Trapping} & Body part caught in articulating joints or mechanisms & Pinch points at joints, insufficient clearance & Joint covers, clearance design\\
\hline
\textbf{Ejection} & Part or tool flies off due to grip failure or breakage & Grip force, centrifugal force, tool condition & Grip verification, part detection, enclosure\\
\hline
\textbf{Other} & Electrical, thermal, noise, radiation hazards & Voltage, heat, welding arc, laser & Proper grounding, barriers, PPE\\
\hline
\end{tabu}
\end{table}

\subsection{Safety Standards}\label{safety-standards}

\begin{table}
\centering
\caption{\label{tab:safety-standards}Key Robot Safety Standards}
\centering
\begin{tabu} to \linewidth {>{\raggedright}X>{\raggedright}X>{\raggedright}X}
\hline
Standard & Scope & Key Requirements\\
\hline
\textbf{ISO 10218-1:2011} & Robot manufacturer requirements & Stop functions, speed limiting, singularity protection\\
\hline
\textbf{ISO 10218-2:2011} & Robot system integrator requirements & Risk assessment, safeguarding, layout, validation\\
\hline
\textbf{ISO/TS 15066:2016} & Collaborative robot safety & Force/pressure limits, speed/separation monitoring\\
\hline
\textbf{ANSI/RIA R15.06} & US equivalent to ISO 10218 & Safeguarding devices, risk assessment\\
\hline
\textbf{ANSI/RIA R15.08} & Industrial mobile robots & Navigation safety, personnel detection\\
\hline
\end{tabu}
\end{table}

\subsection{Safeguarding Methods}\label{safeguarding-methods}

\pandocbounded{\includegraphics[keepaspectratio]{introduction_files/figure-latex/safeguarding-1.pdf}}

\begin{center}\rule{0.5\linewidth}{0.5pt}\end{center}

\section{Collaborative Robots (Cobots)}\label{collaborative-robots-cobots}

\textbf{Collaborative robots} are designed to work safely alongside humans without traditional guarding.

\subsection{Cobot Characteristics}\label{cobot-characteristics}

\begin{table}
\centering
\caption{\label{tab:cobot-comparison}Traditional Robot vs. Collaborative Robot Comparison}
\centering
\begin{tabu} to \linewidth {>{\raggedright}X>{\raggedright}X>{\raggedright}X}
\hline
 & Traditional Industrial Robot & Collaborative Robot\\
\hline
\textbf{Safety Approach} & \cellcolor[HTML]{fadbd8}{Safeguarded cell; humans excluded} & \cellcolor[HTML]{d5f5e3}{Inherently safe; force/speed limited}\\
\hline
\textbf{Payload} & \cellcolor[HTML]{fadbd8}{Up to 2300 kg} & \cellcolor[HTML]{d5f5e3}{Typically 3-35 kg}\\
\hline
\textbf{Speed} & \cellcolor[HTML]{fadbd8}{Very high (>2 m/s)} & \cellcolor[HTML]{d5f5e3}{Limited (<1.5 m/s typically)}\\
\hline
\textbf{Reach} & \cellcolor[HTML]{fadbd8}{Up to 4 m} & \cellcolor[HTML]{d5f5e3}{500-1300 mm typical}\\
\hline
\textbf{Programming} & \cellcolor[HTML]{fadbd8}{Specialized programmers} & \cellcolor[HTML]{d5f5e3}{Intuitive; hand guiding}\\
\hline
\textbf{Guarding} & \cellcolor[HTML]{fadbd8}{Required (fences, curtains)} & \cellcolor[HTML]{d5f5e3}{Often not required}\\
\hline
\textbf{Investment} & \cellcolor[HTML]{fadbd8}{High (\$50K-500K+ with cell)} & \cellcolor[HTML]{d5f5e3}{Lower (\$25K-75K typical)}\\
\hline
\textbf{Applications} & \cellcolor[HTML]{fadbd8}{High-volume, dedicated tasks} & \cellcolor[HTML]{d5f5e3}{Flexible, shared workspace}\\
\hline
\end{tabu}
\end{table}

\subsection{Collaborative Operation Modes (ISO/TS 15066)}\label{collaborative-operation-modes-isots-15066}

\begin{table}
\centering
\caption{\label{tab:cobot-modes}ISO/TS 15066 Collaborative Operation Modes}
\centering
\begin{tabu} to \linewidth {>{\raggedright}X>{\raggedright}X>{\raggedright}X>{\raggedright}X}
\hline
Collaboration Mode & Description & Application & Requirements\\
\hline
\textbf{Safety-Rated Monitored Stop} & Robot stops when human enters collaborative zone; resumes when clear & Loading/unloading where human enters occasionally & Safety-rated sensors; safe zone definition\\
\hline
\textbf{Hand Guiding} & Operator physically guides robot; robot follows input forces & Teaching positions; flexible positioning tasks & Emergency stop; safe torque control\\
\hline
\textbf{Speed and Separation Monitoring} & Robot adjusts speed based on distance to human; stops if too close & Shared workspace with variable human presence & Distance sensing; certified safety functions\\
\hline
\textbf{Power and Force Limiting} & Robot limits force/pressure on contact to safe levels & Direct human-robot collaboration; assembly assist & Compliant design; verified force limits\\
\hline
\end{tabu}
\end{table}

\subsection{Force and Pressure Limits}\label{force-and-pressure-limits}

ISO/TS 15066 specifies maximum contact forces and pressures for different body areas:

\begin{table}
\centering
\caption{\label{tab:force-limits}ISO/TS 15066 Biomechanical Limits (Selected Values)}
\centering
\begin{tabu} to \linewidth {>{\raggedright}X>{\raggedleft}X>{\raggedleft}X>{\raggedleft}X>{\raggedleft}X}
\hline
Body Area & Max Pressure (N/cm²) & Quasi-Static Pressure & Max Force (N) & Quasi-Static Force\\
\hline
\textbf{Skull/Forehead} & 130 & 130 & 130 & 130\\
\hline
\textbf{Face} & 65 & 65 & 65 & 65\\
\hline
\textbf{Neck (front/back)} & 145 & 145 & 150 & 150\\
\hline
\textbf{Chest} & 140 & 140 & 140 & 140\\
\hline
\textbf{Abdomen} & 110 & 110 & 110 & 110\\
\hline
\textbf{Hand/Finger} & 280 & 280 & 140 & 140\\
\hline
\textbf{Arm} & 190 & 190 & 150 & 150\\
\hline
\textbf{Leg} & 220 & 220 & 220 & 220\\
\hline
\end{tabu}
\end{table}

\begin{center}\rule{0.5\linewidth}{0.5pt}\end{center}

\section{Robot Programming Basics}\label{robot-programming-basics}

\subsection{Programming Methods}\label{programming-methods}

\begin{table}
\centering
\caption{\label{tab:programming-methods}Robot Programming Methods Comparison}
\centering
\begin{tabu} to \linewidth {>{\raggedright}X>{\raggedright}X>{\raggedright}X>{\raggedright}X}
\hline
Method & Description & Advantages & Disadvantages\\
\hline
\textbf{Online/Teach Pendant} & Use teach pendant to jog robot to positions and record & Direct; precise; see actual positions & Production downtime; time-consuming for complex paths\\
\hline
\textbf{Lead-Through/Hand Guiding} & Physically move robot arm to desired positions & Intuitive; fast for simple paths; no programming skill needed & Limited to cobot-style robots; not precise\\
\hline
\textbf{Offline Programming} & Create programs on PC without robot; download later & No production downtime; complex paths; optimization & Requires accurate cell model; calibration needed\\
\hline
\textbf{Simulation-Based} & Program and test in virtual environment; transfer to real robot & Risk-free testing; cycle time estimation; collision checking & Model accuracy critical; license costs\\
\hline
\end{tabu}
\end{table}

\subsection{Basic Program Structure}\label{basic-program-structure}

\begin{verbatim}
## Example Robot Program (ABB RAPID-style pseudocode):
\end{verbatim}

\begin{verbatim}
## ────────────────────────────────────────────────────
\end{verbatim}

\begin{verbatim}
## 
## PROGRAM Main()
## 
##   ! Initialize
##   HomePosition()
##   GripperOpen()
## 
##   ! Main cycle
##   WHILE running DO
## 
##     ! Move to pick position
##     MoveJ pApproach, v500, z50, tool1
##     MoveL pPick, v100, fine, tool1
## 
##     ! Pick part
##     GripperClose()
##     WaitTime 0.2
## 
##     ! Move to place position
##     MoveL pApproach, v200, z50, tool1
##     MoveJ pPlaceApproach, v500, z50, tool1
##     MoveL pPlace, v100, fine, tool1
## 
##     ! Place part
##     GripperOpen()
##     WaitTime 0.2
## 
##     ! Return
##     MoveL pPlaceApproach, v200, z50, tool1
## 
##     ! Increment counter
##     PartCount := PartCount + 1
## 
##   ENDWHILE
## 
## ENDPROGRAM
\end{verbatim}

\subsection{Key Programming Concepts}\label{key-programming-concepts}

\begin{table}
\centering
\caption{\label{tab:programming-concepts}Key Robot Programming Concepts}
\centering
\begin{tabu} to \linewidth {>{\raggedright}X>{\raggedright}X>{\raggedright}X}
\hline
Concept & Description & Example\\
\hline
\textbf{Position/Target} & Stored robot position (joint angles or Cartesian coordinates) & pHome, pPick, pPlace\\
\hline
\textbf{Motion Instruction} & Command to move robot (MoveJ, MoveL, MoveC) & MoveL pTarget, v100, fine, tool1\\
\hline
\textbf{Speed Data} & Velocity parameter (mm/s for TCP, \% for joints) & v100 = 100 mm/s, v500 = 500 mm/s\\
\hline
\textbf{Zone Data} & Corner path blending (fine = stop at point, z10 = 10mm blend) & fine = stop, z5 = 5mm radius blend\\
\hline
\textbf{Tool Data} & Definition of tool center point relative to flange & tool1 with TCP offset [0, 0, 150, 0, 0, 0]\\
\hline
\textbf{Work Object} & Definition of work coordinate system for part positions & wobj\_fixture with base offset\\
\hline
\textbf{I/O Commands} & Control external devices (SetDO, WaitDI, etc.) & SetDO doGripper, 1; WaitDI diPartPresent, 1\\
\hline
\end{tabu}
\end{table}

\begin{center}\rule{0.5\linewidth}{0.5pt}\end{center}

\section{Video Resources}\label{video-resources-4}

\subsection{Introduction to Industrial Robots}\label{introduction-to-industrial-robots}

\subsection{Collaborative Robots Explained}\label{collaborative-robots-explained}

\begin{center}\rule{0.5\linewidth}{0.5pt}\end{center}

\section{Summary}\label{summary-13}

Industrial robotics is a fundamental technology in modern manufacturing:

\begin{enumerate}
\def\labelenumi{\arabic{enumi}.}
\tightlist
\item
  \textbf{Robot types} include articulated (most versatile), SCARA (fast assembly), delta (high-speed picking), and cartesian (large work areas)
\item
  \textbf{Key specifications} include payload, reach, repeatability, speed, and degrees of freedom
\item
  \textbf{System components} work together: manipulator, controller, teach pendant, end effector, and safety system
\item
  \textbf{Coordinate systems} (world, tool, user) enable flexible programming and positioning
\item
  \textbf{Applications} span automotive, food \& beverage, and aerospace with industry-specific requirements
\item
  \textbf{Safety} requires comprehensive risk assessment and appropriate safeguarding
\item
  \textbf{Collaborative robots} enable human-robot cooperation with force/speed limiting
\item
  \textbf{Programming} can be online (teach pendant), lead-through, or offline
\end{enumerate}

\begin{center}\rule{0.5\linewidth}{0.5pt}\end{center}

\section{Review Questions}\label{review-questions-14}

\textbf{Question 1}: Compare articulated (6-axis) robots, SCARA robots, and delta robots. For each, describe the typical applications and key advantages.

\textbf{Answer:}

\begin{longtable}[]{@{}
  >{\raggedright\arraybackslash}p{(\linewidth - 6\tabcolsep) * \real{0.1875}}
  >{\raggedright\arraybackslash}p{(\linewidth - 6\tabcolsep) * \real{0.2344}}
  >{\raggedright\arraybackslash}p{(\linewidth - 6\tabcolsep) * \real{0.2500}}
  >{\raggedright\arraybackslash}p{(\linewidth - 6\tabcolsep) * \real{0.3281}}@{}}
\toprule\noalign{}
\begin{minipage}[b]{\linewidth}\raggedright
Robot Type
\end{minipage} & \begin{minipage}[b]{\linewidth}\raggedright
Configuration
\end{minipage} & \begin{minipage}[b]{\linewidth}\raggedright
Key Advantages
\end{minipage} & \begin{minipage}[b]{\linewidth}\raggedright
Typical Applications
\end{minipage} \\
\midrule\noalign{}
\endhead
\bottomrule\noalign{}
\endlastfoot
\textbf{6-Axis Articulated} & 6 revolute joints; human-like arm & Maximum flexibility; can reach any orientation; large payload range & Welding, painting, assembly, material handling, machine tending \\
\textbf{SCARA} & 4 axes; selective compliance & Very fast horizontal motion; rigid vertical; compact; precise & Pick-and-place, assembly, screw driving, packaging \\
\textbf{Delta (Parallel)} & 3-4 axes; parallel linkage & Extremely fast (150+ picks/min); low moving mass & High-speed picking, packaging, sorting, light assembly \\
\end{longtable}

\textbf{Detailed Comparison:}

\textbf{6-Axis Articulated:}
- Most versatile - can position tool at any angle
- Wide payload range (3 kg to 2300+ kg)
- Can reach around obstacles
- Complex programming possible
- \emph{Best for:} Tasks requiring complex orientations, welding seams, painting

\textbf{SCARA:}
- Inherently rigid in Z (vertical) - good for insertion
- Very fast in X-Y plane
- Lower cost than 6-axis
- Compact footprint
- \emph{Best for:} Assembly operations, circuit board population, packaging

\textbf{Delta:}
- Highest speed of any configuration
- Low inertia (motors at base, not moving)
- Limited payload (typically \textless15 kg)
- Requires overhead mounting
- \emph{Best for:} High-speed sorting, packaging lines, food handling

\textbf{Question 2}: A robot has a maximum payload of 15 kg. The gripper weighs 4 kg and the part weighs 8 kg. Is this robot suitable? What other factors should be considered?

\textbf{Answer:}

\textbf{Initial Assessment:}

\begin{verbatim}
Total load = Gripper weight + Part weight
Total load = 4 kg + 8 kg = 12 kg

Robot capacity: 15 kg
Load: 12 kg
Remaining margin: 3 kg (20%)
\end{verbatim}

\textbf{Consideration with Safety Factor:}

\begin{verbatim}
Typical safety factor: 1.2 - 1.5
Required capacity = 12 kg × 1.25 = 15 kg

This equals the robot's maximum capacity - marginal!
\end{verbatim}

\textbf{Additional Factors to Consider:}

\begin{enumerate}
\def\labelenumi{\arabic{enumi}.}
\tightlist
\item
  \textbf{Moment Load (Torque at Wrist)}

  \begin{itemize}
  \tightlist
  \item
    Distance from flange to center of gravity matters
  \item
    If part is held 300mm from flange: Moment = 8 kg × 0.3 m = 2.4 kg·m
  \item
    Robot specs include moment limits - must verify
  \end{itemize}
\item
  \textbf{Inertia}

  \begin{itemize}
  \tightlist
  \item
    Fast motion requires acceleration
  \item
    Higher mass + longer distance = higher inertia
  \item
    May need to derate payload for high-speed applications
  \end{itemize}
\item
  \textbf{Payload Derating with Reach}

  \begin{itemize}
  \tightlist
  \item
    Many robots derate payload at extended reach
  \item
    15 kg at 500mm may become 10 kg at 1500mm
  \item
    Check payload diagram in robot specifications
  \end{itemize}
\item
  \textbf{Orientation}

  \begin{itemize}
  \tightlist
  \item
    Payload may vary with wrist orientation
  \item
    Horizontal reach vs.~vertical lift differ
  \end{itemize}
\item
  \textbf{Acceleration Requirements}

  \begin{itemize}
  \tightlist
  \item
    Higher acceleration = higher effective load
  \item
    May need to reduce speed/acceleration
  \end{itemize}
\end{enumerate}

\textbf{Recommendation:}
The 15 kg robot is marginal for this application. Consider:
- Select next size up (20 kg) for adequate margin
- Use lighter gripper (3 kg or less)
- Reduce part handling distance from flange
- Limit acceleration/speed if using this robot

\textbf{Question 3}: Explain the difference between repeatability and accuracy. Why is repeatability often more important than accuracy for robot applications?

\textbf{Answer:}

\textbf{Definitions:}

\textbf{Repeatability:}
- Ability to return to the \textbf{same taught position} multiple times
- Measures consistency/precision of motion
- Quoted as ± value (e.g., ±0.05 mm at 3σ)
- Affected by: mechanical wear, servo control, thermal effects

\textbf{Accuracy:}
- Ability to reach a \textbf{commanded position} exactly
- Measures how close actual position is to theoretical position
- Typically worse than repeatability (e.g., ±0.5 mm)
- Affected by: manufacturing tolerances, calibration, deflection, backlash

\textbf{Why Repeatability is Usually More Important:}

\begin{enumerate}
\def\labelenumi{\arabic{enumi}.}
\tightlist
\item
  \textbf{Teaching Method}

  \begin{itemize}
  \tightlist
  \item
    Most robots are programmed by teaching (jogging to positions)
  \item
    Robot doesn't need to know where it ``should'' be - just return to where it was taught
  \item
    As long as it returns consistently (repeatability), accuracy doesn't matter
  \end{itemize}
\item
  \textbf{Process Requirements}

  \begin{itemize}
  \tightlist
  \item
    Welding: Need to follow same seam every time → repeatability
  \item
    Assembly: Need to insert part in same hole every time → repeatability
  \item
    Pick-place: Need to pick from same fixture position → repeatability
  \end{itemize}
\item
  \textbf{Compensation Possible}

  \begin{itemize}
  \tightlist
  \item
    Poor accuracy can be compensated by adjusting taught points
  \item
    Poor repeatability cannot be compensated
  \end{itemize}
\end{enumerate}

\textbf{When Accuracy Matters:}

\begin{enumerate}
\def\labelenumi{\arabic{enumi}.}
\tightlist
\item
  \textbf{Offline Programming}

  \begin{itemize}
  \tightlist
  \item
    Positions calculated from CAD data, not taught
  \item
    Robot must go where commanded, first time
  \item
    Common in aerospace (large parts, tight access)
  \end{itemize}
\item
  \textbf{Multi-Robot Systems}

  \begin{itemize}
  \tightlist
  \item
    Robots must agree on positions
  \item
    Calibration to common frame requires accuracy
  \end{itemize}
\item
  \textbf{Frequent Program Changes}

  \begin{itemize}
  \tightlist
  \item
    If programs are frequently changed without re-teaching
  \item
    Offline modifications need accurate execution
  \end{itemize}
\end{enumerate}

\textbf{Bottom Line:}
For most industrial applications, ±0.05 mm repeatability is far more valuable than ±0.5 mm accuracy because taught positions are used.

\textbf{Question 4}: Describe the four collaborative operation modes defined in ISO/TS 15066. Give an example application for each.

\textbf{Answer:}

\textbf{1. Safety-Rated Monitored Stop}

\emph{Description:} Robot operates at normal speed when workspace is clear. When a person enters the collaborative zone, the robot stops and holds position. Robot resumes automatically when person exits.

\emph{Technical Requirements:}
- Safety-rated sensors (light curtains, scanners)
- Safety-rated monitored stop function
- Clearly defined collaborative zone

\emph{Example Application:} Machine tending cell where operator occasionally loads raw material. Robot works at full speed during machining cycle but stops safely when operator enters to load/unload.

\begin{center}\rule{0.5\linewidth}{0.5pt}\end{center}

\textbf{2. Hand Guiding}

\emph{Description:} Operator physically contacts the robot through a hand guiding device and manually moves it. Robot follows operator's force input.

\emph{Technical Requirements:}
- Hand guiding device (force/torque sensing)
- Emergency stop accessible at guiding device
- Speed reduction while guiding
- Safe torque control

\emph{Example Application:} Teaching a paint path on an aerospace component. Operator guides robot along surface contour while robot records positions. Also used for ``collaborative finishing'' where operator guides robot holding a polishing tool.

\begin{center}\rule{0.5\linewidth}{0.5pt}\end{center}

\textbf{3. Speed and Separation Monitoring}

\emph{Description:} Robot continuously monitors distance to the nearest person. Speed adjusts based on separation: full speed when far, reduced speed when closer, stop if minimum distance breached.

\emph{Technical Requirements:}
- Safety-rated distance monitoring (usually area scanners)
- Safety-rated speed control
- Multiple speed/distance zones defined
- Real-time separation calculation

\emph{Example Application:} Shared packaging area where robot palletizes cases while operator moves around the cell. Robot slows when operator approaches, stops if too close, and resumes at appropriate speed as operator moves away.

\begin{center}\rule{0.5\linewidth}{0.5pt}\end{center}

\textbf{4. Power and Force Limiting}

\emph{Description:} Robot is designed so that forces and pressures from any contact are below injury thresholds. Contact is allowed because it cannot cause harm.

\emph{Technical Requirements:}
- Inherently safe design (compliant joints, low mass, rounded surfaces)
- Force/torque sensing or current monitoring
- Verified force limits per ISO/TS 15066 body area tables
- May include padding/soft covers

\emph{Example Application:} Assembly assist where robot holds a heavy part while human worker installs fasteners. Robot and human are in direct contact, with robot providing position support and human providing dexterity. Force is limited so accidental contact causes no injury.

\textbf{Question 5}: Calculate the required gripper force for a robot picking a 2 kg part with a friction coefficient of 0.4. The robot accelerates at 10 m/s². Use a safety factor of 2.0.

\textbf{Answer:}

\begin{Shaded}
\begin{Highlighting}[]
\CommentTok{\# Given values}
\NormalTok{mass }\OtherTok{\textless{}{-}} \FloatTok{2.0}            \CommentTok{\# kg}
\NormalTok{acceleration }\OtherTok{\textless{}{-}} \DecValTok{10}     \CommentTok{\# m/s²}
\NormalTok{gravity }\OtherTok{\textless{}{-}} \FloatTok{9.81}        \CommentTok{\# m/s²}
\NormalTok{friction\_coef }\OtherTok{\textless{}{-}} \FloatTok{0.4}   \CommentTok{\# Coefficient of friction}
\NormalTok{safety\_factor }\OtherTok{\textless{}{-}} \FloatTok{2.0}   \CommentTok{\# Safety factor}

\CommentTok{\# Calculate forces}
\CommentTok{\# Weight force (vertical)}
\NormalTok{F\_gravity }\OtherTok{\textless{}{-}}\NormalTok{ mass }\SpecialCharTok{*}\NormalTok{ gravity}

\CommentTok{\# Inertia force (from acceleration)}
\NormalTok{F\_inertia }\OtherTok{\textless{}{-}}\NormalTok{ mass }\SpecialCharTok{*}\NormalTok{ acceleration}

\CommentTok{\# Combined force (vector sum for worst case)}
\CommentTok{\# Worst case: acceleration horizontal while holding against gravity}
\NormalTok{F\_combined }\OtherTok{\textless{}{-}} \FunctionTok{sqrt}\NormalTok{(F\_gravity}\SpecialCharTok{\^{}}\DecValTok{2} \SpecialCharTok{+}\NormalTok{ F\_inertia}\SpecialCharTok{\^{}}\DecValTok{2}\NormalTok{)}

\CommentTok{\# Required friction force to hold part}
\CommentTok{\# F\_friction = μ × F\_normal}
\CommentTok{\# F\_normal = Grip force}
\CommentTok{\# F\_friction must overcome F\_combined}

\CommentTok{\# Without safety factor:}
\NormalTok{F\_grip\_min }\OtherTok{\textless{}{-}}\NormalTok{ F\_combined }\SpecialCharTok{/}\NormalTok{ friction\_coef}

\CommentTok{\# With safety factor:}
\NormalTok{F\_grip\_required }\OtherTok{\textless{}{-}}\NormalTok{ F\_grip\_min }\SpecialCharTok{*}\NormalTok{ safety\_factor}

\FunctionTok{cat}\NormalTok{(}\StringTok{"Gripper Force Calculation:}\SpecialCharTok{\textbackslash{}n}\StringTok{"}\NormalTok{)}
\end{Highlighting}
\end{Shaded}

\begin{verbatim}
## Gripper Force Calculation:
\end{verbatim}

\begin{Shaded}
\begin{Highlighting}[]
\FunctionTok{cat}\NormalTok{(}\StringTok{"═══════════════════════════════════════}\SpecialCharTok{\textbackslash{}n\textbackslash{}n}\StringTok{"}\NormalTok{)}
\end{Highlighting}
\end{Shaded}

\begin{verbatim}
## ═══════════════════════════════════════
\end{verbatim}

\begin{Shaded}
\begin{Highlighting}[]
\FunctionTok{cat}\NormalTok{(}\StringTok{"Input Parameters:}\SpecialCharTok{\textbackslash{}n}\StringTok{"}\NormalTok{)}
\end{Highlighting}
\end{Shaded}

\begin{verbatim}
## Input Parameters:
\end{verbatim}

\begin{Shaded}
\begin{Highlighting}[]
\FunctionTok{cat}\NormalTok{(}\StringTok{"  Part mass:"}\NormalTok{, mass, }\StringTok{"kg}\SpecialCharTok{\textbackslash{}n}\StringTok{"}\NormalTok{)}
\end{Highlighting}
\end{Shaded}

\begin{verbatim}
##   Part mass: 2 kg
\end{verbatim}

\begin{Shaded}
\begin{Highlighting}[]
\FunctionTok{cat}\NormalTok{(}\StringTok{"  Robot acceleration:"}\NormalTok{, acceleration, }\StringTok{"m/s²}\SpecialCharTok{\textbackslash{}n}\StringTok{"}\NormalTok{)}
\end{Highlighting}
\end{Shaded}

\begin{verbatim}
##   Robot acceleration: 10 m/s²
\end{verbatim}

\begin{Shaded}
\begin{Highlighting}[]
\FunctionTok{cat}\NormalTok{(}\StringTok{"  Friction coefficient:"}\NormalTok{, friction\_coef, }\StringTok{"}\SpecialCharTok{\textbackslash{}n}\StringTok{"}\NormalTok{)}
\end{Highlighting}
\end{Shaded}

\begin{verbatim}
##   Friction coefficient: 0.4
\end{verbatim}

\begin{Shaded}
\begin{Highlighting}[]
\FunctionTok{cat}\NormalTok{(}\StringTok{"  Safety factor:"}\NormalTok{, safety\_factor, }\StringTok{"}\SpecialCharTok{\textbackslash{}n\textbackslash{}n}\StringTok{"}\NormalTok{)}
\end{Highlighting}
\end{Shaded}

\begin{verbatim}
##   Safety factor: 2
\end{verbatim}

\begin{Shaded}
\begin{Highlighting}[]
\FunctionTok{cat}\NormalTok{(}\StringTok{"Force Analysis:}\SpecialCharTok{\textbackslash{}n}\StringTok{"}\NormalTok{)}
\end{Highlighting}
\end{Shaded}

\begin{verbatim}
## Force Analysis:
\end{verbatim}

\begin{Shaded}
\begin{Highlighting}[]
\FunctionTok{cat}\NormalTok{(}\StringTok{"  Gravity force: F\_g = m × g ="}\NormalTok{, mass, }\StringTok{"×"}\NormalTok{, gravity, }\StringTok{"="}\NormalTok{,}
    \FunctionTok{round}\NormalTok{(F\_gravity, }\DecValTok{1}\NormalTok{), }\StringTok{"N}\SpecialCharTok{\textbackslash{}n}\StringTok{"}\NormalTok{)}
\end{Highlighting}
\end{Shaded}

\begin{verbatim}
##   Gravity force: F_g = m × g = 2 × 9.81 = 19.6 N
\end{verbatim}

\begin{Shaded}
\begin{Highlighting}[]
\FunctionTok{cat}\NormalTok{(}\StringTok{"  Inertia force: F\_i = m × a ="}\NormalTok{, mass, }\StringTok{"×"}\NormalTok{, acceleration, }\StringTok{"="}\NormalTok{,}
    \FunctionTok{round}\NormalTok{(F\_inertia, }\DecValTok{1}\NormalTok{), }\StringTok{"N}\SpecialCharTok{\textbackslash{}n}\StringTok{"}\NormalTok{)}
\end{Highlighting}
\end{Shaded}

\begin{verbatim}
##   Inertia force: F_i = m × a = 2 × 10 = 20 N
\end{verbatim}

\begin{Shaded}
\begin{Highlighting}[]
\FunctionTok{cat}\NormalTok{(}\StringTok{"  Combined force: F\_c = √(F\_g² + F\_i²) ="}\NormalTok{, }\FunctionTok{round}\NormalTok{(F\_combined, }\DecValTok{1}\NormalTok{), }\StringTok{"N}\SpecialCharTok{\textbackslash{}n\textbackslash{}n}\StringTok{"}\NormalTok{)}
\end{Highlighting}
\end{Shaded}

\begin{verbatim}
##   Combined force: F_c = √(F_g² + F_i²) = 28 N
\end{verbatim}

\begin{Shaded}
\begin{Highlighting}[]
\FunctionTok{cat}\NormalTok{(}\StringTok{"Grip Force Calculation:}\SpecialCharTok{\textbackslash{}n}\StringTok{"}\NormalTok{)}
\end{Highlighting}
\end{Shaded}

\begin{verbatim}
## Grip Force Calculation:
\end{verbatim}

\begin{Shaded}
\begin{Highlighting}[]
\FunctionTok{cat}\NormalTok{(}\StringTok{"  F\_friction = μ × F\_grip ≥ F\_combined}\SpecialCharTok{\textbackslash{}n}\StringTok{"}\NormalTok{)}
\end{Highlighting}
\end{Shaded}

\begin{verbatim}
##   F_friction = μ × F_grip ≥ F_combined
\end{verbatim}

\begin{Shaded}
\begin{Highlighting}[]
\FunctionTok{cat}\NormalTok{(}\StringTok{"  F\_grip ≥ F\_combined / μ ="}\NormalTok{, }\FunctionTok{round}\NormalTok{(F\_combined, }\DecValTok{1}\NormalTok{), }\StringTok{"/"}\NormalTok{, friction\_coef,}
    \StringTok{"="}\NormalTok{, }\FunctionTok{round}\NormalTok{(F\_grip\_min, }\DecValTok{1}\NormalTok{), }\StringTok{"N}\SpecialCharTok{\textbackslash{}n}\StringTok{"}\NormalTok{)}
\end{Highlighting}
\end{Shaded}

\begin{verbatim}
##   F_grip ≥ F_combined / μ = 28 / 0.4 = 70 N
\end{verbatim}

\begin{Shaded}
\begin{Highlighting}[]
\FunctionTok{cat}\NormalTok{(}\StringTok{"  With safety factor:"}\NormalTok{, }\FunctionTok{round}\NormalTok{(F\_grip\_min, }\DecValTok{1}\NormalTok{), }\StringTok{"×"}\NormalTok{, safety\_factor,}
    \StringTok{"="}\NormalTok{, }\FunctionTok{round}\NormalTok{(F\_grip\_required, }\DecValTok{1}\NormalTok{), }\StringTok{"N}\SpecialCharTok{\textbackslash{}n\textbackslash{}n}\StringTok{"}\NormalTok{)}
\end{Highlighting}
\end{Shaded}

\begin{verbatim}
##   With safety factor: 70 × 2 = 140.1 N
\end{verbatim}

\begin{Shaded}
\begin{Highlighting}[]
\FunctionTok{cat}\NormalTok{(}\StringTok{"═══════════════════════════════════════}\SpecialCharTok{\textbackslash{}n}\StringTok{"}\NormalTok{)}
\end{Highlighting}
\end{Shaded}

\begin{verbatim}
## ═══════════════════════════════════════
\end{verbatim}

\begin{Shaded}
\begin{Highlighting}[]
\FunctionTok{cat}\NormalTok{(}\StringTok{"REQUIRED GRIPPER FORCE:"}\NormalTok{, }\FunctionTok{ceiling}\NormalTok{(F\_grip\_required), }\StringTok{"N minimum}\SpecialCharTok{\textbackslash{}n}\StringTok{"}\NormalTok{)}
\end{Highlighting}
\end{Shaded}

\begin{verbatim}
## REQUIRED GRIPPER FORCE: 141 N minimum
\end{verbatim}

\begin{Shaded}
\begin{Highlighting}[]
\FunctionTok{cat}\NormalTok{(}\StringTok{"Select gripper rated for ≥"}\NormalTok{, }\FunctionTok{ceiling}\NormalTok{(F\_grip\_required}\SpecialCharTok{/}\DecValTok{10}\NormalTok{)}\SpecialCharTok{*}\DecValTok{10}\NormalTok{, }\StringTok{"N}\SpecialCharTok{\textbackslash{}n}\StringTok{"}\NormalTok{)}
\end{Highlighting}
\end{Shaded}

\begin{verbatim}
## Select gripper rated for ≥ 150 N
\end{verbatim}

\textbf{Question 6}: List and describe five safety devices used to protect personnel around industrial robots.

\textbf{Answer:}

\textbf{1. Physical Barriers (Safety Fencing)}

\emph{Description:} Fixed guards or fencing that physically prevent access to the robot work envelope.

\emph{Characteristics:}
- Rigid construction (steel, aluminum, polycarbonate)
- Must withstand expected impact forces
- Minimum height (typically ≥1.8 m)
- Properly anchored to floor
- May include interlocked gates for authorized entry

\emph{Application:} Primary safeguard for most traditional robot cells.

\begin{center}\rule{0.5\linewidth}{0.5pt}\end{center}

\textbf{2. Light Curtains (AOPD - Active Opto-electronic Protective Device)}

\emph{Description:} Transmitter and receiver create a sensing field of infrared beams. Breaking any beam triggers a stop.

\emph{Characteristics:}
- Type 2 or Type 4 (safety-rated)
- Resolution (14mm for finger, 40mm for hand/arm)
- Muting capability for material transfer
- Very fast response (\textless20ms)

\emph{Application:} Access points where frequent entry needed; allows material flow while protecting personnel.

\begin{center}\rule{0.5\linewidth}{0.5pt}\end{center}

\textbf{3. Safety Mats (Pressure-Sensitive Mats)}

\emph{Description:} Floor mats that detect weight/pressure and trigger stop when stepped on.

\emph{Characteristics:}
- Placed in hazardous zones
- Withstand industrial environment (oil, debris)
- Fast response
- Self-monitoring for failures

\emph{Application:} Areas where overhead detection isn't practical; around robot bases.

\begin{center}\rule{0.5\linewidth}{0.5pt}\end{center}

\textbf{4. Area Scanners (Safety Laser Scanners)}

\emph{Description:} Laser scanner detects objects/personnel entering defined zones. Multiple warning and stop zones configurable.

\emph{Characteristics:}
- 270° field of view typical
- Multiple programmable zones
- Warning zone (slow robot) and stop zone (stop robot)
- Type 3 safety rating
- Detects at floor level

\emph{Application:} Open cells requiring flexibility; mobile robot protection; speed/separation monitoring.

\begin{center}\rule{0.5\linewidth}{0.5pt}\end{center}

\textbf{5. Interlocked Gates/Doors}

\emph{Description:} Access gates with safety switches that stop robot when opened.

\emph{Characteristics:}
- Mechanical or magnetic interlock switches
- Trapped key systems for lockout
- May include guard locking (prevent opening while robot moving)
- Escape release from inside
- Meets ISO 14119

\emph{Application:} Authorized entry points for maintenance, setup, troubleshooting.

\begin{center}\rule{0.5\linewidth}{0.5pt}\end{center}

\textbf{Additional Devices:}

\begin{itemize}
\tightlist
\item
  \textbf{E-Stop (Emergency Stop)}: Palm-operated buttons to immediately stop robot
\item
  \textbf{Two-Hand Controls}: Require both hands on buttons to initiate motion
\item
  \textbf{Enabling Devices}: 3-position switches held by personnel in restricted area
\item
  \textbf{Presence-Sensing Devices}: Capacitive, radar, or vision-based detection
\end{itemize}

\textbf{Question 7}: Explain the difference between MoveJ (Joint Move) and MoveL (Linear Move). When would you use each?

\textbf{Answer:}

\textbf{MoveJ (Joint Move)}

\emph{How It Works:}
- Each joint moves from start to end angle independently
- All joints start and stop together (coordinated)
- Path of TCP (Tool Center Point) is unpredictable - curved
- Robot calculates joint interpolation, not TCP path

\emph{Characteristics:}
- Fastest point-to-point motion
- No singularity issues during move
- Path may vary depending on starting configuration
- Speed specified in \% of max or joint velocity (°/s)

\emph{When to Use:}
- Moving between distant points through free space
- Approach movements before precise positioning
- Home position moves
- Any time when path doesn't matter, only endpoints
- Escaping from near-singularity positions

\begin{center}\rule{0.5\linewidth}{0.5pt}\end{center}

\textbf{MoveL (Linear Move)}

\emph{How It Works:}
- TCP follows a straight line from start to end
- Robot calculates joint positions to maintain linear TCP path
- Orientation is interpolated linearly as well
- All axes coordinated to produce straight line

\emph{Characteristics:}
- Guaranteed straight path
- Consistent, predictable motion
- Slower than MoveJ for same endpoint distance
- Can encounter singularities during move
- Speed specified in mm/s at TCP

\emph{When to Use:}
- Process paths (welding, sealing, cutting)
- Insertion/extraction operations
- Approach to pick/place positions
- Any time path must be straight
- When moving close to obstacles

\begin{center}\rule{0.5\linewidth}{0.5pt}\end{center}

\textbf{Practical Example:}

Imagine picking a part from a fixture:

\begin{verbatim}
! Move from home to near pick position - path doesn't matter
MoveJ pApproach, v1000, z50, tool1    ! Joint move - fast

! Move down to pick - must be straight to avoid collision
MoveL pPick, v100, fine, tool1        ! Linear move - controlled

! Pick part, then retract straight up
GripperClose()
MoveL pApproach, v200, z10, tool1     ! Linear move - straight up

! Move to place area - through open space
MoveJ pPlaceApproach, v1000, z50, tool1  ! Joint move - fast
\end{verbatim}

\textbf{Key Decision:}
- If path matters → MoveL
- If only destination matters → MoveJ

\textbf{Question 8}: A manufacturing engineer needs to select a robot for a palletizing application. The requirements are: 25 kg payload, 2.2 m horizontal reach, 10 cycles per minute, floor-mounted. Recommend a robot type and justify your selection.

\textbf{Answer:}

\textbf{Application Analysis:}

\begin{longtable}[]{@{}
  >{\raggedright\arraybackslash}p{(\linewidth - 4\tabcolsep) * \real{0.3939}}
  >{\raggedright\arraybackslash}p{(\linewidth - 4\tabcolsep) * \real{0.2121}}
  >{\raggedright\arraybackslash}p{(\linewidth - 4\tabcolsep) * \real{0.3939}}@{}}
\toprule\noalign{}
\begin{minipage}[b]{\linewidth}\raggedright
Requirement
\end{minipage} & \begin{minipage}[b]{\linewidth}\raggedright
Value
\end{minipage} & \begin{minipage}[b]{\linewidth}\raggedright
Implication
\end{minipage} \\
\midrule\noalign{}
\endhead
\bottomrule\noalign{}
\endlastfoot
Payload & 25 kg & Medium-heavy; rules out delta and most SCARA \\
Reach & 2.2 m & Large work envelope needed \\
Cycle rate & 10/min & 6 seconds/cycle; moderate speed \\
Mounting & Floor & Typical industrial installation \\
Motion type & Palletizing & Mainly vertical and horizontal; some rotation \\
\end{longtable}

\textbf{Robot Type Evaluation:}

\textbf{1. 4-Axis Palletizing Robot (RECOMMENDED)}
- Specifically designed for palletizing
- 4 axes: base rotation, arm, elbow, wrist
- High payload capability (25-700+ kg available)
- Long reach available (up to 3.2m)
- Fast for palletizing motions
- Lower cost than 6-axis
- Examples: FANUC M-410iC, KUKA KR QUANTEC PA, ABB IRB 460

\emph{Verdict: Excellent match for this application}

\textbf{2. 6-Axis Articulated Robot}
- Maximum flexibility
- Can handle complex pallet patterns
- Higher cost than 4-axis for same payload
- May be over-specified for simple palletizing
- Examples: FANUC M-710iC/50, ABB IRB 6700

\emph{Verdict: Capable but potentially over-specified and more expensive}

\textbf{3. Cartesian/Gantry Robot}
- Can cover very large areas
- Simple motion control
- Higher installation complexity
- Less common for palletizing applications

\emph{Verdict: Possible but not ideal for this payload/cycle rate}

\textbf{Recommended Selection:}

\textbf{FANUC M-410iC/185} or similar 4-axis palletizer
- Payload: 185 kg (well above 25 kg requirement with margin for gripper)
- Reach: 3.143 m (exceeds 2.2 m requirement)
- Repeatability: ±0.5 mm
- Axes: 4

\textbf{Justification:}
1. \textbf{Purpose-built}: Designed specifically for palletizing
2. \textbf{Adequate payload}: 185 kg allows for heavy gripper + multiple parts
3. \textbf{Reach}: 3.1 m provides margin for future needs
4. \textbf{Speed}: Optimized kinematics for palletizing motion profile
5. \textbf{Reliability}: Proven in thousands of palletizing applications
6. \textbf{Cost}: Lower than equivalent 6-axis robot
7. \textbf{Simplicity}: 4 axes easier to program and maintain than 6

\textbf{Additional Recommendations:}
- Vacuum or mechanical gripper depending on product
- Vision system for product detection if variety exists
- Safety scanner for operator access during pallet changes
- Interface to conveyor for product delivery

\begin{center}\rule{0.5\linewidth}{0.5pt}\end{center}

\section{References}\label{references-14}

\begin{enumerate}
\def\labelenumi{\arabic{enumi}.}
\item
  Siciliano, B., \& Khatib, O. (Eds.). (2016). \emph{Springer Handbook of Robotics} (2nd ed.). Springer.
\item
  ISO 10218-1:2011. \emph{Robots and Robotic Devices - Safety Requirements for Industrial Robots - Part 1: Robots}.
\item
  ISO 10218-2:2011. \emph{Robots and Robotic Devices - Safety Requirements for Industrial Robots - Part 2: Robot Systems and Integration}.
\item
  ISO/TS 15066:2016. \emph{Robots and Robotic Devices - Collaborative Robots}.
\item
  International Federation of Robotics. (2023). \emph{World Robotics Report}. IFR.
\item
  Niku, S.B. (2020). \emph{Introduction to Robotics: Analysis, Control, Applications} (3rd ed.). Wiley.
\item
  Craig, J.J. (2018). \emph{Introduction to Robotics: Mechanics and Control} (4th ed.). Pearson.
\item
  FANUC Corporation. (2023). \emph{Robot Technical Manuals}. FANUC.
\item
  ABB Robotics. (2023). \emph{Technical Reference Manual - RAPID Instructions}. ABB.
\item
  Robotic Industries Association. (2023). \emph{ANSI/RIA R15.06-2012: Industrial Robots and Robot Systems - Safety Requirements}. RIA.
\end{enumerate}

\bibliography{references.bib}

\end{document}
